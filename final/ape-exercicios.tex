\chapter{Exercícios}
\label{ape:exercicios}

Os exercícios são os mesmos para todos os grupos. O participante, em caso de dúvidas, poderá perguntar ao pesquisador.

\section{Lembrete ao participante}

Caro participante,

Lembre-se que os problemas aqui propostos simulam complicações do mundo real. 
Ao resolvê-los, tenha em mente que esses códigos serão futuramente mantidos
por você ou até por uma equipe maior.

Tente criar o design mais elegante possível em todas as soluções. Por serem problemas
recorrentes, imagine que amanhã esse mesmo problema se repetirá.
Escreva um código flexível o suficiente para que novas mudanças sejam fáceis de serem 
implementadas.

Você tem 50 minutos por exercício. Mas lembre-se de focar na qualidade. Um exercício de 
qualidade pela metade é mais importante do que um exercício completo, sem qualidade.

Lembre-se: dê o melhor de si em ambos os exercícios!

\section{Exercício 1 - Calculadora de Salário}

O participante deve implementar uma calculadora de salário de funcionários. Um
funcionário contém nome, e-mail, salário-base e cargo. De acordo com seu cargo,
a regra para cálculo do salário líquido é diferente:

\begin{enumerate}
	\item Caso o cargo seja DESENVOLVEDOR, o funcionário terá desconto de 20\%
	caso o salário seja maior ou igual que 3.000,00, ou apenas 10\% caso o salário seja menor 
	que isso.
	
	\item Caso o cargo seja DBA, o funcionário terá desconto de 25\%
	caso o salário seja maior ou igual que 2.000,00, ou apenas 15\% caso o salário seja menor 
	que isso.

	\item Caso o cargo seja TESTADOR, o funcionário terá desconto de 25\%
	caso o salário seja maior ou igual que 2.000,00, ou apenas 15\% caso o salário seja menor 
	que isso.
	
	\item Caso o cargo seja GERENTE, o funcionário terá desconto de 30\%
	caso o salário seja maior ou igual que 5.000,00, ou apenas 20\% caso o salário seja menor 
	que isso.
\end{enumerate}

Exemplos de cálculo do imposto:

\begin{itemize}
	\item DESENVOLVEDOR com salário-base 5,000.00. Salário final = 4.000,00
	\item GERENTE com salário-base de 2.500,00. Salário final: 2.000,00
	\item TESTADOR com salário de 550.00. Salário final: 467,50
\end{itemize}


O participante deve criar todo o código responsável para esse cálculo. Uma classe com
o método "main()" deverá ser entregue ao final, com exemplo de uso das classes criadas.

\section{Exercício 2 - Gerador de Nota Fiscal}

O participante deve implementar um sistema de geração de nota fiscal a partir de uma fatura. 
Uma fatura contém o nome e endereço do cliente, tipo do serviço e valor da fatura. O gerador de
nota fiscal deverá gerar uma nota fiscal que contém nome do cliente, valor da nota e valor
do imposto a ser pago.

O valor da nota é o mesmo do valor da fatura. Já o cálculo do imposto a ser pago deve seguir
as seguintes regras:

\begin{enumerate}
	\item Caso o serviço seja do tipo "CONSULTORIA", o valor do imposto é de 25%;
	\item Caso o serviço seja do tipo "TREINAMENTO", o valor do imposto é 15%;
	\item Qualquer outro, o valor do imposto é 6%.
\end{enumerate}

Ao final da geração da nota fiscal, o sistema ainda deve enviar essa nota por e-mail,
para o SAP, e persistir na base de dados. Por simplicidade, o desenvolvedor pode usar
os códigos abaixo, que simulam o comportamento do SMTP, SAP e banco de dados:

\begin{lstlisting}
class NotaFiscalDao {
	public void salva(NotaFiscal nf) { 
		System.out.println("salvando no banco"); 
	}
}

class SAP {
	public void envia(NotaFiscal nf) { 
		System.out.println("enviando pro sap"); 
	}
}

class Smtp {
	public void envia(NotaFiscal nf) { 
		System.out.println("enviando por email"); 
	}
}
\end{lstlisting}

O participante é livre para alterar os métodos, parâmetros recebidos ou qualquer outra coisa das classes acima.

Ao final, o participante deve entregar todo o código responsável por geração e encaminhamento da nota fiscal 
para os processos acima citados. Uma classe com o método "main()" deverá ser entregue ao final, com
exemplo de uso das classes criadas.

\section{Exercício 3 - Processador de Boletos}

Nesse exercício, o participante deverá implementar um processador de boletos. O objetivo desse processador
é verificar todos os boletos e, caso o valor da soma de todos os boletos seja maior que o valor
da fatura, então essa fatura deverá ser considerada como paga.

Uma fatura contém data, valor total e nome do cliente. Um boleto cóntem código do boleto, data, e valor pago.

O processador de boletos, ao receber uma lista de boletos, deve então, para cada boleto, criar um
"pagamento" associado a essa fatura. Esse pagamento contém o valor pago, a data, e o tipo do pagamento efetuado
(que nesse caso é "BOLETO").

Como dito anteriormente, caso a soma de todos os boletos ultrapasse o valor da fatura, a mesma deve
ser marcada como "PAGA".

O participante deve criar todo o código responsável pelo processador de boletos. Uma classe com
o método "main()" deverá ser entregue ao final, com exemplo de uso das classes criadas.

Exemplos de processamento:

\begin{itemize}
	\item Fatura de 1.500,00 com 3 boletos no valor de 500,00, 400,00 e 600,00: fatura marcada como PAGA, e três pagamentos do tipo BOLETO criados 
	\item Fatura de 1.500,00 com 3 boletos no valor de 1000,00, 500,00 e 250,00: fatura marcada como PAGA, e três pagamento do tipo BOLETO criados
	\item Fatura de 2.000,00 com 2 boletos no valor de 500,00 e 400,00: fatura não marcada como PAGA, e dois pagamentos do tipo BOLETO criados 
\end{itemize}

\section{Exercício 4 - Filtro de Faturas}

O participante deverá implementar um filtro de faturas. Uma fatura contém um código, um valor, uma data,
e pertence a um cliente. Um cliente tem um nome, data de inclusão e um estado.

O filtro deverá então, dado uma lista de faturas, remover as que se encaixam em algum dos critérios
abaixo:

\begin{itemize}
	\item Se o valor da fatura for menor que 2000;
	\item Se o valor da fatura estiver entre 2000 e 2500 e a data for menor ou igual a de um mês atrás;
	\item Se o valor da fatura estiver entre 2500 e 3000 e a data de inclusão do cliente for menor ou igual a 2 meses atrás;
	\item Se o valor da fatura for maior que 4000 e pertencer a algum estado da região Sul do Brasil.
\end{itemize}

O participante deve criar todo o código responsável pelo filtro de faturas.
Uma classe com o método "main()" deverá ser entregue ao final, com exemplo 
de uso das classes criadas.