\chapter{Informações ao participante}
\label{ape:informacoes-participante}

\section{Convite}

Meu nome é Mauricio Aniche. Sou aluno de mestrado em Ciência da Computação pelo
Instituto de Matemática e Estatística da Universidade de São Paulo (USP).
Atualmente pesquiso sobre Test-Driven Development e sua influência no processo
de desenvolvimento de software.

Para alcançar esse objetivo, estou realizando entrevistas com desenvolvedores de
diversas empresas do mercado brasileiro.
Este convite permite a você compartilhar suas experiências e sentimentos em
relação à prática e cooperar com as pesquisas na área.

É importante reforçar que a participação é totalmente voluntária e não há nenhum
tipo de remuneração associada. Você pode desistir da sua participação sem nenhum
tipo de consequência.

\section{Qual o objetivo desta pesquisa?}

O objetivo desta pesquisa é entender de maneira mais profunda a influência de DGT
no processo de desenvolvimento de software. Essas informações serão
capturadas baseadas na percepção dos participantes dessa pesquisa.

\section{Qual meu papel dentro dela?}

Como participante dessa pesquisa, você deverá vir ao laboratório em uma
data definida com antecedência e resolver a 2 exercícios usando Java e DGT.
O código-fonte do exercício, bem como a gravação em vídeo do seu monitor
ficarão com o pesquisador.
Após isso, uma nova data será marcada para que você seja entrevistado sobre
os exercícios resolvidos.

\section{Quais são os benefícios?}

Além de cooperar com o avanço da pesquisa na área de engenharia de software, os
resultados obtidos por essa pesquisa são compartilhadas com você, e eu espero
que as informações ali contidas possam ser úteis para a evolução da técnica.

\section{Minha privacidade será garantida?}

Sim, todas as informações gravadas serão mantidas em completo sigilo. Apenas os
pesquisadores participantes desse trabalho terão acesso ao mesmo.

Além disso, nenhum nome será revelado no resultado final da pesquisa.

\section{Qual o tempo de participação na pesquisa?}

O participante gastará em torno de 2 horas para resolver os exercícios.
Além disso, o pesquisador precisará de 1 hora (em um outro dia) para a realização da entrevista. 
Caso uma nova entrevista seja necessária, ele marcará a mesma com antecedência. 

\section{Em caso de dúvidas, o que devo fazer?}

Em caso de dúvida, favor contatar o pesquisador ou o orientador dessa pesquisa.

Mauricio Finavaro Aniche (aniche@ime.usp.br)

Marco Aurélio Gerosa (gerosa@ime.usp.br) 

Departamento de Ciência da Computação - Instituto de Matemática e Estatística - 
Universidade de São Paulo (USP) - Caixa Postal 66.281 - 05.508-090 - São Paulo -
SP  - Brasil

