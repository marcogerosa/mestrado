%% ------------------------------------------------------------------------- %%
\chapter{Resultados Encontrados e Discussão}
\label{cap:discussao}

\section{Particularidades da execução}

na hora do estudo:
- pesquisador tirava dúvidas dos exercícios
- ajudava na configuração das máquinas
- copiava pro hd externo com a nomenclatura "NOME-COMBINACAO"
- dúvidas como "quando escrevo testes depois, posso refatorar também?" --> sim (na próxima, pensar melhor sobre casos extremos)
- sem pressão para o participante.. apenas de tempos em tempos, passava pelas máquinas para garantir que nada de errado havia acontecido;
- avisava o pessoal quando os primeiros 50 minutos acabavam com a mensagem "terminem a linha de raciocinio e partam para o proximo exercicio"
- aproveitava para trabalhar neste documento de licoes aprendidas e etc
- apesar de nao haver nenhuma restricao explicita sobre isso, nao houve conversa entre os participantes, todos eles trabalharam sozinho 100% do tempo.

em casa:
- dava id pro participante
- renomeava pasta para "ID-NOME-COMBINACAO"
- rodava script que gerava paginas para o especialista
- armazenava informações em um mysql para facilitar as consultas

na universidade:
- alunos só conseguiram fazer o 1o exercício. pesquisador deixou já que, na hora de mudar, observando por cima,
percebeu que pouco código havia sido escrito até então (muito diferente da realidade das indústrias, onde
muito código havia sido gerado em 50 minutos de exercício).
- alguns nao se mostraram muito dispostos a participar (diferente da industria)


Even though there are some preliminary indications that students can be used for certain tasks instead of professionals under certain conditions (Höst et al. 2000), it is still unclear how well results from student- based experiments generalize to professional software engineers (Harrison 2000).

discutir q profissionais sao mto variados -- checar paper condutctinr realistic experiements ..


na industria:
- receberam por email antecipadamente o que precisava ser instalado
- alguns participantes não acabaram todo o processo, outros com um pouco de pressa. isso pode influenciar, e foi
levado na análise.
- necessidade de vender a ideia e a promessa da divulgação dos resultados assim que terminar


entrevista:
pesquisador anotava no papel de observacoes, os pontos mais interessantes que o participante levantou no pos-experimento,
como por exemplo, "tdd faz eu gerar metodos pequenos", para perguntar depois.

tb anotava os exercicios que ele resolveu, e a letra do GRUPO da entrevista que deveria ser feito.

notebook aberto com ambos os códigos gerados pelo participante, para que o participante pudesse lembrar,
e o pesquisador usar exemplos na hora de perguntar como o design foi feito... "da onde veio a ideia dessa classe
Fatura, com esse método X(), e a classe ProcessadorDeBoletos com a regra de negócios dentro?"

entrevistas levando em torno de 30 minutos, gravadas e audio importado para o PC.

quando o participante comentava algo interessante, o pesquisador fugia do roteiro para fazer ele falar mais sobre o assunto.

roteiro de entrevista sofreu pequena mudança ao final da primeira entrevista, já que o pesquisador percebeu que
é difícil (e talvez não etico) falar que o código gerado pelo participante "não apresenta bom design". a mudança de foco
é para entender como ele chegou naquele design (mesmo que não ideal)

notei que alguns participantes conversaram com outros para entender como eles resolveram o exercício. e, quando perguntei,
sobre como foi o design, e discutiram outras maneiras de fazer, muitos deles sugeriram a mesma refatoracao.

discuti sempre primeiro o exercicio q fez com tdd, independente da ordem que
ele executou no dia.

dados:
- numero de pessoas que usaram polimorfismo foi baixissimo
- uma pessoa da industria NAO CONSEGUIU FAZER NADA


\section{Descrição dos participantes}

Conforme mencionado, o estudo foi executado em dois ambientes diferentes: em empresas de software
do mercado brasileiro e na academia. Como ambos estudos foram executados de maneira independente,
descrevemos os grupos separados nas sub-seções \ref{findings-desc-industria} e \ref{findings-desc-academia}.

\subsection{Participantes da Indústria}
\label{findings-desc-industria}

Ao todo tivemos 25 participantes, de 6 diferentes empresas.
Os participantes da indústria, em sua maioria, são pessoas com pouca experiência em TDD.
40\% deles disseram utilizar a prática há no máximo um ano. 52\% deles praticam TDD
entre 1 e 3 anos. Apenas 4\% pratica entre três e quatro anos, e nenhum participante
possui mais experiência do que isso. Na Figura \ref{fig:exp-tdd-industria}, mostramos
a distribuição da experiência da prática de TDD entre os participantes.

Os números são um pouco diferentes quando se trata da experiência em desenvolvimento
de software. Por volta de 24\% dos participantes desenvolve software entre 4 e 5 anos.
28\% deles faz isso entre 6 e 10 anos. Entretanto, 20\% possui até 2 anos de experiência.
Na Figura \ref{fig:exp-sw-industria}, mostramos a distribuição.

Entrando em aspectos mais técnicos, 64\% dos participantes afirmam conhecer Java. Entretanto,
36\% dizem não trabalharem com Java no seu dia-a-dia. Todos eles afirmam conhecer JUnit,
e só 12\% diz nunca ter ouvido falar sobre o conceito de objetos dublês. De fato, 64\% deles
aplicam objetos dublês durante suas atividades de desenvolvimento. Com relação a conhecimentos
em orientação a objetos, na pergunta aberta do questionário, grande parte deles 
afirmaram que possuem uma boa experiência e alguns
chegam até a afirmam que dominam o assunto. Poucos disseram que possuem conhecimentos
básicos. Na Tabela \ref{tab:exp-industria},
apresentamos o conhecimento dos participantes em relação a Java, JUnit e objetos dublês.

Esses dados nos mostram que a amostra selecionada se assemelha ao mundo real, onde
encontramos desde desenvolvedores mais experientes até novatos. Em relação a experiência com TDD,
podemos afirmar que metade dos participantes ainda está experimentando a prática, enquanto
outros já a tem mais consolidada. Isso é positivo, já que foi possível capturar informações
da prática de TDD por pessoas com diferentes níveis de maturidade.

Em relação ao alto número de pessoas que não utilizam Java, isso se deve ao fato de uma das
empresas fazer uso de PHP para seu trabalho do dia-a-dia. No entanto, nós conhecemos a equipe
e verificamos que, apesar de não utilizarem a linguagem constantemente, eles não tiveram
problema algum durante a execução dos exercícios.

\subsection{Participantes da Academia}
\label{findings-desc-industria}

Os participantes da academia que participaram do estudo, 21 no total, possuem baixíssima experiência,
tanto em TDD quanto em desenvolvimento de software em geral. Surpreendentemente, 90\%
dos participantes afirmaram não conhecer Java, e 61\% disseram não conhecer TDD.
Em contraste com essa informação, todos eles afirmaram conhecer JUnit, e 76\% afirmaram
conhecer objetos dublês na teoria. 
Na Figura \ref{fig:exp-tdd-academia}, mostramos a experiência dos participantes em TDD, e
na Tabela \ref{tab:exp-academia} apresentamos a experiência deles em Java, JUnit, e objetos
dublês.

Em relação a experiência de desenvolvimento de software, apenas 23\% afirmaram não
ter nenhuma experiência, enquanto que por volta de 43\% afirmaram possuir
entre 2 a 4 anos de experiência. Nenhum participante com experiência superior a isso.
Na Figura \ref{fig:exp-sw-academia}, mostramos a distribuição por completo.

De acordo com os dados, percebemos que grande parte dos participantes não conhecem a linguagem ou
mesmo a prática de TDD. A linguagem foi realmente um problema: durante a execução, percebemos
que muitos deles faziam buscas por elementos simples da linguagem Java, como manipulação de arrays
e listas, ou mesmo por manipulação de texto.

Isso explica o baixo desempenho dos alunos da academia durante
o experimento. Nenhum deles conseguiu terminar os dois exercícios propostos. Por esse
motivo, nenhum aluno foi selecionado para o processo de entrevista. Entretanto,
os códigos-fonte produzidos serão utilizados no processo de análise quantitativa.

\begin{figure}[ht]
  \begin{minipage}[b]{0.45\linewidth}
    \centering
    \includegraphics[scale=.4]{findings/experiencia-tdd-industria.png}
    \caption{Experiência da equipe da indústria com TDD}
    \label{fig:exp-tdd-industria}
  \end{minipage}
  \hspace{0.5cm}
  \begin{minipage}[b]{0.45\linewidth}
    \centering
    \includegraphics[scale=.4]{findings/experiencia-tdd-academia.png}
    \caption{Experiência dos estudantes da academia com TDD}  
    \label{fig:exp-tdd-academia}
  \end{minipage}
\end{figure}

\begin{figure}[ht]
  \begin{minipage}[b]{0.45\linewidth}
    \centering
    \includegraphics[scale=.4]{findings/experiencia-sw-industria.png}
    \caption{Experiência da equipe da indústria com desenvolvimento de software em geral}
    \label{fig:exp-sw-industria}
  \end{minipage}
  \hspace{0.5cm}
  \begin{minipage}[b]{0.45\linewidth}
    \centering
    \includegraphics[scale=.4]{findings/experiencia-sw-academia.png}
    \caption{Experiência dos estudantes da academia com desenvolvimento de software em geral}  
    \label{fig:exp-sw-academia}
  \end{minipage}
\end{figure}

% TODO tabela com experiencia java, junit, bla bla de cada grupo

\section{Análise das Entrevistas}

Diferente do que muitos estudos de TDD afirmam, os participantes, em sua grande maioria, afirmaram que 
a prática de TDD não faria com que o design de classes fosse de alguma forma diferente, caso tivessem
feito ambos os exercícios com a prática.
A principal justificativa dada pelos participantes foi que a experiência e o conhecimento prévio
em orientação a objetos os guiaram durante o processo de criação do design de classes. Nenhum dos
participantes, por exemplo, afirmou que um desenvolvedor sem conhecimento em alguma das áreas
citadas criaria um bom design de classes somente por praticar TDD.

Dois bons exemplos foram dados pelos participantes, que ajudam a reforçar esse ponto. Um deles
comentou que fez uso de um padrão de projetos \cite{gof} que aprendeu apenas alguns dias atrás.
Outro participante mencionou que seus estudos sobre os princípios SOLID (discutidos no Capítulo \ref{cap:design})
o ajudaram durante os exercícios. Além do mais, o único participante da indústria que nunca havia
praticado TDD afirmou que não sentiu diferença nenhuma no processo de criação de classes durante
a prática. 

\textit{"Até foi engraçado, eu estou lendo o Design Patterns (livro), e ele fala de polimorfismo, e foi
lá que eu mirei pra fazer, porque eu nunca tinha feito nada assim (...), aqui dificilmente eu crio
coisa nova, só dou manutenção no código."}

Curioso é que esse mesmo participante que nunca praticou TDD afirmou que "sabia que TDD era uma prática de design",
diferentemente dos participantes mais experientes que sempre afirmavam que TDD não é só uma prática de design,
mas também de testes. Isso indica, de certa forma, que a popularidade dos efeitos de TDD no design, por mais
que nada tenha sido provado, é grande.

uma descricao da galera: \textit{"[TDD] acho que tem muita relação com qualidade do código e testes 
de regrassão. Acho que as duas principais vantagens que eu tenho quando uso TDD é isso: o código
fica melhor e depois eu tenho a segurança dos testes de regressão para refatorar."}

Apesar do TDD não guiar o desenvolvedor diretamente para um bom design,
todos eles afirmaram que veem benefícios na prática de TDD, mesmo do
ponto de vista de design. Nas sub-seções abaixo, apresentamos cada um dos pontos 
levantados pelos participantes, bem como discutimos sobre o tópico.

\textit{"O ideal é somar as duas coisas [experiência e TDD] (...) 
Não acredito que TDD sozinho consiga fazer as coisas ficarem boas. Tem outros conceitos
para as coisas ficarem boas."}



\subsection{Segurança na refatoração}

Os participantes afirmaram que, durante o processo de criação de design, a mudança de ideia é
constante, afinal pouco se conhece do problema, e de como a classe deve ser construída. Segundo eles,
uma vantagem íntriseca do TDD é a suíte de testes gerada. Essa suíte possibilita ao desenvolvedor
muda de ideia e refatorar todo o design de classes com segurança.
A segurança, segundo eles, é fator importante para mudanças de design de classes ou mesmo de implementação.


\textit{"Sim, me dá a chance de aprender pelo caminho e fazer algumas coisas diferentes. (...) O teste te dá segurança."}

\textit{"Porque em um projeto você muda de ideia muitas vezes. (...) Porque grande grande do que você
pensou está especificado (no teste) e, por mais que você esqueça uma coisa ou outra, o teste vai
te lembrar. Isso aconteceu comigo várias vezes. Eu estava mudando alguma coisa no código e não passava
o teste, não passava... Isso aconteceu até eu ver que o errado era eu, e não o teste."}

\textit{"No TCC da pós, eu estava desenvolvendo uma ferramenta que trabalhava com manipulação de código, e fiz
tudo com TDD. Várias vezes eu chegava a apagar todo código do sistema, mantinha os testes, e começava uma nova
linha de raciocínio. Achei que me ajudou muito fazer TDD (...), tanto que no fim que eu fui executar a ferramenta,
antes eu só validava pelos testes."}

refatoracao + experiencia

\textit{"(...) se você não tiver embasamento sobre esses aspectos de única responsabilidade,
coesão, acoplamento, acho que não adianta muito [fazer TDD]. Você precisa ter isso em mente
para conseguir mudar, precisa desse conhecimento para conseguir refatorar."}

\subsection{Espaço para se pensar}

When not doing TDD, participants often said that they are too focused on writing code that they forget to think on the class design. The test makes them think on how the class being created will interact with other objects, and how easy is to use that class.


Participantes afirmaram também que enxergam o teste de unidade, em uma analogia
retirada de um participante, como uma \textit{folha de rascunho},
onde eles podem tentar diferentes abordagens e mudar de ideia constantemente. Segundo eles,
ao começar a escrever um teste, os programadores estão, pela primeira vez, utilizando a sua 
própria classe. Isso faz com que eles busquem por uma maneira melhor e mais clara de invocar
seus comportamentos, e facilitar a utilização da classe.

\textit{"Os testes ajudam nisso. São uma folha de rascunho para você tentar modelar
isso da melhor maneira possível. Se fosse modelar isso direto, é como se você tivesse
uma forma, e se errar, quebrou. Ou se você errar, você vai ter muito trabalho pra consertar.
O lance de você testar e começar a pensar em testes, você está ali com uma folha em branco,
e você pode arrancar qualquer coisa que está ali, pois essa coisa ainda não existe."}

Segundo um participante: \textit{"Acho que o normal das pessoas não é pensar antes. Parece que o natural
é já sair fazendo (até pela pressão interna, que aqui não é tão grande). (...) Poucas pessoas pensam
antes de começar. Com TDD, você é obrigado a pensar, o TDD faz você parar e pensar, estruturar. Não
é meu natural pensar antes, mas com TDD sim."}

\textit{"Como eu primeiro penso no que eu vou precisar a partir dos testes, ou seja, eu preciso que tenho
isso e aquilo, o teste me faz pensar antes de sair desenvolvendo. Com os testes eu paro pra pensar antes.
Aí acredito que nós consigamos pensar melhor, numa solução mais bacana."}

\textit{"Porque sem o TDD, no calor do momento, você vai acoplando, vai herdando, vai agregando, e não pensa
que no futuro isso possa dar algum problema. Com TDD, você é forçado a ir mais devagar, dá tempo de pensar melhor nas
coisas."}

manutencao para o proximo
\textit{"Quando estou escrevendo meu rascunho, eu tenho a liberdade de pensar o máximo possível para quando alguém
pegar isso para entender, ou para debugar, ou mesmo para corrigir bug, ele vai conseguir saber o que uma Fatura é ou faz. Sem
precisar abrir uma Fatura real."}

pensar ate no requisito

\textit{"Algumas vezes ele [o teste] acaba mostrando problemas da regra de negócio. Mostrava problemas
que as vezes o especificador não pegava. (...)."}

O teste é o primeiro cliente da
classe que o programador ainda está por escrever e isso o faz pensar
melhor a respeito do comportamento que ele espera da classe. Além disso,
programadores contemplam e decidem também sobre a interface (como nomes de
classes e métodos, tipos de retorno e exceções lançadas) que a classe terá
\cite{janzen-saiedian}.
Além disso, ao escrever o teste antes, o programador é encorajado a escrever um
código que seja facilmente testável. Códigos como esse possuem algumas
características interessantes, como a facilidade para invocar o comportamento
esperado, a não necessidade de pré-condições complicadas e a explicitação de
todas as dependências que a classe possui.

\subsection{Passos menores e simplicidade}


TDD sugere que o programador dê sempre pequenos passos (conhecidos pelo termo em
inglês, \textit{baby steps}): deve-se escrever testes sempre para a menor
funcionalidade possível, escrever o código mais simples que faça o teste passar
e fazer sempre apenas uma refatoração por vez \cite{TDDByExample}.

Uma justificativa para tal é a de que, quanto maior o passo que o programador dá, mais
tempo ele leva para concluí-lo e, consequentemente, ele fica mais tempo
sem \textit{feedback} sobre o código. Além disso, faz com que o programador não crie
soluções mais complexas do que elas precisam ser, tornando o código, a longo
prazo, o mais simples possível.

Manter o \textit{design} simples não é tarefa fácil, e TDD sugere que o programador
escreva sempre o código mais simples que atenda a necessidade. Somente se a
necessidade crescer, é que o programador deverá evoluir o \textit{design}. Uma decisão de
\textit{design} pode ser mais complicada do que parece e, sem um teste para mostrar isso
rapidamente, o programador dificilmente perceberia o problema \cite{aim-fire}.

\textit{"Porque sem os testes, nós não pensamos em passos menores, mas sim na solução inteira
e acaba por não observar problemas que podem acontecer pelo caminho."}

\textit{"Porque nós começamos a pensar no pequeno e não no todo. Quando faço TDD eu escrevo
uma regra simples (...), aí vou lá e escrevo o método. Se passou, passou! Como você vai aos poucos,
a arquitetura vai ficando legal. (...) Eu tinha mania de pensar no todo (...), as vezes
em vez de você pensar em um negócio pequeno, você pensa em um enorme. Acho que o cérebro funciona
melhor quando você pensa pequeno. Se você pensa grande, pra mim é óbvio que você vai deixar
alguma coisa faltando."}

foco

\textit{"Talvez sejamos pessoas desfocadas naturalmente. Você vê uma coisa e já te dá vontade
de corrigir aquilo. Como eu já tenho o teste escrito, posso fazer qualquer coisa."}


\subsection{Busca pela testabilidade}

\textit{"Eu utilizo isso como uma regra: sempre que está muito complexo [o teste],
acho que nós temos que parar e refatorar, porque, na minha opinião, dá
pra ficar mais simples."}

Quanto mais difícil for a escrita do teste, maior a chance da existência de
algum problema de \textit{design}. Segundo Michael Feathers \cite{feathers-synergy}, 
existe uma sinergia muito
grande entre uma classe com alta testabilidade e um bom \textit{design}: se o
programador busca por testabilidade, acaba criando um bom \textit{design}; se 
busca por um bom \textit{design}, acaba escrevendo um \textit{design} mais
testável.

TDD encoraja o programador a escrever componentes fracamente acoplados, de
maneira que eles possam ser testados de maneira isolada e, em um nível maior,
combinados com outros componentes.
Programar voltado para interfaces é uma prática de orientação a objetos há muito
tempo conhecida. Pensar em classes e dar maior foco à maneira com que
elas se relacionam do que com o modo que determinado comportamento será implementado
torna-se mais natural ao praticar TDD \cite{GOOS}. 

\subsection{Feedback mais rápido}

desenho aqui do feedback

\textit{"Você ia olhar pro teste, e falar: "Está legal? Não está?", e ia fazer de novo."}

\textit{"Quando você faz o teste, você vê logo o que não gostou do método daquele jeito (...), você
não percebe isso até que você use o teste."}

\textit{"[O teste] não é só uma especificação; ele tem que de fato funcionar. Então,
como você diminui muito o tempo entre escrever um programa que funcione e testar aquilo,
você consegue mais rápido ver se aquela parte pequena funciona ou não (...)"}



Em projetos novos, praticantes de TDD afirmam que sentem menos necessidade da
utilização de recursos de depuração de código \cite{george-williams-experiment} 
\cite{janzen-arch-improvement}. 
A quantidade de código
escrita entre um teste e outro tende a ser pequena, e caso um teste falhe
inesperadamente, o programador pode simplesmente reverter as alterações para a 
versão anterior estável e começar novamente. Essa abordagem pode muitas vezes
ser mais produtiva do que a atividade de depuração 
\cite{janzen-arch-improvement}. Por essas e outras razões, desenvolvedores afirmam 
que são mais produtivos quando praticam TDD. Apesar de o custo da escrita do teste
existir, a longo prazo o desenvolvedor gasta menos tempo com depurações ou 
erros de regressão, e com isso tem sua produtividade aumentada
\cite{george-e-williams}.



\subsection{Depois que aprende, dificil ficar sem}

\textit{"Você vai fazer alguma coisa, você acaba pensando já nos testes que você vai fazer. É difícil 
falar assim: "programa sem pensar nos testes!" Depois que você acostuma, você não sabe outra
maneira de programar..."}

curva de aprendizado, nao sabe muito bem o que e como testar no comeco

\textit{"É complicado se disciplinar [a praticar TDD], mas conforme vai passando o tempo, 
você percebe que a curva para se manter o projeto fica bem menos íngreme, 
começa a perceber os benefícios e aí vicia. Você acaba não se sentindo
mais confortável de escrever código sem teste."}

dificil convencer os outros no comeco

seguranca que nao volta atras
\textit{"Eu tenho achado o máximo. Não consigo fazer nada sem utilizar os testes mais. Não tem como.
Só me sinto seguro de entregar a funcionalidade quando ele está testado."}

\section{Padrões de TDD}
A. Unit test feedback patterns
Tests can show coupling issues. When a single test needs a big scenario, or it makes use of many mock objects [13], it may indicate that your production code is highly coupled. A participant said that, in order to easily unit test a code, one need to decouple the code as much as s/he can.
On the other hand, if one production method needs many different unit tests to assure its entire behavior, the tested code may be suffering from cohesion problems.
Lack of abstraction can also be perceived through unit tests. If there are many similar unit tests for different interfaces, it may indicate a lack of a right abstraction for both entities. Also, when a test finds a bug and, in order to fix it, the developer needs to update many classes, it may indicate that the abstraction is not good enough, and information is duplicated.

Classes que obedecem aos princípios SOLID são fáceis de serem testadas, pois
apresentam todas as vantagens citadas acima. Ao inverter o processo, e fazer o
programador pensar em classes facilmente testadas desde o princípio, TDD exige
que o programador faça uso desses princípios, resultando em um código de maior
qualidade.

Como citado no capítulo \ref{cap:tdd}, grande parte do feedback que os testes
dão acontecem no momento em que o programador encontra dificuldades para a
escrita do mesmo.

Classes que violam o SRP, por exemplo, necessitam de grande
esforço por parte do desenvolvedor para ser testada, já que é uma classe que faz
muita coisa, e por consequência, demanda muitos testes.

O DIP também emerge da prática, conforme ilustrado no exemplo do GOOS
\cite{GOOS}. Ao desenhar classes e pensar no relacionamento com outras classes,
programadores pensam apenas no comportamento esperado e sua abstração, e não na
maneira como isso será implementado. A implementação das abstrações são
inclusive feitas em fases posteriores. Isso incentiva as classes a
sempre dependerem de abstrações e não de implementações concretas.

Apesar de conhecidos, esses princípios não são comumente aplicados.
Este trabalho visa entender se TDD, ao forçar o programador a escrever o teste
antes, encoraja o programador a fazer uso das boas práticas citadas neste
capítulo.


do condicional pra frente, nós derivamos

sempre relacionar com SOLID do cap de design

\subsection{Indecisão ao dar o nome do teste}

\subsection{Muitos testes para um método = falta de coesão}

\subsection{Não consegue testar um método inteiro = falta de coesão}

\subsection{Muitos testes para uma classe}

\subsection{Mocks em excesso}

\subsection{Testes com cenários muito grandes}

\subsection{Mesma alteração em diferentes testes}

\subsection{Testes repetidos para entidades diferentes}

\subsection{Desacoplar para conseguir testar}

\subsection{Difícil usar = interface não amigável}


do aniche

\subsection{Condicional no nome do teste}

\subsection{Asserts em objetos que não são alvo da classe de teste}

\subsection{Testes em métodos que não são públicos}

\subsection{Dependência que não é utilizada por um dos testes}

\section{Análise Quantitativa}

\subsection{Métricas de código}

mostrar p-values aqui

\subsection{Especialistas}

mostrar p-values aqui
