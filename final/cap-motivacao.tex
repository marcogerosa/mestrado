%% ------------------------------------------------------------------------- %%
\chapter{Uma análise qualitativa sobre TDD em um evento de métodos ágeis}
\label{cap:motivacao}

Este estudo exploratório e qualitativo, realizado em 2010 e publicado no 
WBMA 2010 \cite{aniche-wbma}, dentro de uma conferência de 
métodos ágeis mostra o quanto a prática de TDD tem crescido em popularidade. No entanto,
os desenvolvedores ainda não compreendem muito bem quais são os efeitos
da prática no projeto de classes.

Este capítulo discute o estudo feito, bem como o que encontramos na análise
da discussão de um grupo de participantes do Encontro Ágil 2010, um
dos primeiros eventos sobre metodologias ágeis do Brasil.

\section{Escopo}

A conferência "Encontro Ágil", que acontece em São Paulo, é uma das
primeiras conferências brasileiras sobre métodos ágeis de desenvolvimento
de software. 

Na edição de 2010, a conferência propôs um formato diferente:
nenhuma palestra. A ideia era justamente aumentar a interação e a troca
de experiência entre os 200 participantes. Por esse motivo, o programa
do evento era composto de sessões com espaços abertos, sessões
de \textit{dojo}, \textit{workshops} e jogos.

\subsection{A Sessão sobre TDD}

Nós conduzimos uma sessão aberta sobre Desenvolvimento Guiado por Testes.
A ideia principal da sessão era entender a visão que as pessoas que
atualmente utilizam TDD tem sobre a prática. O objetivo da sessão
era entender o que uma audiência com praticantes ainda inexperientes
pensam sobre TDD, e suas opiniões em relação aos efeitos da prática 
no projeto de classes.

Na sessão, realizamos um bate-papo informal, usando uma técnica
conhecida por \textit{Fishbowling} 
\footnote{\url{http://en.wikipedia.org/wiki/Fishbowl_(conversation)}. Último acesso em 27 de novembro de 2010.}
Nela, quatro pessoas ficam sentadas, em uma roda, e discutem sobre
o assunto proposto. Uma cadeira extra vazia junto a roda possibilita
que um participante da audiência se junte a conversa. Mas, no momento
em que um novo participante entra na conversa, um dos quatro anteriores
deve sair e voltar para a audiência.

A sessão levou em torno de 1 hora e meia, e nós propusémos sete diferente
questões para discussão. Algumas delas eram intencionalmente provocativas,
justamente para promover uma discussão entre os participantes. A lista
contém as perguntas, na ordem em que foram discutidas:

\begin{enumerate}
\item TDD é uma prática de testes ou de projeto de classe?
\item Discuta a seguinte afirmação: é impossível criar um bom projeto de classes sem fazer TDD.
\item Como os testes te ajudam a criar um bom projeto de classes? Por exemplo, classes com baixo
acoplamento, alta coesão, código simples, e etc?
\item Muitas pessoas aplicam TDD fazendo "o mais simples possível", mesmo se
o código a ser escrito já é muito simples (por exemplo, uma simples calculadora
que executa somas, onde o desenvolvedor cria e retorna constantes e adiciona
muitos condicionais antes de uma grande refatoração). Você pensa que isso
é realmente necessário?
\item Agora que usei meus testes para melhorar o projeto das minhas classes, posso
agora jogá-los fora?
\item Algumas pessoas afirmam que escrever os testes antes diminui a produtividade.
Por exemplo, se o programador escreve 100 linhas de código por dia, apenas 50 delas são
código de produção; o restante é código de teste. O que você pensa sobre isso?
\item Fale sobre a primeira vez que praticou TDD.
\label{questions}
\end{enumerate}

Os participantes tinham por volta de sete minutos para discutir cada questão.
Assim que o tempo acabava, nós solicitávamos a eles que votassem se a discussão
deveria continuar ou não. Para a votação, nós distribuímos cartões com faces
verde e vermelha entre os participantes. A face verde indicava que a discussão
deveria continuar; já a face vermelha indicava que o tópico não era mais 
interessante e que um novo tópico deveria já poderia entrar em discussão.

Dez congressistas participaram da discussão, e todos eles entraram na
roda de discussão pelo menos uma vez. Nós não participamos da discussão, já
que isso poderia enviesar os participantes. Após a sessão, nós pedimos a eles
para preencher um questionário sobre experiência em desenvolvimento de software
e TDD. Na Seção \ref{motiv:sec:participants}, discutimos sobre o 
perfil dos participantes.

Para a análise posterior, um gravador de voz foi colocado no centro
da sala, e todos os participantes estavam cientes de que suas opiniões
estavam sendo gravadas e suas afirmações usadas de maneira anônima e apenas
para o propósito deste estudo.

\subsection{Perfil da Audiência}
\label{motiv:sec:participants}

Nós solicitamos aos participantes que respondessem um curto questionário anônimo
sobre suas experiências em desenvolvimento de software e TDD.
O questionário e suas possíveis respostas estão representadas
na Tabela \ref{motiv:questionnaire}.

\begin{center}
\begin{table}[h]
\centering
\begin{tabular}{ | p{5.5cm} || p{5.5cm} | }
\hline
Questão & Opções
\\ \hline \hline
Experiência em desenvolvimento de software (em anos) &
0-1, 1-2, 2-3, 3-4, 4-5, 5-6, 6+\hspace{0.1cm}
\\ \hline
Experiência em TDD (em anos) &
0-1, 1-2, 2-3, 3-4, 4-5, 5-6, 6+
\\ \hline
\end{tabular}
\caption{Questionário usado para avaliar os participantes da sessão}
\label{motiv:questionnaire}
\end{table}
\end{center}

A maioria dos participantes possuíam entre 0 e 3 anos de experiência em TDD,
a apenas 2 possuíam mais do que isso. Apenas 1 participante nunca havia
praticado TDD antes.
Sobre experiência em desenvolvimento de software, a distribuição era
razoavelmente homogênea. Programadores iniciantes (menos de 2 anos de experiência)
representavam 30\% da audiência, enquanto que os mais experientes (mais de 3 anos de experiência)
representavam 70\%.

Baseado nestes números, a audiência pode ser classificada como um grupo
experiente em desenvolvimento de software, mas que ainda está iniciando
a prática de TDD. Este tipo de perfil é ainda muito comum na indústria,
já que os desenvolvedores ainda estão apenas começando a aplicar práticas
agiles e TDD especificamente.

\section{Research Methodology}

The research on software engineering practices, since it involves humans, is
benefited from the use of qualitative techniques. In particular, when evaluating
the effects of TDD, for example, it is hard to separate it from other agile
practices that are usually done together \cite{agile-manifesto}
\cite{guidelines-qualitative}.
Qualitative analysis is well suited for many kinds of software engineering
research, as the objects of study are contemporary phenomena, which are hard to
study in isolation.
Case studies do not generate the same results on e.g. causal relationships as
controlled experiments do, but they provide deeper understanding of the
phenomena under study \cite{guidelines-qualitative}.
In this proposal the researchers chose some techniques based on Grounded
Theory \cite{grounded-theory}.

The researchers created all the questions and some of them intent to stimulate
the discussion. The idea was to generate a discussion and make people
comfortable to talk about the questions.
During the session the researchers were only observing the discussion without
actually participating on it.
The researchers took notes about participants' feelings, i.e., when they
were agreeing or disagreeing with something.
Although the number of participants was quantitatively small (only 10 people),
the volume of data gathered during this one hour and a half session is
reasonably high and covers different topics about the TDD practice.

As mentioned before, the discussion was entirely recorded. The researchers
transcripted the audio and double listened in order to check for eventual errors
in the transcription. After that, two researchers individually started the
coding process, which is the act of organizing and classifying data into
categories or segments of text before trying to give a meaning to that piece of
information \cite{rossman}. At the first moment, researchers were free to
create any code they want. After that, researchers discussed about each code
created and merged them. The intent of this part of the process is to reduce
bias.
The codes were then grouped into themes by the researchers, which became the
subsections of Section \ref{sec:findings}.

As expected, articulated participants talked more than others. That may
influence the discussion to a participant's point of view. However, when a
stronger position was placed, all participants argued about it until it
converged into something that everyone was comfortable with. Researchers were
aware of it and took this into account during the analysis.
Afterwards, participants were invited to review this paper in order to find any
flaw or bias during the analysis. Their revision can be found in Section
\ref{sec:review}.

\section{Findings}
\label{sec:findings}

\subsection{TDD as a Design Technique}

Although some definitions of TDD focus on its testing perspective, most of the
participants affirmed that they use TDD mainly as a design technique, just like
Robert Martin \cite{bob-martin} and Kent Beck state \cite{aim-fire}
\cite{tdd-by-example}: \textit{``TDD's goal may be to do real testing but in
order to make the test you end up by influencing the design."}

Participants often had the opinion that design is a consequence of testing: 
\textit{``Tests are the means to obtain the design."}
This concept was reinforced by a few participants when they talked about the
effects of TDD on class coupling. They agreed that if programmers do not
decouple their code they would not be able to write a unit test. Also, if
programmers spend too much time trying to write a simple test, it may
indicate that there is something wrong with the design: \textit{``At the moment
you are testing unit by unit and you are writing a unit test and you see that
it is too complex, you already know that you need something. You know that you
are doing something you should not do. Tests show that, if something is really
hard to test, then it is because there's something wrong."}

A participant mentioned the effect of the test feedback. 
As developers receive constant and rapid feedback about the code design,
they are able to find some design smells and fix them when it is still cheap and
easy:
\textit{``TDD makes you start writing the code from the very beginning. After
that, you notice that what you have just written at the beginning is not good
enough and then you start to improve the code. This is how TDD ends up
influencing."}

Another interesting point of view raised by one of the participants was that
when programmers are doing TDD and they write a unit test for a class that
sometimes do not even exist, the test is the first client of that class. It
encourages developers to write simple code:
 \textit{``TDD ends up influencing us to simplify the code. As we are the
first client of our own system, we end up trying to simplify the design and
reduce coupling, and that is why our software design keeps evolving
constantly."}

On the other hand, there was one participant who did not believe that TDD is a
design technique, but a testing technique. He kept asking questions about
testing to the other participants. He believed that the benefits of TDD are the
test suite programmers have at the end. He asked: \textit{``But you need to
have some automated test suite, even without TDD. It will help me change my
design anyway, won't it?"}
Others replied that there is a difference in design when writing tests after,
as it was mentioned through all this section.

Participants gave most of their opinions about the effects of TDD in design 
during questions number 1 and 3. 
A curious fact is that most participants voted for an extra round for
question number 1, but question number 3 ended before the 7 minutes limit.

\subsection{Refactoring Confidence}

The opinion of the majority is that refactoring confidence a great advantage
when doing TDD. Programmers can evolve the code and the design without fear. 
If something goes
wrong the test suite warns them: \textit{``TDD influences a lot on design,
for sure. However, the most important thing in my opinion is the safety that it
gives me at the moment of refactoring the system. Without TDD I don't have a
clear criterion whether my refactoring was successful or not"};
\textit{``I like the refactoring part the most.
When you do your first refactoring and see a lot of red tests and, then, when
you see them turn green, you think: I can change this code with no fear at
all!"}.
All participants agreed immediately with these opinions.

This was deeply discussed also in question number 5. That was a tricky question:
if TDD is a design technique and the programmer used the test solely to improve
the design, as soon as the design is done the programmer could delete all tests.
This question was promptly replied by most of them with phrases like 
\textit{``No way!"} and \textit{``You can't do it in any case!"} followed 
by some laughing. One of them made an interesting comparison: 
\textit{``I've done the source code, it did compile and meet the needs. Can I
delete the source code now?"}

\subsection{Initial Skepticism}

Writing the test before the code goes against the traditional approach of
software programming. 
All participants that were on the discussion said that, when introduced to TDD,
they did not believe in the practice.

One of them suggested that the best way to see TDD's benefits is by
experimenting the practice with some pet project at home or at work.
They all said that after a certain time of practice the benefits become evident
despite the initial skepticism caused by the paradigm shift.
In their opinion, when developers find that the practice helps them making
better software they start to believe in TDD: \textit{``It's hard to believe
because the benefit takes some time to show up. However, if he thought that TDD
would help him develop better software in a long term, he would believe in it.''}

\subsection{Experience Matters}
\label{subsec:experience}

All participants agreed that TDD does not solve all design problems by itself.
They stated that there were thousands of projects in the past that still work
and have a good design. 

In spite of that, they also agreed that doing TDD helps
developers to create a better design in less time than the traditional approach.
This view is represented by the following statement from one of them:
\textit{``TDD influences and helps (the design) but it is not mandatory. It is
possible to do a good design without TDD but we will face other problems.''}.

\subsection{Different Opinions About Baby Steps}

TDD states that developers should do the simplest thing that
makes the test pass. In order to achieve that, programmers do it in baby steps,
which are small changes in the code. They help programmers to avoid
unnecessary and complicated code, not even at implementation level, but also at
design level.
Participants were divided about baby steps. Some of them believed that
programmers should do baby steps all the time and some of them believed that
baby steps all the time are not productive.

The example discussed was the implementation of a simple method
\textit{sum(int a, int b)} that sums two integer numbers. The idea was to
generalize that example to a much more robust algorithm during the discussion.
Baby steps group affirmed that they would only get into \textit{return a+b}
after a few tests with integers, for example. The other group affirmed that
programmers could go right to the final implementation with just one test for
the sum.

Those who believed that baby steps all the time are mandatory argued that when
programmers do not do it, they may forget to test some corner cases and a
refactoring may change the expected behavior of the system.
They would prefer to have 10 unit tests that test the same behavior than forget
one of them.
They also argued that it is really hard to implement the simplest code if
programmers do not do baby steps. If programmers do not do baby steps the chance
to write unnecessary code is higher.

On the other hand, the other group said that experience should be taken into
account when doing baby steps.
They agreed that doing baby steps all the time are not productive. A participant
paraphrased Kent Beck: \textit{``I am not sure if you need baby steps all the
time. Kent Beck says in his XP book that the programmer should use his own
experience. In his TDD book he says that, with experience, you feel whether you
make a bigger step or not."}
Some of them also said that 10 tests for a single case is a waste of time. In
addition, they noticed that if there were 10 tests for a feature it would be
harder to change that behavior: the programmer would have to alter 10 tests
instead of just one.

The interesting part about the last statement is that participants perceived a
possible coupling between test and production code. Although they did not
mentioned the correct term, they are already aware of it.
During another question, a participant commented about testing one case of each
equivalence class. It means that, in the calculator example, the programmer
would test only once a sum of two positive integers, then only once a sum of two
negative integers, and so on.

\subsection{No Productivity at the Beginning}

Almost all participants mentioned that when they started using TDD, they did not
feel too productive. However, in medium terms, the productivity went high as it
was much easier to fix bugs. In their opinion, using traditional approaches,
programmers may deliver code faster but after some time the productivity
decreases as they spend too much time searching for bugs and trying to evolve
the software.
They also tried to find a definition for productivity. In their opinion,
productivity may not be measured in lines of code, but they did not reach a
conclusion for that question.

The discussion turned to a more philosophical discussion: \textit{``We should
write code with quality. Write a code that does not work may help you to achieve
the customer's deadline, but he will call us incompetent!"} They were saying
that as developers, they should do the best they can and write tests (not only
TDD), and this is essential to the success of a project. One of them also cited
the doctor's anecdote: \textit{``If you tell a doctor to do a surgery without
cleaning the tools as you want it faster and cheaper he will not do the surgery;
if the customer says the same to you, you should not develop the software!"}

\subsection{Difficulty in Learning}

A topic that was mentioned more than once was that learning TDD is not easy.
Some participants said that learning how to write good unit tests or how to
mock objects is not simple. Even experienced programmers may feel it too.
One of them said: \textit{``Sometimes I look to a code and I think: I don't know
how to test this!"} He even said that sometimes he had the feeling of losing
productivity.

When the participants were talking about their first time in TDD they started by
mentioning how they have learned it. At least three of them said that they
learned it in Dojo sessions \cite{dojo}. Although they mentioned that they do not usually
like the way programmers do baby steps at Dojo sessions, they agreed that it is
a good way to spread knowledge about the practice.

There were other ways of learning cited during the session. A participant
commented about his participation in discussion lists. Another one talked about
some practical videos he has found. The same participant made lots of references
to books from famous authors like Kent Beck and Robert Martin during the
discussions and, when that happened, other participants made a gesture
indicating that they already read the book. Therefore, it is possible to infer
that books are another way of learning TDD.

A participant commented about an interesting situation. He was forced to do TDD
by a colleague at work. He said he did not believe at the beginning, but as soon
as he himself started practicing and saw improvements in his design and all
automated tests advantages, he started believing in TDD. Thus, peer learning
was the way a participant learned TDD and in his opinion it was very effective.

\section{Discussion}

The main topic in the session was that TDD is actually a design technique,
which matches with what is found in literature. Participants talked about many
effects of the practice, such as the need to manage dependencies, and the design
simplicity achieved by the urge to test a class without much effort. It is
interesting to notice that practitioners that are experienced in software
development but are beginners in TDD, notice these effects. 
The confidence when refactoring was also highly discussed by the participants.
The test suite enables developers to change code with safety. It shows that the
testing part of the technique is also useful.

As participants noticed, experience is fundamental to the process. When
trying to easily write a unit test to a class, a developer needs to use good
object orientation principles. TDD is a technique that gives feedback from the
design, and the practitioner should use it to drive design to better solutions.

Baby steps were also a popular topic. It continued for one more
round as most part of the audience voted for it.
It can be justified by the fact that baby steps are one of the most
misunderstood part of the practice. Baby steps try to prevent developers from
creating a complex solution when a simpler one solves the problem. A programmer
should be driven by his experience: if s/he is comfortable with that part of the
implementation, the step may be bigger; if not, a smaller one should be taken.
In researchers' interpretation, the goal of baby steps is to simplify both
implementation and design, and not to remember programmers about corner case,
which is a software testing activity.

However, developers face many problems when starting to practice TDD. As
presented before, most of developers tend not to believe in the practice at
first. The relation between testability and good design is not clear enough at the
beginning.
It is hard to notice TDD's effects on design when writing small projects that
does not require a flexible design. In order to believe in the practice,
programmers should try to use it so that they can make their own conclusions, 
noticing the substantial improvement in the quality of their code and design.

As participants said, the practice may reduce developers' productivity at the
beginning. However, it is hard to perceive if productivity is reduced when
programmers do not know how to write a unit test or because they were not familiar with OO
principles. But, as noticed by participants, although they do not feel
productive at the beginning, in medium terms productivity grows. 
This also may explain the the learning difficulty presented by the participants.
Besides learning the fundamental tools to write a unit test, developers should
write decoupled code; both activities have a learning curve.

Interestingly, the effects of TDD in external quality was not mentioned. That
can be explained by the fact that most questions were focused on the design
effects. However, the first question was clearly unbiased, and participants only
discussed about the design part. It may indicate that TDD effects are only in
internal quality and external quality is just a side-effect.

\section{Participants' review}
\label{sec:review}

Some participants accepted to review the paper. They all agreed with all
the research findings. 
Some of them also showed their own interpretations of the discussion. For
instance, one participant said that his conclusion about the baby steps discussion was
that there is no need to do small steps for simple code; however, for a more
complex one, baby steps would help him earn experience with the code he is
developing. He also said that, besides the design effect, TDD helps programmers
to understand the business they are dealing with.

\section{Threats to Validity}
\label{sec:threats-to-validity}

Although many participants showed great theoretical and practical knowledge
about TDD, it is hard to know if they do exactly the way they reported.
In addition, a considerable amount of them talked about practices that differ
from what TDD suggests.
As some of them do not follow TDD steps the way they theoretically should, it
may influence their opinion.

A few participants also affirmed that do not practice TDD regularly during their
daily jobs. They often practice TDD in pet projects. It may influence the
opinion as some of them lack TDD experience in real world projects.

The 10 participants represented over 7\% of the event's audience and there were
no pre-requirements to join the session. The sample may not be representative in
order to generalize the findings on this paper.

\section{Conclusions and Future Work}

There are still many experienced programmers adhering to Test-Driven
Development. This paper showed that most of TDD beginners' opinions and concerns
about the practice match with what is reported in literature. 
TDD is a technique that makes design problems more visible, regardless of the
level of experience with the practice; it is up to the developers to see them
and improve the design. Moreover, what makes developers fix design flaws is
their experience in software design, and not their experience in TDD itself. 
This is reinforced by the participants' opinions on the influence of experience
in the process.

Almost all participants agreed that design is a consequence of testing.
Programmers use the feedback from the tests to improve the
design. However, the specific question about how programmers get feedback from
tests did not reach the time limit, which may indicate that programmers do not
know exactly how they get this feedback.
They also talked a lot about design, and researchers were expecting
more citations from good design techniques, but there were only few
mentions to good object-oriented principles. It may indicate that a study
relating unit tests and object-oriented principles need to be done.

When start practicing TDD, beginners are mainly concerned about their
productivity; writing unit tests, constant refactoring, and doing baby steps all
the time suggest that developers will spend too much time to write code.
However, as participants suggested, the best way to see the benefits is by
trying to practice TDD.

This work also contributed with a different way to gather data about any agile
practice. Agile conferences with open spaces are becoming popular and it is a
good place for researchers to interview people from industry and also enable
participants to learn with each other. Also, researchers noticed that proposing
a fishbowling was a good way to make people with diverse experience to discuss
about the same subject and get different answers.
