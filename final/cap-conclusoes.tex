%% ------------------------------------------------------------------------- %%
\chapter{Conclusões e Trabalhos Futuros}
\label{cap:conclusoes}

Neste trabalho, fomos além da validação da afirmação de que a prática de DGT
melhora o projeto de classes. Discutimos e entendemos como a prática pode
fazer a diferença no dia a dia de um desenvolvedor de softwares.
Projetar classes é uma tarefa difícil. A prática de DGT e os
testes de unidade intrísecos ao processo podem ajudar desenvolvedores a antecipar
os problemas do projeto através de um \textit{feedback} rápido. \textit{Feedback}
esses que foram coletados, discutidos e catalogados também por esta pesquisa.

Em outras palavras, o desenvolvedor que pratica DGT escreve os testes antes do código.
Isso faz com que o teste de unidade que está sendo escrito sirva de
rascunho para o desenvolvedor. Ao observar o código do teste de unidade com atenção,
o desenvolvedor pode perceber problemas no projeto de classes que está criando.
Problemas esses como classes que possuem diversas responsabilidades
ou que possuem muitas outras dependências. Outras práticas dão o mesmo \textit{feedback}
para o desenvolvedor, mas a vantagem do teste é que o retorno é praticamente imediato.
Além disso, como o teste é escrito antes, o desenvolvedor pode mudar de ideia sobre
o projeto enquanto o custo de mudança ainda é baixo (afinal, o projeto ainda não
foi implementado).

Mas, ao contrário
do que é comentado pela indústria, a prática de DGT não guia o programador
para um bom projeto de classes de forma automática; a experiência e conhecimento 
do desenvolvedor são fundamentais ao criar software orientado a objetos. 
Divulgar o que foi encontrado por este trabalho é de extrema importância
para times de desenvolvimento de software, especialmente aos que seguem
algum tipo de metodologia ágil pois, ao conhecer os padrões aqui catalogados,
os desenvolvedores poderão perceber problemas de projeto mais cedo e
melhorar seu projeto de classes. 

Todos os padrões comentados neste trabalho foram levantados junto aos 
desenvolvedores da indústria de software brasileira. Apesar do pequeno
número de desenvolvedores entrevistados, muitos padrões emergiram. Isso
pode significar que existam ainda outros padrões de DGT. Um possível trabalho
futuro seria continuar na busca de padrões de \textit{feedback}.

\section{Lições Aprendidas}

Ao longo desta pesquisa, nós aprendemos, na prática, muita coisa sobre planejamento
e execução de estudos em engenharia de software. Alguns destes pontos valem
a pena serem mencionados para que o pesquisador que decidir evoluir o estudo
não cometa os mesmos erros que acabamos por cometer:

\begin{itemize}
	
	\item A execução do nosso estudo inicial exigia um ambiente razoavelmente
	complicado de ser configurado, como software de gravação de tela instalado, plugins do eclipse,
	controlador de versão Git, entre outros. Isso não facilitou as empresas que participaram
	do estudo. A consequência disso foi um certo atraso para iniciar a execução do estudo nas empresas, 
	devido ao alto número de softwares a serem iniciados antes da implementação. Além do mais, em muitas empresas,
	o software de gravação de tela e plugin do Eclipse não funcionaram e, ao fim, abandonamos a ideia de
	obter esses dados.
	
	\item Muitas participantes de uma só vez faz com que o pesquisador não consiga dar atenção
	a todos os participantes. Já que o ambiente era complicado de ser montado, muitos participantes
	esqueciam de determinadas etapas, ou tinham dúvidas sobre o enunciado. Na próxima execução,
	uma alternativa é levar um pesquisador auxiliar, ou até mesmo um ajudante.
	
	\item Caso o estudo seja feito com estudantes, os mesmos precisam de mais tempo para resolver
	a mesma quantidade de exercícios, e mais atenção, pois eles tiveram um maior número
	de dúvidas. Além disso, ambientes de faculdade são mais complicados de serem configurados.
	O pesquisador precisará de mais tempo para preparar o ambiente.
	
	\item Como o tempo dado a todos os participantes de uma empresa era fixo, alguns deles
	acabavam o exercício antes que outros. Em muitos casos, os participantes nos perguntavam
	o que deveriam fazer com o tempo restante. Constantemente eles nos perguntam sobre a possibilidade
	de refatorar o código que haviam escrito, e nós aceitávamos. Na próxima vez, sugerimos
	ao pesquisador que pense melhor nesses casos extremos.
	
	\item Por trabalhar com indústria, os mais diferentes problemas podem acontecer 
	até a execução do estudo. Uma empresa, por exemplo, cancelou sua participação
	dias antes. Em outras, alguns participantes que eram dados como certos, também
	não estavam presentes no dia. Sugerimos ao próximo pesquisador que faça contatos
	com muitas empresas e já trabalhe pensando em possíveis desistentes.
	
	\item Algumas empresas fora da região aceitaram participar do estudo. Em uma delas,
	conseguimos viajar até a cidade e executar o estudo. Nas outras optamos por não
	prosseguir com o estudo, afinal nosso planejamento não contemplava a execução
	do estudo de forma remota. Sugerimos a possibilidade de execução remota nos 
	próximos planejamentos.
	
	\item Apenas um piloto fez o processo por inteiro; todos os outros, por falta de tempo, 
	executaram apenas determinadas partes do estudo. Nos próximos, sugerimos
	que o pesquisador encontre participantes com mais tempo disponível e com experiências
	variadas.  
	
\end{itemize}
