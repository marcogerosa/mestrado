%% ------------------------------------------------------------------------- %%
\chapter{Conclusões}
\label{cap:conclusoes}

Class design is still a hard task. However, the practice of TDD and its unit tests can anticipate design problems through the possible feedbacks cited by the participants. Any tool that can help developers improve their design should be used. On the other hand, TDD does not lead to a good class design by its own, as suggest for some literature. The developer’s experience and knowledge makes the real difference when creating object-oriented software.


%------------------------------------------------------
\section{Sugestões para Pesquisas Futuras} 

A future work would be to a controlled experiment with combinations of other possible practices that help on class design, such as pair programming, would help researchers to find ways to increase class design feedback.

\section{Lições Aprendidas}

licoes aprendidas:
- ambiente complicado de configurar nao ajuda as empresas (gravar video, plugins, etc); atrasos para iniciar devido ao alto número de coisas pra arrumar;
  --> isso fez com que alguns dados não fossem gerados, como plugin do eclipse e video de gravação de tela
- muitos participantes de uma vez faz com que o experimento seja difícil, dado o número de pessoas que o pesquisador tem que cuidar (mesmo com as instruções,
eles as vezes pulam, esquecem.. o pesquisador precisa estar de olho o tempo todo). na próxima, levar ajudante
- estudantes precisam de mais atenção, pois tiveram mais dificuldades inclusive para configurar o ambiente
- ambientes de universidades são difíceis de serem configurados e podem apresentar problemas malucos
- pensar em casos mais extremos como participantes que acabam o segundo exercício mais rápido que o primeiro: podem voltar e continuar? podem refatorar
o código gerado? para todas elas, o pesquisador respondeu "sim"
- industria ocupada, muitos desistentes em cima da hora (1 empresa e alguns participantes das empresas que aceitaram).
- possibilidade de fazer o estudo remoto (mas menos controle ainda)
- apenas 1 piloto fez o processo inteiro. por isso o pesquisador, no proximo piloto, sempre focava no feedback das mudanças feitas. na próxima,
procurar por pilotos com mais disponibilidade, e se possivel, variar a experiencia do piloto para pegar feedback com pessoas experientes e inexperientes.

