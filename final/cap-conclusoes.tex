%% ------------------------------------------------------------------------- %%
\chapter{Conclusões e Trabalhos Futuros}
\label{cap:conclusoes}

Neste trabalho, fomos além da validação da afirmação de que a prática de DGT
melhora o projeto de classes. Discutimos e entendemos como a prática pode
fazer a diferença no dia a dia de um desenvolvedor de softwares.
Projetar classes é uma tarefa difícil. A prática de DGT e os
testes de unidade intrísecos ao processo podem ajudar desenvolvedores a antecipar
os problemas do projeto através de um \textit{feedback} rápido. \textit{Feedback}
esses que foram coletados, discutidos e catalogados também por esta pesquisa.

Em outras palavras, o desenvolvedor que pratica DGT escreve os testes antes do código.
Isso faz com que o teste de unidade que está sendo escrito sirva de
rascunho para o desenvolvedor. Ao observar o código do teste de unidade com atenção,
o desenvolvedor pode perceber problemas no projeto de classes que está criando.
Problemas esses como classes que possuem diversas responsabilidades
ou que possuem muitas outras dependências. Outras práticas dão o mesmo \textit{feedback}
para o desenvolvedor, mas a vantagem do teste é que o retorno é praticamente imediato.
Além disso, como o teste é escrito antes, o desenvolvedor pode mudar de ideia sobre
o projeto enquanto o custo de mudança ainda é baixo (afinal, o projeto ainda não
foi implementado).

Mas, ao contrário
do que é comentado pela indústria, a prática de DGT não guia o programador
para um bom projeto de classes de forma automática; a experiência e conhecimento 
do desenvolvedor são fundamentais ao criar software orientado a objetos. 
Divulgar o que foi encontrado por este trabalho é de extrema importância
para times de desenvolvimento de software, especialmente aos que seguem
algum tipo de metodologia ágil pois, ao conhecer os padrões aqui catalogados,
os desenvolvedores poderão perceber problemas de projeto mais cedo e
melhorar seu projeto de classes. 

Todos os padrões comentados neste trabalho foram levantados junto aos 
desenvolvedores da indústria de software brasileira. Apesar do pequeno
número de desenvolvedores entrevistados, muitos padrões emergiram. Isso
pode significar que existam ainda outros padrões de DGT. Um possível trabalho
futuro seria continuar na busca de padrões de \textit{feedback}.
