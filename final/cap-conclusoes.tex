%% ------------------------------------------------------------------------- %%
\chapter{Conclusões e Trabalhos Futuros}
\label{cap:conclusoes}

\section{Introdução}

Neste trabalho, provemos evidência empírica sobre os benefícios da prática de TDD
no projeto de classes.
Além da validação da afirmação de que a prática de TDD
melhora o projeto de classes, discutimos e entendemos como a prática pode
fazer a diferença no dia a dia de um desenvolvedor de softwares.

Ao revisitarmos as questões levantadas no começo desta pesquisa,
percebemos que as respostas que chegamos são muito parecidas
com as que são encontradas na literatura, com a diferença de 
que conseguimos observar padrões de \textit{feedback} que aparecem
no momento em que o desenvolvedor pratica TDD, e que o guia durante
o desenvolvimento.

Divulgar o que foi encontrado por este trabalho é de extrema importância
para times de desenvolvimento de software, especialmente aos que seguem
algum tipo de metodologia ágil pois, ao conhecer os padrões aqui catalogados,
os desenvolvedores poderão perceber problemas de projeto mais cedo e
melhorar seu projeto de classes.

Nas sub-seções abaixo, respondemos cada uma das questões levantadas
por este trabalho.

\section{Qual a influência de TDD no projeto de classes?}

A prática de TDD \textbf{pode} influenciar no processo de criação do projeto de classes.
No entanto, ao contrário do que é comentado pela indústria,
\textbf{a prática de TDD não guia o desenvolvedor para um bom projeto de classes
de forma automática}; a experiência e conhecimento 
do desenvolvedor são fundamentais ao criar software orientado a objetos. 

A prática, por meio dos seus possíveis \textit{feedbacks} em relação ao
projeto de classes, discutidos
em profundidade na Seção \ref{padroes-tdd}, podem servir de guia
para o desenvolvedor. Esses \textit{feedbacks}, quando observados, fazem
com que o desenvolvedor perceba problemas de projeto de classes de
forma antecipada, facilitando a refatoração do mesmo.

\textbf{Portanto, essa é a forma na qual a prática guia o desenvolvedor para
um melhor projeto de classes: dando retorno constante sobre os possíveis problemas
existentes no atual projeto de classes. É tarefa do desenvolvedor perceber
estes problemas e melhorar o projeto de acordo.}

\section{Qual a relação entre TDD e as tomadas de decisões de projeto
feitas por um desenvolvedor?}

Como discutido acima, a prática de TDD e seus
testes de unidade, intrísecos ao processo, podem ajudar os desenvolvedores a antecipar
os problemas do projeto através de um \textit{feedback} rápido.
Em outras palavras, o desenvolvedor que pratica TDD escreve os testes antes do código.
Isso faz com que o teste de unidade que está sendo escrito sirva de
rascunho para o desenvolvedor. Ao observar o código do teste de unidade com atenção,
o desenvolvedor pode perceber problemas no projeto de classes que está criando.
Problemas esses como classes que possuem diversas responsabilidades
ou que possuem muitas outras dependências. 

Outras práticas dão o mesmo \textit{feedback}
para o desenvolvedor, mas a vantagem do teste é que o retorno é praticamente imediato.
Além disso, como o teste é escrito antes, o desenvolvedor pode mudar de ideia sobre
o projeto enquanto o custo de mudança ainda é baixo (afinal, o projeto ainda não
foi implementado).

\section{Como a prática de TDD influencia o programador no processo de  
projeto de classes, do ponto de vista do acoplamento, coesão e complexidade?}

Ao escrever um teste de unidade para uma determinada classe, o desenvolvedor
é obrigado a passar sempre pelos mesmos passos. Todo teste de unidade é composto
de um conjunto de linhas responsáveis por montar o cenário do teste, um conjunto
de linhas que executam a ação sob teste e, por fim, um conjunto de linhas que
garantem que o comportamento foi executado de acordo com o esperado.

Uma dificuldade na escrita de qualquer um desses conjuntos pode implicar
em problemas no projeto de classes. Por exemplo, uma classe que para
ser testada necessita de grandes cenários, pode nos indicar que a classe
sob teste possui pré-condições muito complicadas. Já dificuldades na hora
de executar a ação sob teste pode nos indicar que a interface pública desta
classe não é amigável. 

Classes pouco coesas, por exemplo, possuem diversas responsabilidades diferentes.
Isso implica em mais pontos a serem testados que, por consequência, implica
em um maior número de testes para aquela unidade. Classes altamente acopladas,
por exemplo, exigem uma grande quantidade de objetos dublês, tornando a escrita
do teste mais difícil.

Seguindo esta linha de pensamento, concordamos com a opinião do Feathers,
citada no começo do trabalho, onde uma classe difícil de ser testada muito provavelmente
não apresenta um bom projeto de classes.

\subsection{Lições Aprendidas}

Ao longo desta pesquisa, nós aprendemos, na prática, muita coisa sobre planejamento
e execução de estudos em engenharia de software. Alguns destes pontos valem
a pena serem mencionados para que o pesquisador que decidir evoluir o estudo
não cometa os mesmos erros que acabamos por cometer:

\begin{itemize}
	
	\item A execução do nosso estudo inicial exigia um ambiente razoavelmente
	complicado de ser configurado, como software de gravação de tela instalado, plugins do eclipse,
	controlador de versão Git, entre outros. Isso não facilitou as empresas que participaram
	do estudo. A consequência disso foi um certo atraso para iniciar a execução do estudo nas empresas, 
	devido ao alto número de softwares a serem iniciados antes da implementação. Além do mais, em muitas empresas,
	o software de gravação de tela e plugin do Eclipse não funcionaram e, ao fim, abandonamos a ideia de
	obter esses dados.
	
	\item Muitas participantes de uma só vez faz com que o pesquisador não consiga dar atenção
	a todos os participantes. Já que o ambiente era complicado de ser montado, muitos participantes
	esqueciam de determinadas etapas, ou tinham dúvidas sobre o enunciado. Na próxima execução,
	uma alternativa é levar um pesquisador auxiliar, ou até mesmo um ajudante.
	
	\item Caso o estudo seja feito com estudantes, os mesmos precisam de mais tempo para resolver
	a mesma quantidade de exercícios, e mais atenção, pois eles tiveram um maior número
	de dúvidas. Além disso, ambientes de faculdade são mais complicados de serem configurados.
	O pesquisador precisará de mais tempo para preparar o ambiente.
	
	\item Como o tempo dado a todos os participantes de uma empresa era fixo, alguns deles
	acabavam o exercício antes que outros. Em muitos casos, os participantes nos perguntavam
	o que deveriam fazer com o tempo restante. Constantemente eles nos perguntam sobre a possibilidade
	de refatorar o código que haviam escrito, e nós aceitávamos. Na próxima vez, sugerimos
	ao pesquisador que pense melhor nesses casos extremos.
	
	\item Por trabalhar com indústria, os mais diferentes problemas podem acontecer 
	até a execução do estudo. Uma empresa, por exemplo, cancelou sua participação
	dias antes. Em outras, alguns participantes que eram dados como certos, também
	não estavam presentes no dia. Sugerimos ao próximo pesquisador que faça contatos
	com muitas empresas e já trabalhe pensando em possíveis desistentes.
	
	\item Algumas empresas fora da região aceitaram participar do estudo. Em uma delas,
	conseguimos viajar até a cidade e executar o estudo. Nas outras optamos por não
	prosseguir com o estudo, afinal nosso planejamento não contemplava a execução
	do estudo de forma remota. Sugerimos a possibilidade de execução remota nos 
	próximos planejamentos.
	
	\item Apenas um piloto fez o processo por inteiro; todos os outros, por falta de tempo, 
	executaram apenas determinadas partes do estudo. Nos próximos, sugerimos
	que o pesquisador encontre participantes com mais tempo disponível e com experiências
	variadas.  
	
\end{itemize}

\section{Trabalhos Futuros}

Todos os padrões comentados neste trabalho foram levantados junto aos 
desenvolvedores da indústria de software brasileira. Apesar do pequeno
número de desenvolvedores entrevistados, muitos padrões emergiram. Isso
pode significar que existam ainda outros padrões de TDD. Um possível trabalho
futuro seria continuar na busca de padrões de \textit{feedback}.

Além disso, um estudo que visa entender se desenvolvedores que conhecem
esses padrões de antemão percebem problemas de projeto antes de desenvolvedores
que não conhecem esses padrões, poderia ser de grande valia para a indústria.
Como observamos neste estudo, nem mesmo os especialistas perceberam diferenças
entre os códigos produzidos com e sem TDD. Uma justificativa para tal é
a falta de conhecimento sobre os padrões aqui discutdos.
É de vital importância que os desenvolvedores, além
de conhecer a mecânica da prática, entendam também como extrair retorno constante
sobre a qualidade do seu projeto de classes. Nós começamos esta pesquisa com
estudantes de diferentes semestres em uma universidade particular em São Paulo. Mas,
devido ao baixo número de participantes, optamos por não discutir os dados
analisados nesta pesquisa.
