%% ------------------------------------------------------------------------- %%
\chapter{Ameaças a Validade}
\label{cap:ameacas}

Este trabalho possui algumas possíveis ameaças a validade. O objetivo
deste capítulo é listá-las, bem como discutir alternativas para diminuir
os possíveis viéses.

\section{Validade interna}

\subsection{Efeitos recentes de DGT na memória}

Por serem entrevistados pouco tempo depois da resolução dos exercícios, os participantes terão
em suas mentes os efeitos recentes de DGT no código. Isso pode fazer com que o participante
não avalie friamente as vantagens e desvantagens do desenvolvimento sem DGT. 

Para diminuir esse viés, os participantes fizeram alguns exercícios também
sem DGT, para que ambos os estilos de desenvolvimento (com e sem DGT) estejam
recentes em sua memória.

\subsection{Exercícios de pequeno porte}

Os exercícios propostos são pequenos perto de um projeto real. Todos os exercícios propostos contém
problemas localizados de projeto de classes. E, uma vez que esta pesquisa tenta avaliar os efeitos de DGT no projeto de classes, 
acreditamos que os problemas conseguem simular de forma satisfatória
problemas de projeto de classes que desenvolvedores encaram no dia a dia de trabalho.

Além disso, ao final do exercício, os participantes responderão uma pergunta sobre a semelhança
entre os problemas de projeto de classes propostos e os problemas encontrados no mundo real.
As respostas serão levadas em consideração no processo de análise.

\subsection{Exercícios inacabados}

Alguns participantes não terminaram suas implementações dos exercícios. Isso
pode influenciar na análise quantitativa, afinal, um projeto de classes que
seria complexo assim que pronto, ao olho da métrica, ele aparenta ser simples.

\subsection{Influência do pesquisador}

Como discutido no capítulo \ref{cap:qualitativo}, o pesquisador possui
um papel fundamental em pesquisas qualitativas. Mas isso pode fazer com que
a interpretação dos resultados seja influenciada pelo contexto, experiências,
e até viéses do próprio pesquisador.
Neste estudo, a nossa opinião teve forte influência na seleção dos candidatas
para a entrevista.
Para diminuir esse problema, revisamos todas as análises,
buscando por conclusões incorretas ou não tão claras. 

\section{Validade externa}

\subsection{Desejabilidade social}

Enviesamento pela desejabilidade social é o termo científico usado para descrever
a tendência de que alguns participantes respondam questões de modo que serão
bem vistos pelos outros membros da comunidade \cite{crowne}.

Métodos ágeis e DGT possuem um discurso forte. A comunidade brasileira de métodos
ágeis ainda é nova e percebe-se de maneira empírica que muitos repetem o discurso
sem grande experiência ou embasamento no assunto.
No caso desta pesquisa, um possível viés é o participante responder o que
a literatura diz sobre DGT, e não exatamente o que ele pratica e sente sobre
os efeitos da prática. 

Para diminuir esse viés, nos comprometemos a eliminar do processo de análise os participantes
que responderam as perguntas de forma superficial, apenas repetindo a literatura. Na prática,
isso não aconteceu. Em sua maioria, poucas foram as respostas nas quais os participantes
foram superficiais. Nestes casos, essas respostas foram eliminadas da análise.

\subsection{Quantidade de participantes insuficiente}

Apesar de termos feito contato
com diversas empresas e grupos de desenvolvimento de software,
objetivando encontrar um bom número de participantes para a pesquisa,
a quantidade de participantes final do estudo pode não ser suficiente para generalizar
os resultados encontrados. 
