%% ------------------------------------------------------------------------- %%
\chapter{Ameaças a Validade}
\label{cap:ameacas}

SEPARAR AQUI EM INTERNAL, EXTERNAL E CONSTRUCT VALIDITY


TDD is very popular among agilists. The brazilian agile community is still young and one can empirically perceive that many developers are really influenced by literature, and not by their own experience or knowledge on the subject. Therefore, the desirability bias may affect this research, as the participant could have said not that he thinks, but what he has read. In order to reduce this problem, the research tried to eliminate answers that were not well explained.
Although almost all participants affirmed that the exer- cises contain real class design issues, they are still small in comparison to real software. The simplicity may have affected the code generated by the participants.
Some participants did not finish the exercises. This may affect the code metrics. That is why specialists were invited to evaluated the code. They were warned about this problem, and they evaluated not only the final generated code, but the final design the participant has intended (which can be easily seen by just looking the code).

* * * * * * 

Este trabalho possui algumas possíveis ameaças a validade. O objetivo
deste capítulo é listá-las, bem como discutir alternativas para diminuir
os possíveis viéses.

\section{Desejabilidade social}

Enviesamento pela desejabilidade social é o termo científico usado para descrever
a tendência de que alguns participantes respondam questões de modo que serão
bem vistos pelos outros membros da comunidade \cite{crowne}.

Métodos ágeis e TDD possuem um discurso forte. A comunidade brasileira de métodos
ágeis ainda é nova e percebe-se de maneira empírica que muitos repetem o discurso
sem grande experiência ou embasamento no assunto.
No caso desta pesquisa, um possível viés é o participante responder o que
a literatura diz sobre TDD, e não exatamente o que ele pratica e sente sobre
os efeitos da prática. 

Para diminuir esse viés, o pesquisador eliminará do processo de análise os participantes
que responderam as perguntas de forma superficial, apenas repetindo a literatura. Além disso,
o código gerado também será utilizado para validar participantes que não executaram aquilo que
falaram durante a entrevista.

\section{Efeitos recentes de TDD na memória}

Por serem entrevistados pouco tempo depois da resolução dos exercícios, os participantes terão
em suas mentes os efeitos recentes de TDD no código. Isso pode fazer com que o participante
não avalie friamente as vantagens e desvantagens do desenvolvimento sem TDD. 

Para diminuir esse viés, os participantes farão alguns exercícios também
sem TDD, para que ambos os estilos de desenvolvimento (com e sem TDD) estejam
recentes em sua memória.

\section{Exercícios de pequeno porte}

Os exercícios propostos são pequenos perto de um projeto real. Todos os exercícios propostos contém
problemas localizados de design. E, uma vez que esta pesquisa tenta avaliar os efeitos de TDD no design, 
acreditamos que os problemas conseguem simular de forma satisfatória
problemas de design que desenvolvedores encaram no dia a dia de trabalho.

Além disso, ao final do exercício, os participantes responderão uma pergunta sobre a semelhança
entre os problemas de design propostos e os problemas encontrados no mundo real.
As respostas serão levadas em consideração no processo de análise.

\section{Quantidade de participantes insuficiente}

A quantidade de participantes no estudo pode não ser suficiente para generalizar
os resultados encontrados. Para diminuir esse problema, o pesquisador fará contato
com diversas empresas e grupos de desenvolvimento de software da cidade de São Paulo,
objetivando encontrar um bom número de participantes para a pesquisa.

Caso o número de participantes ainda não seja suficiente para generalizar alguma afirmação, 
o pesquisador se compromete a deixar bem claro o contexto dos desenvolvedores entrevistados,
explicitando assim em quais casos os resultados encontrados fazem sentido.

\section{Influência do pesquisador}

Como discutido no capítulo \ref{cap:qualitativo}, o pesquisador possui
um papel fundamental em pesquisas qualitativas. Mas isso pode fazer com que
a interpretação dos resultados seja influenciada pelo contexto, experiências,
e até viéses do próprio pesquisador.

Para diminuir esse problema, outro pesquisador verificará todas as análises,
buscando por conclusões incorretas ou não tão claras. Além disso, após todo o
processo de interpretação dos dados, os resultados encontrados serão submetidos
para os participantes, e os mesmos poderão dar suas opiniões sobre os pontos
levantados.




opiniao do pesquisador para selecionar os candidatos para entrevista