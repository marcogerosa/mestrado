%% ------------------------------------------------------------------------- %%
\chapter{Ameaças a Validade e Lições Aprendidas}
\label{cap:ameacas}

\section{Validade interna}

Uma pesquisa tem alta validade interna quando ela é capaz de diminuir o valor das hipóteses alternativas, mostrando
que a hipótese estudada é a explicação mais plausível dos dados. Para isso, a pesquisa precisa controlar as possíveis
variáveis que poderiam influenciar na coleta, análise e interpretação dos dados. A validade interna é portanto
garantida quando o planejamento do estudo nos possibilita ter certeza de que as relações observadas de
forma empírica não podem ser explicadas por outros fatores.

As sub-seções abaixo discutem as possíveis ameaças a validade interna.

\subsection{Efeitos recentes de DGT na memória}

Por serem entrevistados pouco tempo depois da resolução dos exercícios, os participantes terão
em suas mentes os efeitos recentes de DGT no código. Isso pode fazer com que o participante
não avalie friamente as vantagens e desvantagens do desenvolvimento sem DGT. 

Para diminuir esse viés, os participantes fizeram alguns exercícios também
sem DGT, para que ambos os estilos de desenvolvimento (com e sem DGT) estejam
recentes em sua memória.

\subsection{Exercícios de pequeno porte}

Os exercícios propostos são pequenos perto de um projeto real. Todos os exercícios propostos contém
problemas localizados de projeto de classes. E, uma vez que esta pesquisa tenta avaliar os efeitos de DGT no projeto de classes, 
acreditamos que os problemas conseguem simular de forma satisfatória
problemas de projeto de classes que desenvolvedores encaram no dia a dia de trabalho.

Além disso, ao final do exercício, os participantes responderão uma pergunta sobre a semelhança
entre os problemas de projeto de classes propostos e os problemas encontrados no mundo real.
As respostas serão levadas em consideração no processo de análise.

\subsection{Exercícios inacabados}

Alguns participantes não terminaram suas implementações dos exercícios. Isso
pode influenciar na análise quantitativa, afinal, um projeto de classes que
seria complexo assim que pronto, ao olho da métrica, ele aparenta ser simples.

\subsection{Influência do pesquisador}

Como discutido no capítulo \ref{cap:qualitativo-planejamento}, o pesquisador possui
um papel fundamental em pesquisas qualitativas. Mas isso pode fazer com que
a interpretação dos resultados seja influenciada pelo contexto, experiências,
e até viéses do próprio pesquisador.
Neste estudo, a nossa opinião teve forte influência na seleção dos candidatas
para a entrevista.
Para diminuir esse problema, revisamos todas as análises,
buscando por conclusões incorretas ou não tão claras. 

\section{Validade externa}

Uma pesquisa possui validade externa quando ela possibilita ao pesquisador 
generalizar os resultados obtidos à outras populações ou outros contextos, 
mas sim que suas conclusões são verdadeiras também para outros contextos, outras pessoas. 

As sub-seções abaixo discutem as possíveis ameaças à validade externa
desta pesquisa.

\subsection{Desejabilidade social}

Enviesamento pela desejabilidade social é o termo científico usado para descrever
a tendência de que alguns participantes respondam questões de modo que serão
bem vistos pelos outros membros da comunidade \cite{crowne}.

Métodos ágeis e DGT possuem um discurso forte. A comunidade brasileira de métodos
ágeis ainda é nova e percebe-se de maneira empírica que muitos repetem o discurso
sem grande experiência ou embasamento no assunto.
No caso desta pesquisa, um possível viés é o participante responder o que
a literatura diz sobre DGT, e não exatamente o que ele pratica e sente sobre
os efeitos da prática. 

Para diminuir esse viés, nos comprometemos a eliminar do processo de análise os participantes
que responderam as perguntas de forma superficial, apenas repetindo a literatura. Na prática,
isso não aconteceu. Em sua maioria, poucas foram as respostas nas quais os participantes
foram superficiais. Nestes casos, essas respostas foram eliminadas da análise.

\subsection{Quantidade de participantes insuficiente}

Apesar de termos feito contato
com diversas empresas e grupos de desenvolvimento de software,
objetivando encontrar um bom número de participantes para a pesquisa,
a quantidade de participantes final do estudo pode não ser suficiente para generalizar
os resultados encontrados. 

\section{Lições Aprendidas}

Ao longo desta pesquisa, nós aprendemos, na prática, muita coisa sobre planejamento
e execução de estudos em engenharia de software. Alguns destes pontos valem
a pena serem mencionados para que o pesquisador que decidir evoluir o estudo
não cometa os mesmos erros que acabamos por cometer:

\begin{itemize}
	
	\item A execução do nosso estudo inicial exigia um ambiente razoavelmente
	complicado de ser configurado, como software de gravação de tela instalado, plugins do eclipse,
	controlador de versão Git, entre outros. Isso não facilitou as empresas que participaram
	do estudo. A consequência disso foi um certo atraso para iniciar a execução do estudo nas empresas, 
	devido ao alto número de softwares a serem iniciados antes da implementação. Além do mais, em muitas empresas,
	o software de gravação de tela e plugin do Eclipse não funcionaram e, ao fim, abandonamos a ideia de
	obter esses dados.
	
	\item Muitas participantes de uma só vez faz com que o pesquisador não consiga dar atenção
	a todos os participantes. Já que o ambiente era complicado de ser montado, muitos participantes
	esqueciam de determinadas etapas, ou tinham dúvidas sobre o enunciado. Na próxima execução,
	uma alternativa é levar um pesquisador auxiliar, ou até mesmo um ajudante.
	
	\item Caso o estudo seja feito com estudantes, os mesmos precisam de mais tempo para resolver
	a mesma quantidade de exercícios, e mais atenção, pois eles tiveram um maior número
	de dúvidas. Além disso, ambientes de faculdade são mais complicados de serem configurados.
	O pesquisador precisará de mais tempo para preparar o ambiente.
	
	\item Como o tempo dado a todos os participantes de uma empresa era fixo, alguns deles
	acabavam o exercício antes que outros. Em muitos casos, os participantes nos perguntavam
	o que deveriam fazer com o tempo restante. Constantemente eles nos perguntam sobre a possibilidade
	de refatorar o código que haviam escrito, e nós aceitávamos. Na próxima vez, sugerimos
	ao pesquisador que pense melhor nesses casos extremos.
	
	\item Por trabalhar com indústria, os mais diferentes problemas podem acontecer 
	até a execução do estudo. Uma empresa, por exemplo, cancelou sua participação
	dias antes. Em outras, alguns participantes que eram dados como certos, também
	não estavam presentes no dia. Sugerimos ao próximo pesquisador que faça contatos
	com muitas empresas e já trabalhe pensando em possíveis desistentes.
	
	\item Algumas empresas fora da região aceitaram participar do estudo. Em uma delas,
	conseguimos viajar até a cidade e executar o estudo. Nas outras optamos por não
	prosseguir com o estudo, afinal nosso planejamento não contemplava a execução
	do estudo de forma remota. Sugerimos a possibilidade de execução remota nos 
	próximos planejamentos.
	
	\item Apenas um piloto fez o processo por inteiro; todos os outros, por falta de tempo, 
	executaram apenas determinadas partes do estudo. Nos próximos, sugerimos
	que o pesquisador encontre participantes com mais tempo disponível e com experiências
	variadas.  
	
\end{itemize}
