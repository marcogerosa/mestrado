%% ------------------------------------------------------------------------- %%
\chapter{Trabalhos Relacionados}
\label{cap:trabalhos-relacionados}

Muitos estudos empíricos já foram realizados para avaliar os efeitos de DGT.
Muitos deles apresentam definições
de DGT que não levam em conta seus efeitos no projeto de classes, e talvez por isso muitas das 
pesquisas em relação à prática avaliam os efeitos de DGT sobre a qualidade 
externa, algo geralmente avaliado em técnicas de testes.
Além disso, diferentemente
do que esta pesquisa propõe, muitos estudos optaram por um
maior controle no experimento, e os realizaram dentro de ambientes acadêmicos 
com estudantes dos mais diversos cursos de computação.

Janzen \cite{janzen-arch-improvement} mostrou que programadores que usam DGT na 
indústria produziram código que passaram em, aproximadamente, 50\% mais testes 
caixa-preta do que o código produzido por grupos de controle que não usavam DGT.
O grupo que usava DGT gastou menos tempo depurando. Janzen também 
apontou que a complexidade dos algoritmos era muito menor e a quantidade e
cobertura dos testes era maior nos códigos escritos com DGT.

Outros trabalhos realizados na indústria também apresentam resultados parecidos.
Um estudo feito por Maximillien e Williams \cite{max-e-williams} mostrou uma
redução de 40-50\% na quantidade de defeitos e um impacto mínimo na
produtividade quando programadores usaram DGT. Outro estudo feito por Lui e
Chan \cite{lui-e-chan} comparando dois grupos, um utilizando DGT e o outro 
escrevendo testes apenas após a implementação, mostrou uma redução significativa 
no número de defeitos no grupo que utilizava DGT. 
Além disso, os defeitos que foram encontrados eram 
corrigidos mais rapidamente pelo grupo que utilizou DGT. O estudo feito por 
Damm, Lundberg e Olson \cite{damn-lundberg-e-olson} também mostra uma redução
em torno de 40\% a 50\% na quantidade de defeitos.

O estudo feito por George e Williams \cite{george-e-williams} mostrou que,
apesar de DGT poder reduzir inicialmente a produtividade dos desenvolvedores 
mais inexperientes, o código produzido passou entre 18\% a 50\% mais em testes 
caixa-preta do que códigos produzidos por grupos que não utilizavam DGT. Esse
código também apresentou uma cobertura de testes entre 92\% a 98\%. Uma análise
qualitativa mostrou que 87.5\% dos programadores acreditam que DGT facilitou o 
entendimento dos requisitos e 95.8\% acreditam que DGT reduziu o tempo gasto com
debug. 78\% também acreditam que DGT aumentou a produtividade da equipe. 
Entretanto, apenas 50\% dos participantes disseram que DGT ajuda a diminuir o tempo de 
desenvolvimento. Sobre qualidade, 92\% pensam que DGT ajuda a manter um
código de maior qualidade e 79\% acreditam que ele promove um projeto de classes mais simples.

Turnu \textit{et al.} \cite{turnu-tdd-opensouce} discutem produtividade em
projetos de código aberto. Segundo eles, a produtividade caiu quando DGT foi
adotado completamente, mas em compensação o número de problemas diminuiu 
consideravelmente.

Nagappan \cite{nagappan-ms} mostrou um estudo de caso na Microsoft e na IBM e os
resultados indicaram que o número de defeitos de quatro produtos diminuiu de 
40\% a 90\% em relação à projetos similares que não usaram DGT. Entretanto, o 
estudo mostrou também que DGT aumentou o tempo inicial de desenvolvimento entre 15\%
a 35\%. Langr \cite{langr} apontou que DGT aumenta a qualidade código, provê uma 
facilidade maior de manutenção e ajuda a produzir 33\% mais testes comparados  a
abordagens tradicionais.

Um estudo feito por Erdogmus \textit{et al} \cite{erdogmus-morisio} com 24 estudantes de
graduação mostrou que DGT aumenta a produtividade. Entretanto nenhuma diferença 
de qualidade no código foi encontrada.

Outro estudo feito por Janzen \cite{janzen-saiedian} com três diferentes grupos
de alunos (cada um deles usando uma abordagem diferente: DGT, testes depois, sem
testes), mostrou que o código produzido pelo time que fez DGT usou melhor os
conceitos de orientação a objetos e as responsabilidades foram separadas em 13 
diferentes classes, enquanto os outros times produziram um código mais
procedural. O time de DGT também produziu mais código e entregou mais
funcionalidades. Os testes produzidos por esse time teve duas vezes mais
asserções que os outros e cobriu 86\% mais possíveis caminhos no código 
do que o time \textit{test-last}. 
As classes testadas tinham valores de acoplamento 104\% menor do 
que as classes não testadas e os métodos eram, na média, 43\% menos complexos 
do que os não-testados.

Dogsa e Batic \cite{dogsa-batic} também encontraram uma melhora no
projeto de classes feita com DGT. Mas, segundo os autores, esse projeto de classes é 
consequência da simplicidade que a prática de DGT agrega ao processo. Eles
também  afirmaram que a bateria de testes de regressão gerada durante a prática 
possibilita ao desenvolvedor a constante refatoração do código.

Angela Li \cite{angela-li} propôs um estudo qualitativo para
entender a eficácia de DGT. Por meio de um estudo de caso, Angela coletou as 
percepções de benefícios que praticantes de DGT têm sobre a prática. Para isso ela
fez uso de cinco entrevistas semi-estruturadas realizadas em empresas de software de 
Auckland, Nova Zelândia. Os resultados das entrevistas foram analisados e alinhados
com os maiores temas discutidos sobre o assunto na literatura: qualidade de código,
qualidade da aplicação e produtividade do desenvolvedor.
No que diz respeito à qualidade de código, Li chegou a conclusão de
que DGT guia o desenvolvedor para classes mais simples e com melhor projeto de classes. 
Além disso, o código tende a ser mais simples e fácil de ler.
De acordo com o trabalho, os principais fatores que contribuem para esses benefícios
é a maior confiança em refatorar e modificar código, uma maior cobertura de testes,
entendimento mais profundo dos requisitos, maior facilidade na compreensão do código,
grau e escopo de erros reduzidos, além de uma maior satisfação pessoal do desenvolvedor.

O praticante de DGT geralmente faz uso também de outras práticas ágeis, como
programação pareada, que podem dificultar o processo de avaliação dos benefícios
de DGT. Madeyski \cite{madeyski-package-dependencies} observou os resultados
entre grupos que praticavam DGT, grupos que praticavam programação pareada, 
e a combinação entre elas,
e não conseguiu mostrar grande diferença entre equipes que utilizam programação 
pareada e equipes que utilizam DGT, no que diz respeito ao gerenciamento de dependências entre 
pacotes de classes. Entretanto, ao combinar os resultados, Madeyski encontrou que DGT pode 
ajudar no nível de gerenciamento de dependências entre classes. Segundo ele, o 
programador deve utilizar DGT, mas ficar atento a possíveis problemas de projeto de classes.

O estudo de Muller e Hagner \cite{muller-e-hagner} apontou que DGT não resulta
em melhor qualidade ou produtividade. Entretanto, os estudantes perceberam um 
melhor reúso dos códigos produzidos com DGT. Steinberg \cite{steinberg} mostrou
que código produzido com DGT é mais coeso e menos acoplado. Os estudantes também
reportaram que os defeitos eram mais fáceis de serem corrigidos. A pesquisa feita
por Edwards \cite{edwards}, com 59 estudantes, mostrou que o código produzido com
DGT tem 45\% menos defeitos e faz o programador se sentir mais a vontade
com ele.

Aprender DGT também não é tarefa fácil. Mugridge \cite{mugridge} identificou
dois desafios principais em ensinar DGT nos últimos dois anos: fazer os estudantes
pensarem novamente sobre o projeto de classes, e fazê-los se envolver com essa nova
abordagem. Além disso, é difícil ensinar testes de unidade, projeto de classes e refatoração
de maneira explícita para os estudantes. Contudo, segundo Proulx \cite{proulx}, a partir do momento que
o estudante aprende DGT, ele tende a ter uma melhor performance em disciplinas
de orientação a objetos. Segundo ele, essa melhora é percebida inclusive pelos
empregadores desses alunos. 

Como os estudos acima acabam por misturar efeitos da prática de DGT na
qualidade externa e interna, a Tabela \ref{tab:comparativo} mostra
quais trabalhos apontaram efeitos no projeto de classes.

\begin{table}
	\begin{tabular}{ | l | l |}
		
		\hline
		
		Tópico & Trabalho\\

		\hline
		
		Simplicidade & \cite{janzen-arch-improvement}, \cite{janzen-saiedian}, \cite{angela-li} \\
		Facilidade de manutenção & \cite{langr}\\
		Melhor utilização de conceitos de orientação a objetos & \cite{janzen-saiedian}, \cite{angela-li}\\
		Separação de responsabilidades & \cite{janzen-saiedian}, \cite{steinberg}\\
		Menor acoplamento & \cite{janzen-saiedian}, \cite{steinberg}\\
		Maior reuso de classes & \cite{muller-e-hagner} \\

		\hline

	\end{tabular}
	\label{tab:comparativo}
	\caption{Relação entre Conceitos de Orientação a Objetos e Estudos Já Executados com a Prática de DGT}
\end{table}

Outras compilações de estudos sobre DGT também podem ser encontrados no livro
\textit{Test-Driven Development: An Empirical Evaluation of Agile Practice},
escrito por Madeyski \cite{madeyski-livro} ou no trabalho entitulado
\textit{Test driven development: empirical body of evidence}, feito por
Siniaalto \cite{tdd-body-of-evidence}.

%% ------------------------------------------------------------------------- %%
\section{Discussão}

Como visto anteriormente, poucos trabalhos avaliam os efeitos de DGT sobre o
projeto de classes. Quando o fazem, apenas discutem quais os efeitos da prática
e não exatamente \textbf{como} DGT os influencia. Josefsson
\cite{josefsson}, em sua discussão sobre a necessidade de uma fase de projeto
arquitetural e os efeitos de DGT nesse quesito, chega à mesma conclusão. Segundo
ele, os estudos sobre DGT encontrados na literatura atual são muito limitados e
não são generalizáveis. Por esse motivo, os ditos efeitos que DGT têm 
sobre o projeto de classes não podem ser provados. Apesar desse artigo ser datado de 2004, e
muitos trabalhos terem sido realizados após essa data, o autor deste trabalho ainda acredita 
que os efeitos de DGT ainda não foram provados pela literatura atual.

Grande parte desses estudos também não levam em conta a experiência do
programador que está praticando DGT. Geralmente esse ponto é discutido apenas 
na seção de ameaças à validade do estudo. Janzen, em seu doutorado, percebeu que
desenvolvedores mais maduros obtêm mais benefícios de DGT, escrevendo classes
mais simples. Além disso, desenvolvedores maduros que experimentam a prática
tendem a optar por DGT mais facilmente do que desenvolvedores menos experientes
\cite{janzen-phd}.

Os trabalhos que analisam DGT do ponto de vista de projeto de classes, no entanto, não
chegam a resultados conclusivos; muitos deles dizem que os efeitos
de DGT não são tão diferentes daqueles dos times que não praticam DGT.  A própria tese de
doutorado de Janzen foi inconclusiva no que diz respeito à influência de DGT no 
acoplamento e na coesão \cite{janzen-phd}. 

Além disso, outro ponto fortemente relacionado com projeto de classes é a simplicidade e
facilidade de evolução. Um projeto de classes rígido, que não possibilita mudanças de maneira
fácil, é difícil de ser medido de maneira quantitativa. Complexidade
desnecessária também é totalmente subjetiva. 

Portanto, a crítica dessa pesquisa
com relação aos trabalhos relacionados é justamente na análise feita sobre os
efeitos da prática no DGT. É necessário mais do que uma comparação analítica; o
ponto de vista dos desenvolvedores, que atuam naquele código-fonte durante todo
o dia de trabalho deve ser levado em consideração.

%% ------------------------------------------------------------------------- %%
\section{Posição desta pesquisa na literatura atual}

Esta pesquisa se mostra diferente da maioria dos trabalhos encontrados na
literatura atual. Além de observar DGT pelo ponto de vista única e
exclusivamente do projeto de classes, colhe-se informações baseadas no ponto de
vista de desenvolvedores que a praticam.

Talvez o trabalho mais parecido com o que é proposto aqui é o
realizado por Angela Li, em 2009, que apresenta um estudo qualitativo sobre os
efeitos de DGT no processo de desenvolvimento de software \cite{angela-li}. 
A diferença é que esta pesquisa se concentra em entender os
efeitos de DGT no projeto de classes.

O caminho em destaque da Figura \ref{fig:posicao-pesquisa} mostra a nossa posição
em relação ao que já é encontrado na literatura.

\begin{figure}[h!]
  \centering
  \includegraphics[scale=0.35]{posicao-pesquisa.png}
  \caption{Posição desta pesquisa na literatura atual}
  \label{fig:posicao-pesquisa}
\end{figure}

