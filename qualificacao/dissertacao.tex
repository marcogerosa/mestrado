% Arquivo LaTeX de exemplo de dissertação/tese a ser apresentados à CPG
% do IME-USP
% 
% Versão 3: Thu Dec 10 23:57:33 BRST 2009

%
% Criação: Jesús P. Mena-Chalco
% Revisão: Fábio Kon
%  
% Obs: Leia previamente o texto do arquivo README.txt
%

\documentclass[11pt,twoside,a4paper]{book}

% ---------------------------------------------------------------------------- %
% Pacotes 
\usepackage[T1]{fontenc}
\usepackage[brazil]{babel}
\usepackage[utf8]{inputenc}
\usepackage[pdftex]{graphicx}           % usamos arquivos pdf/png como figuras
\usepackage{pifont}
\usepackage{amsfonts}
\usepackage{amssymb} 
\usepackage{setspace}                   % espaçamento flexível
\usepackage[bf,small,compact]{titlesec} % cabeçalhos dos títulos: menores e compactos
\usepackage{indentfirst}                % indentação do primeiro parágrafo
\usepackage{subfigure}                  % uso de várias figuras numa só
\usepackage{makeidx}                    % índice remissivo
\usepackage[nottoc]{tocbibind}          % acrescentamos a bibliografia/indice/conteudo no Table of Contents
\usepackage{courier}                    % usa o Adobe Courier no lugar de Computer Modern Typewriter
\usepackage{type1cm}                    % fontes realmente escaláveis
\usepackage{listings}                   % para formatar código-fonte (ex. em Java)
\usepackage{setspace}
\usepackage{longtable}
\usepackage{titletoc}
\usepackage{lscape}
\usepackage[fixlanguage]{babelbib}
\usepackage[font=small,format=plain,labelfont=bf,up,textfont=it,up]{caption}
\usepackage[usenames,svgnames,dvipsnames]{xcolor}
\usepackage[a4paper,top=2.54cm,bottom=2.0cm,left=2.0cm,right=2.54cm]{geometry} % margens
\usepackage[pdftex,plainpages=false,pdfpagelabels,pagebackref,colorlinks=true,citecolor=black,linkcolor=black,urlcolor=black,filecolor=black,bookmarksopen=true]{hyperref} % links em preto
%\usepackage[pdftex,plainpages=false,pdfpagelabels,pagebackref,colorlinks=true,citecolor=DarkGreen,linkcolor=NavyBlue,urlcolor=DarkRed,filecolor=green,bookmarksopen=true]{hyperref} % links coloridos
\usepackage[all]{hypcap}                % soluciona o problema com o hyperref e capitulos
\usepackage[square,sort,nonamebreak,comma]{natbib}  % citação
                                % bibliográfica alpha (alpha-ime.bst)
\usepackage{float}

\usepackage{type1cm}      % fontes realmente escaláveis
\fontsize{60}{62}\usefont{OT1}{cmr}{m}{n}{\selectfont}

% ---------------------------------------------------------------------------- %
\floatstyle{boxed}
\newfloat{caixa}{htb}{bx}
\floatname{caixa}{Caixa}

% ---------------------------------------------------------------------------- %
% headers similares oa TAOP de Donald E. Knuth
\usepackage{fancyhdr}
\pagestyle{fancy}
\fancyhf{}
\renewcommand{\chaptermark}[1]{\markboth{\MakeUppercase{#1}}{}}
\renewcommand{\sectionmark}[1]{\markright{\MakeUppercase{#1}}{}}
\renewcommand{\headrulewidth}{0pt}

% ---------------------------------------------------------------------------- %
\graphicspath{{./figuras/}}             % path das figuras (recomendável)
\frenchspacing                          % Arruma o espaço: id est (i.e.) e exempli gratia (e.g.) 
\urlstyle{same}                         % URL com o mesmo estilo do texto e nao mono-spaced
\makeindex                              % para o índice remissivo
\raggedbottom                           % para não permitir espaços extra no texto
\fontsize{60}{62}\usefont{OT1}{cmr}{m}{n}{\selectfont}
\cleardoublepage
\normalsize

% ---------------------------------------------------------------------------- %
% Opções de listing usados para o código fonte
% Ref: http://en.wikibooks.org/wiki/LaTeX/Packages/Listings
\lstset{ %
language=Java,                  % choose the language of the code
basicstyle=\footnotesize,       % the size of the fonts that are used for the code
numbers=left,                   % where to put the line-numbers
numberstyle=\footnotesize,      % the size of the fonts that are used for the line-numbers
stepnumber=1,                   % the step between two line-numbers. If it's 1 each line will be numbered
numbersep=5pt,                  % how far the line-numbers are from the code
showspaces=false,               % show spaces adding particular underscores
showstringspaces=false,         % underline spaces within strings
showtabs=false,                 % show tabs within strings adding particular underscores
frame=single,	                % adds a frame around the code
framerule=0.6pt,
tabsize=2,	                    % sets default tabsize to 2 spaces
captionpos=b,                   % sets the caption-position to bottom
breaklines=true,                % sets automatic line breaking
breakatwhitespace=false,        % sets if automatic breaks should only happen at whitespace
escapeinside={\%*}{*)},         % if you want to add a comment within your code
backgroundcolor=\color[rgb]{1.0,1.0,1.0}, % choose the background color.
rulecolor=\color[rgb]{0.8,0.8,0.8},
extendedchars=true,
xleftmargin=10pt,
xrightmargin=10pt,
framexleftmargin=10pt,
framexrightmargin=10pt
}

% ---------------------------------------------------------------------------- %
% Corpo do texto
\begin{document}
\frontmatter 
% headers para as páginas do frontmatter 
\fancyhead[RO]{{\footnotesize\rightmark}\hspace{2em}\thepage}
\setcounter{tocdepth}{2}
\fancyhead[LE]{\thepage\hspace{2em}\footnotesize{\leftmark}}
\fancyhead[RE,LO]{}
\fancyhead[RO]{{\footnotesize\rightmark}\hspace{2em}\thepage}

\onehalfspacing  % espaçamento


% ---------------------------------------------------------------------------- %
% Capa
\thispagestyle{empty}
\begin{center}
  \vspace*{2.3cm}
  \textbf{\Large{Um estudo sobre o Impacto de Test-Driven Development\\
      na Qualidade do Design de Software}}\\
	
  \vspace*{1.2cm} \Large{Mauricio Finavaro Aniche}
    
  \vskip 2cm \textsc{
    Dissertação apresentada\\[-0.25cm]
    ao\\[-0.25cm]
    Instituto de Matemática e Estatística\\[-0.25cm]
    da\\[-0.25cm]
    Universidade de São Paulo}
    
  \vskip 1.5cm
  Programa: Mestrado em Ciência da Computação\\
  Orientador: Prof. Dr. Marco Aurélio Gerosa
	
  \vskip 0.5cm \normalsize{São Paulo, Janeiro de 2011}
\end{center}

% ---------------------------------------------------------------------------- %
% Página de rosto (só para a versão final) \newpage
% \thispagestyle{empty}
%	\begin{center}
%   \vspace*{2.3 cm}
%   \textbf{\Large{Título do trabalho a ser apresentado à \\
%       CPG para a dissertação/tese}}\\
%   \vspace*{2 cm}
%	\end{center}
%
%	\vskip 2cm
%
%	\begin{flushright}
%   Este exemplar corresponde à redação\\
%   final da dissertação devidamente corrigida\\
%   e defendida por Hugo Corbucci\\
%   e aprovada pela Comissão Julgadora.  \vskip 2cm
%
%	\end{flushright}
%	\vskip 4.2cm
%
%	\begin{quote}
%   \noindent Banca Examinadora:
%	
%   \begin{itemize}
%   \item Prof. Dr. Alfredo Goldman (orientador) - IME-USP.
%   \item Prof. Dr. Fabio Kon - IME-USP.
%   \item Prof. Dr. José Carlos Maldonado - ICMC-USP.
%   \end{itemize}
%	  
%	\end{quote}
% \pagebreak

\pagenumbering{roman} % começamos a numerar

% ---------------------------------------------------------------------------- %
% Agradecimentos
\chapter*{Agradecimentos}

Agradeco ...
% TODO Completar

% ---------------------------------------------------------------------------- %
% Resumo
\chapter*{Resumo}

resumo

\noindent \textbf{Palavras-chave:} p1, p2, p3

% ---------------------------------------------------------------------------- %
% Abstract
\chapter*{Abstract}

abstract

\noindent \textbf{Keywords:} k1, k2

% ---------------------------------------------------------------------------- %
% Sumário
\tableofcontents % imprime o sumário

% ---------------------------------------------------------------------------- %
% Listas: abreviaturas, símbolos, figuras e tabelas

\chapter{Lista de Abreviaturas}
\begin{tabular}{ll}
  TDD	& Test-Driven Development.\\
  XP       & Programação Extrema (\emph{Extreme Programming}).\\
\end{tabular}

% \chapter{Lista de Símbolos}
% \begin{tabular}{ll}
%		$\omega$    & Freqüência angular.\\
%\end{tabular}

\listoffigures % lista de Figuras
%\listoftables % lista de Tabelas

% ---------------------------------------------------------------------------- %
% Capítulos
\mainmatter
% cabecalho para as páginas do 'mainmatter'
\fancyhead[RE,LO]{\thesection}

% \singlespacing % espaçamento simples
\onehalfspacing % espaçamento um e meio
% \doublespacing % espaçamento duplo

\input cap-introducao % associado ao arquivo: 'cap-introducao.tex'

% cabecalho para os apêndices
\renewcommand{\chaptermark}[1]{\markboth{\MakeUppercase{\appendixname\ \thechapter}} {\MakeUppercase{#1}} }
\fancyhead[RE,LO]{}
\appendix

% APENDICES
%\include{ape-pesquisaEA}

% ---------------------------------------------------------------------------- %
% Bibliografia
\backmatter \singlespacing   % espaçamento simples

\bibliographystyle{alpha-ime}% citação bibliográfica alpha
\bibliography{bibliografia}  % associado ao arquivo: 'bibliografia.bib'

% ---------------------------------------------------------------------------- %
% Índice remissivo
%\index{TBP|see{periodicidade região codificante}}

%\printindex   % imprime o índice remissivo no documento 

\end{document}

