% ---------------------------------------------------------------------------- %
% Influência de Test-Driven Development no Design Orientado a Objetos: Um      %
% Estudo Qualitativo com Desenvolvedores                                       %
%                                                                              %
% Mauricio Finavaro Aniche                                                     %
% mauricioaniche@gmail.com                                                     %
%                                                                              % 
% ---------------------------------------------------------------------------- %

\documentclass[11pt,openany,twoside,a4paper]{book}

% ---------------------------------------------------------------------------- %
% Pacotes 
\usepackage[T1]{fontenc}
\usepackage[brazil]{babel}
\usepackage[utf8]{inputenc}
\usepackage[pdftex]{graphicx}           % usamos arquivos pdf/png como figuras
\usepackage{pifont}
\usepackage{amsfonts}
\usepackage{amssymb} 
\usepackage{setspace}                   % espaçamento flexível
\usepackage[bf,small,compact]{titlesec} % cabeçalhos dos títulos: menores e compactos
\usepackage{indentfirst}                % indentação do primeiro parágrafo
\usepackage{subfigure}                  % uso de várias figuras numa só
\usepackage{makeidx}                    % índice remissivo
\usepackage[nottoc]{tocbibind}          % acrescentamos a bibliografia/indice/conteudo no Table of Contents
\usepackage{courier}                    % usa o Adobe Courier no lugar de Computer Modern Typewriter
\usepackage{type1cm}                    % fontes realmente escaláveis
\usepackage{listings}                   % para formatar código-fonte (ex. em Java)
\usepackage{setspace}
\usepackage{longtable}
\usepackage{multirow}
\usepackage{titletoc}
\usepackage{lscape}
\usepackage[fixlanguage]{babelbib}
\usepackage[font=small,format=plain,labelfont=bf,up,textfont=it,up]{caption}
\usepackage[usenames,svgnames,dvipsnames,table]{xcolor}
\usepackage[a4paper,top=2.54cm,bottom=2.0cm,left=2.0cm,right=2.54cm]{geometry}% margens
\usepackage{float}
\usepackage[pdftex,plainpages=false,pdfpagelabels,pagebackref,colorlinks=true,citecolor=black,linkcolor=black,urlcolor=black,filecolor=black,bookmarksopen=true]{hyperref} % links em preto
\usepackage[all]{hypcap}                % soluciona o problema com o hyperref e capitulos
\usepackage[numbers,square,sort,nonamebreak,comma]{natbib}  % citação
                                % bibliográfica alpha (alpha-ime.bst)

\usepackage{type1cm}      % fontes realmente escaláveis
\fontsize{60}{62}\usefont{OT1}{cmr}{m}{n}{\selectfont}

% ---------------------------------------------------------------------------- %
% Cabeçalhos similares ao TAOCP de Donald E. Knuth
\usepackage{fancyhdr}
\pagestyle{fancy}
\fancyhf{}
\renewcommand{\chaptermark}[1]{\markboth{\MakeUppercase{#1}}{}}
\renewcommand{\sectionmark}[1]{\markright{\MakeUppercase{#1}}{}}
\renewcommand{\headrulewidth}{0pt}

% ---------------------------------------------------------------------------- %
\graphicspath{{./figuras/}}             % caminho das figuras (recomendável)
\frenchspacing                          % arruma o espaço: id est (i.e.) e exempli gratia (e.g.) 
\urlstyle{same}                         % URL com o mesmo estilo do texto e não mono-spaced
\makeindex                              % para o índice remissivo
\raggedbottom                           % para não permitir espaços extra no texto
\fontsize{60}{62}\usefont{OT1}{cmr}{m}{n}{\selectfont}
\cleardoublepage
\normalsize

% ---------------------------------------------------------------------------- %
% Opções de listing usados para o código fonte
% Ref: http://en.wikibooks.org/wiki/LaTeX/Packages/Listings
\lstset{ %
language=Java,                  % choose the language of the code
basicstyle=\footnotesize,       % the size of the fonts that are used for the code
numbers=left,                   % where to put the line-numbers
numberstyle=\footnotesize,      % the size of the fonts that are used for the line-numbers
stepnumber=1,                   % the step between two line-numbers. If it's 1 each line will be numbered
numbersep=5pt,                  % how far the line-numbers are from the code
showspaces=false,               % show spaces adding particular underscores
showstringspaces=false,         % underline spaces within strings
showtabs=false,                 % show tabs within strings adding particular underscores
frame=single,	                % adds a frame around the code
framerule=0.6pt,
tabsize=2,	                    % sets default tabsize to 2 spaces
captionpos=b,                   % sets the caption-position to bottom
breaklines=true,                % sets automatic line breaking
breakatwhitespace=false,        % sets if automatic breaks should only happen at whitespace
escapeinside={\%*}{*)},         % if you want to add a comment within your code
backgroundcolor=\color[rgb]{1.0,1.0,1.0}, % choose the background color.
rulecolor=\color[rgb]{0.8,0.8,0.8},
extendedchars=true,
xleftmargin=10pt,
xrightmargin=10pt,
framexleftmargin=10pt,
framexrightmargin=10pt
}

% ---------------------------------------------------------------------------- %
% Corpo do texto
\begin{document}
\frontmatter 
% cabeçalho para as páginas das seções anteriores ao capítulo 1 (frontmatter)
\fancyhead[RO]{{\footnotesize\rightmark}\hspace{2em}\thepage}
\setcounter{tocdepth}{2}
\fancyhead[LE]{\thepage\hspace{2em}\footnotesize{\leftmark}}
\fancyhead[RE,LO]{}
\fancyhead[RO]{{\footnotesize\rightmark}\hspace{2em}\thepage}

\onehalfspacing  % espaçamento

% ---------------------------------------------------------------------------- %
% Capa
% ---------------------------------------------------------------------------- %
\thispagestyle{empty}
\begin{center}
    \vspace*{2.3cm}
    \textbf{\Large{Influência de Test-Driven Development nas Decisões de Design \\
	em Sistemas Orientado a Objetos: Estudos com Desenvolvedores}}\\
    
    \vspace*{1.2cm}
    \Large{Mauricio Finavaro Aniche}
    
    \vskip 2cm
    \textsc{
    Dissertação apresentada\\[-0.25cm] 
    ao\\[-0.25cm]
    Instituto de Matemática e Estatística\\[-0.25cm]
    da\\[-0.25cm]
    Universidade de São Paulo\\[-0.25cm]
    para\\[-0.25cm]
    obtenção do título\\[-0.25cm]
    de\\[-0.25cm]
    Mestre em Ciência da Computação}
    
    \vskip 1.5cm
    Programa: Mestrado em Ciência da Computação\\
    Orientador: Prof. Dr. Marco Aurélio Gerosa

   	\vskip 1.5cm
    \normalsize{São Paulo, abril de 2011}
\end{center}

% ---------------------------------------------------------------------------- %
% Página de rosto (só para a versão final)
% ---------------------------------------------------------------------------- %
%\newpage
%\thispagestyle{empty}
%    \begin{center}
%        \vspace*{2.3 cm}
%        \textbf{\Large{Título do trabalho a ser apresentado à \\
%        CPG para a dissertação/tese}}\\
%        \vspace*{2 cm}
%    \end{center}

%    \vskip 2cm

%    \begin{flushright}
    % Este exemplar corresponde à redação\\
    % final da dissertação/tese devidamente corrigida\\
    % e defendida por (Nome Completo do Aluno)\\
    % e aprovada pela Comissão Julgadora.
    %
%	Esta versão definitiva da tese/dissertação\\
%	contém as correções e alterações sugeridas pela\\
%	Comissão Julgadora durante a defesa realizada\\
%    por (Nome Completo do Aluno) em 4/5/2010.

%    \vskip 2cm

%    \end{flushright}
%    \vskip 4.2cm

%    \begin{quote}
%    \noindent Comissão Julgadora:
    
%    \begin{itemize}
%		\item Profa. Dra. Nome Completo (orientadora) - IME-USP [sem ponto final]
%		\item Prof. Dr. Nome Completo - IME-USP [sem ponto final]
%		\item Prof. Dr. Nome Completo - IMPA [sem ponto final]
%    \end{itemize}
      
%    \end{quote}
%\pagebreak

\pagenumbering{roman}     % começamos a numerar 

% ---------------------------------------------------------------------------- %
% TODO: agradecimentos
%\chapter*{Agradecimentos}
%Agradecimentos ...

% ---------------------------------------------------------------------------- %
\chapter*{Resumo}

Test-Driven Development (TDD) é uma das práticas sugeridas na Programação
Extrema (XP).
A mecânica da prática é simples: o programador escreve o teste antes
de escrever o código. Apesar de parecer que TDD é uma prática focada em testes
de software (afinal, seu nome é composto da palavra ``teste''), na verdade sua
grande contribuição é o \textit{feedback} em relação à qualidade do design produzido.
Na opinião de muitos autores conhecidos, TDD além de guiar o programador na
criação do design, ainda ajuda a manter o código mais simples,
mais claro etc.

É realmente difícil entender como TDD influencia no processo de desenvolvimento
de software. Boa parte dos experimentos da
academia sobre a prática verificam os efeitos dela sobre a qualidade externa. 
Apesar da pouca quantidade, alguns experimentos mostram que TDD tem uma influência
positiva no design de classes, diminuindo o grau de acoplamento, aumentando
a coesão e a simplicidade de classes e módulos. Entretanto, poucos trabalhos visam
entender a razão pela qual a prática leva os desenvolvedores a obterem bons resultados.

Este trabalho tem por objetivo compreender melhor os efeitos de TDD e como sua prática 
influencia o 
desenvolvedor durante o processo de design de sistemas orientados a objetos. Mais ainda,
como TDD influencia nas decisões de design tomadas por um desenvolvedor durante o
seu dia a dia de desenvolvimento.

Este estudo faz uso de uma combinação entre um estudo quantitativo inicial, na qual participantes serão
convidados a resolver exercícios pré-preparados utilizando TDD e, a partir dos dados colhidos nesse estudo, um outro
estudo qualitativo entrará em detalhes objetivando entender como a prática influenciou as decisões de design dos participantes.

\noindent \textbf{Palavras-chave:} Test-Driven Development, Sistemas Orientados
a Objetos, Design de Classes, Métodos Qualitativos de Pesquisa.

% ---------------------------------------------------------------------------- %
\chapter*{Abstract}

Test-Driven Development (TDD) is one of the most known practices of Extreme
Programming (XP). The mechanical is simple: the developer writes a test before
writing the implementation. Although TDD may look like a testing practice
(after all, TDD has the word ``test'' in its name), its main contribution is the
feedback in the class design. Most well-known authors claim that TDD, besides
driving the design, it still helps programmers to maintain the code simpler, 
cleaner, etc.

It is really hard to understand how TDD influences on the software development
process. Most part of the experiments in academia evaluate the effects of TDD
in the external quality. Despite the small quantity, some experiments
show that TDD has a positive influence on class design, reducing the coupling, and
increasing the cohesion and simplicity of classes and modules. However, only a
few works aim to understand the reason in which the practice guides practitioners
to obtain good results.

This work aims to understand better the effects and how TDD influences the
practitioner during the class design process in object-oriented systems. 
Even more, how TDD influences the design decisions taken by developers
during their development tasks.

This work uses a combination of a initial quantitative study, in which the participants
are invited to solve a few prepared exercises using TDD and, based on data gathered on this,
another qualitative study will explore, in details, the way TDD influenced the design decisions
taken by the participants.

This work analyses the effects of the practice through the
point of view of developers that practice it. In order
to achieve this goal, this work proposes the use of qualitative methods and
suggests a experiment followed by suite of interviews  
with developers in the brazilian software industry.

\noindent \textbf{Keywords:} Test-Driven Development, Object-Oriented
Systems, Class Design, Qualitative Research Methods.

% ---------------------------------------------------------------------------- %
% Sumário
\tableofcontents    % imprime o sumário

% ---------------------------------------------------------------------------- %
\chapter{Lista de Abreviaturas}
\begin{tabular}{ll}
         TDD         & \emph{Test-Driven Development}\\ 
         
         XP          & Programação Extrema (\emph{Extreme Programming})\\
		 
		 BDUF		 & Big Design Up-Front\\
		 
		 TFD         & Design Testando Primeiro (\emph{Test-First Design})\\
		 
		 SRP		 & Princípio da Responsabilidade Única (\emph{Single Responsibility
		 Principle})\\
		 
		 DIP		 & Princípio da Inversão de Dependências (\emph{Single Responsibility
		 Principle})\\ 
		 
		 OCP		 & Princípio do Aberto-Fechado (\emph{Open-CLosed Principle})\\
		 
		 LSP	 	 & Princípio da Substituição de Liskov (\emph{Liskov Substitution
		 Principle})\\
		 
		 ISP		 & Princípio da Segregação de Interfaces (\emph{Interface Segregation
		 Principle})\\
		 
		 OO		 	 & Orientação a Objetos
		 
\end{tabular}

% ---------------------------------------------------------------------------- %
% Listas de figuras e tabelas criadas automaticamente
\listoffigures            
\listoftables            

% ---------------------------------------------------------------------------- %
% Capítulos do trabalho
\mainmatter

% cabeçalho para as páginas de todos os capítulos
\fancyhead[RE,LO]{\thesection}

%\singlespacing              % espaçamento simples
\onehalfspacing            % espaçamento um e meio

\input cap-introducao
\input cap-tdd
\input cap-trabrelacionados
\input cap-qualitativo
\input cap-planejamento
\input cap-findings
\input cap-ameacas
\input cap-cronograma
\input cap-conclusoes

% cabeçalho para os apêndices
\renewcommand{\chaptermark}[1]{\markboth{\MakeUppercase{\appendixname\ \thechapter}} {\MakeUppercase{#1}} }
\fancyhead[RE,LO]{}
\appendix

\chapter{Exercícios}
\label{ape:exercicios}

Os exercícios são os mesmos para todos os grupos. O participante, em caso de dúvidas, poderá perguntar ao pesquisador.

\section{Lembrete ao participante}

Caro participante,

Lembre-se que os problemas aqui propostos simulam complicações do mundo real. 
Ao resolvê-los, tenha em mente que esses códigos serão futuramente mantidos
por você ou até por uma equipe maior.

Tente criar o design mais elegante possível em todas as soluções. Por serem problemas
recorrentes, imagine que amanhã esse mesmo problema se repetirá.
Escreva um código flexível o suficiente para que novas mudanças sejam fáceis de serem 
implementadas.

\section{Exercício 1 - Calculadora de Salário}

O participante deve implementar uma calculadora de salário de funcionários. Um
funcionário contém nome, e-mail, salário-base e cargo. De acordo com seu cargo,
a regra para cálculo do salário líquido é diferente:

\begin{enumerate}
	\item Caso o cargo seja DESENVOLVEDOR, o funcionário terá desconto de 20\%
	caso o salário seja maior ou igual que 3.000,00, ou apenas 10\% caso o salário seja menor 
	que isso.
	
	\item Caso o cargo seja DBA, o funcionário terá desconto de 25\%
	caso o salário seja maior ou igual que 2.000,00, ou apenas 15\% caso o salário seja menor 
	que isso.

	\item Caso o cargo seja TESTADOR, o funcionário terá desconto de 25\%
	caso o salário seja maior ou igual que 2.000,00, ou apenas 15\% caso o salário seja menor 
	que isso.
	
	\item Caso o cargo seja GERENTE, o funcionário terá desconto de 30\%
	caso o salário seja maior ou igual que 5.000,00, ou apenas 20\% caso o salário seja menor 
	que isso.
\end{enumerate}

Exemplos de cálculo do imposto:

\begin{itemize}
	\item DESENVOLVEDOR com salário-base 5,000.00. Salário final = 4.000,00
	\item GERENTE com salário-base de 2.500,00. Salário final: 2.000,00
	\item TESTADOR com salário de 550.00. Salário final: 467,50
\end{itemize}


O participante deve criar todo o código responsável para esse cálculo. Uma classe com
o método "main()" deverá ser entregue ao final, com exemplo de uso das classes criadas.

\section{Exercício 2 - Gerador de Nota Fiscal}

O participante deve implementar um sistema de geração de nota fiscal a partir de uma fatura. 
Uma fatura contém o nome e endereço do cliente, tipo do serviço e valor da fatura. O gerador de
nota fiscal deverá gerar uma nota fiscal que contém nome do cliente, valor da nota e valor
do imposto a ser pago.

O valor da nota é o mesmo do valor da fatura. Já o cálculo do imposto a ser pago deve seguir
as seguintes regras:

\begin{enumerate}
	\item Caso o serviço seja do tipo "CONSULTORIA", o valor do imposto é de 25%;
	\item Caso o serviço seja do tipo "TREINAMENTO", o valor do imposto é 15%;
	\item Qualquer outro, o valor do imposto é 6%.
\end{enumerate}

Ao final da geração da nota fiscal, o sistema ainda deve enviar essa nota por e-mail,
para o SAP, e persistir na base de dados. Por simplicidade, o desenvolvedor pode usar
os códigos abaixo, que simulam o comportamento do SMTP, SAP e banco de dados:

class NotaFiscalDao {
	public void salva(NotaFiscal nf) { System.out.println("salvando no banco"); }
}
class SAP {
	public void envia(NotaFiscal nf) { System.out.println("enviando pro sap"); }
}
class Smtp {
	public void envia(NotaFiscal nf) { System.out.println("enviando por email"); }
}

O participante é livre para alterar os métodos, parâmetros recebidos ou qualquer outra coisa das classes acima.

Ao final, o participante deve entregar todo o código responsável por geração e encaminhamento da nota fiscal 
para os processos acima citados. Uma classe com o método "main()" deverá ser entregue ao final, com
exemplo de uso das classes criadas.

\section{Exercício 3 - Processador de Boletos}

O participante deve implementar um processador de boletos. Esse processador receberá uma lista de boletos 
(que contém basicamente código do boleto, data e valor pago) e a fatura respectiva (que contém data, valor total e nome do cliente). 
O processador deve então, para cada boleto, criar um pagamento associado nessa fatura,
guardando o valor pago, a data e o tipo do pagamento (nesse caso, "BOLETO").
Além disso, caso a soma de todos os boletos ultrapasse o valor da fatura, a mesma deve ser marcada
como "PAGA".

O participante deve criar todo o código responsável pelo processador de boletos. Uma classe com
o método "main()" deverá ser entregue ao final, com exemplo de uso das classes criadas.

Exemplos de processamento:

\begin{itemize}
	\item Fatura de 1.500,00 com 3 boletos no valor de 500,00, 400,00 e 600,00: fatura marcada como PAGA, e três pagamentos do tipo BOLETO criados 
	\item Fatura de 1.500,00 com 2 boletos no valor de 500,00 e 400,00: fatura não marcada como PAGA, e dois pagamentos do tipo BOLETO criados 
\end{itemize}

\section{Exercício 4 - Saída do Quebra-Cabeça Numérico}

O participante deve de alguma forma imprimir a saída do quebra-cabeça numérico. Esse quebra-cabeça gera
uma sequência de números, que devem ser impressos no seguinte formato: "[1 -> 2 -> 3 -> 4 ->5 ->6]" (incluindo os colchetes).

O código do quebra-cabeça maluco encontra-se abaixo:

\begin{lstlisting}
public class QuebraCabecaNumerico {

	private int entrada;
	private int saida;
	private List<Numero> fila;
	private Set<Integer> visitados;
	private Numero solucao;
	
	public QuebraCabecaNumerico() {
		this.fila = new ArrayList<Numero>();
		this.visitados = new HashSet<Integer>();
	}

	public void geraCaminho(int entrada, int saida) {
		this.entrada = entrada;
		this.saida = saida;
		
		this.solucao = buscaSolucao();
	}
	
	private Numero buscaSolucao() {
		 
		adicionaRaizNaFila();
		
		while(existemNumerosNaFila()) {
			Numero numeroAtual = removeDaFila();
			
			if(encontrouSaida(numeroAtual)) return numeroAtual;
			adicionaNaFila(
				multiplicaPorDois(numeroAtual),
				(ehPar(numeroAtual)?dividePorDois(numeroAtual):null),
				somaDois(numeroAtual)
			);
		}
		
		return null;
	}

	private boolean ehPar(Numero numeroAtual) {
		return numeroAtual.getValor()%2==0;
	}

	private boolean encontrouSaida(Numero numeroAtual) {
		return numeroAtual.getValor() == saida;
	}

	private boolean existemNumerosNaFila() {
		return fila.size()!=0;
	}

	private void adicionaRaizNaFila() {
		fila.add(new Numero(entrada, null));
	}
	
	private void adicionaNaFila(Numero... numeros) {
		for(Numero numero : numeros) {
			if(numero!=null) {
				if(!visitados.contains(numero.getValor())) {
					fila.add(numero);
					visitados.add(numero.getValor());
				}
			}
		}
	}
	
	private Numero multiplicaPorDois(Numero numero) {
		return new Numero(numero.getValor()*2, numero);
	}

	private Numero dividePorDois(Numero numero) {
		return new Numero(numero.getValor()/2, numero);
	}
	
	private Numero somaDois(Numero numero) {
		return new Numero(numero.getValor()+2, numero);
	}

	private Numero removeDaFila() {
		Numero topoDaFila = fila.get(0);
		fila.remove(0);
		return topoDaFila;
	}

}

class Numero {
	private final int valor;
	private final Numero pai;
	
	public Numero(int valor, Numero pai) {
		this.valor = valor;
		this.pai = pai;
	}
	public int getValor() {
		return valor;
	}

	public Numero getPai() {
		return pai;
	}
}
\end{lstlisting}

Repare que o único método público existente "buscaSolucao()", invoca o algoritmo e guarda a solução
dentro do atributo "solucao". Um exemplo de código que visita a árvore de números gerada pelo algoritmo é:

\begin{lstlisting}
	while(solucao!=null) {
		int valor = solucao.getValor(); // esse eh o valor a ser impresso
		solucao = solucao.getPai();
	}
\end{lstlisting}

Exemplos de saídas do algoritmo:

\begin{itemize}
	\item Entrada: 2, 2 Saída: [2]
	\item Entrada: 2, 4 Saída: [2 -> 4]
	\item Entrada: 2,10 Saída: [2 -> 4 -> 8 -> 10]
	\item Entrada: 3, 10 Saída: [3 -> 5 -> 10]
\end{itemize}

O participante deve criar todo o código responsável pela saída do quebra-cabeça numérico. Uma classe com
o método "main()" deverá ser entregue ao final, com exemplo de uso das classes criadas.


\chapter{Entrevista}
\label{ape:entrevista}

\section{Dados básicos}

\begin{enumerate}
	\item Nome completo?
	\item Empresa em que atua?
	\item E-mail para contato?
	\item Autoriza o uso da entrevista para a pesquisa?
	\item Tempo total de entrevista?
\end{enumerate}

\section{Caracterização do desenvolvedor}

\begin{enumerate}
	\item Qual a sua formação?
	\item Há quanto tempo atua na área de desenvolvimento de software?
	\item Você desenvolve sistemas orientados a objetos há quanto tempo?	
	\item Fale-me um pouco sobre os projetos que já desenvolveu e os desafios neles encontrados?
	\item Em sua opinião, qual a maior dificuldade na criação de sistemas OO de qualidade?
	\item Com que frequência você costuma ler livros ou ir a congressos para se atualizar sobre as novas práticas? Quais?
\end{enumerate}

\section{A Prática de TDD}

\begin{enumerate}
	\item Você pratica TDD há quanto tempo?
	\item Como tem sido sua relação com a prática?
	\item O que é TDD em sua opinião?
	\item Poderia me explicar como você pratica TDD no dia-a-dia?
	\item Você pratica TDD o tempo todo?
	\item Em sua opinião, quais as vantagens de praticar TDD?
\end{enumerate}

\section{Relação entre TDD e Design}

\begin{enumerate}
	\item{Em sua experiência, você acredita que classes muito acopladas são realmente prejudiciais?}
		\begin{enumerate}
			\item Como você combate esse problema?
			\item TDD te ajuda de alguma forma? Como?
			\item Você acha que mesmo com TDD é possível criar classes com alto acoplamento? 
		\end{enumerate}
	\item{E classes com baixa coesão? Você costuma ver muito?}
		\begin{enumerate}
			\item Em sua opinião, por que elas aparecem?
			\item Você acha que TDD te ajudaria a resolver esse problema? Como?
		\end{enumerate}
	\item Você acredita que TDD influencia na qualidade do design de classes?
	\begin{enumerate}
		\item Como ele influencia no seu design?
		\item Escrever o teste antes ajuda nesse processo?
		\item E por que esse efeito não aconteceria caso você escrevesse o teste depois?
		\item Como seria criar designs sem a ajuda dos testes?
		\item Poderia dar um exemplo (usando código se preferir) de como o TDD influencia o seu design? Algum outro exemplo?
		\item Além de receber as dependências pelo construtor, você vê alguma outra maneira aonde TDD te ajude?
			\begin{enumerate}
				\item Um problema famoso ao começar a praticar TDD é o construtor começar a receber muitos parâmetros. Quando você começa a perceber isso?
				\item E como resolve esse problema?
			\end{enumerate}
	\end{enumerate}
	\item Sabemos que classes com muitos métodos ou métodos com muitas linhas é prejudicial para o código. Como você faz para não escrever classes com esses problemas?
	\item Você conhece os princípios SOLID do Robert Martin? Se sim,
	\begin{enumerate}
		\item você sempre os utilizou?
		\item TDD teve alguma influencia nisso?
	\end{enumerate}
\end{enumerate}

\section{Relação entre TDD e experiência}

\begin{enumerate}
	\item Como você compararia as primeiras vezes que você praticou TDD com agora?
	\item Você acha que a experiência influencia no resultado final?
	\item Em sua opinião, como você acha que seria o design de uma pessoa sem experiência em desenvolvimento de software e sem experiência em OO, mas praticando TDD?
	\item O programador às vezes comete desvios na prática, como não ver o teste falhar, esquecer de refatorar, não rodar a bateria de testes completa, refatorar outro trecho de código enquanto está no vermelho, e etc. Você vê isso como um problema?
	\item TDD fala sobre \textit{baby steps} (ou passos de bebê). O que você entende por dar passos de bebê?
	\begin{enumerate}
		\item É muito comum, principalmente em atividades como dojos, as pessoas ficarem criando if's e mais if's (que é o código mais simples possível que faz o teste passar), até o momento em que o código se torna o complexo suficiente para merecer uma refatoração. O que pensa sobre isso?
	\end{enumerate}
\end{enumerate}

\section{TDD e outras práticas ágeis}

\begin{enumerate}
	\item Você acha que outras práticas ágeis influenciam na qualidade do seu design?
	\item Vocês praticam programação pareada? Com que frequência? Quando vocês fazem programação pareada, você acha que isso ajuda mais no design do que TDD?
	\item Se só praticassem programação pareada, por exemplo, você acha que mesmo assim precisaria de TDD?
	\item Quais outras práticas ágeis você acha que ajuda você a melhorar o design?
\end{enumerate}

\section{TDD e experiência pessoal}

(Nesse ponto o entrevistado poderá mostrar algum trecho de código do projeto em que atua.)

\begin{enumerate}
	\item Tem algum exemplo em mente aonde TDD ajudou a resolver o problema de maneira elegante?
	\item E aonde TDD atrapalhou?
		\begin{enumerate}
			\item Você acha que TDD se aplica à todos os casos? Se não, em quais ele não se aplica e porquê?
		\end{enumerate}
	\item Poderia me mostrar algum trecho de código que você acha que possui um bom design e me explicar o porquê da sua opinião?
		\begin{enumerate}
			\item Você fez esse trecho de código usando TDD?
			\item Você estava sozinho ou pareando com outro desenvolvedor?
			\item Esse design surgiu naturalmente ou tiveram problemas?
		\end{enumerate}
\end{enumerate}

\section{TDD e produtividade}
\begin{enumerate}
	\item Na sua opinião, qual a relação entre TDD e produtividade?
	\item Algumas pessoas dizem que escrever o teste gasta muito tempo. Outros dizem que não. O que você acha?
	\item Você vê alguma relação entre TDD e o sucesso de um produto?
\end{enumerate}

\section{Opiniões finais}

\begin{enumerate}
	\item Você acha que TDD resolve de vez o problema de designs que degradam ao longo do tempo?
	\item O que mais você faria para tentar resolver esse problema?
	\item Existem algumas críticas em relação à TDD, como "TDD é improdutivo, afinal você gasta muito tempo escrevendo os testes". O que você acha disso?
		\begin{enumerate}
			\item A longo prazo, você vê vantagens em gastar esse tempo utilizando TDD?
		\end{enumerate}
	\item Gostaria de dizer mais algo sobre TDD que não disse nas perguntas anteriores?
	\item Com quem posso falar para saber mais sobre o assunto?
\end{enumerate}
\chapter{Informações ao partipante}
\label{ape:informacoes-participante}

\section{Convite}

Meu nome é Mauricio Aniche. Sou aluno de mestrado em Ciência da Computação pelo
Instituto de Matemática e Estatística da Universidade de São Paulo (USP).
Atualmente pesquiso a influência de Test-Driven Development no design de
sistemas orientados a objetos.

Para alcançar esse objetivo, estou realizando entrevistas com desenvolvedores de
diversas empresas do mercado brasileiro que já praticam TDD há pelo menos 1 ano.
Este convite permite a você compartilhar suas experiências e sentimentos em
relação à prática e cooperar com as pesquisas na área.

É importante reforçar que a participação é totalmente voluntária e não há nenhum
tipo de remuneração associada. Você pode desistir da sua participação sem nenhum
tipo de consequência.

\section{Qual o objetivo desta pesquisa?}

O objetivo desta pesquisa é entender de maneira mais profunda como TDD
influencia no design de sistemas orientados a objetos. Essas informações serão
capturadas baseadas na percepção de programadores que realizam a prática no seu
dia-a-dia de trabalho. Além disso, essa pesquisa se propõe a comparar os
resultados encontrados com os dados existentes na literatura.

\section{Qual meu papel dentro dela?}

Como participante dessa pesquisa, você deverá participar das entrevistas. Elas
ocorrerão dento do seu ambiente de trabalho e serão gravadas. Além disso, o
pesquisador poderá fazer uso de anotações durante esse período.

\section{Quais são os benefícios?}

Além de cooperar com o avanço da pesquisa na área de engenharia de software, os
resultados obtidos por essa pesquisa são compartilhadas com você, e eu espero
que as informações ali contidas possam ser úteis para a evolução da técnica.

\section{Minha privacidade será garantida?}

Sim, todas as informações gravadas serão mantidas em completo sigilo. Apenas os
pesquisadores participantes desse trabalho terão acesso ao mesmo.

Além disso, nenhum nome será revelado no resultado final da pesquisa.

\section{Quais são os custos de participação na pesquisa?}

Nenhum. O pesquisador precisará de 1 hora para a realização da entrevista. Caso
uma nova entrevista seja necessária, ele marcará a mesma com antecedência. 

\section{Em caso de dúvidas, o que devo fazer?}

Em caso de dúvida, favor contatar o pesquisador ou o orientador dessa pesquisa.

Mauricio Finavaro Aniche (aniche@ime.usp.br)

Marco Aurélio Gerosa (gerosa@ime.usp.br) 

Departamento de Ciência da Computação - Instituto de Matemática e Estatística - 
Universidade de São Paulo (USP) - Caixa Postal 66.281 - 05.508-090 - São Paulo -
SP  - Brasil


\chapter{Autorização}
\label{ape:autorizacao}

\section{Consentimento de Participação na Pesquisa}

Caro participante, por favor preencha atentamente as instruções abaixo:

\begin{itemize}
  \item (  ) Eu recebi, li e entendi as informações sobre essa pesquisa;
  \item (  ) Eu tive a oportunidade de tirar dúvidas sobre a pesquisa;
  \item (  ) Eu entendo que anotações serão feitas durantes as entrevistas e que
  elas serão gravadas e transcritas;
  \item (  ) Eu entendo que eu posso desistir da minha participação ou de
  qualquer informação que eu provi a qualquer momento antes da finalização do processo de
  coleta de dados, sem qualquer tipo de dano ou perda;
  \item (  ) Eu entendo que, em caso de desistência, a gravação, transcrição ou
  qualquer outra informação persistida será destruída;
  \item (  ) Eu aceito fazer parte desta pesquisa;
  \item (  ) Eu gostaria de receber uma cópia do resultado final da pesquisa
\end{itemize}

Assine este documento, informando seu nome e data corrente.

% ---------------------------------------------------------------------------- %
% Bibliografia
\backmatter \singlespacing   % espaçamento simples
\bibliographystyle{alpha-ime}% citação bibliográfica alpha
\bibliography{bibliografia}  % associado ao arquivo: 'bibliografia.bib'


\end{document}
