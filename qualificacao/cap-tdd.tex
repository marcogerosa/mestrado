%% ------------------------------------------------------------------------- %%
\chapter{Test-Driven Development}
\label{cap:tdd}

Test-Driven Development (TDD) é a arte de produzir testes automatizados para código de produção, usando
esse processo para guiar o design e a programação. Para cada pequeno pedaço de funcionalidade, o desenvolvedor
deve primeiro escrever um teste que especifique e valide o que o código irá fazer. O programador então produz
somente o código necessário para fazer esse teste passar. Então ele refatora (simplifica e clareia) tanto o
código de produção quanto o código de testes \cite{agilealliance-tdd} \cite{tdd-taxonomy}.

TDD é uma importante prática na Programação Extrema (XP) \cite{XPExplained} já que, como sugerido pelas práticas
ágeis, o design de um software deve emergir conforme o software cresce. E, para responder rapidamente a essas
alterações, é necessário um constante \textit{feedback} sobre a qualidade interna e externa do código, e TDD
provê isso já que TDD faz com que o teste de unidade seja o primeiro cliente real da classe que o programador ainda
vai implementar, e isso faz com que ele pense melhor a respeito do comportamento que ele espera dessa classe. Além disso,
a suíte de testes gerada dá segurança ao programador no momento de uma refatoração pois, caso ele cometa um erro,
algum dos testes provavelmente irá falhar, avisando-o imediatamente sobre o problema.

TDD faz com que esse simples processo de escrever um teste antes de escrever o código
sirva para ajudar o time a entender as funcionalidades que os usuários do sistema precisam e 
a entregar essas funcionalidades de maneira confiável e previsível \cite{GOOS}. TDD usa os testes para clarificar as expectativas
sobre o que um pedaço de código deve fazer [TODO 3]. Além disso, ele cria uma medida concreta de falha, que são os testes de 
unidade que estão falhando. Com essa medida em mãos, o objetivo do programador naquele momento é simplesmente fazer o teste passar. É um
objetivo claro e bem definido, e permite ao programador focar naquele pequeno pedaço de funcionalidade e no trecho de código que ele 
deve escrever para fazer o teste passar \cite{TDDByExample}. 

Na prática, TDD obriga que o programador escreva
um código novo apenas se houver um teste automatizado falhando, e faz com que ele remova duplicação de dados e de código constantemente. 
Essas duas regras implicam no seguinte ciclo: o programador primeiro deve escrever um pequeno teste que falhe (ou que até mesmo não compile), 
em seguida deve fazer o teste passar de maneira mais rápida possível e, por fim, deve refatorar toda a duplicação criada \cite{TDDByExample}. 
Esse ciclo é também conhecido como Vermelho-Verde-Refatora (ou Red-Green-Refactor), já que lembram as cores que o programador geralmente 
vê quando faz TDD: o vermelho geralmente significa que o teste está falhando e o verde quando o teste foi executado com sucesso. 

% TODO figura red-green-refactor

TDD apareceu para o mundo de desenvolvimento de softwares em 2001, quando Kent Beck publicou seu trabalho sobre Programação Extrema \cite{XPExplained},
onde TDD era uma das práticas citadas por ele como essenciais para o processo de desenvolvimento. Conhecida também
por \textit{Test-First Programming} (Programação Testando Primeiro), \textit{Test-Driven Design} (Design Guiado pelos Testes) ou 
\textit{Test-First Design} (Design Testando Primeiro), sua popularidade cresceu junto com o aumento de popularidade da própria Programação Extrema. 
Mas, nos últimos tempos, TDD tem recebido uma grande atenção entre os pesquisadores e profissionais da indústria, mesmo em equipes não-ágeis, já que
é dito que a sua prática aumenta tanto a qualidade interna do código, aumentando sua coesão, diminuindo seu acoplamento e tornando
o código mais simples, quanto a qualidade externa, diminuindo o número de defeitos.


\section{Definições incompletas} 
\label{sec:tdd-definicoes-incompletas}

Apesar de sua popularidade, existem muitas definições incompletas sobre TDD. A maioria delas de maneira geral levam em conta
apenas a parte do teste, e dizem apenas
que "TDD é uma prática de desenvolvimento de software na qual o programador deve escrever um teste antes de escrever o código".
Definições como essa não definem todos os pontos-chave da prática, como a utilização desse processo para guiar o design e
a obrigação de sempre escrever o código mais simples possível.

Janzen \cite{tdd-taxonomy} exemplica esse problema mostrando a definição no livro \textit{JUnit in Action} \cite{junit-in-action}:
\textit{Test-Driven Development é uma prática de programação que instrui desenvolvedores a escrever código novo apenas se um teste
automatizado estiver falhando, e a eliminar duplicação. O objetivo de TDD é "código claro que funcione"}. 
Entretanto, muitos autores lembram que TDD é na verdade uma prática de design \cite{tdd-taxonomy} \cite{aim-fire}, 
e citam sempre a famosa frase do Ward Cunningham, um dos pioneiros da Programação Extrema:
\textit{Programação Teste Primeiro (Test-First Programming) não é uma prática de testes}. 

Entender que TDD é na verdade sobre análise e design seja talvez a parte mais complicada e desafiadora para novos adotantes da prática. 

\section{Diferentes perspectivas}

TODO discussao do janzen sobre abordagem test, driven e development 

\section{O Ciclo} 
\label{sec:tdd-ciclo}

De forma mais ampla, podemos definir o ciclo de TDD da seguinte maneira:

\begin{enumerate}
	\item adicione um teste; 
	\item rode todos os testes e veja o novo teste falhar; 
	\item escreva o código mais simples que faça o teste passar; 
	\item rode todos os testes e veja o novo teste passar; 
	\item refatore para remover duplicação de dados e de código.
\end{enumerate}

As sub-seções abaixo visam explicar cada passo do ciclo de forma mais detalhada, explicando a motivação e o esperado 
em cada um deles.

\subsection{Adicionar um teste}

O primeiro passo do ciclo é escrever um teste para uma nova funcionalidade. Para isso, o programador deve primeiro entender a funcionalidade 
que ele deseja implementar, e dessa forma entender melhor o requisito antes de escrever qualquer linha de código.
Esse teste falhará, afinal a funcionalidade ainda não foi implementada. 

Esse é um passo muito importante do processo, já que o teste que o programador vai escrever é o primeiro cliente da classe que está
prestes a implementar. Nesse momento o programador decide não somente sobre a interface que essa classe terá (como nomes dos métodos,
parâmetros, tipos de retorno e exceções a serem lançadas) como também sobre o comportamento que ele espera dessa classe (como 
os resultados de acordo com as diferentes entradas) \cite{GOOS}.

\subsection{Rodar todos os testes e ver o novo teste falhar}

Nesse momento o programador roda todos os seus testes de unidade e verifica se o teste escrito no passo anterior realmente falhou. Em um
primeiro momento esse passo pode parecer desnecessário, pois o programador escreveu um teste para uma funcionalidade que ele ainda nem
implementou, e portanto o teste irá falhar. 

Mas, na prática, esse teste pode não falhar, indicando alguns possíveis problemas, como: talvez o código da classe que está sendo testada
não está reagindo da maneira que o programador pensou que estava (afinal ele esperava por um determinado comportamento quando na verdade a
classe apresentou outro); problemas na implementação do teste, afinal teste também é código, e assim como o código de produção, ele é
totalmente suscetível a falhas.
Portanto, pode-se concluir que ver o novo teste falhar ajuda o programar a validar tanto o código de teste quanto o seu conhecimento
sobre a classe que está sendo testada. 

\subsection{Escreva o código mais simples que faça o teste passar}

O programador deve então escrever o código mais simples possível que faça o teste passar. Nesse momento o programador não deve estar
focado em escrever a solução mais clara (pois isso será feito em um passo posterior), e sim realmente a mais simples.

O objetivo desse passo é evitar que o programador, ao invés de resolver o problema de maneira simples, por algum motivo
implemente uma solução mais complexa, e com isso acrescentando complexidade desnecessária no código. Algumas técnicas como
\textit{Implementação Óbvia} e \textit{Triangularização} \cite{TDDByExample}, discutidas com mais detalhes na seção \ref{sec:tdd-patterns}, 
podem ajudar o programador na tarefa de escrever a implementação mais simples possível.

Muitas vezes esse passo é mal entendido pelos programadores, que acabam por confundir o "mais simples" pelo o "mais simplório". 
A ideia de passos de bebê \cite{TDDByExample} é melhor detalhada na seção \ref{sec:tdd-baby-steps}.

\subsection{Rode todos os testes e veja o novo teste passar}

Assim que o programador escrever o código mais simples que faça o teste passar, ele deve rodar a suíte de testes novamente
para verificar que a nova implementação realmente faz o teste passar. Nesse ponto é importante que o programador rode a suíte 
completa de testes, pois a implementação pode ter feito o novo teste passar, mas pode ter quebrado testes antigos que já existiam na suíte. 
Caso algum teste antigo quebre, o programador deve então voltar um passo atrás, e tentar novamente escrever o código mais simples
que faça o teste que está quebrado passar.

Esse passo traz segurança ao programador durante todo o ciclo de desenvolvimento, pois rodando a suíte de testes inteira, 
ele rapidamente descobre se introduziu um erro em alguma outra parte do sistema. Essa é uma técnica conhecida na área de testes e é
chamada de \textit{testes de regressão} \cite{art-of-sw-testing}. Em TDD, e execução constante desses testes de regressão 
trazem outra vantagem que é a detecção do erro acontece basicamente no momento em que esse erro é introduzido, facilitando assim
o processo de correção do mesmo. Alguns estudos inclusive mostram que o programador ao fazer TDD passa menos tempo depurando o código
do que programadores que utilizam a abordagem tradicional (TODO janzen 5, george williams 9 artigo aniche).

\subsection{Refatore para remover duplicação de dados e de código}

No último passo do ciclo o programador agora deve refatorar o código escrito, procurando remover as duplicações de código ou de dados. 
Conforme comentado acima, o código escrito anteriormente era o mais simples, mas não necessariamente o mais claro possível. O programador
nesse momento tem total liberdade para refatorar, já que tem uma suíte de testes para garantir que o comportamento não será alterado.
Esse é um dos principais passos no ciclo, já que é aqui que o programador se preocupa em melhorar o seu código, tornando-o
mais claro e, por consequência, facilitando sua evolução.

Como TDD faz com que a prática de testes seja utilizada para receber informações sobre o design, 
a prática de refatoração é parte vital do processo já que o programador muda de ideia muitas vezes sobre a interface que determinada
classe deve ter, e para isso, faz uso de práticas de refatoração, já muito discutida na literatura \cite{fowler-refactoring} \cite{joshua-refactoring}.

\section{Passos de bebê}
\label{sec:tdd-baby-steps}

TDD sugere que o programador dê sempre pequenos passos (conhecidos pelo termo em inglês, \textit{baby steps}): deve-se escrever testes 
sempre para a menor funcionalidade possível, escrever o código mais simples que faça o teste passar e fazer sempre 
apenas uma refatoração por vez. 

A justificativa para tal é que quanto maior o passo que o programador dá, mais tempo ele leva para concluir esse passo e, 
consequentemente ele fica mais tempo sem feedback sobre o código. Além disso, faz com que o programador não crie 
soluções mais complexas do que elas precisam ser, tornando o código à longo prazo o mais simples possível.

Em seu livro sobre TDD, Kent Beck sugere a seguinte abordagem inicial para uma classe \textit{Dinheiro}, que permite uma operação de 
multiplicação \cite{TDDByExample}:

\begin{lstlisting}[frame=trbl]
    public void testaMultiplicacao() {
		Dinheiro cinco = new Dinheiro(5);
		cinco.vezes(2);
		assertEquals(10, dinheiro.valor);
	}
\end{lstlisting}

Nesse momento ocorrem alguns erros de compilação já que a classe Dinheiro ainda não foi escrita. Em seguida, Beck escreve
a seguinte classe:

\begin{lstlisting}[frame=trbl]
	class Dinheiro {
		public int valor;
		
		Dinheiro(int valor) {
		}
		
		void vezes(int multiplicador) {
		}
	}
\end{lstlisting}

Nesse momento não existem mais erros de compilação, e é possível executar e ver o teste falhar. O teste falha pois ele esperava
que o atributo \textit{valor} retornasse 10, mas retornou 0. Nesse momento, Beck faz o teste passar da forma mais simples possível,
que é simplesmente inicializar o atributo \textit{valor} da classe \textit{Dinheiro} com o valor 10.

\begin{lstlisting}[frame=trbl]
	public int valor = 10;
\end{lstlisting}

Agora com o teste passando, Beck começa então o passo de refatoração, visando remover duplicação de dados e de código. No exemplo,
existe a clara duplicação de dados entre a classe e o código de teste: o valor 10, que surgiu da multiplicação do 5 por 2. 

\begin{lstlisting}[frame=trbl]
	public int valor = 5 * 2;
\end{lstlisting}

Nesse momento, Beck percebe que não há como remover a duplicação em apenas um passo. Ele então, move a multiplicação para dentro
do método \textit{vezes}:

\begin{lstlisting}[frame=trbl]
	public int valor;
	
	void vezes(int multiplicador) {
		valor = 5 * 2;
	}
\end{lstlisting}

Finalmente ele extrai o valor 5 do parâmetro que é passado para o construtor:

\begin{lstlisting}[frame=trbl]
class Dinheiro {
	public int valor;
	
	Dinheiro(int valor) {
		this.valor = valor;
	}
	
	void vezes(int multiplicador) {
		valor = valor * 2;
	}
}
\end{lstlisting}

Por fim, ele remove o 2 e multiplica agora pelo \textit{multiplicador}:

\begin{lstlisting}[frame=trbl]
class Dinheiro {
	public int valor;
	
	Dinheiro(int valor) {
		this.valor = valor;
	}
	
	void vezes(int multiplicador) {
		valor = valor * multiplicador;
	}
}
\end{lstlisting}

O único teste ainda continua passando, provando que a refatoração foi feita com sucesso. Pode-se observar através
do exemplo acima que os passos dados foram sempre os menores e mais simples possíveis, ou \textit{passos de bebê}, por assim dizer. 
Veja que, mesmo após diversos passos de refatoração, ainda é possível encontrar problemas de design, como o atributo \textit{valor}
declarado como público. Seriam necessários outras rodadas de refatoração para eliminar esse problema.

Uma confusão comum de programadores que experimentam TDD pela primeira vez é a de que fazer passos de bebê o tempo todo faz
com que a produtividade diminua. De fato, fazer passos de bebê durante todo o ciclo faz com que o tempo de desenvolvimento aumente,
já que muitos desses passos são simples demais e poderiam ser pulados por um programador mais experiente. Mas TDD não força o programador
a dar passos de bebê o tempo todo. TDD permite que o mesmo dê passos de bebê quando achar necessário \cite{TDDByExample}: caso
o programador esteja confiante sobre o trecho de código que está escrevendo naquele momento, ele pode dar um passo maior; mas caso ele
não esteja tão confiante, TDD permite a ele ir mais devagar e dar passos de bebê, obtendo feedback mais rápido sobre o código que está escrevendo. 

No exemplo acima, o método \textit{vezes} poderia ser considerado simples por um programador experiente, que optaria por chegar à
mesma solução acima de maneira mais rápida, tomando passos maiores.

\section{Padrões para a Prática de TDD}
\label{sec:tdd-patterns}

tdd patterns e red/green bar patterns (beck) sec:tdd-patterns

\section{Feedback e Big-Design Up Front}

feedback (exemplo do MortalityTable do Aim, fire), nada de bduf

---



lista de proximas coisas a fazer

definir codigo de producao?

exemplo parecido com o do money (simples e dpeois completo)

história do tdd

tdd = tfd + refatoracao

refatoracao tanto de baixo nivel quanto de alto nivel

tdd é documentação?

visao do martin sobre tdd

ron jefrries clean code that works

