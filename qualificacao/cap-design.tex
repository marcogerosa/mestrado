%% ------------------------------------------------------------------------- %%
\chapter{TDD e sua influ�ncia no design}
\label{cap:tdd-design}

Texto texto texto texto texto texto texto texto texto
foco em feedback do design
2 vantagens gratuitas; gerenciamento de dependencias e codigo facilmente invocavel
experiencia do programador para refatoracao de nivel maior
falar sobre feedback do design
sinergia entre alta testabilidade e codigo bom

% levar esses paragrafos para parte sobre design
TDD permite tamb�m que o desenvolvedor obtenha feedback sobre o design do sistema. Como sugerido pelas pr�ticas �geis,
o design de um software deve emergir junto com sua evolu��o. E, para responder rapidamente � essas mudan�as, � necess�rio
um feedback constante sobre a qualidade do design. Por esse motivo, TDD � considerado uma importante pr�tica na Programa��o 
Extrema (XP) \cite{XPExplained} e uma estrat�gia essencial para designs emergentes pois, 
ao escrever um teste antes do c�digo, os programadores contemplam e decidem n�o somente sobre a interface (como nomes de classes e 
m�todos, tipos de retorno e exce��es lan�adas), mas tamb�m no comportamento que se espera de uma determinada classe 
(como o resultado esperado de acordo com determinadas entradas) [TODO 13]. 

Conforme o sistema � desenvolvido, TDD possibilita um feedback sobre a qualidade tanto da implementa��o quanto do design. Segundo 
Freeman \cite{GOOS}, escrever testes (i) clarifica o crit�rio de aceita��o para o pr�ximo peda�o de software; (ii) encoraja o programador
a escrever componentes fracamente acoplados, de maneira que eles possam ser testados de maneira isolada, e em um n�vel maior, combinados
com outros componentes; (iii) cria uma especifica��o execut�vel sobre o que o c�digo faz; e (iv) cria uma su�te de testes de regress�o.
J� o processo de rodar os testes possibilita (i) a detec��o de erros enquanto o contexto est� recente na mente do programador, e (ii)
faz com que o programador saiba quando terminou seu trabalho, diminuindo assim o risco de implementar funcionalidades desnecess�rias.
