%% ------------------------------------------------------------------------- %%
\chapter{Ameaças a Validade}
\label{cap:ameacas}

Este trabalho possui algumas possíveis ameaças a validade. O objetivo
deste capítulo é listá-las, bem como discutir alternativas para diminuir
os possíveis viéses.

\section{Enviesamento pela Desejabilidade Social}

Enviesamento pela desejabilidade social é o termo científico usado para descrever
a tendência de que alguns participantes respondam questões de modo que serão
bem vistos pelos outros membros da comunidade \cite{crowne}.

Métodos ágeis e TDD possuem um discurso forte. A comunidade brasileira de métodos
ágeis ainda é nova e percebe-se de maneira empírica que muitos repetem o discurso
sem grande experiência ou embasamento no assunto.
No caso desta pesquisa, um possível viés é o participante responder o que
a literatura diz sobre TDD, e não exatamente o que ele pratica e sente sobre
os efeitos da prática. 

Para diminuir esse viés, o pesquisador eliminou do processo de análise os participantes
que responderam as perguntas de forma superficial, apenas repetindo a literatura.

\section{Exercícios de pequeno porte}

Os exercícios propostos são pequenos perto de um projeto real. Todos os exercícios propostos contém
problemas localizados de design. E, uma vez que esta pesquisa tenta avaliar os efeitos de TDD no design, 
acreditamos que os problemas conseguem simular de forma satisfatória
problemas de design que desenvolvedores encaram no dia a dia de trabalho.

Além disso, ao final do exercício, os participantes responderão uma pergunta sobre a semelhança
entre os problemas de design propostos dos exercícios e os problemas encontrados no mundo real.
As respostas serão levadas em consideração no processo de análise.