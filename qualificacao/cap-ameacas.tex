%% ------------------------------------------------------------------------- %%
\chapter{Ameaças a Validade}
\label{cap:ameacas}

Esse trabalho possui algumas possíveis ameaças a validade. O objetivo
deste capítulo é listá-las, bem como discutir alternativas para diminuir
os possíveis viéses.

\section{Enviesamento pela Desejabilidade Social}

Enviesamento pela desejabilidade social é o termo científico usado para descrever
a tendência de que alguns participantes respondam questões de modo que serão
bem vistos pelos outros membros da comunidade \cite{crowne}.

Métodos ágeis e TDD possuem um discurso forte. A comunidade brasileira de métodos
ágeis ainda é nova e, percebe-se de maneira empírica que muitos repetem o discurso
sem grande experiência ou embasamento no assunto.
No caso dessa pesquisa, um possível viés é o participante responder o quê
a literatura diz sobre TDD, e não exatamente o quê ele pratica e sente sobre
os efeitos da prática. 

Para diminuir esse viés, o pesquisador eliminou do processo de análise os participantes
que responderam as perguntas de forma superficial, apenas repetindo a literatura.

\section{Mudança de Atitude Durante a Observação}

Durante o processo de observação, participantes que não levam em consideração os efeitos
de TDD no processo podem passar a reparar neles apenas por estarem sendo observados.
Apesar disso, o pesquisador acredita que se o participante não tiver a experiência necessária,
tanto com a prática de TDD quanto com design de software, ele raramente conseguirá utilizar
o feedback que o ciclo de TDD provê para melhorar o design.

Caso o pesquisador perceba que o participante está atuando diferentemente apenas por
estar sendo observado, o pesquisador irá remover os detalhes dessa observação do
processo de análise final.

