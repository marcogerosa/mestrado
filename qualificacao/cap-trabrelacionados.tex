%% ------------------------------------------------------------------------- %%
\chapter{Trabalhos Relacionados}
\label{cap:trabalhos-relacionados}

% TODO: traduzir e incrementar trab relacionados
Muitos estudos empíricos já foram realizados para avaliar os efeitos de TDD.
Curiosamente, apesar de ser uma prática focada em design, muitos estudos avaliam
os efeitos de TDD na qualidade externa do sistema. Além disso, ao
contrário do que essa pesquisa propõe, muitos desses estudos optaram por um
maior controle no experimento, e os realizaram dentro de ambientes acadêmicos 
com estudantes dos mais diversos cursos de computação.

É notável o interesse em TDD por parte de pesquisadores famosos na área de
métodos ágeis, como é o caso da prof. Dra. Laurie Williams, famosa por seus
estudos com programação pareada.

As sub-seções abaixo avaliam experimentos que envolvem a prática, focados em
qualidade interna ou externa, feitos na indústria ou na academia.

\subsection{Qualidade Externa}

Janzen \cite{janzen-arch-improvement} mostrou que programadores que usam TDD na indústria
produziram código que passaram em aproximadamente 50\% mais testes caixa-preta
do que o código produzido por grupos de controle que não usavam TDD. Além do
mais, o grupo que usava TDD gastou menos tempo debugando. Janzen também mostrou 
que a complexidade dos algoritmos era muito menor e a quantidade e cobertura dos
testes era maior nos códigos escritos com TDD.

Outros estudos realizados na indústria também apresentam resultados parecidos.
Um estudo feito pelo Maximillien e Williams \cite{max-e-williams}, mostrou uma
redução de 40-50\% na quantidade de defeitos e um impacto mínimo na
produtividade  quando programadores usaram TDD. Outro estudo feito por Lui e
Chan  \cite{lui-e-chan} comparando dois grupos, um utilizando TDD e o outro 
escrevendo testes apenas após a implementação, mostrou uma redução significante 
no número defeitos. Além do mais, os defeitos que foram encontrados eram 
corrigidos mais rapidamente pelo grupo que utilizou TDD. O estudo feito por 
Damm, Lundberg e Olson \cite{damn-lundberg-e-olson} também mostra uma redução
significante nos defeitos.

O estudo feito por George e Williams \cite{george-e-williams} mostrou que,
apesar de TDD poder reduzir inicialmente a produtividade dos desenvolvedores 
mais inexperientes, o código produzido passou entre 18\% a 50\% mais em testes 
caixa-preta do que códigos produzidos por grupos que não utilizavam TDD. Esse
código também apresentou uma cobertura entre 92\% a 98\%. Uma análise
qualitativa mostrou que 87.5\% dos programadores acreditam que TDD facilitou o 
entendimento dos requisitos e 95.8\% acreditam que TDD reduziu o tempo gasto com debug. 78\%
também acreditam que TDD aumentou a produtividade da equipe. Entretanto, apenas 
50\% acreditam que TDD ajuda a diminuir o tempo de desenvolvimento. Sobre
qualidade, 92\% acreditam que TDD ajuda a manter um código de maior qualidade e 
79\% acreditam que ele promove um design mais simples.

Nagappan \cite{nagappan-ms} mostrou um estudo de caso na Microsoft e na IBM e os
resultados indicaram que o número de defeitos de quatro produtos diminuir entre 
40\% a 90\% em relação à projetos similares que não usaram TDD. Entretanto, o 
estudo mostrou também TDD aumentou o tempo inicial de desenvolvimento entre 15\%
a 35\%. Langr \cite{langr} mostrou que TDD aumenta a qualidade código, provê uma 
facilidade maior de manutenção e ajuda a produzir 33\% mais testes comparados  a
abordagens tradicionais.

na academia


Um estudo feito por Erdogmus et al \cite{erdogmus-morisio} com 24 estudos de
graduação mostrou que TDD aumenta a produtividade. Entretanto nenhuma diferença 
de qualidade no código foi encontrada.

Outro estudo feito por Janzen \cite{janzen-saiedian} com três diferentes grupos
de alunos (cada um deles usando uma abordagem diferente: TDD, test-last, sem
testes), mostrou que o código produzido pelo time que fez TDD usou melhor
conceitos  de orientação a objetos e as responsabilidades foram separadas em 13 
diferentes classes enquanto que os outros times produziram um código mais
procedural. O time de TDD também produziu mais código e entregou mais features.
Os testes produzidos por esse time teve duas vezes mais asserções que os outros 
e cobriu 86\% mais branches do que o time test-last. Além do mais, as classes 
testadas tinham valores de acoplamento 104\% menor do que as classes não
testadas e os métodos eram, na média, 43\% menos complexos do que os
não-testados.

O estudo de Müller e Hagner \cite{muller-e-hagner} mostrou que TDD não resulta
em melhor qualidade ou produtividade. Entretanto, os estudantes perceberam um 
melhor reuso dos códigos produzidos com TDD. Steinberg \cite{steinberg} mostrou
que código produzido com TDD é mais coeso e menos acoplado. Os estudantes também
reportaram que os defeitos eram mais fáceis de serem corrigidos. O estudo feito
por Edwards \cite{edwards} com  59 estudantes mostrou que código produzido com
TDD tem 45\% menos defeitos e faz com que o programador se sinta mais a vontade
com ele.

The learning difficulty presented by the participants was also evaluated by
researchers. Mugridge \cite{mugridge} identified two main challenges in
teaching TDD over the last two years: to get students to rethink about design, 
and to really engage with this new approach. Also, it is
hard to explicitly develop students' skills in testing, design and refactoring.

citar tabela do madeyski com os experimentos (e colocar o trabalho dele na
tabela)
empirical body of evidence sinialtoo \cite{tdd-body-of-evidence}

% TODO: completar com os artigos sobre design baixados

% TODO: comentar da dissertacao de mestrado da australiana, que eh bem parecido

%% ------------------------------------------------------------------------- %%
\section{Discussão}
% TODO: discutir trabalhos relacionados
critica aos trabalhos relacionados

comentar do problema com a medicao de acoplamento (em TDD ele na verdade cresce, pois acopla com mais classes pequenas e coesas).
as metricas devem ser avaliadas juntas

%% ------------------------------------------------------------------------- %%
\section{Posição desta pesquisa na literatura atual}
% TODO: posicionar este trabalha na literatura

grafico posicionando meu trabalho em relacao aos outros
