\chapter{Entrevista}
\label{ape:entrevista}

O pesquisador deve primeiramente se apresentar e dizer que ele será entrevista
sobre Test-Driven Development.

\section{Dados básicos}

\begin{enumerate}
	\item Nome completo?

	\item Qual seu cargo na empresa atual?
	
	\item Em que projetos trabalha atualmente?
	
	\item Há quanto tempo está na empresa?

	\item E-mail para contato?

\end{enumerate}

\section{Caracterização do desenvolvedor}

\begin{enumerate}
	\item Qual a sua formação?

	\item Há quanto tempo atua na área de desenvolvimento de software?

	\item Como você classificaria seu conhecimento em orientação a objetos?

	\item Você desenvolve sistemas orientados a objetos há quanto tempo?	

	\item Fale-me um pouco sobre os projetos que já desenvolveu e os desafios 
	neles encontrados?

	\item Em sua opinião, qual a maior dificuldade na criação de sistemas OO de
	qualidade?

	\item Com que frequência você costuma ler livros ou ir a congressos para se 
	atualizar sobre as novas práticas? Quais?

\end{enumerate}

\section{A Prática de TDD}

\begin{enumerate}
	\item Você pratica TDD há quanto tempo?

	\item Em sua opinião, o que é TDD?
	
	\item Quais as suas referências em TDD (livros, blogs, etc.)? 
		  Leu os livros do Kent Beck, Dave Astels ou Freeman sobre TDD?

	\item Você crê em tudo que eles falam?
	
	\item Você já leu o meu blog?
	
	\item Por que você pratica TDD?

	\item Como tem sido sua relação com a prática?

	\item Poderia me explicar como você pratica TDD no dia-a-dia?

	\item Você acha que TDD influencia alguma parte do processo de desenvolvimento
	de software, como requisitos, design, implementação, testes, ou operação?
\end{enumerate}

\section{Relação entre TDD e Design}

Nesse ponto o pesquisador deve repetir esse conjunto de perguntas
para os 4 exercícios propostos.
O pesquisador deve discutir um pouco sobre o design do código do participante e, em seguida,
fazer as perguntas abaixo.

Caso o exercício tenha sido implementado com TDD e a qualidade foi satisfatória (GRUPO A),
o conjunto de perguntas deve ser o abaixo:

\begin{enumerate}
	
	\item No exercício X, como você chegou nesse design?
	
	\item Da onde surgiu a ideia dessa modelagem de classes?
	
	\item E o TDD, aonde entra nessa história?
	
	\item Como especificamente o teste te ajudou nisso?
	
	\item Isso é recorrente? Acontece sempre?
	
	\item Como você sabe que o motivo desse bom design é o TDD, e não apenas sua experiência
	com design OO?
	
	\item Qual seria a diferença de não usar TDD nesse exercício?
\end{enumerate}

Caso o exercício tenha sido implementado com TDD e a qualidade não foi satisfatória (GRUPO B),
o conjunto de perguntas deve ser o abaixo:

\begin{enumerate}
	
	\item No exercício X, como você chegou nesse design?
	
	\item Da onde surgiu a ideia dessa modelagem de classes?
	
	\item E o TDD, aonde entra nessa história?
	
	\item Como especificamente o teste te ajudou nisso?
	
	\item Uma outra solução seria <discutir solução alternativa>, que apresentaria um bom design também. Por que você
	acha que não chegou a essa solução?
	
	\item O TDD poderia te guiar a essa solução? O que faltou para isso?
	
	\item Qual seria a diferença de não usar TDD nesse exercício?

\end{enumerate}

Caso o exercício tenha sido implementado sem TDD e a qualidade foi satisfatória (GRUPO C),
o conjunto de perguntas deve ser o abaixo:

\begin{enumerate}
	
	\item No exercício X, como você chegou nesse design?
	
	\item Da onde surgiu a ideia dessa modelagem de classes?
	
	\item Mesmo sem TDD, você chegou em um design elegante. Por que não precisou da prática?
	
	\item Se tivesse resolvido o exercício utilizando TDD, qual seria a diferença?
	
\end{enumerate}

Caso o exercício tenha sido implementado sem TDD e a qualidade não foi satisfatória (GRUPO D),
o conjunto de perguntas deve ser o abaixo:

\begin{enumerate}
	
	\item No exercício X, como você chegou nesse design?
	
	\item Da onde surgiu a ideia dessa modelagem de classes?
	
	\item Uma outra solução seria <discutir solução alternativa>, que apresentaria um bom design também. Por que você
	acha que não chegou a essa solução?
	
	\item Se tivesse resolvido o exercício utilizando TDD, qual seria a diferença?
	
	\item O TDD poderia te guiar a essa solução? O que faltou para isso?
	
\end{enumerate}

\section{Relação entre TDD e experiência}

\begin{enumerate}
	\item Como você compararia as primeiras vezes que você praticou TDD com agora?

	\item Você acha que a experiência influencia no resultado final?

	\item Em sua opinião, como você acha que seria o design de uma pessoa sem
	experiência em desenvolvimento de software e sem experiência em OO, mas praticando TDD?

	\item O programador às vezes comete desvios na prática, como não ver o teste
	falhar,  esquecer de refatorar, não rodar a bateria de testes completa,
	refatorar outro trecho de código enquanto está no vermelho, e etc. Você vê isso
	como um problema?
	
	\item TDD fala sobre \textit{baby steps} (ou passos de bebê). O que você 
	entende por dar passos de bebê?

\end{enumerate}

\section{Opiniões finais}

\begin{enumerate}
	\item Você acha que TDD resolve de vez o problema de designs que degradam ao longo do tempo?

	\item Gostaria de dizer mais algo sobre TDD que não disse nas perguntas anteriores?
	
	\item Como você compararia a qualidade de códigos escritos com TDD e sem TDD? 

	\item Com quem posso falar para saber mais sobre o assunto?
\end{enumerate}
