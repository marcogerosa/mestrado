\chapter{Entrevista}
\label{ape:entrevista}

O pesquisador deve primeiramente se apresentar e dizer que ele será entrevista
sobre Test-Driven Development.

\section{Dados básicos}
\label{entrevista:dados-basicos}

\begin{enumerate}
	\item Nome completo?

	\item Empresa em que atua?

	\item E-mail para contato?

\end{enumerate}

\section{Caracterização do desenvolvedor}
\label{entrevista:caracterizacao}

\begin{enumerate}
	\item Qual a sua formação?

	\item Há quanto tempo atua na área de desenvolvimento de software?

	\item Como você classificaria seu conhecimento em orientação a objetos?

	\item Você desenvolve sistemas orientados a objetos há quanto tempo?	

	\item Fale-me um pouco sobre os projetos que já desenvolveu e os desafios 
	neles encontrados?

	\item Em sua opinião, qual a maior dificuldade na criação de sistemas OO de
	qualidade?

	\item Com que frequência você costuma ler livros ou ir a congressos para se 
	atualizar sobre as novas práticas? Quais?
	
	\item Você já leu o meu blog?
\end{enumerate}

\section{A Prática de TDD}
\label{entrevista:pratica}

\begin{enumerate}
	\item Você pratica TDD há quanto tempo?

	\item Em sua opinião, o que é TDD?
	
	\item Quais as suas referências em TDD (livros, blogs, etc)?

	\item Por que você pratica TDD?

	\item Como tem sido sua relação com a prática?

	\item Poderia me explicar como você pratica TDD no dia-a-dia?

\end{enumerate}

\section{Relação entre TDD e Design}

\subsection{TDD e acoplamento}
\label{entrevista:acoplamento}

\begin{enumerate}
	\item{Em sua experiência, você acredita que classes muito acopladas são realmente prejudiciais?}
		\begin{enumerate}
			\item Como você combate esse problema?

			\item Como TDD te ajuda nisso?
			
			\item Você percebe alguma diferença no acoplamento que você cria quando
			pratica TDD, do acoplamento que você cria quando não pratica TDD?
			
			\item E por que você atribui essa diferença ao TDD, ou seja, o bom design
			criado não pode ter sido influenciado apenas pela sua experiência?

			\item Você acha que mesmo com TDD é possível criar classes com alto acoplamento? 
		\end{enumerate}
\end{enumerate}

\subsection{TDD e coesão}
\label{entrevista:coesao}

\begin{enumerate}
	\item{Classes com baixa coesão (que têm muitas responsabilidade)? Você
	costuma ver muitas delas?}
		\begin{enumerate}
			\item Em sua opinião, por que elas aparecem?

			\item Como você combate esse problema?

			\item Como TDD te ajuda nisso?
			
			\item Você percebe alguma diferença na coesão das classes que você cria quando
			pratica TDD, da coesão das classes que você cria quando não pratica TDD?
			
			\item E por que você atribui essa diferença ao TDD, ou seja, o bom design
			criado não pode ter sido influenciado apenas pela sua experiência?

			\item Você acha que mesmo com TDD é possível criar classes com baixa coesão? 

		\end{enumerate}
\end{enumerate}

\subsection{TDD e simplicidade}
\label{entrevista:simplicidade}


\begin{enumerate}
	\item{Classes complexas, com métodos e classes longas. Você
	costuma ver muitas delas?}
		\begin{enumerate}
			\item Em sua opinião, por que elas aparecem?

			\item Como você combate esse problema?

			\item Como TDD te ajuda nisso?
			
			\item Você percebe alguma diferença da simplicidade das classes que você cria quando
			pratica TDD, da simplicidade das classes que você cria quando não pratica TDD?
			
			\item E por que você atribui essa diferença ao TDD, ou seja, o bom design
			criado não pode ter sido influenciado apenas pela sua experiência?

			\item Você acha que mesmo com TDD é possível criar classes com códigos difíceis de serem entendidos? 

		\end{enumerate}
\end{enumerate}

\section{Relação entre TDD e experiência}
\label{entrevista:experiencia}

\begin{enumerate}
	\item Como você compararia as primeiras vezes que você praticou TDD com agora?

	\item Você acha que a experiência influencia no resultado final?

	\item Em sua opinião, como você acha que seria o design de uma pessoa sem
	experiência em desenvolvimento de software e sem experiência em OO, mas praticando TDD?

	\item O programador às vezes comete desvios na prática, como não ver o teste
	falhar,  esquecer de refatorar, não rodar a bateria de testes completa,
	refatorar outro trecho de código enquanto está no vermelho, e etc. Você vê isso
	como um problema?
	
	\item TDD fala sobre \textit{baby steps} (ou passos de bebê). O que você 
	entende por dar passos de bebê?
	\begin{enumerate}
		\item É muito comum, principalmente em atividades como dojos, as pessoas
		ficarem  criando if's e mais if's (que é o código mais simples possível que
		faz o teste passar),  até o momento em que o código se torna o complexo
		suficiente para merecer uma refatoração. O que pensa sobre isso?

		\item Como você evita que esse código fique complicado?
	\end{enumerate}
\end{enumerate}

\section{TDD e outras práticas ágeis}
\label{entrevista:tdd-e-praticas-ageis}

\begin{enumerate}
	\item Você acha que outras práticas ágeis influenciam na qualidade do seu design?

	\item Vocês praticam programação pareada? Com que frequência? Quando vocês
	fazem programação pareada, você acha que isso ajuda mais no design do que TDD?

	\item Se só praticassem programação pareada, por exemplo, você acha que mesmo
	assim precisaria de TDD?

	\item Quais outras práticas ágeis você acha que ajuda você a melhorar o design?
\end{enumerate}

\section{Opiniões finais}

\begin{enumerate}
	\item Você acha que TDD resolve de vez o problema de designs que degradam ao longo do tempo?

	\item Gostaria de dizer mais algo sobre TDD que não disse nas perguntas anteriores?
	
	\item Como você compararia a qualidade de códigos escritos com TDD e sem TDD? 

	\item Com quem posso falar para saber mais sobre o assunto?
\end{enumerate}

