%% ------------------------------------------------------------------------- %%
\chapter{Métodos qualitativos de pesquisa} 
\label{cap:qualitativo}

Conduzir um estudo experimental em engenharia de software sempre foi uma
atividade difícil. Uma das razões para isso é o fator humano, muito presente 
no processo de desenvolvimento de software, como sugerido por métodos ágeis  em
geral \cite{AgileManifesto}. Dessa maneira, o paradigma de pesquisa analítico 
não é suficiente para investigar casos reais complexos envolvendo pessoas e 
suas interações com a tecnologia \cite{guidelines-case-study}.

Conforme levantado por Seaman \cite{seaman}, esses problemas já foram levantados
por muitos pesquisadores e finalmente tem-se levado em consideração a influência de
problemas não-técnicos e a intersecção entre esses problemas e a parte técnica
dentro da engenharia de software. 
Apesar disso, o número de estudos empíricos é ainda muito pequeno dentro da área
de pesquisa em ciência da computação: Sjoberg et al \cite{sjoberg} encontrou
apenas 103 experimentos em 5.453 artigos e Ramesh et al. \cite{ramesh}
identificou menos de 2\% de experimentos envolvendo humanos e apenas 0.16\% 
estudos em campo dentre 628 artigos.

Uma pesquisa qualitativa é um meio para se explorar e entender a influência que 
indivíduos ou grupos atribuem a um problema social ou humano. O processo de
pesquisa envolve questões emergentes e procedimentos, dados geralmente colhidos
sob o ponto de vista do participante, com a análise feita de maneira indutiva
indo geralmente de um tema específico para um tema geral e com o pesquisador
fazendo interpretações do significado desses dados. Dados capturados por estudos
qualitativos são representados por palavras e figuras, e não por números.
O relatório final tem uma estrutura flexível e os pesquisadores que se
dedicam a esta forma de pesquisa apoiam uma maneira de olhar para a pesquisa que
honra o estilo indutivo, o foco em termos individuais, e a importância de mostrar a 
complexidade de uma situação \cite{creswell}. 

%% ------------------------------------------------------------------------- %%
\section{Características de pesquisas qualitativas}

Métodos qualitativos de busca possuem diversas características, que juntas fazem
com que a pesquisa se torne rica em detalhes. Creswell \cite{creswell} lista
alguma delas:

\begin{enumerate}
  
  \item \textbf{Ambiente real}. Diferentemente da maneira tradicional,
  pesquisas qualitativas não levam indíviduos para laboratórios e fazem experimentos
  controlados; elas são geralmente realizadas em ambientes reais, aonde o objeto
  sob estudo acontece. Essa é a principal vantagem da pesquisa qualitativa, já
  que ela acontece no ambiente na qual o objeto sob estudo atua na prática. O
  contato pessoal do pesquisador com os indíviduos permite com que ele colha
  informações que raramente são repetidas em laboratório;
  
  \item \textbf{Pesquisador como instrumento chave de pesquisa}. O pesquisador
  tem papel fundamental no processo, já que ele é o responsável pela captura dos
  dados, através da examinação de documentos, entrevistas ou observações feitas
  no mundo real. Pesquisadores tendem a não utilizar questionários ou
  instrumentos desenvolvidos por outros pesquisadores;
  
  \item \textbf{Múltiplas fontes de dados}. Pesquisas qualitativas geralmente
  colhem informações de múltiplas fontes de dados, como entrevistas,
  observações e documentos, ao invés de confiar em apenas uma fonte de dados;
  
  \item \textbf{Análise dos dados indutiva}. Os dados são analisados de dentro
  para fora, através da categorização dos mesmos em unidades de informação cada
  vez mais abstratas. Esse processo indutivo gera diversas idas e vindas entre
  os temas encontrados e a base de dados, até o momento em que os pesquisadores
  estabeleçam um conjunto compreensivo de temas;
  
  \item \textbf{Visão do participante}. Trabalhos qualitativos focam na visão do
  participante sobre o objeto em estudo, e não na visão que o pesquisador ou a
  litetura tem à respeito do mesmo;
  
  \item \textbf{Design emergente}. O processo de pesquisa qualitativa é
  emergente. Isso significa que o processo não deve ser completamente descrito
  desde o começo, mas sim modificado de acordo com o início da coleta dos dados
  no campo de pesquisa. A ideia chave por trás da pesquisa qualitativa é
  aprender sobre o problema com os participantes e direcionar a pesquisa para
  obter aquela informação;
  
  \item \textbf{Interpretativa}. Pesquisadores fazem uma interpretação daquilo
  que veem, ouve e entendem. As interpretações do pesquisador não podem ser
  separadas do seu conhecimento, história, contexto e entendimentos anteriores
  do problema. Ao final do relatório da pesquisa, leitores também fazem suas
  interpretações, oferecendo ainda novas interpretações para o estudo. Com os
  leitores, participantes e pesquisadores fazendo interpretações, múltiplas
  visões do problema podem emergem;
  
\end{enumerate} 

%% ------------------------------------------------------------------------- %%
\section{Papel do pesquisador}

Devido à sua característica interpretativa, o pesquisador tem papel fundamental
em uma pesquisa qualitativa. A princípio, o pesquisador deve refletir e
deixar explícito qualquer possível viés da pesquisa e como isso foi tratado
durante a mesma. O pesquisador deve também deixar claro qual a sua relação
pessoal com os participantes, já que isso pode influenciar o resultado final.

O roteiro da entrevista, anotações, documentos e materiais visuais, bem como o
protocolo para coletar informações devem ser preparados pelo pesquisador.
Diferentemente do sugerido em pesquisas quantativas, aonde a amostra é
geralmente randomica, o ambiente utilizado na pesquisa deve ser cuidadosamente
escolhido pelo pesquisador, afim de melhor obter informações sobre o objeto em
estudo. Durante a escolha do ambiente, quatro aspectos devem ser levados em
conta \cite{miles-and-huberman}: o ambiente (aonde a pesquisa acontecerá), os
atores (participantes que serão observados), os eventos (o que os atores serão
observados ou entrevistados fazendo) e o processo (o caráter evolutivo dos
eventos realizados pelos atores dentro do ambiente).

%% ------------------------------------------------------------------------- %%
\section{Formas de colheta de dados}

O pesquisador também deve indicar os tipos de dados a serem colhidos. A Tabela
\ref{tab:coleta-de-dados}, baseada em Creswell \cite{creswell} discute as
principais maneiras de se colher informações, bem como suas vantagens e
desvantagens. Mas isso não quer dizer que o pesquisador deve ficar limitado a
esses instrumentos de coleta -- ele pode usar de sua criatividade para encontrar
novos instrumentos, enriquecendo ainda mais o trabalho.

\begin{table}
	\begin{tabular}{ | p{3cm} | p{6cm} | p{6cm} | }
		\hline
		Observações &
		  Pesquisador tem experiência em primeira mão com o participante;
		  Pesquisador pode gravar dados na medida em que eles ocorrem;
		  Aspectos não usuais podem ser notados durante a observação;
		  Útil para explorar informações que os participantes não se sentem a
		  vontade para discutir.
		&
		  Pesquisador pode ser visto como um intruso;
		  Informações confidenciais, que não podem ser relatadas, podem ser
		  vistas pelo pesquisador;
		  Pesquisador pode não ser um bom observador;
		\\
		\hline
		Entrevistas &
		  Útil quando os participantes não podem ser observados diretamente;
		  Participantes podem prover dados históricos sobre o objeto em estudo;
		  Permite ao pesquisador o controle sobre as questões a serem feitas;
		&
		  Provê informações indiretas, filtradas através da visão dos
		  participantes;
		  Informações não são colhidas em seu ambiente natural;
		  A presença do pesquisador pode enviesar as respostas;
		  Nem todos os participantes são articulados e perceptivos;
		\\
		\hline
		Documentos &
		  Fonte de dados não obstrusiva, ou seja, pode ser acessado a qualquer
		  momento pelo pesquisador;
		  São dados nos quais os participantes deram atenção para compilá-los;
		  Como já estão escritas, poupam o tempo da transcrição que
		  seria gasto pelo pesquisador;
		&
		  Podem possuir informações privadas, que não podem ser relatadas pelo
		  pesquisador;
		  Podem exigir que o pesquisador busque por informações em lugares
		  difíceis de procurar;
		  Podem ser incompletos;
		\\			
		\hline
	\end{tabular}
	\caption{Métodos de coleta de dados, suas vantagens e desvantagens}
	\label{tab:coleta-de-dados}
\end{table}

Todas as práticas citadas acima podem ter variações. Observações, por exemplo,
podem ser feitas através da participação ativa do pesquisador ou tendo o
pesquisador como mero observador. Entrevistas podem ser face-a-face, através de
telefones ou grupos focais. 

Entrevistas podem ser ainda estruturadas, onde o
pesquisador tem perguntas definidas e o participante apenas as responde, ou
não-estruturadas, onde o pesquisador apenas cita temas e o participante discute
a respeito dos mesmos. As perguntas feitas pelo pesquisador podem ser abertas,
onde o participante pode responder o que quiser, ou fechadas, na qual o mesmo
deve apenas escolher entre opções dadas pelo pesquisador \cite{seaman}.

Todos os instrumentos de coleta também devem ser acompanhados
de um protocolo. Observações, por exemplo, podem ser divididas em notas descritivas
(como trechos mencionados importantes) e notas reflexivas (pensamentos
pessoais do pesquisador, especulações ou sentimentos). Já um protocolo de
entrevista pode contar um cabeçalho para preenchimento de dados básicos (como data, tempo
de duração, nome do participante), instruções para o entrevistador seguir em
todas as entrevistas, 3 ou 4 questões para diminuir o medo do participante em
falar com o entrevistador, e assim por diante. 

%% ------------------------------------------------------------------------- %%
\section{Análise dos dados}

Por fim, o processo de análise dos dados é responsável por dar sentido e
interpretação à todos os dados colhidos. Esse processo envolve a preparação dos
dados e o aprofundamento no entendimento dos mesmos. É um processo contínuo que
envolve reflexão contínua sobre os dados colhidos, gerando questões analíticas e
escrevendo notas em todo o estudo. Segundo Creswell \cite{creswell}, essa
análise pode ser feita enquanto as entrevistas ainda estão acontecendo; o
pesquisador pode ir analisando os dados colhetados até aquele momento,
escrever notas e ir organizando a estrutura do relatório final.
É possível ainda adicionar procedimentos à esse processo. Teoria fundamentada em
dados, por exemplo, possue passos sistemáticos para a realização da análise;
estudos de caso e etnografia envolvem uma descrição detalhada do ambiente e dos
participantes, seguido da análise dos dados para os temas \cite{stake}.

De maneira geral, pesquisadores seguem os passos abaixo durante o processo de
análise dos dados:

\begin{itemize}

	\item \textbf{Organização e Preparação dos Dados}: Este prcesso
	envolve a transcrição de entrevistas e das notas geradas durante todo o 
	processo de entrevistas;
	
	\item \textbf{Leitura dos dados}: Leitura de todos os dados gerados até o
	momento. Nesse momento há uma reflexão sobre os principais pontos e opiniões 
	colhidos em todas as fontes de dados;
	
	\item \textbf{Codificação e Agrupamento por Temas}:	Codificação é o ato de
	organizar e classificar os dados em pequenas categorias ou segmentos de textos 
	antes de trazer qualquer significado àquela informação \cite{rossman}. O
	pesquisador usualmente seleciona frases ou parágrafos e atribui a ele uma
	categoria. Após o processo de codificação, agrupa-se esses dados em
	temas que possuem uma granularidade maior. Esses temas são aqueles que 
	aparecerão como maiores contribuições da pesquisa qualitativa.
	
	\item \textbf{Interpretação do resultado}: O pesquisador interpreta os dados, 
	basendo-se tanto na sua experiência pessoal com o assunto quanto com dados 
	retirados da literatura.

\end{itemize}

%% ------------------------------------------------------------------------- %%
\section{Validade e confiabilidade}

O pesquisador deve ter também grande preocupação com a validade e a
confiabilidade do estudo qualitativo. O termo \textit{validade} significa que o
pesquisador se preocupou com a exatidão das informações geradas pela pesquisa.
Já o termo \textit{confiabilidade} indica que a abordagem do pesquisador é
consistente entre diferentes pesquisadores e diferentes projetos
\cite{gibbs-2007}.

Para garantir a confiabilidade de seu trabalho, Gibbs \cite{gibbs-2007} sugere
que o pesquisador tome as seguintes precauções:

\begin{itemize}
	\item \textbf{Checar as transcrições}. O pesquisador garante que não foram
	cometidos erros óbvios durante a transcrição;

	\item \textbf{Má interpretação dos códigos}. A pesquisa certifica que não há um
	desvio na definição dos códigos ou no significado dos códigos durante o processo 
	de codificação, já que os mesmos foram alcançados através de comparação e
	cruzamento entre dados de diferentes pesquisadores;
	
	\item \textbf{Rastreabilidade dos dados}. Todos os dados colhidos devem ser
	preservados e podem ser consultados pelos pesquisadores a qualquer momento
	durante todo o processo de pesquisa.

\end{itemize}

Para garantir a validade, o pesquisador deve aumentar a precisão do seu estudo.
A listagem abaixo discute algumas das possíveis estratégias para tal:

\begin{itemize}
	\item \textbf{Triangulação}. A pesquisa deve contar com diferentes fontes
	de dados e usá-las como uma justificativa coerente para os temas que emergiram
	dessas fontes de dados;

	\item \textbf{Apresentar os resultados da pesquisa para os participantes e
	verificar se eles concordam com os resultados gerados}. Garantir que os
	participantes validem os dados encontrados diminui o viés do pesquisador;

	\item \textbf{Prover descrição rica e detalhada sobre o ambiente}. A riqueza
	dos detalhes mostra a qualidade do estudo, além de possibilitar a repetição do
	experimetno por outros pesquisadores;

	\item \textbf{Esclarecer todos os possíveis vieses da pesquisa}. A pesquisa
	deve deixar claro e ser honesta sobre quais são suas limitações;
	
	\item \textbf{Discutir pontos discrepantes}. Pesquisas que discutem pontos
	discrepantes tornam-se mais realistas, e por consequência, aumentam em
	credibilidade;
	
	\item \textbf{Auditor externo}. Um auditor externo, que não é familiar com a
	pesquisa ou com o pesquisador, pode prover informações objetivas sobre a
	pesquisa e o processo como um todo.

\end{itemize} 
