%% ------------------------------------------------------------------------- %%
\chapter{Planejamento do Experimento}
\label{cap:planejamento}

Conforme discutido no Capítulo \ref{cap:trabalhos-relacionados}, muitos 
trabalhos avaliaram TDD, e alguns deles relatam uma melhora
no design de classes, como um menor acoplamento, uma maior coesão, e até mesmo
mais simplicidade nas classes quando desenvolvidas utilizando-se a prática. 

Grande parte deles estão interessados em avaliar quais os efeitos da prática
no código final produzido, mas poucos estudos tentam entender como TDD e a
prática de escrever o teste antes do código real realmente guiam o programador 
em direção à essas melhorias.

Para entender como TDD influencia o desenvolvedor durante o desenvolvimento,
este trabalho propõe estudos de caso em empresas de desenvolvimento de
software brasileiras, utilizando métodos qualitativos de pesquisa.
Este capítulo detalha o planejamento do experimento, bem como o processo de
análise dos dados colhidos.

%% ------------------------------------------------------------------------- %%
\section{Métodos qualitativos de pesquisa} 
\label{sec:planejamento-qualitativa}

Conduzir um estudo experimental em engenharia de software sempre foi uma
atividade difícil. Uma das razões para isso é o fator humano, muito presente 
no processo de desenvolvimento de software, como sugerido por métodos ágeis  em
geral \cite{AgileManifesto}. Dessa maneira, o paradigma de pesquisa analítico 
não é suficiente para investigar casos reais complexos envolvendo pessoas e 
suas interações com a tecnologia \cite{guidelines-case-study}.

Conforme levantado por Seaman, esses problemas já foram levantados por muitos
pesquisadores e finalmente tem-se levado em consideração a influência de
problemas não-técnicos e a intersecção entre esses problemas e a parte técnica
dentro da engenharia de software \cite{seaman}. 

Apesar disso, o número de estudos empíricos é ainda muito pequeno dentro da área
de pesquisa em ciência da computação: Sjoberg et al encontrou apenas 103
experimentos em 5.453 artigos \cite{sjoberg} e Ramesh et al. identificou menos
de 2\% de experimentos envolvendo humanos e apenas 0.16\% estudos em campo 
dentre 628 artigos \cite{ramesh} .

Uma pesquisa qualitativa é um meio para se explorar e entender a influência que 
indivíduos ou grupos atribuem a um problema social ou humano. O processo de
pesquisa envolve questões emergentes e procedimentos, dados geralmente colhidos
sob o ponto de vista do participante, com a análise feita de maneira indutiva
indo  geralmente de um tema específico para um tema geral e com o pesquisador
fazendo interpretações do significado desses dados. Dados capturados por estudos
qualitativos são representados por palavras e figuras, e não por números.

O relatório final tem uma estrutura flexível e os pesquisadores que se
dedicam a esta forma de pesquisa apoiam uma maneira de olhar para a pesquisa que
honra o estilo indutivo, o foco em termos individuais, e a importância de mostrar a 
complexidade de uma situação \cite{creswell}. 

Existem muitas maneiras de se conduzir um estudo qualitativo, como etnografia,
teoria fundamentada nos dados (em inglês, \textit{Grounded Theory}), estudos de 
caso, pesquisa de fenômenos (em inglês, \textit{phenomenological research}) e 
pesquisa narrada \cite{creswell}. 

Baseando-se na definição de Robson \cite{robson}, Yin \cite{yin} e Benbasat et
al. \cite{benbasat}, Runeson e Host mostram que todas as definições concordam
que um estudo de caso investiga um fenômeno contemporâneo no seu contexto. 
Robson afirma que um estudo de caso estressa o uso de múltiplas fontes de
evidência, Yin denomina estudo de caso um "inquérito" e destaca que os limites 
entre o fenômeno e seu contexto podem não ser claros, enquanto que Benbasat et
al. menciona que um estudo de caso recolhe informações de poucas entidades 
(como pessoas, grupos ou organizações) e que também é caracterizado pela  falta
de controle experimental \cite{guidelines-case-study}.

Baseando-se no fato de que o processo de desenvolvimento de software envolve 
diversos fatores humanos e é totalmente sensível ao contexto em que ele está 
inserido, este trabalho é composto por um conjunto de estudos de caso, onde são 
utilizados métodos baseados em teoria fundamentada nos dados para colheta e
análise dos dados \cite{grounded-theory}.

Conduzir uma pesquisa no mundo real implica em um equilíbrio entre nível de
controle e grau de realismo. Uma situação realística é geralmente complexa e 
não-determinística, impedindo o entendimento sobre o que acontece. Por outro
lado, aumentar o controle sobre o experimento reduz o grau de realismo, muitos
vezes fazendo com que os reais fatores de influência fiquem fora do escopo do 
estudo. Estudos de caso são, por definição, conduzidos no mundo real e por isso 
têm um alto grau de realismo, mas dessa forma sacrificando o nível de  controle
\cite{guidelines-case-study}.

, o que
significa que o pesquisador possui um roteiro de perguntas a serem respondidas, 
mas pode eventualmente alterar o rumo da entrevista de acordo com as respostas 
dadas pelos entrevistados \cite{guidelines-case-study}. 

O objetivo básico de uma análise de dados qualitativa é derivar conclusões a
partir dos dados, mantendo uma clara cadeia de evidências. Isso significa que  o
leitor deve ser capaz de seguir a derivação de resultados e conclusões pelos 
dados coletados \cite{yin}.

Os termos validade e confiabilidade em estudos qualitativos não possuem o mesmo 
significado dos estudos quantitativos. Confiabilidade em um estudo qualitativo 
indica que a abordagem do pesquisador é consistente entre diferentes
pesquisadores e diferentes projetos. 
sugestao do gibbs

Já o termo validade em um estudo qualitativo significa que o pesquisador  se
preocupou com a exatidão das informações geradas pela pesquisa. 


%% ------------------------------------------------------------------------- %%
\section{Participantes da pesquisa}
\label{sec:planejamento-participantes}

Para a condução do estudo de caso, algumas empresas do mercado de software
brasileiro foram selecionadas de acordo com alguns critérios.
Os critérios que foram levados em consideração no momento de convidar empresas a
participarem da pesquisa foram:

\begin{itemize}
	\item Utilização de TDD há mais de 1 ano;

	\item Localizadas em São Paulo (SP), para facilitar o processo de
	entrevista e observação do ambiente;

	\item Desenvolvedores com diferentes níveis de experiência em TDD e em 
	Orientação a Objetos, desde iniciantes até experientes;

	\item Projetos que utilizem a tecnologia Java e sejam baseados na web;

	\item Metodologias ágeis como processo de desenvolvimento de
	software;
\end{itemize}

Ao final, três empresas foram selecionadas. Todas elas desenvolvem softwares de
pequeno e médio porte (e por isso entenda-se software entre 100 a 500 mil linhas
de código), utilizando a plataforma Java para desenvolvimento. Elas também 
possuem profissionais com os mais variados níveis de experiência na área 
trabalhando juntos no mesmo time.

A repetição do experimento em diferentes empresas com diferentes projetos,
equipes e contextos serve como estratégia de validação dos resultados
encontrados.

%% ------------------------------------------------------------------------- %%
\section{Estratégia de pesquisa} 
\label{sec:planejamento-estrategia}


Os dados serão coletados principalmente através de entrevistas com
desenvolvedores que praticam TDD no dia-a-dia de trabalho e que podem avaliar
com propriedade o efeito da prática sobre o design orientado a objetos dos seus 
projetos. Além disso, o ambiente será observado pelo pesquisador afim de
capturar informações do cotidiano relevantes para a pesquisa.

As sub-seções abaixo detalham como os dados serão colhidos em cada etapa da pesquisa.

\subsection{Entrevistas}
\label{sec:planejamento-estrategia-entrevistas}

O primeiro passo da pesquisa é uma série de entrevistas com os desenvolvedores 
das empresas participantes. As entrevistas serão semi-estruturadas, dando
liberdade ao pesquisador para mudar o rumo das perguntas, caso se faça
necessário.

Além disso, todas as perguntas serão abertas, permitindo que o desenvolvedor dê
uma resposta ampla sobre o assunto. O principal objetivo dessas entrevistas é
capturar informações sobre:

\begin{itemize}
	\item \textbf{Caracterizar a experiência do programador com desenvolvimento de
	software, boas práticas de desenvolvimento e TDD}. Vital para caracterizar o
	público-alvo da pesquisa.

	\item \textbf{A relação entre a experiência do programador e os efeitos de TDD
	na qualidade do design}. Por ser um outro possível fator de influência, o
	pesquisador colherá informações sobre a experiência do programador com
	desenvolvimento de softwares, conceitos de orientação a objetos e TDD. 

	\item \textbf{Como o desenvolvedor vê a influência que a utilização de TDD
	junto com outras práticas ágeis, como programação pareada, tem sobre o design}.
	Por ser um possível fator de influência, o pesquisador colherá informações para
	diferenciar efeitos da programação pareada dos efeitos de TDD.

	\item \textbf{A visão do desenvolvedor sobre a influência de TDD no design}.
	O pesquisador alinhará as definições da prática com todos os participantes da
	pesquisa.

	\item \textbf{Os efeitos de TDD sobre o acoplamento e a coesão geral das
	classes do sistema}. Entender a opinião do participante sobre os efeitos de TDD
	nesses dois pontos importantes em sistemas orientados a objetos. 

 	\item \textbf{Os efeitos de TDD sobre a gerenciamento de dependências em
 	sistemas orientados a objetos}. Em um nível maior, entender qual a possível
 	influência da prática no gerenciamento de dependências.
	
	\item \textbf{Como o desenvolvedor utiliza o feedback dos testes para guiar o
	design}. Entender como o desenvolvedor utiliza os testes para melhorar o design
	é a questão-chave dessa pesquisa, e para isso o pesquisador perguntará desde
	questões conceituais até questões bem práticas com discussões de trechos de
	código.

	\item \textbf{A relação entre TDD e boas práticas de orientação a objetos}.
	O pesquisador colherá informações sobre a relação entre a prática e
	bons princípios de orientação a objetos, como SOLID \cite{bob-martin}.

	\item \textbf{A visão do programador sobre a simplicidade que TDD agrega ao
	processo}. Entender como TDD leva o programador à criação de códigos mais
	simples.

\end{itemize} 

As entrevistas serão feitas em dias diferentes para que, ao final de cada dia,
o pesquisador possa discutir sobre os pontos levantados no dia
corrente e sugerir melhorias no instrumento de pesquisa. A
versão final do roteiro de entrevista pode ser encontrado no apêndice 
\ref{ape:entrevista} deste trabalho.

O processo de codificação \cite{seaman} será feito por dois
pesquisados separadamente, e só então ambos juntarão seus resultados, após uma 
série de comparações e discussões sobre as diferenças encontradas.

%% ------------------------------------------------------------------------- %%
\subsection{Métricas de código}
\label{sec:planejamento-metricas}



%% ------------------------------------------------------------------------- %%
\section{Análise dos dados}
\label{sec:planejamento-analise}

O processo de análise, baseado em \cite{creswell}, que está representado na
figura (TODO fig), ilustra o processo de análise que será utilizado nessa
pesquisa. Apesar de parecer uma abordagem em cascata, ele é na prática 
interativo, já que os passos são interconectados. Cada um dos níveis é 
detalhado a seguir:

\begin{itemize}

	\item \textit{Organização e Preparação dos Dados}: Transcrição das entrevistas 
	e das notas geradas durante todo o processo de entrevistas; transcrição das
	notas das observações;
	
	\item \textit{Leitura dos dados}: Leitura de todos os dados gerados até o
	momento. Nesse momento há uma reflexão sobre os principais pontos e opiniões 
	colhidos em todas as fontes de dados;
	
	\item \textit{Codificação e Agrupamento por Temas}:	Codificação é o ato de
	organizar e classificar os dados em pequenas categorias ou segmentos de textos 
	antes de trazer qualquer significado àquela informação \cite{rossman}. Esse 
	processo será feito por dois pesquisadores e os resultados serão confrontados. 
	Após o processo de codificação, agrupa-se os dados codificados em temas que 
	possuem uma granularidade maior. Esses temas são aqueles que aparecerão como 
	maiores contribuições da pesquisa qualitativa.
	
	\item \textit{Interpretação do resultado}: O pesquisador interpreta os dados, 
	basendo-se tanto na sua experiência pessoal com o assunto quanto com dados 
	retirados da literatura.

\end{itemize}
 
 %% ------------------------------------------------------------------------- %%
\section{Validade e Confiabilidade do Estudo}
\label{sec:planejamento-validacao}

Para garantir a confiabilidade deste estudo, os pesquisadores seguem as
orientações de Gibbs \cite{gibbs} e realizam os seguintes procedimentos:

\begin{itemize}
	\item \textbf{Checar as transcrições}. O pesquisador garante que não foram
	cometidos erros óbvios durante a transcrição;

	\item \textbf{Má interpretação dos códigos}. A pesquisa certifica que não há um
	desvio na definição dos códigos ou no significado dos códigos durante o processo 
	de codificação, já que os mesmos foram alcançados através de comparação e
	cruzamento entre dados de diferentes pesquisadores;
	
	\item \textbf{Rastreabilidade dos dados}. Todos os dados colhidos estão
	preservados e podem ser consultados pelos pesquisadores a qualquer momento
	durante todo o processo de pesquisa.

\end{itemize}

A validade do estudo será garantida por alguns procedimentos executados pelos
pesquisadores, dentre eles:

\begin{itemize}
	\item \textbf{Triangularização diferentes fontes de dados}. A pesquisa é
	realizada em diferentes ambientes, com diferentes participantes e projetos,
	trazendo diferentes pontos de vista para dentro da análise;

	\item \textbf{Apresentar os resultados da pesquisa para os participantes e
	verificar se eles concordam com os resultados gerados}. Garantir que os
	participantes validem os dados encontrados diminui o viés do pesquisador;

	\item \textbf{Prover descrição rica e detalhada sobre o ambiente}. A riqueza
	dos detalhes mostra a qualidade do estudo, além de possibilitar a repetição do
	experimetno por outros pesquisadores;

	\item \textbf{Esclarecer todos os possíveis vieses da pesquisa}. A pesquisa
	deixa claro quais são suas limitações.

\end{itemize}

%% ------------------------------------------------------------------------- %%
\section{Papel do Pesquisador}
\label{sec:planejamento-papel}

O pesquisador tem como papel fundamental participar do processo de  captura de
todos os dados comentados nas seções acima. No processo de entrevista, o 
pesquisador fará o papel de entrevistador principal, enquanto outro  pesquisador
convidado fará o papel de observador da entrevista.

Creswell \cite{creswell}, citando Locke \cite{locke}, lembra
que a contribuição do investigador para o contexto da pesquisa pode ser útil e
positiva ao invés de prejudicial. Além do mais, o pesquisador é responsável por
identificar todos os valores pessoais, pressuposições e vieses desse estudo.

%% ------------------------------------------------------------------------- %%
\section{Problemas Éticos}
\label{sec:planejamento-etica}

Essa pesquisa poderá revelar problemas nas empresas selecionadas, como problemas
no design do software, na qualidade dos seus desenvolvedores, entre outros. 

Por esse motivo, todos os dados colhidos pelo pesquisador serão mantidos em
sigilo e todos os nomes de desenvolvedores e projetos omitidos, conforme acordo 
assinado entre o pesquisador e a empresa.

%% ------------------------------------------------------------------------- %%
\section{Resultados Esperados}
\label{sec:planejamento-resultados-esperados}

O objetivo desta pesquisa é entender, de maneira satisfatória, a influência de 
TDD no design de sistemas orientados a objetos, obtendo o ponto de vista dos 
desenvolvedores que a praticam diariamente. Conforme discutido nos trabalhos 
relacionados, muitos pesquisas já foram feitas, mas poucas discutem os motivos 
e as razões de TDD influenciar no design.

%% ------------------------------------------------------------------------- %%
\section{Estudo piloto}

% TODO: estudo piloto