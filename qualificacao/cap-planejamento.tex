%% ------------------------------------------------------------------------- %%
\chapter{Planejamento do Experimento}
\label{cap:planejamento}

Conforme discutido no Capítulo \ref{cap:trabalhos-relacionados}, muitos 
trabalhos avaliaram TDD, e alguns deles relatam uma melhora
no design de classes, como um menor acoplamento, uma maior coesão, e até mesmo
mais simplicidade nas classes quando desenvolvidas utilizando-se a prática. 
Grande parte deles estão interessados em avaliar quais os efeitos da prática
no código final produzido, mas poucos estudos tentam entender como TDD e a
prática de escrever o teste antes do código real realmente guiam o programador 
em direção à essas melhorias.

Para entender como TDD influencia o desenvolvedor durante o desenvolvimento,
este trabalho propõe estudos de caso em empresas de desenvolvimento de
software brasileiras, utilizando métodos qualitativos de pesquisa.
Este capítulo detalha o planejamento do experimento, bem como o processo de
análise dos dados colhidos.

%% ------------------------------------------------------------------------- %%
\section{Métodos qualitativos de pesquisa} 
\label{sec:planejamento-qualitativa}

Conduzir um estudo experimental em engenharia de software sempre foi uma
atividade difícil. Uma das razões para isso é o fator humano, muito presente 
no processo de desenvolvimento de software, como sugerido por métodos ágeis  em
geral \cite{AgileManifesto}. Dessa maneira, o paradigma de pesquisa analítico 
não é suficiente para investigar casos reais complexos envolvendo pessoas e 
suas interações com a tecnologia \cite{guidelines-case-study}.

Conforme levantado por Seaman, esses problemas já foram levantados por muitos
pesquisadores e finalmente tem-se levado em consideração a influência de
problemas não-técnicos e a intersecção entre esses problemas e a parte técnica
dentro da engenharia de software \cite{seaman}. 
Apesar disso, o número de estudos empíricos é ainda muito pequeno dentro da área
de pesquisa em ciência da computação: Sjoberg et al encontrou apenas 103
experimentos em 5.453 artigos \cite{sjoberg} e Ramesh et al. identificou menos
de 2\% de experimentos envolvendo humanos e apenas 0.16\% estudos em campo 
dentre 628 artigos \cite{ramesh} .

Uma pesquisa qualitativa é um meio para se explorar e entender a influência que 
indivíduos ou grupos atribuem a um problema social ou humano. O processo de
pesquisa envolve questões emergentes e procedimentos, dados geralmente colhidos
sob o ponto de vista do participante, com a análise feita de maneira indutiva
indo geralmente de um tema específico para um tema geral e com o pesquisador
fazendo interpretações do significado desses dados. Dados capturados por estudos
qualitativos são representados por palavras e figuras, e não por números.
O relatório final tem uma estrutura flexível e os pesquisadores que se
dedicam a esta forma de pesquisa apoiam uma maneira de olhar para a pesquisa que
honra o estilo indutivo, o foco em termos individuais, e a importância de mostrar a 
complexidade de uma situação \cite{creswell}. 

Métodos qualitativos de busca possuem diversas características, que juntas fazem
com que a pesquisa se torne rica em detalhes. Creswell \cite{creswell} lista
alguma delas:

\begin{enumerate}
  
  \item \textbf{Ambiente real}. Diferentemente da maneira tradicional,
  pesquisas qualitativas não levam indíviduos para laboratórios e fazem experimentos
  controlados; elas são geralmente realizadas em ambientes reais, aonde o objeto
  sob estudo acontece. Essa é a principal vantagem da pesquisa qualitativa, já
  que ela acontece no ambiente na qual o objeto sob estudo atua na prática. O
  contato pessoal do pesquisador com os indíviduos permite com que ele colha
  informações que raramente são repetidas em laboratório;
  
  \item \textbf{Pesquisador como instrumento chave de pesquisa}. O pesquisador
  tem papel fundamental no processo, já que ele é o responsável pela captura dos
  dados, através da examinação de documentos, entrevistas ou observações feitas
  no mundo real. Pesquisadores tendem a não utilizar questionários ou
  instrumentos desenvolvidos por outros pesquisadores;
  
  \item \textbf{Múltiplas fontes de dados}. Pesquisas qualitativas geralmente
  colhem informações de múltiplas fontes de dados, como entrevistas,
  observações e documentos, ao invés de confiar em apenas uma fonte de dados;
  
  \item \textbf{Análise dos dados indutiva}. Os dados são analisados de dentro
  para fora, através da categorização dos mesmos em unidades de informação cada
  vez mais abstratas. Esse processo indutivo gera diversas idas e vindas entre
  os temas encontrados e a base de dados, até o momento em que os pesquisadores
  estabeleçam um conjunto compreensivo de temas;
  
  \item \textbf{Visão do participante}. Trabalhos qualitativos focam na visão do
  participante sobre o objeto em estudo, e não na visão que o pesquisador ou a
  litetura tem à respeito do mesmo;
  
  \item \textbf{Design emergente}. O processo de pesquisa qualitativa é
  emergente. Isso significa que o processo não deve ser completamente descrito
  desde o começo, mas sim modificado de acordo com o início da coleta dos dados
  no campo de pesquisa. A ideia chave por trás da pesquisa qualitativa é
  aprender sobre o problema com os participantes e direcionar a pesquisa para
  obter aquela informação;
  
  \item \textbf{Interpretativa}. Pesquisadores fazem uma interpretação daquilo
  que veem, ouve e entendem. As interpretações do pesquisador não podem ser
  separadas do seu conhecimento, história, contexto e entendimentos anteriores
  do problema. Ao final do relatório da pesquisa, leitores também fazem suas
  interpretações, oferecendo ainda novas interpretações para o estudo. Com os
  leitores, participantes e pesquisadores fazendo interpretações, múltiplas
  visões do problema podem emergem;
  
\end{enumerate} 

Devido à sua característica interpretativa, o pesquisador tem papel fundamental
em uma pesquisa qualitativa. A princípio, o pesquisador deve refletir e
deixar explícito qualquer possível viés da pesquisa e como isso foi tratado
durante a mesma. O pesquisador deve também deixar claro qual a sua relação
pessoal com os participantes, já que isso pode influenciar o resultado final.

O roteiro da entrevista, anotações, documentos e materiais visuais, bem como o
protocolo para coletar informações devem ser preparados pelo pesquisador.
Diferentemente do sugerido em pesquisas quantativas, aonde a amostra é
geralmente randomica, o ambiente utilizado na pesquisa deve ser cuidadosamente
escolhido pelo pesquisador, afim de melhor obter informações sobre o objeto em
estudo. Durante a escolha do ambiente, quatro aspectos devem ser levados em
conta \cite{miles-and-huberman}: o ambiente (aonde a pesquisa acontecerá), os
atores (participantes que serão observados), os eventos (o que os atores serão
observados ou entrevistados fazendo) e o processo (o caráter evolutivo dos
eventos realizados pelos atores dentro do ambiente).

O pesquisador também deve indicar os tipos de dados a serem colhidos. A Tabela
\ref{tab:coleta-de-dados}, baseada em Creswell \cite{creswell} discute as
principais maneiras de se colher informações, bem como suas vantagens e
desvantagens. Mas isso não quer dizer que o pesquisador deve ficar limitado a
esses instrumentos de coleta -- ele pode usar de sua criatividade para encontrar
novos instrumentos, enriquecendo ainda mais o trabalho.

\begin{table}
	\begin{tabular}{ | p{3cm} | p{6cm} | p{6cm} | }
		\hline
		Observações &
		  Pesquisador tem experiência em primeira mão com o participante;
		  Pesquisador pode gravar dados na medida em que eles ocorrem;
		  Aspectos não usuais podem ser notados durante a observação;
		  Útil para explorar informações que os participantes não se sentem a
		  vontade para discutir.
		&
		  Pesquisador pode ser visto como um intruso;
		  Informações confidenciais, que não podem ser relatadas, podem ser
		  vistas pelo pesquisador;
		  Pesquisador pode não ser um bom observador;
		\\
		\hline
		Entrevistas &
		  Útil quando os participantes não podem ser observados diretamente;
		  Participantes podem prover dados históricos sobre o objeto em estudo;
		  Permite ao pesquisador o controle sobre as questões a serem feitas;
		&
		  Provê informações indiretas, filtradas através da visão dos
		  participantes;
		  Informações não são colhidas em seu ambiente natural;
		  A presença do pesquisador pode enviesar as respostas;
		  Nem todos os participantes são articulados e perceptivos;
		\\
		\hline
		Documentos &
		  Fonte de dados não obstrusiva, ou seja, pode ser acessado a qualquer
		  momento pelo pesquisador;
		  São dados nos quais os participantes deram atenção para compilá-los;
		  Como já estão escritas, poupam o tempo da transcrição que
		  seria gasto pelo pesquisador;
		&
		  Podem possuir informações privadas, que não podem ser relatadas pelo
		  pesquisador;
		  Podem exigir que o pesquisador busque por informações em lugares
		  difíceis de procurar;
		  Podem ser incompletos;
		\\			
		\hline
	\end{tabular}
	\caption{Métodos de coleta de dados, suas vantagens e desvantagens}
	\label{tab:coleta-de-dados}
\end{table}

Todas as práticas citadas acima podem ter variações. Observações, por exemplo,
podem ser feitas através da participação ativa do pesquisador ou tendo o
pesquisador como mero observador. Entrevistas podem ser face-a-face, através de
telefones ou grupos focais. 

Entrevistas podem ser ainda estruturadas, onde o
pesquisador tem perguntas definidas e o participante apenas as responde, ou
não-estruturadas, onde o pesquisador apenas cita temas e o participante discute
a respeito dos mesmos. As perguntas feitas pelo pesquisador podem ser abertas,
onde o participante pode responder o que quiser, ou fechadas, na qual o mesmo
deve apenas escolher entre opções dadas pelo pesquisador \cite{seaman}.

Todos os instrumentos de coleta também devem ser acompanhados
de um protocolo. Observações, por exemplo, podem ser divididas em notas descritivas
(como trechos mencionados importantes) e notas reflexivas (pensamentos
pessoais do pesquisador, especulações ou sentimentos). Já um protocolo de
entrevista pode contar um cabeçalho para preenchimento de dados básicos (como data, tempo
de duração, nome do participante), instruções para o entrevistador seguir em
todas as entrevistas, 3 ou 4 questões para diminuir o medo do participante em
falar com o entrevistador, e assim por diante. 

Por fim, o processo de análise dos dados é responsável por dar sentido e
interpretação à todos os dados colhidos. Esse processo envolve a preparação dos
dados e o aprofundamento no entendimento dos mesmos. É um processo contínuo que
envolve reflexão contínua sobre os dados colhidos, gerando questões analíticas e
escrevendo notas em todo o estudo. Segundo Creswell \cite{creswell}, essa
análise pode ser feita enquanto as entrevistas ainda estão acontecendo; o
pesquisador pode ir analisando os dados colhetados até aquele momento,
escrever notas e ir organizando a estrutura do relatório final.
É possível ainda adicionar procedimentos à esse processo. Teoria fundamentada em
dados, por exemplo, possue passos sistemáticos para a realização da análise;
estudos de caso e etnografia envolvem uma descrição detalhada do ambiente e dos
participantes, seguido da análise dos dados para os temas \cite{stake}.

De maneira geral, pesquisadores seguem os passos abaixo durante o processo de
análise dos dados:

\begin{itemize}

	\item \textbf{Organização e Preparação dos Dados}: Este prcesso
	envolve a transcrição de entrevistas e das notas geradas durante todo o 
	processo de entrevistas;
	
	\item \textbf{Leitura dos dados}: Leitura de todos os dados gerados até o
	momento. Nesse momento há uma reflexão sobre os principais pontos e opiniões 
	colhidos em todas as fontes de dados;
	
	\item \textbf{Codificação e Agrupamento por Temas}:	Codificação é o ato de
	organizar e classificar os dados em pequenas categorias ou segmentos de textos 
	antes de trazer qualquer significado àquela informação \cite{rossman}. O
	pesquisador usualmente seleciona frases ou parágrafos e atribui a ele uma
	categoria. Após o processo de codificação, agrupa-se esses dados em
	temas que possuem uma granularidade maior. Esses temas são aqueles que 
	aparecerão como maiores contribuições da pesquisa qualitativa.
	
	\item \textbf{Interpretação do resultado}: O pesquisador interpreta os dados, 
	basendo-se tanto na sua experiência pessoal com o assunto quanto com dados 
	retirados da literatura.

\end{itemize}

O pesquisador deve ter também grande preocupação com a validade e a
confiabilidade do estudo qualitativo. O termo \textit{validade} significa que o
pesquisador se preocupou com a exatidão das informações geradas pela pesquisa.
Já o termo \textit{confiabilidade} indica que a abordagem do pesquisador é
consistente entre diferentes pesquisadores e diferentes projetos
\cite{gibbs-2007}.

Para garantir a confiabilidade de seu trabalho, Gibbs \cite{gibbs-2007} sugere
que o pesquisador tome as seguintes precauções:

\begin{itemize}
	\item \textbf{Checar as transcrições}. O pesquisador garante que não foram
	cometidos erros óbvios durante a transcrição;

	\item \textbf{Má interpretação dos códigos}. A pesquisa certifica que não há um
	desvio na definição dos códigos ou no significado dos códigos durante o processo 
	de codificação, já que os mesmos foram alcançados através de comparação e
	cruzamento entre dados de diferentes pesquisadores;
	
	\item \textbf{Rastreabilidade dos dados}. Todos os dados colhidos devem ser
	preservados e podem ser consultados pelos pesquisadores a qualquer momento
	durante todo o processo de pesquisa.

\end{itemize}

Para garantir a validade, o pesquisador deve aumentar a precisão do seu estudo.
A listagem abaixo discute algumas das possíveis estratégias para tal:

\begin{itemize}
	\item \textbf{Triangularização}. A pesquisa deve contar com diferentes fontes
	de dados e usá-las como uma justificativa coerente para os temas que emergiram
	dessas fontes de dados;

	\item \textbf{Apresentar os resultados da pesquisa para os participantes e
	verificar se eles concordam com os resultados gerados}. Garantir que os
	participantes validem os dados encontrados diminui o viés do pesquisador;

	\item \textbf{Prover descrição rica e detalhada sobre o ambiente}. A riqueza
	dos detalhes mostra a qualidade do estudo, além de possibilitar a repetição do
	experimetno por outros pesquisadores;

	\item \textbf{Esclarecer todos os possíveis vieses da pesquisa}. A pesquisa
	deve deixar claro e ser honesta sobre quais são suas limitações;
	
	\item \textbf{Discutir pontos discrepantes}. Pesquisas que discutem pontos
	discrepantes tornam-se mais realistas, e por consequência, aumentam em
	credibilidade;
	
	\item \textbf{Auditor externo}. Um auditor externo, que não é familiar com a
	pesquisa ou com o pesquisador, pode prover informações objetivas sobre a
	pesquisa e o processo como um todo.

\end{itemize} 

%% ------------------------------------------------------------------------- %%
\section{Design da pesquisa}

Baseando-se no fato de que o processo de desenvolvimento de software envolve 
diversos fatores humanos e é totalmente sensível ao contexto em que ele está 
inserido, este trabalho é composto por um conjunto de estudos de caso, onde são 
utilizados métodos qualitativos de pesquisa para o processo de colheta e análise
dos dados.

Baseando-se na definição de Robson \cite{robson}, Yin \cite{yin} e Benbasat et
al. \cite{benbasat}, Runeson e Host mostram que todas as definições concordam
que um estudo de caso investiga um fenômeno contemporâneo no seu contexto. 
Robson afirma que um estudo de caso estressa o uso de múltiplas fontes de
evidência, Yin denomina estudo de caso um "inquérito" e destaca que os limites 
entre o fenômeno e seu contexto podem não ser claros, enquanto que Benbasat et
al. menciona que um estudo de caso recolhe informações de poucas entidades 
(como pessoas, grupos ou organizações) e que também é caracterizado pela falta
de controle experimental \cite{guidelines-case-study}.

A intenção dos estudos de caso é obter informações relevantes sobre como os
praticantes de TDD utilizam a prática para influenciar o design de suas
classes. Na tentativa de capturar o máximo de informações possíveis, esse estudo
inclui uma análise profunda através de entrevistas, observações no mundo real e 
avaliação de métricas de código.

%% ------------------------------------------------------------------------- %%
\section{Participantes da pesquisa}
\label{sec:planejamento-participantes}

Para a condução do estudo de caso, algumas empresas do mercado de software
brasileiro foram selecionadas de acordo com alguns critérios.
Os critérios que foram levados em consideração no momento de convidar empresas a
participarem da pesquisa foram:

\begin{itemize}
	\item Utilização de TDD há mais de 1 ano;

	\item Localizadas em São Paulo (SP), para facilitar o processo de
	entrevista e observação do ambiente;

	\item Desenvolvedores com diferentes níveis de experiência em TDD e em 
	Orientação a Objetos, desde iniciantes até experientes;

	\item Projetos que utilizem a tecnologia Java e sejam baseados na web;

	\item Metodologias ágeis como processo de desenvolvimento de
	software;
\end{itemize}

Ao final, três empresas foram selecionadas. Todas elas desenvolvem softwares de
pequeno e médio porte (e por isso entenda-se software entre 100 a 500 mil linhas
de código), utilizando a plataforma Java para desenvolvimento. Elas também 
possuem profissionais com os mais variados níveis de experiência na área 
trabalhando juntos no mesmo time.

A repetição do experimento em diferentes empresas com diferentes projetos,
equipes e contextos serve como estratégia de validação dos resultados
encontrados.

%% ------------------------------------------------------------------------- %%
\section{Estratégia de pesquisa} 
\label{sec:planejamento-estrategia}


Os dados serão coletados principalmente através de entrevistas com
desenvolvedores que praticam TDD no dia-a-dia de trabalho e que podem avaliar
com propriedade o efeito da prática sobre o design orientado a objetos dos seus 
projetos, durante os meses de Junho e Agosto de 2011. 
Além disso, o ambiente será observado pelo pesquisador afim de capturar
informações do cotidiano relevantes para a pesquisa.

As sub-seções abaixo detalham como os dados serão colhidos em cada etapa da pesquisa.

\subsection{Entrevistas}
\label{sec:planejamento-estrategia-entrevistas}

O primeiro passo da pesquisa é uma série de entrevistas com os desenvolvedores 
das empresas participantes, que serão realizadas entre os meses de Junho e
Julho de 2011. As entrevistas serão semi-estruturadas, dando liberdade ao
pesquisador para mudar o rumo das perguntas, caso se faça necessário.

Além disso, todas as perguntas serão abertas, permitindo que o desenvolvedor dê
uma resposta ampla sobre o assunto. O principal objetivo dessas entrevistas é
capturar informações sobre:

\begin{itemize}
	\item \textbf{Caracterizar a experiência do programador com desenvolvimento de
	software, boas práticas de desenvolvimento e TDD}. Vital para caracterizar o
	público-alvo da pesquisa.

	\item \textbf{A relação entre a experiência do programador e os efeitos de TDD
	na qualidade do design}. Por ser um outro possível fator de influência, o
	pesquisador colherá informações sobre a experiência do programador com
	desenvolvimento de softwares, conceitos de orientação a objetos e TDD. 

	\item \textbf{Como o desenvolvedor vê a influência que a utilização de TDD
	junto com outras práticas ágeis, como programação pareada, tem sobre o design}.
	Por ser um possível fator de influência, o pesquisador colherá informações para
	diferenciar efeitos da programação pareada dos efeitos de TDD.

	\item \textbf{A visão do desenvolvedor sobre a influência de TDD no design}.
	O pesquisador alinhará as definições da prática com todos os participantes da
	pesquisa.

	\item \textbf{Os efeitos de TDD sobre o acoplamento e a coesão geral das
	classes do sistema}. Entender a opinião do participante sobre os efeitos de TDD
	nesses dois pontos importantes em sistemas orientados a objetos. 

 	\item \textbf{Os efeitos de TDD sobre a gerenciamento de dependências em
 	sistemas orientados a objetos}. Em um nível maior, entender qual a possível
 	influência da prática no gerenciamento de dependências.
	
	\item \textbf{Como o desenvolvedor utiliza o feedback dos testes para guiar o
	design}. Entender como o desenvolvedor utiliza os testes para melhorar o design
	é a questão-chave dessa pesquisa, e para isso o pesquisador perguntará desde
	questões conceituais até questões bem práticas com discussões de trechos de
	código.

	\item \textbf{A relação entre TDD e boas práticas de orientação a objetos}.
	O pesquisador colherá informações sobre a relação entre a prática e
	bons princípios de orientação a objetos, como SOLID \cite{bob-martin}.

	\item \textbf{A visão do programador sobre a simplicidade que TDD agrega ao
	processo}. Entender como TDD leva o programador à criação de códigos mais
	simples.

\end{itemize} 

% TODO relacionar pontos acima com as perguntas

O processo de entrevista será composto por uma breve introdução da pesquisa, de
modo a não enviesar o participante, seguida de algumas perguntas que visam
caracterizar o perfil do participante; perguntas como qual a experiência do
desenvolvedor em desenvolvimento de software e TDD são necessárias para ajudar o
pesquisador no entendimento das respostas dadas. Além disso, perguntas sobre
referencias, livros e outros pontos de informação nas quais o participante lê a
respeito da prática servem para que o programador entenda o embasamento teórico
do praticante sobre TDD.

Em seguida, o pesquisador perguntará os pontos discutidos na listagem anterior.
Para isso, o pesquisador fará uso não só de perguntas abertas, mas também
solicitará exemplos de código, para que as respostas se tornem técnicas e
específicas.
Por fim, o pesquisador pedirá referências ao pesquisador sobre
possíveis próximos entrevistados, afim de descobrir os programadores referencias 
na prática para aquele determinado participante.

Todas as entrevistas serão gravadas, para que o pesquisador possa fazer a
transcrição e rever os dados a qualquer momento durante o processo. Além disso,
o pesquisador também tomará notas, capturando informações como reações dos 
participantes a determinadas perguntas, ou qualquer outra informação relevante. 

As entrevistas serão feitas em dias diferentes para que, ao final
de cada dia, o pesquisador possa discutir sobre os pontos levantados no dia
corrente e sugerir melhorias no instrumento de pesquisa. A
versão final do roteiro de entrevista pode ser encontrado no apêndice 
\ref{ape:entrevista} deste trabalho.
O processo de codificação será feito por dois
pesquisados separadamente, e só então ambos juntarão seus resultados, após uma 
série de comparações e discussões sobre as diferenças encontradas.

%% ------------------------------------------------------------------------- %%
\subsection{Observação do ambiente real}
\label{sec:planejamento-observacao}

O pesquisador observará os participantes durante suas atividades diárias. As
visitas serão pré-agendadas com os líderes técnicos de cada equipe, de modo a
otimizar o tempo, e evitar a presença do pesquisador durante atividades
não que não fazem uso da prática, como reuniões de trabalho.

O período mínimo de observação será de 2 horas. Após esse período, o observador
decidirá se o tempo de observação deverá ser prolongado. O número mínimo de
observações será de 10 por equipe, garantindo que cada equipe será observada por
no mínimo 20 horas de trabalho.

Para auxiliar o processo de observação, o pesquisador tomará notas durante o
processo, e fará as perguntas que achar necessário para os participantes. Além
disso, o pesquisador observará também o código final produzido pelas equipes,
possibilitando assim o relacionamento entre a prática e a qualidade do design
final produzido.


%% ------------------------------------------------------------------------- %%
\subsection{Métricas de código}
\label{sec:planejamento-metricas}

Após o processo de entrevistas e observação, o pesquisador fará uso de métricas
de código para validar as informações obtidas através das entrevistas e
observações. 

Para facilitar ainda mais o entendimento, essas métricas serão
calculadas em diferentes pontos do código-fonte dentro do controlador de versão,
possibilitando ao pesquisador uma análise sobre os efeitos de TDD no design ao
longo do tempo. Para auxiliar esse trabalho, o pesquisador fará uso de
ferramentas, como a X e a Y.
% TODO ferramentas acima.

%% ------------------------------------------------------------------------- %%
\section{Análise dos dados}
\label{sec:planejamento-analise}

O processo de análise, baseado em \cite{creswell}, que está representado na
figura (TODO fig), ilustra o processo de análise que será utilizado nessa
pesquisa. Apesar de parecer uma abordagem em cascata, ele é na prática 
interativo, já que os passos são interconectados. 

A princípio, os dados serão preparados, através das transcrição das
entrevistas e das notas geradas durante as observações, além das
visualizações das métricas de código colhidas. A seguir, a leitura completa dos
dados colhidos para que o pesquisador possa buscar por erros no processo de
transcrição e para que reflita também sobre os principais pontos e opiniões em
todas as fontes de dados.

O processo de codificação e o agrupamento
por temas será então realizado, que por fim aparecerão como maiores
contribuições da pesquisa qualitativa, durante o processo de interpretação dos
dados pelo pesquisador. O software utilizado para o processo de codificação será
o Atlas.ti \footnote{\url{http://www.atlasti.com/}. Último acesso em 3 de maio
de 2011.}, produzido na Alemanha, que permite ao pesquisador organizar textos,
gráficos, aúdios, e arquivos visuais, e juntá-los com códigos e anotações. 

Durante todo o processo, o pesquisador constantemente validará toda e qualquer
informação colhida e, caso seja necessário, a colheta de qualquer entrevista,
observação ou métrica poderá ser refeita. Os participantes das entrevistas
são comunicados de que o pesquisador poderá entrar em contato
novamente para alguma eventual dúvida.

 %% ------------------------------------------------------------------------- %%
\section{Validade e Confiabilidade do Estudo}
\label{sec:planejamento-validacao}

Para garantir a confiabilidade deste estudo, o pesquisador realizará os
seguintes procedimentos:

\begin{itemize}
	\item \textbf{Checar as transcrições}. O objetivo é garantir que nenhum erro
	óbvio foi cometido;

	\item \textbf{Verificação de pesquisador auxiliar}. Um pesquisador auxiliar
	checará se há algum desvio na definição dos códigos ou no significado dos códigos 
	durante o processo de codificação;
	
	\item \textbf{Rastreabilidade dos dados}. Todos os dados colhidos serão
	preservados em forma eletrônica.

\end{itemize}

A validade do estudo será garantida por alguns procedimentos executados pelos
pesquisadores, dentre eles:

\begin{itemize}
	\item \textbf{Triangularização diferentes fontes de dados}. A pesquisa é
	realizada em diferentes ambientes, com diferentes participantes e projetos,
	trazendo diferentes pontos de vista para dentro da análise;

	\item \textbf{Apresentar os resultados da pesquisa para os participantes e
	verificar se eles concordam com os resultados gerados}. Garantir que os
	participantes validem os dados encontrados diminui o viés do pesquisador;

	\item \textbf{Prover descrição rica e detalhada sobre o ambiente}. A riqueza
	dos detalhes mostra a qualidade do estudo, além de possibilitar a repetição do
	experimetno por outros pesquisadores;

	\item \textbf{Esclarecer todos os possíveis vieses da pesquisa}. A pesquisa
	deixa claro quais são suas limitações.

\end{itemize}

Em resumo, o principal meio de validação do estudo será o rico detalhamento dos
participantes, dos dados colhidos e instrumentos de colheta, de forma
que qualquer pesquisador interessado em replicar o experimento terá um
arcabouço sólido para comparação \cite{merriam-1988}. A análise de
dados também será relatada em detalhes para que os leitores tenham uma visão
acurada sobre o método utilizado na pesquisa. 
Além disso, essa pesquisa também é acompanhada pelo orientador do pesquisador,
que constantemente valida e discute os pontos levantados nesse planejamento.

%% ------------------------------------------------------------------------- %%
\section{Papel do Pesquisador}
\label{sec:planejamento-papel}

O pesquisador tem como papel fundamental participar do processo de captura dos
dados, bem como seu preparo e interpretação final.
Creswell \cite{creswell}, citando Locke \cite{locke}, lembra
que a contribuição do investigador para o contexto da pesquisa pode ser útil e
positiva ao invés de prejudicial. Além do mais, o pesquisador é responsável por
identificar todos os valores pessoais, pressuposições e vieses desse estudo.

O pesquisador tem formação em Ciência da Computação, e desenvolve software há 8
anos. Pratica TDD diariamente nos últimos 3 anos, e possui profundos
conhecimentos teóricos e práticos sobre orientação a objetos e métodos ágeis.
Além disso, o pesquisador palestrou sobre TDD em eventos da indústria brasileira
de desenvolvimento de software, como a Agile Brazil 2010, o .NET Architects
2010, e o QCON São Paulo 2010. O pesquisador acredita que sua experiência nessas
áreas aumentam sua capacidade de análise dos efeitos de TDD no design de sistemas 
orientados a objetos.

%% ------------------------------------------------------------------------- %%
\section{Problemas Éticos}
\label{sec:planejamento-etica}

Essa pesquisa poderá revelar problemas nas empresas selecionadas, como problemas
no design do software, na qualidade dos seus desenvolvedores, entre outros. 
Por esse motivo, todos os dados colhidos pelo pesquisador serão mantidos em
sigilo e todos os nomes de desenvolvedores e projetos omitidos, conforme acordo 
assinado entre o pesquisador e a empresa.

%% ------------------------------------------------------------------------- %%
\section{Resultados Esperados}
\label{sec:planejamento-resultados-esperados}

O objetivo desta pesquisa é entender, de maneira satisfatória, a influência de 
TDD no design de sistemas orientados a objetos, obtendo o ponto de vista dos 
desenvolvedores que a praticam diariamente. Conforme discutido nos trabalhos 
relacionados, muitos pesquisas já foram feitas, mas poucas discutem os motivos 
e as razões de TDD influenciar no design.

%% ------------------------------------------------------------------------- %%
\section{Estudo piloto}

% TODO: estudo piloto