%% ------------------------------------------------------------------------- %%
\chapter{Planejamento do Experimento}
\label{cap:planejamento}

\section{Propósito do Estudo} 
\label{sec:planejamento-proposito}

O propósito deste estudo de caso é avaliar a influência de Test-Driven Development na qualidade do 
design do software durante toda a sua evolução. Este estudo visa entender como os programadores utilizam o 
feedback dos testes para guiar o design, e dessa maneira influenciar na qualidade final do mesmo. 
Além disso, este estudo busca entender os efeitos da prática na qualidade do design quando executados 
por pessoas com diferentes níveis técnicos dentro de projetos reais.

Para estudar essa influência, este estudo faz uso de métodos qualitativos de pesquisa.

\section{Métodos qualitativos de pesquisa} 
\label{sec:planejamento-qualitativa}

O estudo experimental de engenharia de software sempre foi uma atividade difícil. Uma das razões para isso é o
fator humano, muito presente no processo de desenvolvimento de software, como sugerido por métodos ágeis em geral \cite{AgileManifesto}. 
Dessa maneira, o paradigma de pesquisa analítico não é suficiente para investigar casos reais complexos envolvendo pessoas
e suas interações com a tecnologia \cite{guidelines-case-study}.

Esses problemas já foram levantados por muitos pesquisadores (TODO seaman 1 4 5 6) e finalmente tem-se 
levado em consideração a influência de problemas não-técnicos e a intersecção entre esses problemas e a parte técnica 
dentro da engenharia de software \cite{seaman}.
Apesar disso, o número de estudos empíricos é ainda muito pequeno dentro da área de pesquisa em ciência da computação: 
Sjoberg et al \cite{sjoberg} encontrou apenas 103 experimentos em 5.453 artigos e Ramesh et al. \cite{ramesh} 
identificou menos de 2\% de experimentos envolvendo humanos e apenas 0.16\% estudos em campo dentre 628 artigos.

Uma pesquisa qualitativa é um meio para se explorar e entender a influência que indivíduos ou grupos atribuem a um
problema social ou humano. O processo de pesquisa envolve questões emergentes e procedimentos, dados geralmente colhidos
sob o ponto de vista do participante, com a análise feita de maneira indutiva indo geralmente de um tema específico 
para um tema geral e com o pesquisador fazendo interpretações do significado desses dados. 
O relatório final escrito tem uma estrutura flexível e os pesquisadores que se dedicam a esta forma de pesquisa 
apoiam uma maneira de olhar para a pesquisa que honra o estilo indutivo, o foco em termos individuais, 
e a importância de mostrar a complexidade de uma situação \cite{creswell}. Dados capturados por estudos qualitativos
são representados por palavras e figuras, e não por números (TODO seaman 8). Apesar disso, uma combinação de dados
qualitativos e quantitativos frequentemente provêm um melhor entendimento sobre o fenômeno estudado \cite{seaman}.  

Existem muitas maneiras de se conduzir
um estudo qualitativo, como etnografia, teoria fundamentada nos dados (em inglês, \textit{Grounded Theory}), estudos de caso,
pesquisa de fenômenos (em inglês, \textit{phenomenological research}) e pesquisa narrada \cite{creswell}. 
Baseando-se na definição de Robson \cite{robson}, Yin \cite{yin} e Benbasat et al. \cite{benbasat}, \cite{guidelines-case-study} 
mostra que todas as definições concordam que um estudo de caso investiga um fenônemo contemporâneo no seu contexto. Robson afirma
que um estudo de caso que estressa o uso de múltiplas fontes de evidência, Yin denomina estudo de caso um "inquérito" e
destaca que os limites entre o fenomêno e seu contexto podem não ser claros, enquanto que Benbasat et al. menciona que um estudo
de caso recolhe informações de poucas entidades (como pessoas, grupos ou organizações) e que também é caracterizado pela
falta de controle experimental.

\section{Participantes da pesquisa}
\label{sec:planejamento-participantes}

O critério usado para convite de empresas para participação na pesquisa foram:

\begin{itemize}
	\item Empresa de desenvolvimento de software que utiliza TDD há mais de 1 ano;
	\item Localizada em São Paulo, para facilitar o processo de entrevista e observação do ambiente;
	\item Desenvolvedores com diferentes níveis de experiência em TDD, desde iniciantes até experientes; 
\end{itemize}

De acordo com esses critérios, três empresas do mercado brasileiro de software foram selecionadas. 
Todas elas desenvolvem softwares de pequeno e médio porte (TODO definir) utilizando a plataforma Java para desenvolvimento.

TODO detalhar mais as empresas

\section{Estratégia de pesquisa} 
\label{sec:planejamento-estrategia}

Conduzir uma pesquisa no mundo real implica em um equilíbrio entre nível de controle e grau de realismo. 
Uma situação realística é geralmente complexa e não-determinística, impedindo o entendimento sobre o que
acontece. Por outro lado, aumentar o controle sobre o experimento reduz o grau de realismo, muitos
vezes fazendo com que os reais fatores de influência fiquem fora do escopo do estudo. Estudos de caso são,
por definição, conduzidos no mundo real e por isso têm um alto grau de realismo, mas dessa forma sacrificando
o nível de controle \cite{guidelines-case-study}.

Esta pesquisa, portanto, é um estudo de caso da influência de TDD na qualidade do design do software no 
contexto de três empresas brasileiras de desenvolvimento de software. 
Podemos classifica-lá como explanatória \cite{robson}, já que seu principal objetivo é
procurar por uma explicação para a influência que TDD exerce sobre a qualidade do design.

As sub-seções abaixo mostram como os dados serão colhidos.

\subsection{Entrevistas}
\label{sec:planejamento-estrategia-entrevistas}

O primeiro passo da pesquisa é uma série de entrevistas com os desenvolvedores das empresas participantes. 
As entrevistas serão semi-estruturadas, o que significa que o pesquisador possui um roteiro de perguntas a serem respondidas, mas pode eventualmente
alterar o rumo da entrevista de acordo com as respostas dados pelos entrevistados \cite{guidelines-case-study}. 
Além disso, todas as perguntas serão abertas, permitindo que o desenvolvedor dê uma resposta ampla sobre o assunto.

O principal objetivo dessas entrevistas é capturar informações sobre:

\begin{itemize}
	\item a visão do desenvolvedor sobre a influência de TDD no design
	\item como o desenvolvedor utiliza o feedback dos testes para guiar o design
	\item como o desenvolvedor vê a influência que a utilização de TDD junto com outras práticas ágeis, como programação pareada, tem sobre o design
	\item a relação entre a experiência do programador e os efeitos de TDD na qualidade do design
	\item os efeitos de TDD sobre a gerênciamento de depêndencias em sistemas orientados a objetos
	\item a relação entre TDD e boas práticas de orientação a objetos, com os princípios SOLID \cite{bob-martin}
	\item os efeitos de TDD sobre o acoplamento e a coesão geral das classes do sistema
	\item A visão do programador sobre a simplicidade que TDD agrega ao processo
\end{itemize} 

TODO: Mapear perguntas com os objetivos acima

Todas as entrevistas serão gravadas para que possam ser consultadas pelo pesquisador em um momento futuro. Além disso,
as pesquisas serão sempre feitas por dois observadores, possibilitando que ambos tomem notas separadas e comparem
suas anotações. O objetivo é diminuir o viés do processo.

As entrevistas serão feitas em dias diferentes. Ao final de cada dia, os pesquisadores se reunirão para discutir
sobre os pontos levantados no dia corrente e para sugerir melhorias no instrumento de entrevista. 

A versão final do roteiro de entrevista pode ser encontrado nos apêndices desse trabalho.

\subsection{Observação do mundo real}
\label{sec:planejamento-estrategia-observacao}

O pesquisador observará os desenvolvedores em seu ambiente natural para investigar como a 
prática é conduzida no dia-a-dia pelos desenvolvedores. Através da observação, o pesquisador poderá avaliar se
a respostas dadas na entrevista conferem com o que eles realmente realizam durante o dia de trabalho. Além disso, o pesquisador
poderá fazer perguntas sobre determinada situação que aconteceu naquele momento.

Os objetivos das observações são:

\begin{itemize}
	\item descobrir como os programadores praticam TDD (Aniche e Gerosa \cite{aniche-mistakes} mostraram que os desenvolvores tendem a cometer desvios durante o processo).
	\item descobrir como eles realmente utilizam o feedback dos testes para guiar o design
	\item verificar se outras práticas ágeis, como programação pareada, influenciam a prática de TDD
	\item outros maneiras na qual a utilização de TDD influenciou no design
\end{itemize}

O pesquisador fará anotações sobre o que achar relevante. Ele anotará também o dia e hora que o fato ocorreu, bem como os desenvolvedores
que estavam participando naquele momento.

\subsection{Engenharia reversa}
\label{sec:planejamento-estrategia-engenharia-reversa}

O pesquisador utilizará técnicas de engenharia reversa \cite{eng-reversa}, como: "observar o código todo em uma hora",
aonde a ideia é fazer com que o pesquisador leia o código por uma hora e detalhe os pontos claros e os pontos não-claros do código; 
"fazer uma visita guiada ao código", aonde um dos desenvolvedores da aplicação apresenta trechos de código e tenta 
explicar o motivo, vantagens e desvantagens daquela decisão de design.

Os objetivos da utilização dessas técnicas são:

\begin{itemize}
	\item fazer com que o pesquisador tenha um conhecimento mais profundo sobre o código na qual os desenvolvedores trabalham no dia-a-dia e agregar mais informação ao pesquisador para o processo de análise
	\item gerar questões sobre possíveis problemas de design e tentar entender o motivo dos mesmos.
\end{itemize}

Para isso, o código-fonte completo das aplicações será disponibilizado para uso do pesquisador.

\subsection{Métricas de Código}
\label{sec:planejamento-estrategia-metricas}

Conhecidas métricas de qualidade de design, como acoplamento e coesão, serão calculadas para os códigos em que os desenvolvedores
atuam. Além disso, \cite{lanza} propõe uma maneira para se descobrir maus cheiros de design através de métricas baseadas nas métricas
já apresentadas.

Dessa maneira, os dados colhidos de maneira qualitativa nos passos anteriores poderão ser triangularizados com os valores gerados
por essas métricas.

Os objetivos de calcular essas métricas é:

\begin{itemize}
	\item verificar o comportamento das métricas de acoplamento;
	\item verificar o comportamento das métricas de coesão;
	\item verificar o comportamento das métricas de simplicidade;
	\item verificar o comportamento das métricas de instabilidade e abstração \cite{bobmartin-oodmetrics};
	\item verificar o comportamento das métricas de testabilidade;
	\item verificar a presença de heurísticas de design \cite{lanza};
\end{itemize}

Alguns dos sistemas foram desenvolvidos utilizando-se uma estratégia de desenvolvimento mista, aonde TDD foi praticado em apenas algumas partes
do código. Durante a medição, os valores obtidos das partes do sistema que foram desenvolvidas utilizando-se TDD serão separados dos valores 
dos trechos de código desenvolvidos sem TDD. Isso possibilitará uma comparação dessas métricas, 
dentro do mesmo sistema, em pontos em que TDD foi praticado e em pontos aonde TDD não foi praticado.

Para o cálculo das métricas, serão utilizadas ferramentas como JDepend \cite{jdepend}, Eclipse Metrics \cite{eclipse-metrics} e (TODO ferramenta gall),
todas disponíveis para uso gratuito.

\section{Análise dos dados}
\label{sec:planejamento-analise}

O objetivo básico de uma análise de dados qualitativa é derivar conclusões a partir dos dados, mantendo uma clara cadeia de evidências.
Isso significa que o leitor deve ser capaz de seguir a derivação de resultados e conclusões pelos dados coletados \cite{yin}.

O processo de análise, baseado em \cite{creswell}, que está representado na figura (TODO fig), ilustra o processo de análise que será utilizado nessa pesquisa. Apesar de parecer uma abordagem em cascata, ele é na prática interativo, já que os passos são interconectados. Cada um dos níveis é detalhado a seguir:

\begin{itemize}

	\item \textit{Organização e Preparação dos Dados}: Transcrição das entrevistas e das notas geradas durante todo o processo de entrevistas; transcrição das notas das observações; transcrição das informações obtidas no processo de engenharia reversa; cálculo das métricas em diferentes momentos do código-fonte dos projetos
	
	\item \textit{Leitura dos dados}: Leitura de todos os dados gerados até o momento. Nesse momento há uma reflexão sobre os principais pontos e opiniões colhidos em todas as fontes de dados.
	
	\item \textit{Codificação e Agrupamento por Temas}:	Codificação é o ato de organizar e classificar os dados em pequenas categorias ou segmentos de textos antes de trazer qualquer significado àquela informação \cite{rossman}. Esse processo será feito por dois pesquisadores e os resultados serão confrontados. Após o processo de codificação, agrupa-se os dados codificados em temas que possuem uma granularidade maior. Esses temas são aqueles que aparecerão como maiores contribuições da pesquisa qualitativa.
	
	\item \textit{Interpretação do resultado}: O pesquisador interpreta os dados, basendo-se tanto na sua experiência pessoal com o assunto quanto com dados retirados da literatura.

\end{itemize}
 
%TODO figura

\section{Validade e Confiabilidade do Estudo}
\label{sec:planejamento-validacao}

Os termos validade e confiabilidade em estudos qualitativos não possuem o mesmo significado dos estudos quantitativos. 
Confiabilidade em um estudo qualitativo indica que a abordadem do pesquisador é consistente entre diferentes pesquisadores
e diferentes projetos. Gibbs \cite{gibbs} sugere alguns procedimentos, dentre eles:

\begin{itemize}
	\item Checar as transcrições para garantir que não foram cometidos erros óbvios durante a transcrição;
	\item Certificar que não há um desvio na definição dos códigos ou no significado dos códigos durante o processo de codificação. Isso pode ser alcançado comparando os dados com os códigos e com as anotações constantemente;
	\item Fazer uma comparação cruzada entre os códigos criados pelos pesquisadores
\end{itemize}

O processo de codificação será feito por dois pesquisadores diferentes, que irão constantemente checar e comparar os resultados
obtidos. Além disso, as entrevistas serão gravadas e o pesquisador poderá revisar a transcrição durante todo a pesquisa.

Já o termo validade em um estudo qualitativo significa que o pesquisador se preocupou com a exatidão das informações
geradas pela pesquisa. Isso pode ser verificado por alguns procedimentos, dentre eles:

\begin{itemize}
	\item Triangularização diferentes fontes de dados;
	\item Apresentar os resultados da pesquisa para os participantes e verificar se eles concordam com os resultados gerados;
	\item Prover descrição rica e detalhada sobre o ambiente;
	\item Esclarecer todos os possíveis vieses da pesquisa;
	\item Auditoria externa para revisar o projeto;
\end{itemize}

Os dados colhidos pela entrevista serão triangularizados com os dados obtidos das métricas de código e design. 
Como exemplo, os entrevistados serão perguntados sobre os possíveis efeitos de TDD no design, como redução do acoplamento ou
aumento da coesão, e esses dados serão comparados com as métricas de acoplamento e coesão, abstração e instabilidade 
calculadas sobre o código.
Além disso, os dados observados no ambiente dos programadores também servirão para aumentar a validade dos 
dados colhidos nas entrevistas.

\section{Papel do pesquisador}
\label{sec:planejamento-papel}

O pesquisador tem como papel fundamental participar do processo de captura de todos os dados comentados
nas seções acima. No processo de entrevista, o pesquisador fará o papel de entrevistador principal, enquanto outro
pesquisador convidado fará o papel de observador da entrevista.

O pesquisador também observará o ambiente real dos desenvolvedores nas três empresas selecionadas
para a pesquisa. A experiência e vivência do pesquisador na área de desenvolvimento de software 
nesse ponto será importante para ajudar na compreensão dos ambientes. 

Creswell \cite{creswell}, citando Locke \cite{locke}, lembra
que a contribuição do investigador para o contexto da pesquisa pode ser útil e positiva ao invés de prejudicial.
Além do mais, o pesquisador é responsável por identificar todos os valores pessoais, pressuposições e vieses desse estudo.

\section{Problemas éticos}
\label{sec:planejamento-etica}

Essa pesquisa poderá revelar problemas nas empresas selecionadas, como problemas no design do software, na
qualidade dos seus desenvolvedores, entre outros. Por esse motivo, todos os dados colhidos pelo pesquisador
serão mantidos em sigilo e todos os nomes de desenvolvedores e projetos omitidos, conforme acordo assinado entre o pesquisador e a empresa.


\section{Resultados esperados}
\label{sec:planejamento-resultados-esperados}

O objetivo desta pesquisa é entender, de maneira satisfatória, o efeito da influência de TDD sobre a qualidade
do design do software, obtendo o ponto de vista dos desenvolvedores que a praticam diariamente. 
Conforme discutido nos trabalhos relacionados, muitos pesquisas já foram feitas, 
mas poucas discutem os motivos e as razões de TDD influenciar no design.

