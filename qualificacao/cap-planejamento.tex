%% ------------------------------------------------------------------------- %%
\chapter{Planejamento do Experimento}
\label{cap:planejamento}

Conforme discutido no capítulo \ref{cap:trabalhos-relacionados}, muitos 
trabalhos avaliaram TDD e alguns deles perceberam uma melhora (mas muitas vezes
sem nenhuma significância estatística) no design de classes, como um menor 
acoplamento, uma maior coesão, e até mesmo mais simplicidade nas classes quando
desenvolvidas utilizando-se a prática. 

Grande parte deles estão interessados em avaliar quais os efeitos da prática
no código final produzido, mas poucos estudos tentam entender como TDD e a
prática de escrever o teste antes do código real realmente guiam o programador 
em direção à essas melhorias.

Para entender como TDD influencia o desenvolvedor durante o desenvolvimento,
este trabalho propõe estudos de caso em 3 empresas de desenvolvimento de
software brasileiras, utilizando métodos qualitativos de pesquisa.

Este capítulo detalha o planejamento do experimento, bem como o processo de
análise dos dados colhidos.

%% ------------------------------------------------------------------------- %%
\section{Métodos qualitativos de pesquisa} 
\label{sec:planejamento-qualitativa}

Conduzir um estudo experimental em engenharia de software sempre foi uma
atividade difícil. Uma das razões para isso é o fator humano, muito presente 
no processo de desenvolvimento de software, como sugerido por métodos ágeis  em
geral \cite{AgileManifesto}. Dessa maneira, o paradigma de pesquisa analítico 
não é suficiente para investigar casos reais complexos envolvendo pessoas e 
suas interações com a tecnologia \cite{guidelines-case-study}.

Conforme levantado por Seaman, esses problemas já foram levantados por muitos
pesquisadores e finalmente tem-se levado em consideração a influência de
problemas não-técnicos e a intersecção entre esses problemas e a parte técnica
dentro da engenharia de software \cite{seaman}. 

Apesar disso, o número de estudos empíricos é ainda muito pequeno dentro da área
de pesquisa em ciência da computação: Sjoberg et al encontrou apenas 103
experimentos em 5.453 artigos \cite{sjoberg} e Ramesh et al. identificou menos
de 2\% de experimentos envolvendo humanos e apenas 0.16\% estudos em campo 
dentre 628 artigos \cite{ramesh} .

Uma pesquisa qualitativa é um meio para se explorar e entender a influência que 
indivíduos ou grupos atribuem a um problema social ou humano. O processo de
pesquisa envolve questões emergentes e procedimentos, dados geralmente colhidos
sob o ponto de vista do participante, com a análise feita de maneira indutiva
indo  geralmente de um tema específico para um tema geral e com o pesquisador
fazendo interpretações do significado desses dados. Dados capturados por estudos
qualitativos são representados por palavras e figuras, e não por números.

O relatório final escrito tem uma estrutura flexível e os pesquisadores que se
dedicam a esta forma de pesquisa apoiam uma maneira de olhar para a pesquisa que
honra o estilo indutivo, o foco em termos individuais, e a importância de mostrar a 
complexidade de uma situação \cite{creswell}. 

Existem muitas maneiras de se conduzir um estudo qualitativo, como etnografia,
teoria fundamentada nos dados (em inglês, \textit{Grounded Theory}), estudos de 
caso, pesquisa de fenômenos (em inglês, \textit{phenomenological research}) e 
pesquisa narrada \cite{creswell}. 

Baseando-se na definição de Robson \cite{robson}, Yin \cite{yin} e Benbasat et
al. \cite{benbasat}, Runeson e Host mostram que todas as definições concordam
que um estudo de caso investiga um fenônemo contemporâneo no seu contexto. 
Robson afirma que um estudo de caso estressa o uso de múltiplas fontes de
evidência, Yin denomina estudo de caso um "inquérito" e destaca que os limites 
entre o fenomêno e seu contexto podem não ser claros, enquanto que Benbasat et
al. menciona que um estudo de caso recolhe informações de poucas entidades 
(como pessoas, grupos ou organizações) e que também é caracterizado pela  falta
de controle experimental \cite{guidelines-case-study}.

Baseando-se no fato de que o processo de desenvolvimento de software envolve 
diversos fatores humanos e é totalmente sensível ao contexto em que ele está 
inserido, este trabalho é composto por um conjunto de estudos de caso, onde são 
utilizados métodos baseados em teoria fundamentada nos dados para colheta e
análise dos dados.

%% ------------------------------------------------------------------------- %%
\section{Participantes da pesquisa}
\label{sec:planejamento-participantes}

Para a condução do estudo de caso, algumas empresas do mercado de software
brasileiro foram selecionadas de acordo com alguns critérios.
Os critérios que foram levados em consideração no momento de convidar empresas a
participarem da pesquisa foram:

\begin{itemize}
	\item Utilizem de TDD há mais de 1 ano;

	\item Sejam localizadas em São Paulo (SP), para facilitar o processo de
	entrevista e observação do ambiente;

	\item Possuam desenvolvedores com diferentes níveis de experiência em TDD e em 
	Orientação a Objetos, desde iniciantes até experientes;

	\item Possuam projetos que utilizem a tecnologia Java e sejam baseados na web;

	\item Utilizem alguma metodologia ágil como processo de desenvolvimento de
	software;
\end{itemize}

Ao final, três empresas foram selecionadas. Todas elas desenvolvem softwares de
pequeno e médio porte (e por isso entenda-se software entre 100 a 500 mil linhas
de código), utilizando a plataforma Java para desenvolvimento. Elas também 
possuem profissionais com os mais variados níveis de experiência na área 
trabalhando juntos no mesmo time.

A repetição do experimento em diferentes empresas com diferentes projetos,
equipes e contextos serve como estratégia de validação dos resultados
encontrados.

%% ------------------------------------------------------------------------- %%
\section{Estratégia de pesquisa} 
\label{sec:planejamento-estrategia}

Conduzir uma pesquisa no mundo real implica em um equilíbrio entre nível de
controle e grau de realismo. Uma situação realística é geralmente complexa e 
não-determinística, impedindo o entendimento sobre o que acontece. Por outro
lado, aumentar o controle sobre o experimento reduz o grau de realismo, muitos
vezes fazendo com que os reais fatores de influência fiquem fora do escopo do 
estudo. Estudos de caso são, por definição, conduzidos no mundo real e por isso 
têm um alto grau de realismo, mas dessa forma sacrificando o nível de  controle
\cite{guidelines-case-study}.

Optando por um maior realismo, os dados serão coletados principalmente através 
de entrevistas com desenvolvedores que praticam TDD no dia-a-dia de trabalho e
que podem avaliar com propriedade o efeito da prática sobre o design orientado 
a objetos dos seus projetos. Além disso, o ambiente será observado pelo
pesquisador afim de capturar informações do cotidiano relevantes para a pesquisa. 

As sub-seções abaixo detalham como os dados serão colhidos em cada etapa da pesquisa.

\subsection{Entrevistas}
\label{sec:planejamento-estrategia-entrevistas}

O primeiro passo da pesquisa é uma série de entrevistas com os desenvolvedores 
das empresas participantes. As entrevistas serão semi-estruturadas, o que
significa que o pesquisador possui um roteiro de perguntas a serem respondidas, 
mas pode eventualmente alterar o rumo da entrevista de acordo com as respostas 
dadas pelos entrevistados \cite{guidelines-case-study}. 

Além disso, todas as perguntas serão abertas, permitindo que o desenvolvedor dê
uma resposta ampla sobre o assunto. O principal objetivo dessas entrevistas é
capturar informações sobre:

\begin{itemize}
	\item Caracterizar a experiência do programador com desenvolvimento de
	software, boas práticas de desenvolvimento e TDD;

	\item A visão do desenvolvedor sobre a influência de TDD no design;

	\item Como o desenvolvedor utiliza o feedback dos testes para guiar o design;

	\item Como o desenvolvedor vê a influência que a utilização de TDD junto com 
	outras práticas ágeis, como programação pareada, tem sobre o design;

	\item A relação entre a experiência do programador e os efeitos de TDD na 
	qualidade do design;

	\item Os efeitos de TDD sobre a gerênciamento de depêndencias em sistemas 
	orientados a objetos;

	\item A relação entre TDD e boas práticas de orientação a objetos;

	\item Os efeitos de TDD sobre o acoplamento e a coesão geral das classes  do
	sistema;

	\item A visão do programador sobre a simplicidade que TDD agrega ao processo;
\end{itemize} 

Todas as entrevistas serão gravadas para que possam ser consultadas pelo
pesquisador em momentos futuros. Além disso, as pesquisas serão sempre feitas
por dois observadores, possibilitando que ambos tomem notas separadas e comparem
suas anotações. O objetivo é diminuir o viés do processo.

As entrevistas serão feitas em dias diferentes. Ao final de cada dia, os
pesquisadores se reunirão para discutir sobre os pontos levantados no dia
corrente, com o objetivo de sugerir melhorias no instrumento de entrevista. A
versão final do roteiro de entrevista pode ser encontrado no apêndice 
\ref{ape:entrevista} deste trabalho.

Além disso, o processo de codificação \cite{seaman} será feito por ambos os
pesquisados separadamente, e só então ambos juntarão seus resultados, após uma 
série de comparações e discussões sobre as diferenças encontradas.

\subsection{Observação do mundo real}
\label{sec:planejamento-estrategia-observacao}

O pesquisador observará os desenvolvedores em seu ambiente natural para
investigar como a prática é conduzida no dia-a-dia pelos desenvolvedores. O
período de observação será pré-agendado com os líderes de cada equipe para que
o pesquisador esteja observando o ambiente em um dia em que atividades de 
design provavelmente acontecerão (o que significa que o pesquisador não estará 
presente em momentos de  correções de erros ou criação de interfaces de usuário,
por exemplo). O tempo de observação é indeterminado, sendo decidido pelo 
pesquisador se existe a real necessidade de continuar a observação.

Através da observação, o pesquisador poderá avaliar se
a respostas dadas na entrevista conferem com o que eles realmente realizam 
durante o dia de trabalho. Além disso, o pesquisador poderá fazer perguntas 
sobre determinada situação que aconteceu naquele momento.

Os objetivos das observações são:

\begin{itemize}
	\item Descobrir como os programadores praticam TDD (Aniche e Gerosa 
	\cite{aniche-mistakes} mostraram que os desenvolvores tendem a cometer desvios
	durante o processo);

	\item Descobrir como eles realmente utilizam o feedback dos testes para guiar
	o design;

	\item Verificar se outras práticas ágeis, como programação pareada, 
	influenciam a prática de TDD;

	\item Outras maneiras na qual a utilização de TDD influenciou no design;
\end{itemize}

O pesquisador fará anotações sobre o que achar relevante. Ele anotará também o 
dia e hora que o fato ocorreu, bem como os desenvolvedores que estavam
participando naquele momento, para que possa eventualmente checar com as
respostas dadas na entrevista.

%% ------------------------------------------------------------------------- %%
\section{Análise dos dados}
\label{sec:planejamento-analise}

O objetivo básico de uma análise de dados qualitativa é derivar conclusões a
partir dos dados, mantendo uma clara cadeia de evidências. Isso significa que  o
leitor deve ser capaz de seguir a derivação de resultados e conclusões pelos 
dados coletados \cite{yin}.

O processo de análise, baseado em \cite{creswell}, que está representado na
figura (TODO fig), ilustra o processo de análise que será utilizado nessa
pesquisa. Apesar de parecer uma abordagem em cascata, ele é na prática 
interativo, já que os passos são interconectados. Cada um dos níveis é 
detalhado a seguir:

\begin{itemize}

	\item \textit{Organização e Preparação dos Dados}: Transcrição das entrevistas 
	e das notas geradas durante todo o processo de entrevistas; transcrição das
	notas das observações;
	
	\item \textit{Leitura dos dados}: Leitura de todos os dados gerados até o
	momento. Nesse momento há uma reflexão sobre os principais pontos e opiniões 
	colhidos em todas as fontes de dados;
	
	\item \textit{Codificação e Agrupamento por Temas}:	Codificação é o ato de
	organizar e classificar os dados em pequenas categorias ou segmentos de textos 
	antes de trazer qualquer significado àquela informação \cite{rossman}. Esse 
	processo será feito por dois pesquisadores e os resultados serão confrontados. 
	Após o processo de codificação, agrupa-se os dados codificados em temas que 
	possuem uma granularidade maior. Esses temas são aqueles que aparecerão como 
	maiores contribuições da pesquisa qualitativa.
	
	\item \textit{Interpretação do resultado}: O pesquisador interpreta os dados, 
	basendo-se tanto na sua experiência pessoal com o assunto quanto com dados 
	retirados da literatura.

\end{itemize}
 
 %% ------------------------------------------------------------------------- %%
\section{Validade e Confiabilidade do Estudo}
\label{sec:planejamento-validacao}

Os termos validade e confiabilidade em estudos qualitativos não possuem o mesmo 
significado dos estudos quantitativos. Confiabilidade em um estudo qualitativo 
indica que a abordadem do pesquisador é consistente entre diferentes
pesquisadores e diferentes projetos. Gibbs \cite{gibbs} sugere alguns
procedimentos,  dentre eles:

\begin{itemize}
	\item Checar as transcrições para garantir que não foram cometidos erros 
	óbvios durante a transcrição;

	\item Certificar que não há um desvio na definição dos códigos ou no
	significado dos  códigos durante o processo de codificação. Isso pode ser
	alcançado comparando os dados com os códigos e com as anotações constantemente;

	\item Fazer uma comparação cruzada entre os códigos criados pelos
	pesquisadores;
\end{itemize}

O processo de codificação será feito por dois pesquisadores diferentes, que 
irão constantemente checar e comparar os resultados obtidos. Além disso, as 
entrevistas serão gravadas e o pesquisador poderá revisar a transcrição  durante
todo a pesquisa.

Já o termo validade em um estudo qualitativo significa que o pesquisador  se
preocupou com a exatidão das informações geradas pela pesquisa. Isso pode ser 
verificado por alguns procedimentos, dentre eles:

\begin{itemize}
	\item Triangularização diferentes fontes de dados;

	\item Apresentar os resultados da pesquisa para os participantes  e verificar
	se eles concordam com os resultados gerados;

	\item Prover descrição rica e detalhada sobre o ambiente;

	\item Esclarecer todos os possíveis vieses da pesquisa;

	\item Auditoria externa para revisar o projeto;
\end{itemize}

Os dados colhidos pela entrevista serão triangularizados com os dados
verificados durante a observação do ambiente real de trabalho, na
tentativa de aumentar a validade dos dados colhidos nas entrevistas.

%% ------------------------------------------------------------------------- %%
\section{Papel do Pesquisador}
\label{sec:planejamento-papel}

O pesquisador tem como papel fundamental participar do processo de  captura de
todos os dados comentados nas seções acima. No processo de entrevista, o 
pesquisador fará o papel de entrevistador principal, enquanto outro  pesquisador
convidado fará o papel de observador da entrevista.

O pesquisador também observará o ambiente real dos desenvolvedores nas três 
empresas selecionadas para a pesquisa. A experiência e vivência do pesquisador 
na área de desenvolvimento de software nesse ponto será importante para  ajudar
na compreensão dos ambientes.

Creswell \cite{creswell}, citando Locke \cite{locke}, lembra
que a contribuição do investigador para o contexto da pesquisa pode ser útil e
positiva ao invés de prejudicial. Além do mais, o pesquisador é responsável por
identificar todos os valores pessoais, pressuposições e vieses desse estudo.

%% ------------------------------------------------------------------------- %%
\section{Problemas Éticos}
\label{sec:planejamento-etica}

Essa pesquisa poderá revelar problemas nas empresas selecionadas, como problemas
no design do software, na qualidade dos seus desenvolvedores, entre outros. 

Por esse motivo, todos os dados colhidos pelo pesquisador serão mantidos em
sigilo e todos os nomes de desenvolvedores e projetos omitidos, conforme acordo 
assinado entre o pesquisador e a empresa.

%% ------------------------------------------------------------------------- %%
\section{Resultados Esperados}
\label{sec:planejamento-resultados-esperados}

O objetivo desta pesquisa é entender, de maneira satisfatória, a influência de 
TDD no design de sistemas orientados a objetos, obtendo o ponto de vista dos 
desenvolvedores que a praticam diariamente. Conforme discutido nos trabalhos 
relacionados, muitos pesquisas já foram feitas, mas poucas discutem os motivos 
e as razões de TDD influenciar no design.

