%% ------------------------------------------------------------------------- %%
\chapter{Planejamento e Execução do Estudo} 
\label{cap:qualitativo-planejamento}

Conduzir um estudo experimental em engenharia de software sempre foi uma
atividade difícil. Uma das razões para isso é o fator humano, muito presente 
no processo de desenvolvimento de software, como sugerido por métodos ágeis  em
geral \cite{AgileManifesto}. Dessa maneira, o paradigma de pesquisa analítico 
não é suficiente para investigar casos reais complexos envolvendo pessoas e 
suas interações com a tecnologia \cite{guidelines-case-study}.

Esses problemas já foram levantados por muitos pesquisadores, e hoje tem-se 
considerado melhor a influência de problemas não-técnicos e a intersecção entre 
eles e a parte técnica dentro da engenharia de software \cite{seaman}.
Apesar disso, o número de estudos empíricos é ainda muito pequeno dentro da área
de pesquisa em ciência da computação: Sjoberg \textit{et al.} \cite{sjoberg} encontraram
apenas 103 experimentos em 5.453 artigos, e Ramesh \textit{et al.} \cite{ramesh}
identificaram menos de 2\% de experimentos envolvendo humanos e apenas 0.16\% 
estudos em campo dentre 628 artigos.

Uma pesquisa qualitativa é um meio para se explorar e entender a influência que 
indivíduos ou grupos atribuem a um problema social ou humano. O processo de
pesquisa envolve questões emergentes e procedimentos, dados geralmente colhidos
sob o ponto de vista do participante, com a análise feita de maneira indutiva
indo geralmente de um tema específico para um tema geral e com o pesquisador
fazendo interpretações do significado desses dados. Dados capturados por estudos
qualitativos são representados por palavras e figuras, e não por números.
O relatório final tem uma estrutura flexível e os pesquisadores que se
dedicam a esta forma de pesquisa apoiam uma maneira de olhar para a pesquisa que
honra o estilo indutivo, o foco em termos individuais e a importância de mostrar a 
complexidade de uma situação \cite{creswell}. 

Conforme discutido na Seção \ref{cap:trabalhos-relacionados}, muitos 
trabalhos avaliaram TDD, e alguns deles relatam inclusive uma melhora
no projeto de classes, como um menor acoplamento, uma maior coesão, e até mesmo
mais simplicidade. 
Grande parte deles focam nos efeitos da prática no código final, mas poucos 
estudos tentam entender a possível influência da experiência
nos resultados encontrados, e como TDD e a prática de escrever o teste 
antes do código real realmente guiam o programador 
em direção a essas melhorias.

Para entendê-las, neste trabalho optamos por um estudo exploratório
essencialmente qualitativo,
no qual participantes foram convidados a resolver exercícios 
pré-preparados utilizando TDD e, a partir 
dos dados colhidos nessa primeira parte, detalhes sobre como a prática influenciou as 
decisões de projeto de classes foram extraídos dos participantes através de 
entrevistas.
Este capítulo detalha o planejamento do estudo, bem como o processo 
de análise dos dados colhidos.


%% ------------------------------------------------------------------------- %%
\section{Características de pesquisas qualitativas}

Métodos qualitativos de pesquisa possuem diversas características, que juntas fazem
com que a pesquisa se torne rica em detalhes. Creswell \cite{creswell} lista
algumas delas:

\begin{enumerate}
  
  \item \textbf{Pesquisador como instrumento chave de pesquisa}. O pesquisador
  tem papel fundamental no processo, visto que ele é o responsável pela captura dos
  dados, por meio da examinação de documentos, entrevistas ou observações feitas
  no mundo real. Pesquisadores tendem a não utilizar questionários ou
  instrumentos desenvolvidos por outros pesquisadores;
  
  \item \textbf{Múltiplas fontes de dados}. Pesquisas qualitativas geralmente
  colhem informações de múltiplas fontes de dados, como entrevistas,
  observações e documentos;
  
  \item \textbf{Análise dos dados indutiva}. Os dados são analisados de dentro
  para fora, por meio da categorização dos mesmos em unidades de informação cada
  vez mais abstratas. Esse processo indutivo gera diversas idas e vindas entre
  os temas encontrados e a base de dados, até o momento em que os pesquisadores
  estabeleçam um conjunto extensivo de temas;
  
  \item \textbf{Visão do participante}. Trabalhos qualitativos focam na visão do
  participante sobre o objeto em estudo, e não na visão que o pesquisador ou a
  literatura tem a respeito do mesmo;
  
  \item \textbf{Projeto emergente}. O processo de pesquisa qualitativa é
  emergente. Isso significa que o processo não deve ser completamente descrito
  desde o começo, mas sim modificado de acordo com o início da coleta dos dados. 
  A ideia chave por trás da pesquisa qualitativa é
  aprender sobre o problema com os participantes e direcionar a pesquisa para
  obter aquela informação;
  
  \item \textbf{Interpretativa}. Pesquisadores fazem uma interpretação daquilo
  que veem, ouvem e entendem. As interpretações do pesquisador não podem ser
  separadas do seu conhecimento, história, contexto e entendimentos anteriores
  do problema. Ao final do relatório da pesquisa, leitores também fazem suas
  críticas, oferecendo ainda novas interpretações para o estudo. Com os
  leitores, participantes e pesquisadores fazendo interpretações, múltiplas
  visões do problema podem emergir.
  
\end{enumerate} 

\subsection{Estudos mistos}

Abordagens que combinam tanto métodos qualitativos quanto quantitativos são conhecidos por métodos mistos. 
É mais do que apenas coletar e analisar ambos tipos de dados; é também fazer interpretações que unam ambos
resultados encontrados, de forma que a força do estudo seja maior do que se ambos os métodos fossem
usados separadamente \cite{creswell}.

Possíveis diferentes abordagens podem ser levadas em conta em estudos mistos. Eles podem começar com estudos
qualitativos de exploração, seguidos de um estudo quantitativo com uma população maior de forma a
generalizar os resultados para a população. Alternativamente, o estudo pode começar com um estudo quantitativo
no qual uma teoria ou conceito é testado, seguido de um estudo qualitativo, envolvendo exploração detalhada
de alguns casos ou indivíduos \cite{creswell}.

Reconhecendo que todos possuem limitações, pesquisadores perceberam que um viés de um método pode ser reduzido
por um outro método, e para isso devem sempre tentar triangular conjuntos de dados diferentes.
Por exemplo, os resultados de um método podem ajudar a identificar participantes a serem estudados por um outro método.
Dados qualitativos e quantitativos podem ser unidos em um único conjunto de dados ou seus resultados usados
lado a lado para que um reforçe as ideias do outro \cite{creswell}.

%% ------------------------------------------------------------------------- %%
\section{Questões de pesquisa}

Conforme já mencionado na introdução, 
o objetivo principal deste estudo é \textbf{entender a relação da prática de TDD 
e as decisões de projeto de classes tomadas pelo programador durante o processo de 
projeto de sistemas orientados a objetos}.
Para compreendê-la, tenta-se responder às questões listadas
abaixo:

\begin{enumerate}

	\item Qual a influência de TDD no projeto de classes?

	\item Qual a relação entre TDD e as tomadas de decisões de projeto de classes
	feitas por um desenvolvedor?

	\item Como a prática de TDD influencia o programador no processo de  
	projeto de classes, do ponto de vista do acoplamento, coesão e complexidade?

\end{enumerate}

%% ------------------------------------------------------------------------- %%
\section{Projeto da pesquisa}

Participantes de diferentes empresas de desenvolvimento de software do mercado
brasileiro foram selecionados. Todos eles foram solicitados a resolver 
alguns problemas utilizando Java, dentro de um período de tempo limitado. 
Os participantes utilizaram TDD em um problema, e não o utilizaram
no outro. Os problemas resolvidos bem como em qual deles o participante
deveria utilizar TDD foram aleatorizados, a fim de diminuir o problema do aprendizado.

Todas as implementações feitas foram salvas, para posterior
cálculo de métricas de código. Ao final do exercício, todos participantes
também responderam um questionário, sobre seu desempenho na resolução dos problemas.
Em seguida, uma análise filtrou os candidatos
mais interessantes, que foram posteriormente entrevistados. 
Todos os dados gerados no processo, 
como código produzido e as entrevistas, foram analisados.

As sub-seções a seguir detalham cada um dos pontos levantados. A ordem das
sub-seções também representam a ordem de execução dos passos do estudo, uma vez
que o executamos de maneira sequêncial.
Um roteiro mais resumido e pronto para ser utilizado em replicações também pode
ser encontrado no Apêndice \ref{ape:protocolo-resumido}.

%% ------------------------------------------------------------------------- %%
\subsection{Participantes da pesquisa}
\label{sec:planejamento-participantes}

Desenvolvedores atuantes no mercado de 
software brasileiro foram selecionados para participarem da pesquisa.
Dada a dificuldade de se
encontrar desenvolvedores e empresas interessadas em participar de
estudos científicos, todos os que se candidataram, foram utilizados no estudo.

Para análise futura, os participantes foram avaliados de acordo com
certos critérios:

\begin{itemize}
	\item \textbf{Experiência em TDD.} Eles foram categorizados em programadores inexperientes 
	em TDD (pouco conhecimento teórico e prático) e programadores com experiência
	em TDD (praticantes frequentes há no mínimo 3 anos).
	
	\item \textbf{Experiência em desenvolvimento de software.} Participantes podem ser
	experientes (com no mínimo 3 anos de desenvolvimento e bons conhecimentos em orientação a objetos) ou 
	inexperientes (com no máximo 1 ano de desenvolvimento e pouco conhecimento de orientação a objetos).

	\item \textbf{Conhecimentos em Java.} Nível de conhecimento na linguagem Java.
	
	\item \textbf{Conhecimentos em Testes de Unidade.} Conhecimento em testes
	de unidade e na prática de TDD.

\end{itemize}

Esses pontos foram avaliados por meio de um questionário, 
respondido por todos os participantes antes do início do estudo. 
Este questionário, além de perguntar qual a experiência
do participante (de maneira quantitativa, em anos), 
continha questões nas quais o participante
podia falar sobre sua expêriencia em projeto orientado a objetos,
Java e TDD de forma mais aberta.
Uma cópia deste questionário
pode ser encontrada no Apêndice \ref{ape:questionario-inicio}.

O objetivo de trazer participantes com as mais diferentes experiências em desenvolvimento
de software e TDD é fazer análises para os seguintes grupos:

\begin{itemize}
	\item \textbf{Experientes em desenvolvimento de software e em TDD:} 
	Por ser composto de participantes com experiência tanto
	em desenvolvimento de software quanto em TDD, devemos entender por que
	pessoas com alta experiência optam por utilizar a prática;
	
	\item \textbf{Experientes em desenvolvimento de software, mas não em TDD:} 
	Por serem participantes com experiência em desenvolvimento
	de software, mas não os praticantes de TDD, devemos entender a diferença entre
	praticar TDD e não praticar TDD;
		
	\item \textbf{Inexperientes tanto em desenvolvimento de software, quanto em TDD:} 
	Por serem participantes sem nenhuma experiência, é esperado que
	a prática ajude na qualidade do código. Caso isso não aconteça, devemos
	entender o motivo de TDD não ter auxiliado os desenvolvedores durante a 
	criação do projeto de classes.
\end{itemize}

%% ------------------------------------------------------------------------- %%
\subsection{Resolução dos problemas propostos}
\label{sec:execucao}	

Todos os participantes foram convidados a resolver os exercícios preparados. 
Para isso, criamos um caderno de exercícios que foi seguido pelo participante.
Esse caderno de exercícios continha nada mais do que os exercícios selecionados
para aquele participante, com a instrução de ``praticar ou não'' TDD.

Os participantes tiveram duas horas para resolver todos os exercícios. Embora
não houvesse nenhuma regra definida, ao final da primeira hora, nós os
avisavámos para que pudessem controlar melhor o tempo e sugeríamos que eles 
partissem para o segundo exercício. 

Todos os códigos foram salvos ao final do estudo para que pudessem ser 
analisados junto com os dados das entrevistas.
O tempo foi considerado suficiente para que o participante resolvesse todos os
exercícios (através do questionário respondido após a resolução dos exercícios). 

Os participantes não podiam se comunicar durante o exercício, e cada um deles recebeu
os exercícios em ordens diferentes, para tentar diminuir o fator de aprendizado que 
pudesse ocorrer durante a resolução dos problemas. 

Cada participante recebeu dois exercícios. Em um deles, o participante praticou TDD; no outro,
ele programou sem a prática. A razão disso
é fazer com que o participante exercite ambos estilos de desenvolvimento (com e sem TDD)
e tenha mais embasamento para ser entrevistado nas próximas etapas do estudo. 
Cada participante recebeu instruções claras no caderno de exercícios 
sobre quais exercícios deveriam ser feitos
com TDD. A escolha desses exercícios também foi randomizada na tentativa de diminuir
o efeito do aprendizado.

Ao final, todos responderam a um questionário online, 
que continha perguntas sobre a qualidade
do código que acabaram de produzir. Esse questionário
é melhor detalhado na Seção \ref{sec:questionario}.

%% ------------------------------------------------------------------------- %%
\subsection{Problemas Propostos}
\label{sec:exercicios}

Foram propostos quatro problemas que deveriam ser resolvidos pelos participantes, utilizando
linguagem Java. O objetivo desses exercícios foi simular problemas de projeto de classes 
recorrentes em diversos projetos de software. Os enunciados encontram-se no Apêndice 
\ref{ape:exercicios}.
Na Tabela \ref{tab:problemas-exercicios}, apresentamos a relação entre uma má
implementação dos exercícios e os princípios de projeto de classes feridos por
ela. As boas práticas de projeto de classes que foram utilizados ao longo deste
estudo são baseadas nos príncipios catalogados por Martin \cite{bob-martin} e
conhecidos pelo acrônimo \textit{SOLID}. No Apêndice \ref{ape:design},
discutimos esses princípios em detalhes.

Foi dito ao participante que os exercícios simulam problemas do mundo real, e ele deveria
ter em mente que as soluções geradas supostamente seriam mantidas por uma outra equipe.
Por esse motivo, foi solicitado ao participante que implemente a solução mais elegante e flexível 
possível.

O primeiro exercício pede ao participante que implemente uma calculadora de salário, em que
o algoritmo de cálculo varia de acordo com o cargo do funcionário. Em uma implementação
procedural e mais difícil de ser mantida, esse problema seria resolvido por meio de uma
sequência de "ifs"; todo novo cargo obrigaria o desenvolvedor a acrescentar mais um "if" 
nessa classe. Uma implementação mais flexível teria cada algoritmo de cálculo em uma 
classe separada.

O segundo exercício pede que o participante implemente o processo de geração de uma nota fiscal e, após
esse processo, a nota gerada deve passar por diversos outros processos, como envio por e-mail, envio
para um sistema externo, persistir na base de dados, etc. Possíveis más implementações incluem a 
implementação de uma única classe que faria todo o processo, ou uma classe altamente acoplada.
Uma solução mais elegante seria extrair cada responsabilidade em uma classe diferente e compô-las
por meio, por exemplo, da implementação do padrão \textit{Observer} \cite{gof}.

O terceiro exercício pede ao participante a implementação de um simples processador de boletos que
deve marcar a fatura como paga, caso a soma de pagamentos seja maior ou igual ao valor da fatura. 
Em uma implementação elegante, o comportamento de marcar a fatura como paga deveria estar encapsulado, e 
ficar dentro da classe "Fatura", ou entidade similar criada pelo participante.

No quarto exercício, o participante deveria escrever um algoritmo responsável por filtrar faturas de
acordo com diferentes critérios. Em uma implementação procedural, esse único algoritmo seria
responsável por validar todos os critérios. Mas, por serem complexos, esses filtros deveriam ser divididos em 
várias classes, em vez de ficarem em uma única classe responsável por todos os critérios.


\begin{table}
	\centering
	\begin{tabular}{| l | l | l | }
		\hline
		\textbf{Exercício} & \textbf{Mau Cheiro} & \textbf{Princípios A Serem Seguidos}\\
		
		\hline
		
		Exercício 1 & Rigidez, Complexidade Desnecessária & PRU, PAF \\
		Exercício 2 & Fragilidade, Viscosidade, Imobilidade & PRU, PID, PAF \\
		Exercício 3 & Rigidez, Fragilidade & PRU\\
		Exercício 4 & Fragilidade, Viscosidade, Imobilidade & PAF, PRU, PID \\
		
		\hline
	\end{tabular}
	\caption{Exercícios propostos e mau cheiros de projeto de classes}
	\label{tab:problemas-exercicios}
\end{table}

Os exercícios propostos são baseados em um workshop criado pelo autor desta pesquisa, e o mesmo
foi aplicado para 2 turmas diferentes, uma delas dentro do Agile Brazil 2011, o
maior evento brasileiro de métodos ágeis, que tinha um público heterogêneo, e uma delas para
uma das turmas do curso de Ciência da Computação do Instituto de Matemática e Estatística da Universidade
de São Paulo, na qual o público era constituído em sua maioria de alunos de graduação. 

Neste workshop, os participantes, além de receberem os mesmos
exercícios, também possuíam um código-fonte inicial do exercício, e a tarefa era
apenas finalizar a solução. No entanto, o código-fonte inicial enviesava o participante a gerar
uma má implementação. O resultado de ambas as turmas foram semelhantes; alguns participantes
não perceberam a má qualidade do código inicial e apenas deram prosseguimento ao código
de má qualidade. Outros perceberam os problemas e refatoraram os códigos em busca
de um melhor projeto de classes. Ambas as turmas avaliaram positivamente os
exercícios propostos.


%% ------------------------------------------------------------------------- %%
\subsection{Questionário pós-experimento}
\label{sec:questionario}

Como mencionado anteriormente, ao final dos exercícios o participante responderam a um questionário.
Algumas perguntas foram abertas, nas quais o participante podia dar uma opinião mais embasada sobre o assunto,
e outras foram fechadas, escolhendo um valor dentro de uma escala
Likert com 5 valores.

Para melhor explicar cada questão encontrada no questionário, dividimo-as em pequenos blocos. 
Na Tabela \ref{tab:questionario-pos}, apresentamos
as informações extraídas desse questionário, e qual o objetivo de cada uma delas. Uma cópia
do mesmo pode ser encontrada no Apêndice \ref{ape:questionario-pos}.

\begin{table}
	\centering
	\begin{tabular}{ | p{5cm} | p{5cm} |}
		
		\hline
		
		\textbf{Bloco} & \textbf{Objetivo} \\
		
		\hline
		
		O estudo &
		Este bloco objetiva obter a opinião dos participantes sobre o estudo, como clareza dos
		exercícios. O objetivo é aumentar a validade do estudo.
		Além disso, entender se os exercícios propostos são parecidos com os problemas encontrados no mundo
		real ajudam a aumentar a possibilidade de generalização do estudo.
		\\ \hline
		
		Código gerado &
		O objetivo é obter a visão do desenvolvedor sobre o próprio código gerado e como ele
		faz para obter \textit{feedback} sobre a qualidade do código que escreve.
		\\ \hline
		
		Prática de TDD &
		O objetivo deste bloco é entender como a prática pode ter influenciado
		nas decisões de projeto de classes feitas pelo programador durante o exercício.\\
		
		\hline
		
	\end{tabular}
	\caption{Informações que são extraídas do questionário pós-experimento.}
	\label{tab:questionario-pos}
\end{table}

%% ------------------------------------------------------------------------- %%
\subsection{Escolha de candidatos para a entrevista}

Após a implementação, os dados colhidos foram parcialmente analisados. O intuito
foi encontrar, dentre todos os participantes, aqueles com dados mais relevantes
e que mereçam ser aprofundados.
Para isso, informações como qualidade dos códigos gerados, grupo que participa
do estudo e respostas no questionário influenciaram na escolha.

O procedimento adotado para escolha dos candidatos foi:

\begin{enumerate}
	\item Leitura dos questionário inicial e pós-experimento respondidos pelo participante.
	
	\item Avaliação do código gerado, levando-se em conta qual foi resolvido com
	TDD e qual foi resolvido sem TDD.
	
	\item Geração de relatório para cada participante, descrevendo as opiniões do participante
	e o nosso ponto de vista sobre os dados colhidos até então.
	
	\item Candidatos que apresentaram códigos de qualidade ou alguma divergência entre as respostas
	no questionário e os códigos feitos com TDD e sem TDD 
	(por exemplo, participante comentou no questionário que TDD o ajudou no projeto de classes,
	mas não percebemos uma melhora no projeto de classes no código que ele produziu), que merecesse
	atenção especial, foram selecionados para a entrevista.
\end{enumerate}

%% ------------------------------------------------------------------------- %%
\subsection{Entrevistas}
\label{sec:planejamento-estrategia-entrevistas}

A entrevista foi semi-estruturada, dando liberdade ao
pesquisador para mudar o rumo das perguntas, caso se fizesse necessário.
Além disso, todas as perguntas foram abertas, permitindo que o desenvolvedor desse
uma resposta ampla sobre o assunto. Uma cópia do roteiro da entrevista pode ser encontrada
no Apêndice \ref{ape:entrevista}.

O processo de entrevista é composto por uma breve introdução da pesquisa, tomando
o cuidado para não enviesar o participante, seguida de algumas perguntas que visam
caracterizar o perfil do participante; perguntas como qual a experiência do
desenvolvedor em desenvolvimento de software e TDD são necessárias para ajudar o
pesquisador no entendimento das respostas dadas. Além disso, perguntas sobre
referências, livros e outros pontos de informação nas quais o participante lê a
respeito da prática servem para que entendamos o embasamento teórico
dos praticantes sobre TDD. Apesar das perguntas já terem sido feitas durante
o questionário inicial, essa parte inicial é importante para tranquilizar
o participante, e possibilitar com que ele esteja mais confiante e fale
mais durante as perguntas mais cruciais.

Em seguida, nós perguntamos os principais pontos da pesquisa.
Para isso, fizemos uso não só de perguntas abertas, mas também
voltamos aos códigos gerados durante o exercício, para que as respostas se tornassem técnicas e
específicas, caso necessário. A ideia foi fazer com que o participante nos explicasse
como que o projeto de classes daquele exercício foi concebido.

Uma vez que as decisões tomadas por um programador durante a atividade de projeto de classes
podem ser influenciadas por vários diferentes fatores, 
as perguntas foram feitas de modo que o participante triangule suas respostas,
e tente isolar o máximo possível a atividade de TDD dos outros possíveis fatores
de influência. Participantes que não articulassem bem suas respostas seriam eliminados
durante o processo de análise.

Todas as entrevistas foram gravadas para que nós pudéssemos fazer a
transcrição e rever os dados a qualquer momento durante o processo. Além disso,
nós também tomamos notas, capturando informações como reações dos 
participantes a determinadas perguntas, ou qualquer outra informação relevante. 
As entrevistas também foram feitas em dias diferentes de acordo com a disponibilidade
de cada participante.

Entrevistas, de maneira geral, tendem a ser úteis já que os participantes podem prover dados históricos 
sobre o objeto em estudo.
Além disso, entrevistas nos permitem o controle sobre as questões a serem feitas.
Mas, um possível problema é que as entrevistas geralmente provêm informações indiretas, 
filtradas por meio da visão dos participantes. Além disso, a presença do pesquisador pode 
intimidar o participante ou enviesar as respostas.
Outro possível problema é que nem todos os participantes são articulados e perceptivos, e conseguem
formalizar, em palavras, o que conhecem ou estão pensando.


%% ------------------------------------------------------------------------- %%
\subsection{Métricas de código}

Com o código-fonte em mãos, é possível utilizar-se de métricas de código
para avaliar sua qualidade.

As métricas utilizadas foram:

\begin{enumerate}
	\item \textbf{Complexidade Ciclomática}: Optamos por utilizar o algoritmo de complexidade ciclomática criado
	por McCabe \cite{mccabe}. Uma simples explicação desse algoritmo seria que, para cada método, um contador
	é incrementado sempre que um if, for, while, case, catch, E lógico, OU lógico, ou if ternário aparece.
	Todos os métodos tem ainda seus contadores iniciados com 1. 
	
	\item \textbf{\textit{Fan-Out}}: Essa métrica conta o número de classes que uma classe conhece e faz uso \cite{lorenz}.
	
	\item \textbf{Falta de Coesão dos Métodos}: A versão implementada do algoritmo de falta de coesão dos métodos 
	(ou, do inglês, \textit{Lack of Cohesion of Methods (LCOM)}) foi a sugerida por Henderson-Sellers \cite{lcom-hs}.
	Neste algoritmo, uma classe é considerada altamente coesa se e somente se todos os seus métodos usam
	todos seus atributos de instância. Neste caso, a métrica resulta em zero. 
	
	\item \textbf{Quantidade de Linhas por Método}: Essa métrica conta o número de linhas em cada método de
	cada classe. Linhas em branco dentro dos métodos também entram na conta.
	
	\item \textbf{Quantidade de Métodos}: A métrica conta o número de métodos por classe.
	
\end{enumerate}

Todas as métricas citadas já são de uso conhecido na academia e indústria, e de fácil implementação. 
Para calcular essas
métricas, nós implementamos nossa própria ferramenta. O motivo para tal é que
grande parte das ferramentas existentes fazem uso de código compilado, e não
apenas do código-fonte. Nossa ferramenta possui bateria de testes automatizados
e código-fonte aberto \footnote{\url{http://www.github.com/mauricioaniche/msr-asserts}. 
Último acesso em 10 de Fevereiro de 2012.}.

%% ------------------------------------------------------------------------- %%
\subsection{Avaliação do Especialista}
\label{sec:planejamento-especialista}

Dois especialistas foram convidados a analisar os códigos-fonte e a dar notas para cada
um deles. Apesar das métricas de código nos darem informações
preciosas sobre a qualidade do código, a opinião de um especialista, baseada
em sua experiência passada, pode ser bastante enriquecedora.

As categorias nas quais eles deveriam avaliar eram: \textit{Simplicidade}, \textit{Testabilidade} e
\textit{Qualidade do Projeto de Classes}.
Em cada uma dessas categorias, os especialistas puderam dar notas entre
1 (ruim) e 5 (bom), ou optar por não avaliar aquele exercício.
Como alguns participantes não terminaram o exercício, o especialista
foi avisado de que ele deve avaliar inclusive a "intenção" de projeto de classes criado
pelo participante, e não só o código atual. 

Para que a opinião do especialista fosse imparcial, ele \textbf{não} sabia a qual grupo
pertencia e como cada código-fonte analisado foi desenvolvido (com ou sem a prática de TDD).

%% ------------------------------------------------------------------------- %%
\section{Análise dos dados}
\label{sec:planejamento-analise}

O procedimento de análise, baseado em \cite{creswell}, que está representado na
Figura \ref{fig:analise-dados}, ilustra o processo de análise utilizado
nessa pesquisa. Apesar de parecer uma abordagem em cascata, ele é na prática 
iterativo, visto que os passos são interconectados. 

\begin{figure}[h]
  \centering
  \includegraphics[scale=0.5]{analisedados}
  \caption{Processo de análise dos dados}
  \label{fig:analise-dados}
\end{figure}


A princípio, todos os dados recolhidos foram reunidos e preparados. As entrevistas
foram transcritas, assim como as observações feitas sobre as gravações das implementações.
A seguir, para que nós possamos buscar por erros no processo de transcrição, todos
os dados foram re-lidos. Essa primeira leitura também serviu para nós termos
uma ideia inicial das informações ali presentes.

O processo de codificação e o agrupamento por temas foi então realizado. 
Os códigos gerados eram derivados da opinião dos participantes, e os mesmos 
eram constantemente revistos e unidos, quando dois códigos eram similares.
Além disso, quando o participante mencionava algum padrão de \textit{feedback},
um código diferente era dado para esse trecho, para que nós o pudéssemos encontrá-lo
facilmente durante a escrita deste texto.
Os códigos que mais foram relacionados apareceram como maiores
contribuições da pesquisa qualitativa, durante o processo de interpretação dos
dados. 

O software utilizado para o processo de codificação foi
o \textit{Atlas.ti} \footnote{\url{http://www.atlasti.com/}. Último acesso em 3
de maio de 2011.}, produzido na Alemanha, que nos possibilitou 
organizar textos e juntá-los com códigos e anotações. 

Além disso, as métricas calculadas foram utilizadas na busca por alguma diferença
estatisticamente significante entre códigos produzidos com e sem TDD. 
Testes estatísticos para análise de variância foram utilizados
para verificar se TDD influencia nas mesmas.
A variância dos dados obtidos na opinião do especialista sobre cada código-fonte gerado
foi utilizada também como entrada para algoritmos estatísticos. 

Como esses dados não seguem uma distribuição normal, o \textbf{teste
de Wilcoxon} foi escolhido. Este é um teste de hipotése não-paramétrico,
usado para comparar duas amostras e verificar se há diferença na média das populações.
O teste de Wilcoxon recebe dois conjuntos de dados como entrada, que são os
conjuntos na qual ele procurará por diferenças nas médias.

Executamos o teste de Wilcoxon diversas vezes, uma para cada métrica de código
que calculamos: complexidade ciclomática, acoplamento eferente, falta
de coesão dos métodos, número de linhas por métodos, e quantidade de métodos
por classe. Para cada uma dessas métricas, separamos o conjunto de valores
encontrados pela métrica em códigos não produzidos com TDD, do conjunto de
valores para códigos produzidos com TDD.

Durante o processo, o pesquisador constantemente validou toda e qualquer
informação colhida e, quando se fez necessário, a coleta de qualquer entrevista,
observação ou métrica foi refeita. 

 %% ------------------------------------------------------------------------- %%
\section{Validade e Confiabilidade do Estudo}
\label{sec:planejamento-validacao}

Para garantir a confiabilidade deste estudo, nós realizamos os
seguintes procedimentos:

\begin{itemize}
	\item \textbf{Checar as transcrições}. O objetivo foi garantir que nenhum erro
	óbvio tenha sido cometido;

	\item \textbf{Verificação de pesquisador auxiliar}. Um pesquisador auxiliar
	checou a interpretação dos dados gerada por esta pesquisa;
	
	\item \textbf{Rastreabilidade dos dados}. Todos os dados colhidos foram
	preservados em forma eletrônica.

\end{itemize}

A validade do estudo foi buscada por alguns procedimentos executados pelos
pesquisadores, dentre eles:

\begin{itemize}

	\item \textbf{Prover descrição rica e detalhada sobre o ambiente}. A riqueza
	dos detalhes mostra a qualidade do estudo, além de possibilitar a repetição do
	experimento por outros pesquisadores;

	\item \textbf{Esclarecer todos os possíveis vieses da pesquisa}. A pesquisa
	deixa claro quais são suas limitações. Todas elas são discutidas no Capítulo
	\ref{cap:ameacas}.

\end{itemize}

Em resumo, o principal meio de validação do estudo foi o rico detalhamento dos
participantes, dos dados colhidos e instrumentos de coleta, de forma
que qualquer pesquisador interessado em replicar o experimento terá um
arcabouço sólido para comparação \cite{merriam-1998}. A análise de
dados também foi relatada em detalhes para que os leitores tenham uma visão
clara sobre o método utilizado na pesquisa. 
Além disso, esta pesquisa também foi acompanhada pelo meu orientador,
que constantemente validou e discutiu os pontos levantados nesse planejamento.

%% ------------------------------------------------------------------------- %%
\section{Papel do Pesquisador}
\label{sec:planejamento-papel}

Em um estudo qualitativo, o pesquisador tem como papel fundamental participar do 
processo de captura dos dados, bem como seu preparo e interpretação final.
Creswell \cite{creswell}, citando Locke \cite{locke}, lembra
que a contribuição do investigador para o contexto da pesquisa pode ser útil e
positiva. Além do mais, o pesquisador é responsável por
identificar todos os valores pessoais, pressuposições e vieses do estudo.

O autor desta pesquisa tem formação em Ciência da Computação, e desenvolve software há 9
anos, pratica TDD diariamente nos últimos 3 anos e possui profundos
conhecimentos teóricos e práticos sobre orientação a objetos e métodos ágeis.
Além disso, o autor palestrou sobre TDD em eventos da indústria brasileira
de desenvolvimento de software, como a Agile Brazil 2010, o .NET Architects
2010 e o QCON São Paulo 2010. O autor desta pesquisa acredita que sua experiência nessas
áreas aumentam sua capacidade de análise dos efeitos de TDD no projeto de classes de sistemas 
orientados a objetos.

%% ------------------------------------------------------------------------- %%
\section{Questões Éticas}
\label{sec:planejamento-etica}

Esta pesquisa pode revelar desenvolvedores que produzem projeto de classes não
satisfátorios ou não utilizam a prática corretamente.
Por esse motivo, todos os dados colhidos pelo pesquisador foram mantidos em
sigilo e todos os nomes de desenvolvedores e projetos omitidos, conforme acordo 
assinado entre o pesquisador e a participante.

%% ------------------------------------------------------------------------- %%
\section{Estudo piloto}
\label{sec:estudo-piloto}

Antes da execução do estudo com participantes reais, um estudo piloto foi
executado para que o pesquisador pudesse validar todos os instrumentos de pesquisa,
como exercícios, gravação de vídeo, protocolo e roteiro de entrevista.

Com os resultados do estudo piloto, o pesquisador fez melhorias
nos diversos instrumentos de pesquisa.
Vale ressaltar que as pessoas que participaram do estudo piloto não foram reutilizadas no
estudo final.

Na primeira versão, o participante deveria implementar todos os quatro exercícios em 2 horas.
Mas, após a execução do primeiro piloto, o participante nos contou que se sentiu muito cansado, e
que, ao final, não estava mais trabalhando direito. Por esse motivo, decidimos que
os participantes resolveriam apenas 2 exercícios.

No segundo piloto, o participante teve dificuldades para configurar a área de trabalho no Eclipse e
para entender o que deveria fazer em cada exercício. Para resolver este problema, adotamos um 
caderno de questões bem explicado, além de sugerir ao participante o \textit{download} de uma
área de trabalho do Eclipse previamente configurada.
O mesmo participante também comentou que os exercícios poderiam ser simplificados. Essa sugestão
não foi aceita, já que queríamos que os exercícios fossem parecidos com os do mundo real. No entanto,
passamos a avisar aos participantes que eles não precisavam necessariamente terminar o exercício,
mas sim trabalhar com qualidade.

Já o terceiro piloto nos ajudou a melhorar o roteiro de entrevista. Percebemos a existência de diversas perguntas
repetidas. Após o término, removemos essas questões e deixamos o roteiro de entrevistas mais simples.

Infelizmente não conseguimos executar mais pilotos, devido a falta de
tempo e disponibilidade de possíveis participantes.

\section{Execução do estudo}
\label{subsec:particularidades-execucao}

Como os estudos foram executados fisicamente em muitas das empresas selecionadas,
nós acabamos por ajudar na organização do ambiente, mesmo a equipe tendo 
recebido o caderno de questões com as instruções da instalação
alguns dias antes. A ideia era também gravar a implementação dos alunos,
mas dificuldades em se encontrar um software de vídeo para as diferentes
plataformas, e arquivos muito grandes, impossibilitaram a gravação.

Todos os participantes eram avisados de que tinham por volta de 50 minutos
por exercício. Eles também sabiam que, mesmo que não terminassem o exercício,
deveriam focar sempre na qualidade do código gerado. Eles também eram solicitados
a implementar um projeto de classes flexível para os problemas. A frase que dizíamos para
eles era geralmente: \textit{"Levem os exercícios para o mundo real, onde um outro
desenvolvedor deverá manter o código gerado. Lembrem-se de implementar o código mais fácil possível
para evoluir. As regras de negócio que existem hoje no enunciado tendem a aumentar
de número e, portanto, deixem a manutenção do código de vocês mais simples."}

Durante a execução, nós tirávamos diversas dúvidas sobre enunciados dos
exercícios, e até mesmo sobre procedimentos que os participantes podiam
adotar durante a execução. Um deles, por exemplo, perguntou se poderia
refatorar o código durante a implementação sem TDD. 
Nós também não os pressionávamos em nenhum momento. Eles ficavam,
cada um em suas máquinas, trabalhando na implementação. Não ficávamos 
passando atrás das máquinas para ver como estavam indo, na tentativa
de evitar qualquer possível alteração de comportamento pela nossa presença.
Ao final de cada intervalo de 50 minutos, nós avisávamos para eles finalizarem
a linha de raciocínio e partir para o próximo exercício.

Todo o experimento ocorreu bem, com exceção do que foi executado
dentro da universidade. Além de diversos problemas de infraestrutura,
como a falta de espaço em disco disponível para alguns alunos, que impedia até mesmo
o JUnit de executar, todos os participantes conseguiram fazer apenas
1 exercício. Diante dessa situação, optamos por deixá-los implementar
o mesmo exercício até o final da aula já que, após os 50 minutos iniciais,
nós observamos que pouco código havia sido escrito. Outro problema levantado
foi que alguns alunos não se mostraram muito dispostos a participar
do estudo.
Ao contrário, a grande maioria dos participantes da indústria conseguiram
implementar o exercício no tempo delimitado, e todos se mostraram
muito receptivos para o estudo. Um ponto que se mostrou bem útil
para convencer participantes da indústria a participar foi a proposta
posterior de nós apresentarmos os resultados encontrados em uma palestra.

Ao final da execução do estudo em cada empresa, nós guardávamos
os dados gerados (código-fonte e caderno de questões assinado),
e dávamos o nome da pasta do participante, de acordo com
o seguinte formato: \textit{id-nome-combinação}. O id aponta
para um número único do participante no estudo, e a combinação
aponta quais exercícios ele resolveu, bem como em qual deles
ele utilizou TDD.

As entrevistas foram, em grande parte, realizadas pessoalmente com 
o desenvolvedor. Quando o participante não estava disponível (por estar
localizado em outra cidade), a entrevista era realizada por Skype.
Durante toda a entrevista, o participante podia observar o código que
produziu. Para isso, nós criamos uma simples aplicação web para facilitar
a exibição dos códigos-fonte. O objetivo do participante ver o código
era lembrar sobre suas decisões, e nos possibilitar perguntas específicas
sobre o projeto de classes gerado.

Em média, as entrevistas levavam 30 minutos. Quando o participante comentava
algo interessante, nós adaptávamos o roteiro para permitir que ele falasse mais
do assunto, e anotávamos o ponto para que, ao final, fosse possível discutir novamente sobre o assunto.
O roteiro de entrevistas sofreu uma pequena mudança ao final da primeira entrevista,
já que percebemos que comunicar ao participante que o código que ele produziu
\textit{não apresenta um bom projeto de classes} não era uma tarefa fácil, e talvez, não ética. 
Optamos por perguntar sobre como aquele projeto de classes foi construído, mesmo que ele
não estivesse bem construído em nossa opinião.

Para manter um padrão, nós sempre perguntávamos primeiro sobre o exercício que ele
fez com TDD, independente da ordem que ele implementou no dia da execução. Notamos
que muitos participantes discutiram sobre os exercícios e possíveis implementações com seus colegas
após a realização do exercício.

Dos três especialistas convidados a avaliar os códigos produzidos, apenas um não
completou a tarefa. Sugerimos a todos eles, antes do início da avaliação, que avaliassem
não só a quantidade de código escrito, mas as decisões de projeto de classes tomadas por aquele
participante. Para que os especialistas avaliassem cada código gerado, nós implementamos
uma aplicação web, que eles tinham acesso a qualquer momento, e podiam parar ou continuar
a avaliar na hora que preferissem.


\section{Descrição dos participantes}
\label{sec:desc-participantes}

Ao todo tivemos 25 participantes, de 6 diferentes empresas de pequeno porte do mercado
brasileiro \footnote{Consideramos empresas de pequeno porte aquelas que tem menos de 50 funcionários.}.
Os participantes da indústria, em sua maioria, eram pessoas com pouca experiência em TDD.
40\% deles disseram utilizar a prática há no máximo um ano. 52\% deles praticam TDD
entre 1 e 3 anos. Apenas 4\% praticou entre três e quatro anos, e nenhum participante
possuía mais experiência do que isso. Na Figura \ref{fig:exp-tdd-industria}, mostramos
a distribuição da experiência da prática de TDD entre os participantes.

Os números são um pouco diferentes quando se trata da experiência em desenvolvimento
de software. 24\% dos participantes desenvolve software entre 4 e 5 anos.
28\% deles faz isso entre 6 e 10 anos. 20\% possui até 2 anos de experiência.
Na Figura \ref{fig:exp-sw-industria}, mostramos a distribuição.


\begin{figure}[h!]
  \centering
  \includegraphics[scale=0.6]{findings/experiencia-tdd-industria.png}
  \caption{Experiência dos participantes da indústria com TDD}
  \label{fig:exp-tdd-industria}
\end{figure}

\begin{figure}[h!]
  \centering
  \includegraphics[scale=0.5]{findings/experiencia-sw-industria.png}
  \caption{Experiência dos participantes da indústria com desenvolvimento de software em geral}
  \label{fig:exp-sw-industria}
\end{figure}

Entrando em aspectos mais técnicos, 64\% dos participantes afirmaram programar em Java. Entretanto,
36\% disseram que não trabalham com Java no seu dia a dia. Todos eles afirmam conhecer JUnit,
e só 12\% diz nunca ter ouvido falar sobre o conceito de objetos dublês\footnote{Objetos dublê ou, do inglês, 
\textit{mock objects}, são objetos criados durante um teste de unidade, e que imitam o comportamento de um
outro objeto concreto. Geralmente são muito utilizados quando queremos isolar nosso teste de outras classes
do sistema. Mais informações sobre objetos dublês podem ser encontradas em \cite{mocks}.}. De fato, 64\% deles
aplicam objetos dublês durante suas atividades de desenvolvimento. Com relação a conhecimentos
em orientação a objetos, na pergunta aberta do questionário, grande parte deles 
afirmou que possuem uma boa experiência e alguns
chegam até a afirmar que dominam o assunto. Poucos disseram que possuem conhecimentos
básicos. Na Tabela \ref{tab:exp-industria},
apresentamos o conhecimento dos participantes em relação a Java, JUnit e objetos dublês.


\begin{table}
	\begin{tabular}{ | p{5cm} | p{5cm} | p{5cm} | }
		\hline
		\textbf{Ferramenta} & \textbf{Participantes que conhecem} & \textbf{Participantes que não conhecem}\\
		\hline
		Java & 16 & 9\\
		JUnit & 25 & 0\\
		Objetos Dublê & 16 (utilizam no dia a dia), 6 (na teoria) & 3\\
		\hline
	\end{tabular}
	\caption{Experiência em Java, JUnit, e Objetos Dublê dos participantes da indústria}
	\label{tab:exp-industria}
\end{table}

Em relação à experiência com TDD,
podemos afirmar que metade dos participantes ainda está experimentando a prática, enquanto
outros já a tem mais consolidada. Isso é positivo, já que foi possível capturar informações
da prática de TDD por pessoas com diferentes níveis de maturidade.

Em relação ao alto número de pessoas que não utilizam Java, isso se deve ao fato de uma das
empresas fazer uso de PHP para seu trabalho do dia a dia. No entanto, nós conhecemos a equipe
e verificamos que, apesar de não utilizarem a linguagem constantemente, eles não tiveram
problema algum durante a execução dos exercícios.