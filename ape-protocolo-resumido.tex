\chapter{Protocolo do estudo}
\label{ape:protocolo-resumido}

\section{Execução}

Nesta seção, apresentamos o protocolo utilizado na execução do estudo.
Como este estudo não precisa ser executado em grupo, entende-se que este
roteiro deve ser feito de maneira sequêncial para cada um dos participantes
do estudo.
 
\begin{itemize}
  
  \item Convite para possível participante do estudo;
  \item Envio do questionário inicial para levantamento do perfil do
  participante (encontra-se no Apêndice \ref{ape:questionario-inicio});
  \item Entrega do caderno de exercícios (encontrados no Apêndice
  \ref{ape:exercicios}) já devidamente randomizado e ínicio da resolução dos problemas por parte do participante;
  \item Envio do questionário final (que se encontra no Apêndice
  \ref{ape:questionario-pos}), que deve ser respondido imediamente após a
  finalização da implementação;
  \item Colheta do código-fonte gerado pelo participante;
  \item Análise do código-fonte e dos dados em ambos os questionários
  preenchidos para possível seleção para entrevista;
  \item Caso o participante tenha sido selecionado, executar entrevista,
  detalhada no Apêndice \ref{ape:entrevista}.

\end{itemize}

\section{Análise}

Aqui apresentamos o processo de análise utilizado ao longo do estudo.

\begin{itemize}
  \item Cálculo das métricas de código para uso na análise quantitativa;
  \item Execução do teste estatístico sobre os valores encontrados;
  \item Transcrição dos roteiros de entrevista;
  \item Revisão dos roteiros de entrevista;
  \item Processo de codificação da entrevista;
  \item Análise dos dados encontrados pelo teste estatístico e pelas
  entrevistas, e redação dos resultados encontrados.
\end{itemize}
