% ---------------------------------------------------------------------------- %
% Mauricio Finavaro Aniche                                                     %
% mauricioaniche@gmail.com                                                     %
% ---------------------------------------------------------------------------- %

\documentclass[11pt,openany,twoside,a4paper]{book}

% ---------------------------------------------------------------------------- %
% Pacotes 
\usepackage[T1]{fontenc}
\usepackage[brazil]{babel}
\usepackage[utf8]{inputenc}
\usepackage[pdftex]{graphicx}           % usamos arquivos pdf/png como figuras
\usepackage{pifont}
\usepackage{amsfonts}
\usepackage{amssymb} 
\usepackage{setspace}                   % espaçamento flexível
\usepackage[bf,small,compact]{titlesec} % cabeçalhos dos títulos: menores e compactos
\usepackage{indentfirst}                % indentação do primeiro parágrafo
\usepackage{subfigure}                  % uso de várias figuras numa só
\usepackage{makeidx}                    % índice remissivo
\usepackage[nottoc]{tocbibind}          % acrescentamos a bibliografia/indice/conteudo no Table of Contents
\usepackage{courier}                    % usa o Adobe Courier no lugar de Computer Modern Typewriter
\usepackage{type1cm}                    % fontes realmente escaláveis
\usepackage{listings}                   % para formatar código-fonte (ex. em Java)
\usepackage{setspace}
\usepackage{longtable}
\usepackage{framed}
\usepackage{multirow}
\usepackage{titletoc}
\usepackage{lscape}
\usepackage[final]{pdfpages}
\usepackage[fixlanguage]{babelbib}
\usepackage[font=small,format=plain,labelfont=bf,up,textfont=it,up]{caption}
\usepackage[usenames,svgnames,dvipsnames,table]{xcolor}
\usepackage[a4paper,top=2.54cm,bottom=2.0cm,left=2.0cm,right=2.54cm]{geometry}% margens
\usepackage{float}
\usepackage[pdftex,plainpages=false,pdfpagelabels,pagebackref,colorlinks=true,citecolor=black,linkcolor=black,urlcolor=black,filecolor=black,bookmarksopen=true]{hyperref} % links em preto
\usepackage[all]{hypcap}                % soluciona o problema com o hyperref e capitulos
\usepackage[numbers,square,sort,nonamebreak,comma]{natbib}  % citação
\usepackage{longtable}
                                % bibliográfica alpha (alpha-ime.bst)

\usepackage{type1cm}      % fontes realmente escaláveis
\fontsize{60}{62}\usefont{OT1}{cmr}{m}{n}{\selectfont}

% ---------------------------------------------------------------------------- %
% Cabeçalhos similares ao TAOCP de Donald E. Knuth
\usepackage{fancyhdr}
\pagestyle{fancy}
\fancyhf{}
\renewcommand{\chaptermark}[1]{\markboth{\MakeUppercase{#1}}{}}
\renewcommand{\sectionmark}[1]{\markright{\MakeUppercase{#1}}{}}
\renewcommand{\headrulewidth}{0pt}

% ---------------------------------------------------------------------------- %
\graphicspath{{./figuras/}}             % caminho das figuras (recomendável)
\frenchspacing                          % arruma o espaço: id est (i.e.) e exempli gratia (e.g.) 
\urlstyle{same}                         % URL com o mesmo estilo do texto e não mono-spaced
\makeindex                              % para o índice remissivo
\raggedbottom                           % para não permitir espaços extra no texto
\fontsize{60}{62}\usefont{OT1}{cmr}{m}{n}{\selectfont}
\cleardoublepage
\normalsize

% ---------------------------------------------------------------------------- %
% Opções de listing usados para o código fonte
% Ref: http://en.wikibooks.org/wiki/LaTeX/Packages/Listings
\lstset{ %
language=Java,                  % choose the language of the code
basicstyle=\footnotesize,       % the size of the fonts that are used for the code
numbers=left,                   % where to put the line-numbers
numberstyle=\footnotesize,      % the size of the fonts that are used for the line-numbers
stepnumber=1,                   % the step between two line-numbers. If it's 1 each line will be numbered
numbersep=5pt,                  % how far the line-numbers are from the code
showspaces=false,               % show spaces adding particular underscores
showstringspaces=false,         % underline spaces within strings
showtabs=false,                 % show tabs within strings adding particular underscores
frame=single,	                % adds a frame around the code
framerule=0.6pt,
tabsize=2,	                    % sets default tabsize to 2 spaces
captionpos=b,                   % sets the caption-position to bottom
breaklines=true,                % sets automatic line breaking
breakatwhitespace=false,        % sets if automatic breaks should only happen at whitespace
escapeinside={\%*}{*)},         % if you want to add a comment within your code
backgroundcolor=\color[rgb]{1.0,1.0,1.0}, % choose the background color.
rulecolor=\color[rgb]{0.8,0.8,0.8},
extendedchars=true,
xleftmargin=10pt,
xrightmargin=10pt,
framexleftmargin=10pt,
framexrightmargin=10pt
}

% ---------------------------------------------------------------------------- %
% Corpo do texto
\begin{document}
\frontmatter 
% cabeçalho para as páginas das seções anteriores ao capítulo 1 (frontmatter)
\fancyhead[RO]{{\footnotesize\rightmark}\hspace{2em}\thepage}
\setcounter{tocdepth}{2}
\fancyhead[LE]{\thepage\hspace{2em}\footnotesize{\leftmark}}
\fancyhead[RE,LO]{}
\fancyhead[RO]{{\footnotesize\rightmark}\hspace{2em}\thepage}

\onehalfspacing  % espaçamento

% ---------------------------------------------------------------------------- %
% Capa
% ---------------------------------------------------------------------------- %
\thispagestyle{empty}
\begin{center}
    \vspace*{2.3cm}
    \textbf{\Large{Como a prática de TDD influencia o \\projeto de classes em sistemas orientados a objetos}}\\
    
    \vspace*{1.2cm}
    \Large{Mauricio Finavaro Aniche}
    
    \vskip 2cm
    \textsc{
    Dissertação apresentada\\[-0.25cm] 
    ao\\[-0.25cm]
    Instituto de Matemática e Estatística\\[-0.25cm]
    da\\[-0.25cm]
    Universidade de São Paulo\\[-0.25cm]
    para\\[-0.25cm]
    obtenção do título\\[-0.25cm]
    de\\[-0.25cm]
    Mestre em Ciência da Computação}
    
    \vskip 1.5cm
    Programa: Mestrado em Ciência da Computação\\
    Orientador: Prof. Dr. Marco Aurélio Gerosa

   	\vskip 1.5cm
    \normalsize{São Paulo, Março de 2012}
\end{center}

\newpage
\thispagestyle{empty}
    \begin{center}
        \vspace*{2.3 cm}
        \textbf{\Large{Como a prática de TDD influencia o \\projeto de classes em sistemas orientados a objetos}}\\
        \vspace*{2 cm}
    \end{center}

    \vskip 2cm

    \begin{flushright}
     Este exemplar corresponde à redação\\
     final da dissertação/tese devidamente corrigida\\
     e defendida por Mauricio Finavaro Aniche\\
     e aprovada pela Comissão Julgadora.
    
	Esta versão definitiva da tese/dissertação\\
	contém as correções e alterações sugeridas pela\\
	Comissão Julgadora durante a defesa realizada\\
    por Mauricio Finavaro Aniche em 25/4/2012.

    \vskip 2cm

    \end{flushright}
    \vskip 4.2cm

    \begin{quote}
    \noindent Comissão Julgadora:
    
    \begin{itemize}
		\item Prof. Dr. Marco Aurélio Gerosa (orientador) - IME-USP
		\item Prof. Dr. Ismar Frango Silveira - Mackenzie
		\item Prof. Dr. Rafael Prikladnicki - PUC-RS
    \end{itemize}
      
    \end{quote}
\pagebreak

\pagenumbering{roman}     % começamos a numerar 

% ---------------------------------------------------------------------------- %
\chapter*{Agradecimentos}

Em primeiro lugar, gostaria de agradecer a meu orientador, Prof. Dr. Marco Aurélio
Gerosa, que ao longo destes três anos me ensinou mais do que eu poderia imaginar.
Entrei no mestrado com o intuito de aprender sobre TDD, e saí de lá entendendo
melhor sobre ciência. Lições essas que levarei para toda minha vida na academia e indústria.
Aproveito também para agradecer aos professores Dr. Ismar Frango Silveira e Dr. Rafael
Prikladnicki, que aceitaram participar da banca avaliadora
e deram excelentes sugestões durante a qualificação.

Gostaria também de agradecer aos amigos Gustavo Oliva e Mauricio de Diana, que
criticaram a pesquisa durante todo o tempo, fazendo-me pensar novamente sobre
várias das minhas crenças em engenharia de software. Gostaria muito que continuássemos
nosso grupo de pesquisa, pois devo a vocês grande parte do que aprendi no mestrado.

Agradeço também a minha família, amigos e namorada, por terem me dado todo o suporte
emocional que precisei ao longo desta caminhada. Uma menção especial ao meu pai,
que me presenteou com um livro de programação para crianças quando eu tinha por volta
de 9 anos. Talvez, sem esse presente, essa pesquisa nunca teria acontecido.

Agradeço aos meus amigos de trabalho da Locaweb e Caelum Ensino e Inovação, 
por aguentar meus discursos e palestras sobre TDD, e mostrar diferentes
pontos de vista. Isso me ajudou a entender mais sobre a prática, o que resultou
em uma discussão mais rica ao longo do trabalho.

Por fim, agradeço às empresas que aceitaram participar do meu estudo, Bluesoft,
Amil, e WebGoal (São Paulo e Poços de Caldas). Além disso, obrigado aos meus colegas
de profissão Rafael Werner, Murilo Amêndola e Juan Lopes, por também terem
participado da pesquisa de maneira independente. Fico muito feliz por ter sido
bem recebido pela indústria, e espero que possamos continuar essa parceria entre
academia e indústria.

Um forte abraço a todos!

% ---------------------------------------------------------------------------- %
\chapter*{Resumo}

Desenvolvimento Guiado por Testes (TDD) é uma das práticas sugeridas na Programação
Extrema. A mecânica da prática é simples: o programador escreve o teste antes
de escrever o código. É, portanto, possível inferir que a prática de TDD é uma
prática de testes de software. Entretanto, muitos autores de livros 
conhecidos pela indústria e academia
afirmam que os efeitos da prática vão além. Segundo eles,
TDD ajuda o desenvolvedor durante o processo de criação do projeto classes,
fazendo-os criar classes menos acopladas e mais coesas.

Entretanto, grande parte dos trabalhos da literatura são voltados a descobrir se
a prática faz diferença na qualidade do código gerado, mas poucos são os 
autores que discutem como a prática realmente auxilia.
Mesmo os praticantes não entendem ou conseguem expressar bem como a prática
os guia.

Este trabalho tem por objetivo compreender melhor os efeitos de TDD e como sua prática 
influencia o desenvolvedor durante o processo de projeto de sistemas orientados a objetos.
Para alcançar o objetivo, este estudo faz uso de uma combinação entre um experimento controlado 
inicial, na qual participantes foram
convidados a resolver exercícios pré-elaborados utilizando TDD e, a partir dos dados colhidos nesse estudo, um outro
estudo qualitativo detalhou como a prática influenciou as decisões de projeto.

Ao final, observamos que a prática de TDD pode guiar o desenvolvedor durante o processo
de criação do projeto de classes por meio de constantes \textit{feedbacks} sobre a qualidade
do projeto. Esses \textit{feedbacks} alertam desenvolvedores sobre possíveis problemas,
como alto acoplamento ou baixa coesão. Os desenvolvedores, por sua vez, devem interpretar
e melhorar o projeto de classes. Este trabalho catalogou e nomeou os padrões de
\textit{feedback} percebidos pelos participantes.

\noindent \textbf{Palavras-chave:} Desenvolvimento Guiado por Testes, Sistemas Orientados
a Objetos, Projeto de Classes, Qualidade Interna de Código.

% ---------------------------------------------------------------------------- %
\chapter*{Abstract}

Test-Driven Development (TDD) is one of the suggested practices in Extreme
Programming (XP). The mechanical is simple: the developer writes a test before
writing the implementation. Thus, TDD is often seen as
a software testing technique. However, many famous book authors suggest that
TDD can help developers during the class design creation
process, enabling developers to create less coupled highly cohesive classes.

Most of the academic studies are interested on finding the difference between
a TDD'd and a non-TDD'd code. Only a few of them discuss how the practice
really supports class design. Even practitioners
do not understand how the practice guides them.

This work aims to understand better the effects of TDD and how the practice influences the
practitioner during the class design process in object-oriented systems. 
This work uses a combination of an initial controlled experiment, in which the participants
were invited to solve a few prepared exercises using TDD and, based on gathered data,
a qualitative study explored, in detail, the way TDD influenced the design decisions
taken by the participants.

At the end, we observed that the practice of TDD can guide developers during the 
class design creation process through constant feedback about its quality.
These feedbacks alert developers about possible problems, such as high coupling or
low cohesion. Developers then should interpret and improve the class design accordingly.
This study also catalogues the TDD feedback patterns perceived by the participants.

\noindent \textbf{Keywords:} Test-Driven Development, Object-Oriented
Systems, Class Design, Internal Code Quality.

% ---------------------------------------------------------------------------- %
% Sumário
\tableofcontents    % imprime o sumário

% ---------------------------------------------------------------------------- %
\chapter{Lista de Abreviaturas}
\begin{tabular}{ll}
         TDD         & Desenvolvimento Guiado por Testes (do inglês, \emph{Test-Driven Development})\\ 
         
         XP          & Programação Extrema (do inglês, \emph{Extreme Programming})\\
		 
		 BDUF		 & Projeto de Classe Criado de Uma Só Vez (\textit{Big Design Up-Front})\\
		 
		 PRU		 & Princípio da Responsabilidade Única (\emph{Single Responsibility
		 Principle})\\
		 
		 PID		 & Princípio da Inversão de Dependências (\emph{Single Responsibility
		 Principle})\\ 
		 
		 PAF		 & Princípio do Aberto-Fechado (\emph{Open-Closed Principle})\\
		 
		 PSL	 	 & Princípio da Substituição de Liskov (\emph{Liskov Substitution
		 Principle})\\
		 
		 PSI		 & Princípio da Segregação de Interfaces (\emph{Interface Segregation
		 Principle})\\
		 
		 OO		 	 & Orientação a Objetos\\
		
		 WBMA		 & Workshop Brasileiro de Métodos Ágeis
		 
\end{tabular}

% ---------------------------------------------------------------------------- %
% Listas de figuras e tabelas criadas automaticamente
\listoffigures
\begingroup
\let\cleardoublepage\relax
\let\clearpage\relax
\listoftables
\endgroup

% ---------------------------------------------------------------------------- %
% Capítulos do trabalho
\mainmatter

% cabeçalho para as páginas de todos os capítulos
\fancyhead[RE,LO]{\thesection}

%\singlespacing              % espaçamento simples
\onehalfspacing            % espaçamento um e meio

\input cap-introducao
\input cap-tdd
\input cap-quali-planejamento
\input cap-findings
\input cap-findings-quantitativo
\input cap-ameacas
\input cap-conclusoes

% cabeçalho para os apêndices
\renewcommand{\chaptermark}[1]{\markboth{\MakeUppercase{\appendixname\ \thechapter}} {\MakeUppercase{#1}} }
\fancyhead[RE,LO]{}
\appendix

%% ------------------------------------------------------------------------- %%
\chapter{Projeto de Classes em Sistemas Orientados a Objetos}
\label{ape:design}

Conforme sugerido por esta pesquisa, para escrever classes com alta
testabilidade, o profissional da prática de TDD acaba por fazer uso de boas práticas de
desenvolvimento de software. 
Esse apêndice discute algumas dessas boas práticas, que
foram utilizadas durante o processo de avaliação desta pesquisa.

%% ------------------------------------------------------------------------- %%
\section{Sintomas de Projetos de Classes em Degradação}
\label{sec:design-degradacao}

Diz-se que um projeto de classes está \textit{degradando}
quando o mesmo começa a ficar difícil de evoluir, o reúso de código se 
torna mais complicado do que repetir o trecho de código, ou o custo de se fazer 
qualquer alteração no projeto de classes se torna alto.

Robert Martin \cite{bob-martin} enumerou alguns sintomas de projeto de classes em degradação, 
chamados também de \textit{"maus cheiros"} de projeto de classes. Esses sintomas são parecidos com os 
maus cheiros de código (\textit{"code smells"}), mas em um nível mais alto: eles
estão presentes na estrutura geral do software ao invés de estarem localizados
em apenas um pequeno trecho de código.

Esses sintomas podem ser medidos de forma subjetiva e algumas vezes de forma 
até objetiva. Geralmente, esses sintomas são causados por violações de um ou 
mais princípios de projeto de classes, apresentados na seção \ref{sec:design-oo-principios}.
Muitos desses problemas são relacionados à gerência de dependências. Quando essa
atividade não é feita corretamente, o código gerado torna-se difícil de manter e
reusar. Entretanto, quando bem feita, o software tende a ser flexível, robusto 
e suas partes reusáveis.

\subsection{Rigidez}
\label{subsec:rigidez}

Rigidez é a tendência do software em se tornar difícil de mudar, mesmo de 
maneiras simples. Toda mudança causa uma cascata de mudanças subsequentes  em
módulos dependentes. Quanto mais módulos precisam ser modificados, maior é a
rigidez do projeto de classes. 

Quando um projeto de classes está muito rígido, não se sabe com segurança quando uma 
mudança terá fim. Mudanças simples passam a demorar muito tempo até serem 
aplicadas no código e frequentemente acabam superando em várias vezes a 
estimativa de esforço inicial. 
Frases como \textit{"isto foi muito mais complicado do que eu imaginei"} 
tornam-se populares. Neste momento, gerentes de desenvolvimento começam a ficar
receosos em permitir que os desenvolvedores consertem problemas não críticos.

\subsection{Fragilidade}
\label{subsec:fragilidade}

Fragilidade é a tendência do software em quebrar em muitos lugares diferentes 
toda vez que uma única mudança acontece. Frequentemente, os novos problemas
ocorrem  em áreas não relacionadas conceitualmente com a área que foi mudada, 
tornando o processo de manutenção demasiadamente custoso, complexo e tedioso. 

Consertar os novos problemas usualmente passa a resultar em outros novos
problemas e assim por diante. Infelizmente, módulos frágeis são comuns em
sistemas de software. São estes os módulos que sempre aparecem na lista de defeitos
a serem corrigidos. 
Além disto, desenvolvedores começam a ficar receosos de alterar certos trechos 
de código, pois sabem que estes estão tão frágeis que qualquer mudança simples 
fatalmente acarretará na introdução de problemas inesperados e de naturezas
diversas.

\subsection{Imobilidade}

Imobilidade é a impossibilidade de se reusar software de outros projetos ou de
partes do mesmo projeto. Neste cenário, o módulo que se deseja reutilizar 
frequentemente tem uma bagagem muito grande de dependências e não possui código 
claro. 
Depois de muita investigação, os arquitetos descobrem que o trabalho e o risco 
de separar as partes desejáveis das indesejáveis são tão grandes, que o módulo 
acaba sendo reescrito ao invés de reutilizado.

\subsection{Viscosidade}

Quando uma mudança deve ser realizada, usualmente há várias opções para realizar
tal mudança. Quando as opções que preservam o projeto de classes são mais difíceis de serem
implementadas do que aquelas que não o preservam, há alta viscosidade 
de projeto de classes. Neste cenário, é fácil fazer a "coisa errada" e é difícil fazer a
"coisa certa", ou seja, é difícil preservar e aprimorar o projeto de classes.

\subsection{Complexidade Desnecessária}

Detecta-se complexidade desnecessária no projeto de classes quando ele contém muitos 
elementos inúteis ou não utilizados (\textit{dead code}). Geralmente ocorre
quando há muito projeto inicial (\textit{up-front design}) e não se segue uma 
abordagem de desenvolvimento iterativa e incremental, de modo que os projetistas
tentam prever uma série de futuros requisitos para o sistema e concebem um 
projeto de classes demasiadamente flexível ou desnecessariamente sofisticado. 

Frequentemente apenas algumas previsões acabam se concretizando ao longo do
tempo e, neste meio período, o projeto de classes carrega o peso de elementos e construções 
não utilizados. O software então se torna complexo e difícil de ser entendido. 
Projetos com complexidade muito alta comumente afetam a produtividade, porque 
quando os desenvolvedores herdam tal projeto, eles gastam muito tempo 
aprendendo as nuances do projeto de classes antes que possam efetivamente  estendê-lo ou
mantê-lo confortavelmente \cite{kerievsky}.

\subsection{Repetição Desnecessária}

Quando há repetição de trechos de código, é sinal de que uma abstração
apropriada não foi capturada durante o processo de projeto de classes (ou inclusive na
análise). Esse problema é frequente e é comum encontrar softwares que contenham 
dezenas e até centenas de elementos com códigos repetidos. 

Descobrir a melhor abstração para eliminar a repetição de código geralmente não 
está na lista de itens de alta prioridade dos desenvolvedores, de maneira que a 
resolução do problema acaba sendo eternamente postergada. Também, o sistema se
torna cada vez mais difícil de entender e principalmente de manter, pois os 
problemas encontrados em uma unidade de repetição devem ser corrigidos
potencialmente  em toda repetição, com o agravante de que uma repetição pode
ter forma ligeiramente diferente de outra.

\subsection{Opacidade}

Opacidade é a tendência de um módulo ser difícil de ser entendido. Códigos podem
ser escritos de maneira clara e expressiva ou de maneira "opaca" e complicada. A
tendência de um código é se tornar mais e mais opaco à medida que o tempo passa
e, para que isso seja evitado, é necessário um esforço constante em manter esse 
código claro e expressivo. 

Uma maneira para prevenir isso é fazer com que os desenvolvedores se ponham no
papel de leitores do código e refatorem esse código de maneira que qualquer
outro  leitor poderia entender. Além disso, revisões de código feita por outros
desenvolvedores é também uma possível solução para manter o código menos opaco.

%% ------------------------------------------------------------------------- %%
\section{Princípios de Projeto de Classes}
\label{sec:design-oo-principios}

Todos os problemas citados na seção \ref{sec:design-degradacao} podem ser
evitados pelo uso puro e simples de orientação a objetos. A máxima da
programação orientada a objetos diz que classes devem possuir um baixo
acoplamento e uma alta coesão.

Alcançando esse objetivo, mudanças no código seriam executadas mais facilmente;
alterações seriam feitas em pontos únicos e a propagação de mudanças seria bem
menor. Com as abstrações bem estabelecidas, novas funcionalidades seriam
implementadas através de novo código, sem a necessidade de alterações no código
já existente. Necessidades de evolução do projeto de classes seriam feitas com pouco
esforço, já que módulos dependeriam apenas de abstrações.

Mas, alcançar tal objetivo não é tarefa fácil. Criar classes pouco acopladas e
altamente coesas demanda um grande esforço por parte do desenvolvedor e requer
grande conhecimento e experiência no paradigma da orientação a objetos.

Os princípios comentados nesta seção são muito discutidos por Robert Martin
em vários de seus livros e artigos publicados \cite{bob-martin}.
Esses princípios são produto de décadas de experiência em engenharia de
software. Segundo ele, esses princípios não são produto de uma única 
pessoa, mas sim a integração de pensamentos e trabalhos de um grande número de 
desenvolvedores de software  e pesquisadores, e visam combater todos os sintomas
de degradação discutidos na Seção \ref{sec:design-degradacao}.

Conhecidos pelo acrônimo \textit{SOLID} (sólido, em português), são eles:

\begin{itemize}
	\item Princípio da Responsabilidade Única (\textit{Single-Responsibility 
	Principle (SRP)})
	\item Princípio do Aberto-Fechado (\textit{Open-Closed Principle (OCP)})
	\item Princípio de Substituição de Liskov (\textit{Liskov Substitution 
	Principle (LSP)})
	\item Princípio da Inversão de Dependência (\textit{Dependency Inversion
	Principle (DIP)})
	\item Princípio da Segregação de Interfaces (\textit{Interface Segregation 
	Principle (ISP)})
\end{itemize}

\subsection{Princípio da Responsabilidade Única}
\label{subsec:principio-srp}

O termo coesão define a relação entre os elementos de um mesmo módulo
\cite{demarco} \cite{pagejones}. Isso significa que os todos elementos de uma 
classe que tem apenas uma responsabilidade tendem a se relacionar. Diz-se que
uma classe como essa é uma classe que possui alta coesão (ou que é coesa). 
Já em uma classe com muitas responsabilidades diferentes, os elementos tendem a
se relacionar apenas em "grupos", ou seja, com os elementos que tratam de uma 
das responsabilidades da classe. A esse tipo de classe, diz-se que ela possui 
uma baixa coesão (ou que não é coesa).
Robert Martin altera esse conceito de coesão e a relaciona com as forças
que  causam um módulo ou uma classe a mudar. No caso, o Princípio de
Responsabilidade Única diz que uma classe deve ter apenas uma única razão para 
mudar \cite{bob-martin}.

Esse princípio é importante no momento em que há uma alteração em alguma 
funcionalidade do software. Quando isso ocorre, o programador precisa procurar 
pelas classes que possuem a responsabilidade a ser modificada. Supondo uma
classe  que possua mais de uma razão para mudar, isso significa que ela é
acessada  por duas partes do software que fazem coisas diferentes.  Fazer uma
alteração em uma das responsabilidades dessa classe pode, de maneira não
intencional, quebrar a outra parte de maneira inesperada. Isso torna o projeto de classes
frágil, como comentado na sub-seção \ref{subsec:fragilidade}.

\subsection{Princípio do Aberto-Fechado}
\label{subsec:ocp}

O Princípio do Aberto-Fechado, cunhado por Bertrand Meyer, diz que as entidades
do software (como classes, módulos, funções, etc) devem ser abertas para
extensão, mas fechadas para alteração \cite{meyer-ocp}. 
Se uma simples alteração resulta em uma cascata de alterações em módulos
dependentes, isso cheira à rigidez, conforme descrito na sub-seção 
\ref{subsec:rigidez}. O princípio pede então para que o programador sempre 
refatore as classes de modo que mudanças desse tipo não causem mais modificações.

Quando esse princípio é aplicado de maneira correta, novas alterações fazem com
que o programador adicione novo código, e não modifique o anterior. Isso é
alcançado através da criação de abstrações para o problema. Linguagens
orientadas a objetos possuem mecanismos para criá-las (conhecido com interfaces
em linguagens como Java ou C\#). Através dessas abstrações, o programador consegue 
descrever a maneira em que uma determinada classe deve se portar, mas sem se
preocupar em como essa classe faz isso.

\subsection{Princípio de Substituição de Liskov}
\label{subsec:lsp}

Esse princípio, que discute sobre tipos e sub-tipos, criado por Barbara Liskov
em 1988 \cite{liskov}, é importante já que herança é uma das maneiras para se 
suportar abstrações e polimorfismo em linguagens orientadas a objetos e, como 
visto na seção \ref{subsec:ocp}, o Princípio do Aberto-Fechado se baseia 
fortemente na utilização desses recursos.

O problema é que utilizar herança não é tarefa fácil, pois o acoplamento criado
entre classe filha e classe pai é grande. Fazer as classes filhas respeitarem o
contrato do pai, e ainda permitir que mudanças na classe pai não influenciem nas
classes filhas requer trabalho.

O princípio de Liskov diz que, se um tipo S é sub-classe de um tipo T,
então objetos do tipo T podem ser substituídos por objetos do tipo S, sem
alterar nenhuma das propriedades desejadas daquele programa.

Um clássico exemplo sobre Princípio de Substituição de Liskov é o exemplo dos
Quadrados e Retângulos. Imagine uma classe Retângulo. Um retângulo possui dois 
lados de tamanhos diferentes. Imagine agora uma classe Quadrado (figura
geométrica que possui todos os lados com o mesmo tamanho) que herde de
Retângulo. A única alteração é fazer com que os dois lados tenham o mesmo 
tamanho. 
Apesar de parecer lógico, afinal um Quadrado é um Retângulo com apenas uma
condição diferente, a classe Quadrado quebra o Princípio de Liskov: a
pré-condição dela é mais forte do que a do quadrado, afinal os dois lados devem 
ter o mesmo tamanho.

Quebras do princípio de Liskov geralmente levam o programador a quebrar o
princípio do OCP também. Ele percebe que, para determinados sub-tipos, ele 
precisa fazer um tratamento especial, e acaba escrevendo condições nas classes 
clientes que fazem uso disso.

\subsection{Princípio da Inversão de Dependências}
\label{subsec:dip}

Classes de baixo nível, que fazem uso de infraestrutura ou de outros detalhes
de implementação podem facilmente sofrer modificações. E, se classes de mais alto
nível dependerem dessas classes, essas modificações podem se propagar, tornando
o código frágil.

O Princípio de Inversão de Dependências se baseia em duas afirmações:

\begin{itemize}
	\item Módulos de alto nível não devem depender de módulos de baixo nível. 
	Ambos devem depender de abstrações
	\item Abstrações não devem depender de detalhes. Detalhes devem depender de
	abstrações
\end{itemize}

Em resumo, as classes devem, na medida do possível, acoplar-se sempre com módulos mais
estáveis do que ela própria, já que, como as mudanças em módulos estáveis são
menos prováveis, raramente essa classe precisará ser alterada por mudanças em
suas dependências \cite{bobmartin-oodmetrics}.

\subsection{Princípio da Segregação de Interfaces}
\label{subsec:isp}

Acoplar-se com uma interface de baixa granularidade (ou gordas, do termo
em inglês \textit{fat interfaces}) pode ser perigoso, já que qualquer alteração
que um outro cliente forçar nessa interface poderá ser propagada para essa
classe.

O princípio da segregação de interfaces diz que classes cliente não devem ser
forçados a depender de métodos que eles não usam. Quando uma interface não é
coesa, ela contém métodos que são usados por um grupo de clientes, e outros 
métodos que são usados por outro grupo de clientes. Apesar de uma classe poder 
implementar mais de uma interface, o princípio diz que o cliente da classe deve
apenas depender de interfaces coesas.

%% ------------------------------------------------------------------------- %%
\section{Conclusão}

Todos os princípios discutidos na seção \ref{sec:design-oo-principios} tentam
diminuir os possíveis problemas de projeto de classes que possam eventualmente aparecer.
Discutir o que é um bom projeto de classes é algo difícil; mas é possível enumerar algumas
das características desejáveis: isolar elementos reusáveis de elementos não
reusáveis, diminuir a propagação de alterações em caso de uma nova
funcionalidade.

Esses serão os princípios de projeto de classes levados em conta no momento da análise
dos dados colhidos.
\chapter{Questionário inicial}
\label{ape:questionario-inicio}

\begin{enumerate}

\item Seu e-mail?

\item Empresa em que atua?

\item Seu nome?

\item Experiência em desenvolvimento de software \textit{(0, Entre 0 e 1 anos, Entre 1 e 2 anos, Entre 2 e 3 anos, Entre 3 e 4 anos, Entre 4 e 5 anos, Entre 5 e 6 anos, Entre 6 e 7 anos, Entre 7 e 8 anos, Entre 8 e 9 anos, Entre 9 e 10 anos, Mais que 10 anos)}

\item Experiência com TDD \textit{(0, Entre 0 e 1 anos, Entre 1 e 2 anos, Entre 2 e 3 anos, Entre 3 e 4 anos, Entre 4 e 5 anos, Entre 5 e 6 anos, Entre 6 e 7 anos, Entre 7 e 8 anos, Entre 8 e 9 anos, Entre 9 e 10 anos, Mais que 10 anos)}

\item Java é sua principal linguagem de programação? \textit{(Sim, Não)}

\item Como você avalia seus conhecimentos em Java? Fale um pouco sobre ele.	

\item Você conhece JUnit? \textit{(Sim, Não)}

\item Conhece o conceito de Mock Objects? \textit{(Sim e utilizo no meu dia a dia, Sim só na teoria, Não)}

\item Como você avalia seus conhecimentos em TDD? Fale um pouco sobre ele.	

\item Como você avalia seus conhecimentos em orientação a objetos e em design de sistemas OO? Fale um pouco sobre ele.	

\item Como você avalia sua experiência no processo de desenvolvimento de software em geral? Fale um pouco sobre ele.	

\end{enumerate}
\chapter{Exercícios}
\label{ape:exercicios}

Os exercícios são os mesmos para todos os grupos. O participante, em caso de dúvidas, poderá perguntar ao pesquisador.

\section{Lembrete ao participante}

Caro participante,

Lembre-se que os problemas aqui propostos simulam complicações do mundo real. 
Ao resolvê-los, tenha em mente que esses códigos serão futuramente mantidos
por você ou até por uma equipe maior.

Tente criar o design mais elegante possível em todas as soluções. Por serem problemas
recorrentes, imagine que amanhã esse mesmo problema se repetirá.
Escreva um código flexível o suficiente para que novas mudanças sejam fáceis de serem 
implementadas.

\section{Exercício 1 - Calculadora de Salário}

O participante deve implementar uma calculadora de salário de funcionários. Um
funcionário contém nome, e-mail, salário-base e cargo. De acordo com seu cargo,
a regra para cálculo do salário líquido é diferente:

\begin{enumerate}
	\item Caso o cargo seja DESENVOLVEDOR, o funcionário terá desconto de 20\%
	caso o salário seja maior ou igual que 3.000,00, ou apenas 10\% caso o salário seja menor 
	que isso.
	
	\item Caso o cargo seja DBA, o funcionário terá desconto de 25\%
	caso o salário seja maior ou igual que 2.000,00, ou apenas 15\% caso o salário seja menor 
	que isso.

	\item Caso o cargo seja TESTADOR, o funcionário terá desconto de 25\%
	caso o salário seja maior ou igual que 2.000,00, ou apenas 15\% caso o salário seja menor 
	que isso.
	
	\item Caso o cargo seja GERENTE, o funcionário terá desconto de 30\%
	caso o salário seja maior ou igual que 5.000,00, ou apenas 20\% caso o salário seja menor 
	que isso.
\end{enumerate}

Exemplos de cálculo do imposto:

\begin{itemize}
	\item DESENVOLVEDOR com salário-base 5,000.00. Salário final = 4.000,00
	\item GERENTE com salário-base de 2.500,00. Salário final: 2.000,00
	\item TESTADOR com salário de 550.00. Salário final: 467,50
\end{itemize}


O participante deve criar todo o código responsável para esse cálculo. Uma classe com
o método "main()" deverá ser entregue ao final, com exemplo de uso das classes criadas.

\section{Exercício 2 - Gerador de Nota Fiscal}

O participante deve implementar um sistema de geração de nota fiscal a partir de uma fatura. 
Uma fatura contém o nome e endereço do cliente, tipo do serviço e valor da fatura. O gerador de
nota fiscal deverá gerar uma nota fiscal que contém nome do cliente, valor da nota e valor
do imposto a ser pago.

O valor da nota é o mesmo do valor da fatura. Já o cálculo do imposto a ser pago deve seguir
as seguintes regras:

\begin{enumerate}
	\item Caso o serviço seja do tipo "CONSULTORIA", o valor do imposto é de 25%;
	\item Caso o serviço seja do tipo "TREINAMENTO", o valor do imposto é 15%;
	\item Qualquer outro, o valor do imposto é 6%.
\end{enumerate}

Ao final da geração da nota fiscal, o sistema ainda deve enviar essa nota por e-mail,
para o SAP, e persistir na base de dados. Por simplicidade, o desenvolvedor pode usar
os códigos abaixo, que simulam o comportamento do SMTP, SAP e banco de dados:

class NotaFiscalDao {
	public void salva(NotaFiscal nf) { System.out.println("salvando no banco"); }
}
class SAP {
	public void envia(NotaFiscal nf) { System.out.println("enviando pro sap"); }
}
class Smtp {
	public void envia(NotaFiscal nf) { System.out.println("enviando por email"); }
}

O participante é livre para alterar os métodos, parâmetros recebidos ou qualquer outra coisa das classes acima.

Ao final, o participante deve entregar todo o código responsável por geração e encaminhamento da nota fiscal 
para os processos acima citados. Uma classe com o método "main()" deverá ser entregue ao final, com
exemplo de uso das classes criadas.

\section{Exercício 3 - Processador de Boletos}

O participante deve implementar um processador de boletos. Esse processador receberá uma lista de boletos 
(que contém basicamente código do boleto, data e valor pago) e a fatura respectiva (que contém data, valor total e nome do cliente). 
O processador deve então, para cada boleto, criar um pagamento associado nessa fatura,
guardando o valor pago, a data e o tipo do pagamento (nesse caso, "BOLETO").
Além disso, caso a soma de todos os boletos ultrapasse o valor da fatura, a mesma deve ser marcada
como "PAGA".

O participante deve criar todo o código responsável pelo processador de boletos. Uma classe com
o método "main()" deverá ser entregue ao final, com exemplo de uso das classes criadas.

Exemplos de processamento:

\begin{itemize}
	\item Fatura de 1.500,00 com 3 boletos no valor de 500,00, 400,00 e 600,00: fatura marcada como PAGA, e três pagamentos do tipo BOLETO criados 
	\item Fatura de 1.500,00 com 2 boletos no valor de 500,00 e 400,00: fatura não marcada como PAGA, e dois pagamentos do tipo BOLETO criados 
\end{itemize}

\section{Exercício 4 - Saída do Quebra-Cabeça Numérico}

O participante deve de alguma forma imprimir a saída do quebra-cabeça numérico. Esse quebra-cabeça gera
uma sequência de números, que devem ser impressos no seguinte formato: "[1 -> 2 -> 3 -> 4 ->5 ->6]" (incluindo os colchetes).

O código do quebra-cabeça maluco encontra-se abaixo:

\begin{lstlisting}
public class QuebraCabecaNumerico {

	private int entrada;
	private int saida;
	private List<Numero> fila;
	private Set<Integer> visitados;
	private Numero solucao;
	
	public QuebraCabecaNumerico() {
		this.fila = new ArrayList<Numero>();
		this.visitados = new HashSet<Integer>();
	}

	public void geraCaminho(int entrada, int saida) {
		this.entrada = entrada;
		this.saida = saida;
		
		this.solucao = buscaSolucao();
	}
	
	private Numero buscaSolucao() {
		 
		adicionaRaizNaFila();
		
		while(existemNumerosNaFila()) {
			Numero numeroAtual = removeDaFila();
			
			if(encontrouSaida(numeroAtual)) return numeroAtual;
			adicionaNaFila(
				multiplicaPorDois(numeroAtual),
				(ehPar(numeroAtual)?dividePorDois(numeroAtual):null),
				somaDois(numeroAtual)
			);
		}
		
		return null;
	}

	private boolean ehPar(Numero numeroAtual) {
		return numeroAtual.getValor()%2==0;
	}

	private boolean encontrouSaida(Numero numeroAtual) {
		return numeroAtual.getValor() == saida;
	}

	private boolean existemNumerosNaFila() {
		return fila.size()!=0;
	}

	private void adicionaRaizNaFila() {
		fila.add(new Numero(entrada, null));
	}
	
	private void adicionaNaFila(Numero... numeros) {
		for(Numero numero : numeros) {
			if(numero!=null) {
				if(!visitados.contains(numero.getValor())) {
					fila.add(numero);
					visitados.add(numero.getValor());
				}
			}
		}
	}
	
	private Numero multiplicaPorDois(Numero numero) {
		return new Numero(numero.getValor()*2, numero);
	}

	private Numero dividePorDois(Numero numero) {
		return new Numero(numero.getValor()/2, numero);
	}
	
	private Numero somaDois(Numero numero) {
		return new Numero(numero.getValor()+2, numero);
	}

	private Numero removeDaFila() {
		Numero topoDaFila = fila.get(0);
		fila.remove(0);
		return topoDaFila;
	}

}

class Numero {
	private final int valor;
	private final Numero pai;
	
	public Numero(int valor, Numero pai) {
		this.valor = valor;
		this.pai = pai;
	}
	public int getValor() {
		return valor;
	}

	public Numero getPai() {
		return pai;
	}
}
\end{lstlisting}

Repare que o único método público existente "buscaSolucao()", invoca o algoritmo e guarda a solução
dentro do atributo "solucao". Um exemplo de código que visita a árvore de números gerada pelo algoritmo é:

\begin{lstlisting}
	while(solucao!=null) {
		int valor = solucao.getValor(); // esse eh o valor a ser impresso
		solucao = solucao.getPai();
	}
\end{lstlisting}

Exemplos de saídas do algoritmo:

\begin{itemize}
	\item Entrada: 2, 2 Saída: [2]
	\item Entrada: 2, 4 Saída: [2 -> 4]
	\item Entrada: 2,10 Saída: [2 -> 4 -> 8 -> 10]
	\item Entrada: 3, 10 Saída: [3 -> 5 -> 10]
\end{itemize}

O participante deve criar todo o código responsável pela saída do quebra-cabeça numérico. Uma classe com
o método "main()" deverá ser entregue ao final, com exemplo de uso das classes criadas.


\chapter{Questionário pós-experimento}
\label{ape:questionario-pos}

\chapter{Entrevista}
\label{ape:entrevista}

\section{Dados básicos}

\begin{enumerate}
	\item Nome completo?
	\item Empresa em que atua?
	\item E-mail para contato?
	\item Autoriza o uso da entrevista para a pesquisa?
	\item Tempo total de entrevista?
\end{enumerate}

\section{Caracterização do desenvolvedor}

\begin{enumerate}
	\item Qual a sua formação?
	\item Há quanto tempo atua na área de desenvolvimento de software?
	\item Você desenvolve sistemas orientados a objetos há quanto tempo?	
	\item Fale-me um pouco sobre os projetos que já desenvolveu e os desafios neles encontrados?
	\item Em sua opinião, qual a maior dificuldade na criação de sistemas OO de qualidade?
	\item Com que frequência você costuma ler livros ou ir a congressos para se atualizar sobre as novas práticas? Quais?
\end{enumerate}

\section{A Prática de TDD}

\begin{enumerate}
	\item Você pratica TDD há quanto tempo?
	\item Como tem sido sua relação com a prática?
	\item O que é TDD em sua opinião?
	\item Poderia me explicar como você pratica TDD no dia-a-dia?
	\item Você pratica TDD o tempo todo?
	\item Em sua opinião, quais as vantagens de praticar TDD?
\end{enumerate}

\section{Relação entre TDD e Design}

\begin{enumerate}
	\item{Em sua experiência, você acredita que classes muito acopladas são realmente prejudiciais?}
		\begin{enumerate}
			\item Como você combate esse problema?
			\item TDD te ajuda de alguma forma? Como?
			\item Você acha que mesmo com TDD é possível criar classes com alto acoplamento? 
		\end{enumerate}
	\item{E classes com baixa coesão? Você costuma ver muito?}
		\begin{enumerate}
			\item Em sua opinião, por que elas aparecem?
			\item Você acha que TDD te ajudaria a resolver esse problema? Como?
		\end{enumerate}
	\item Você acredita que TDD influencia na qualidade do design de classes?
	\begin{enumerate}
		\item Como ele influencia no seu design?
		\item Escrever o teste antes ajuda nesse processo?
		\item E por que esse efeito não aconteceria caso você escrevesse o teste depois?
		\item Como seria criar designs sem a ajuda dos testes?
		\item Poderia dar um exemplo (usando código se preferir) de como o TDD influencia o seu design? Algum outro exemplo?
		\item Além de receber as dependências pelo construtor, você vê alguma outra maneira aonde TDD te ajude?
			\begin{enumerate}
				\item Um problema famoso ao começar a praticar TDD é o construtor começar a receber muitos parâmetros. Quando você começa a perceber isso?
				\item E como resolve esse problema?
			\end{enumerate}
	\end{enumerate}
	\item Sabemos que classes com muitos métodos ou métodos com muitas linhas é prejudicial para o código. Como você faz para não escrever classes com esses problemas?
	\item Você conhece os princípios SOLID do Robert Martin? Se sim,
	\begin{enumerate}
		\item você sempre os utilizou?
		\item TDD teve alguma influencia nisso?
	\end{enumerate}
\end{enumerate}

\section{Relação entre TDD e experiência}

\begin{enumerate}
	\item Como você compararia as primeiras vezes que você praticou TDD com agora?
	\item Você acha que a experiência influencia no resultado final?
	\item Em sua opinião, como você acha que seria o design de uma pessoa sem experiência em desenvolvimento de software e sem experiência em OO, mas praticando TDD?
	\item O programador às vezes comete desvios na prática, como não ver o teste falhar, esquecer de refatorar, não rodar a bateria de testes completa, refatorar outro trecho de código enquanto está no vermelho, e etc. Você vê isso como um problema?
	\item TDD fala sobre \textit{baby steps} (ou passos de bebê). O que você entende por dar passos de bebê?
	\begin{enumerate}
		\item É muito comum, principalmente em atividades como dojos, as pessoas ficarem criando if's e mais if's (que é o código mais simples possível que faz o teste passar), até o momento em que o código se torna o complexo suficiente para merecer uma refatoração. O que pensa sobre isso?
	\end{enumerate}
\end{enumerate}

\section{TDD e outras práticas ágeis}

\begin{enumerate}
	\item Você acha que outras práticas ágeis influenciam na qualidade do seu design?
	\item Vocês praticam programação pareada? Com que frequência? Quando vocês fazem programação pareada, você acha que isso ajuda mais no design do que TDD?
	\item Se só praticassem programação pareada, por exemplo, você acha que mesmo assim precisaria de TDD?
	\item Quais outras práticas ágeis você acha que ajuda você a melhorar o design?
\end{enumerate}

\section{TDD e experiência pessoal}

(Nesse ponto o entrevistado poderá mostrar algum trecho de código do projeto em que atua.)

\begin{enumerate}
	\item Tem algum exemplo em mente aonde TDD ajudou a resolver o problema de maneira elegante?
	\item E aonde TDD atrapalhou?
		\begin{enumerate}
			\item Você acha que TDD se aplica à todos os casos? Se não, em quais ele não se aplica e porquê?
		\end{enumerate}
	\item Poderia me mostrar algum trecho de código que você acha que possui um bom design e me explicar o porquê da sua opinião?
		\begin{enumerate}
			\item Você fez esse trecho de código usando TDD?
			\item Você estava sozinho ou pareando com outro desenvolvedor?
			\item Esse design surgiu naturalmente ou tiveram problemas?
		\end{enumerate}
\end{enumerate}

\section{TDD e produtividade}
\begin{enumerate}
	\item Na sua opinião, qual a relação entre TDD e produtividade?
	\item Algumas pessoas dizem que escrever o teste gasta muito tempo. Outros dizem que não. O que você acha?
	\item Você vê alguma relação entre TDD e o sucesso de um produto?
\end{enumerate}

\section{Opiniões finais}

\begin{enumerate}
	\item Você acha que TDD resolve de vez o problema de designs que degradam ao longo do tempo?
	\item O que mais você faria para tentar resolver esse problema?
	\item Existem algumas críticas em relação à TDD, como "TDD é improdutivo, afinal você gasta muito tempo escrevendo os testes". O que você acha disso?
		\begin{enumerate}
			\item A longo prazo, você vê vantagens em gastar esse tempo utilizando TDD?
		\end{enumerate}
	\item Gostaria de dizer mais algo sobre TDD que não disse nas perguntas anteriores?
	\item Com quem posso falar para saber mais sobre o assunto?
\end{enumerate}
\chapter{Informações ao partipante}
\label{ape:informacoes-participante}

\section{Convite}

Meu nome é Mauricio Aniche. Sou aluno de mestrado em Ciência da Computação pelo
Instituto de Matemática e Estatística da Universidade de São Paulo (USP).
Atualmente pesquiso a influência de Test-Driven Development no design de
sistemas orientados a objetos.

Para alcançar esse objetivo, estou realizando entrevistas com desenvolvedores de
diversas empresas do mercado brasileiro que já praticam TDD há pelo menos 1 ano.
Este convite permite a você compartilhar suas experiências e sentimentos em
relação à prática e cooperar com as pesquisas na área.

É importante reforçar que a participação é totalmente voluntária e não há nenhum
tipo de remuneração associada. Você pode desistir da sua participação sem nenhum
tipo de consequência.

\section{Qual o objetivo desta pesquisa?}

O objetivo desta pesquisa é entender de maneira mais profunda como TDD
influencia no design de sistemas orientados a objetos. Essas informações serão
capturadas baseadas na percepção de programadores que realizam a prática no seu
dia-a-dia de trabalho. Além disso, essa pesquisa se propõe a comparar os
resultados encontrados com os dados existentes na literatura.

\section{Qual meu papel dentro dela?}

Como participante dessa pesquisa, você deverá participar das entrevistas. Elas
ocorrerão dento do seu ambiente de trabalho e serão gravadas. Além disso, o
pesquisador poderá fazer uso de anotações durante esse período.

\section{Quais são os benefícios?}

Além de cooperar com o avanço da pesquisa na área de engenharia de software, os
resultados obtidos por essa pesquisa são compartilhadas com você, e eu espero
que as informações ali contidas possam ser úteis para a evolução da técnica.

\section{Minha privacidade será garantida?}

Sim, todas as informações gravadas serão mantidas em completo sigilo. Apenas os
pesquisadores participantes desse trabalho terão acesso ao mesmo.

Além disso, nenhum nome será revelado no resultado final da pesquisa.

\section{Quais são os custos de participação na pesquisa?}

Nenhum. O pesquisador precisará de 1 hora para a realização da entrevista. Caso
uma nova entrevista seja necessária, ele marcará a mesma com antecedência. 

\section{Em caso de dúvidas, o que devo fazer?}

Em caso de dúvida, favor contatar o pesquisador ou o orientador dessa pesquisa.

Mauricio Finavaro Aniche (aniche@ime.usp.br)

Marco Aurélio Gerosa (gerosa@ime.usp.br) 

Departamento de Ciência da Computação - Instituto de Matemática e Estatística - 
Universidade de São Paulo (USP) - Caixa Postal 66.281 - 05.508-090 - São Paulo -
SP  - Brasil


\chapter{Autorização}
\label{ape:autorizacao}

\section{Consentimento de Participação na Pesquisa}

Caro participante, por favor preencha atentamente as instruções abaixo:

\begin{itemize}
  \item (  ) Eu recebi, li e entendi as informações sobre essa pesquisa;
  \item (  ) Eu tive a oportunidade de tirar dúvidas sobre a pesquisa;
  \item (  ) Eu entendo que anotações serão feitas durantes as entrevistas e que
  elas serão gravadas e transcritas;
  \item (  ) Eu entendo que eu posso desistir da minha participação ou de
  qualquer informação que eu provi a qualquer momento antes da finalização do processo de
  coleta de dados, sem qualquer tipo de dano ou perda;
  \item (  ) Eu entendo que, em caso de desistência, a gravação, transcrição ou
  qualquer outra informação persistida será destruída;
  \item (  ) Eu aceito fazer parte desta pesquisa;
  \item (  ) Eu gostaria de receber uma cópia do resultado final da pesquisa
\end{itemize}

Assine este documento, informando seu nome e data corrente.
\chapter{Valores Utilizados na Análise Quantitativa}
\label{ape:valores}

Neste apêndice, listamos os valores utilizados durante a 
análise quantitativa. Para facilitar a visualização, nós os separamos em duas tabelas.
Na Tabela \ref{tab:valores-academia}, listamos os valores da academia, e na Tabela
\ref{tab:valores-industria} os valores da indústria.

\begin{longtable}{| p{2cm} | p{6.5cm} | p{6.5cm} | }
		\hline
		Métrica & Valores com TDD & Valores sem TDD\\
		\hline
		Complexi- dade Ciclomática &	
		2, 9, 1, 8, 1, 3, 1, 1, 1, 0, 5, 9, 2, 1, 3, 0, 6, 8, 6, 5, 0, 0, 0, 0, 10, 3, 3, 1, 0, 2, 1, 7, 7, 9, 7, 3, 14, 13 &	
		
		3, 0, 5, 10, 0, 3, 0, 1, 22, 6, 5, 0, 5, 4, 2, 2, 11, 2, 2, 4, 4, 4, 4, 7, 9, 10, 3, 1, 9, 2, 7, 3, 7, 8, 7, 7, 0, 1, 0, 0, 12, 4, 7, 6, 12, 2, 7, 7, 9, 7, 7, 2, 12, 4, 9, 0, 1, 1, 1, 2\\
		
		\hline
		
		Falta de Coesão dos Métodos &	
		1, 0.722222222, 1, 0.8, 1, 0.666666667, 1, 1, 1, 1, 0.266666667, 0.694444444, 0.5, 1, 1, 1, 0.666666667, 0.75, 0.666666667, 0, 1, 1, 1, 1, 0.7, 1, 0, 1, 1, 0.75, 1, 0.571428571, 0.571428571, 0.694444444, 0.619047619, 1, 0.25, 1 &	
		
		0.666666667, 1, 0.6, 1, 1, 1, 1, 0, 0.85, 0.761904762, 0.6, 1, 0.6, 1, 1, 1, 0.783333333, 1, 1, 0, 0, 0, 0, 0.571428571, 0.666666667, 1, 0.666666667, 1, 0.694444444, 0.625, 0.714285714, 1, 0.571428571, 0.75, 0.571428571, 0.5, 1, 1, 1, 1, 1, 0.75, 0.571428571, 0, 0.65, 1, 0.571428571, 0.571428571, 0.666666667, 0.571428571, 0.4, 1, 0.75, 0.75, 0.777777778, 1, 1, 1, 1, 1\\
		
		\hline
		
		Linhas por Método &	
		2, 2, 2, 2, 2, 2, 2, 2, 2, 5, 2, 2, 9, 2, 4, 2, 2, 2, 2, 2, 5, 3, 4, 5, 5, 2, 2, 2, 2, 2, 2, 2, 2, 3, 7, 3, 3, 2, 2, 2, 2, 2, 2, 2, 2, 2, 2, 2, 2, 2, 2, 2, 2, 2, 2, 2, 2, 18, 2, 2, 2, 2, 2, 2, 2, 2, 7, 6, 6, 2, 3, 5, 2, 2, 2, 2, 2, 2, 2, 2, 2, 2, 2, 2, 2, 2, 2, 2, 2, 3, 2, 2, 2, 2, 2, 2, 2, 2, 10, 26, 30
		 &	
		
		2, 2, 2, 2, 2, 2, 2, 9, 11, 13, 11, 11, 9, 2, 2, 2, 2, 2, 39, 2, 2, 2, 9, 9, 9, 27, 9, 3, 2, 2, 2, 2, 2, 2, 2, 2, 7, 3, 3, 2, 2, 2, 0, 2, 2, 7, 2, 2, 3, 3, 4, 4, 4, 15, 4, 4, 4, 15, 4, 4, 4, 15, 4, 4, 4, 15, 2, 2, 2, 2, 2, 2, 2, 2, 2, 2, 2, 2, 2, 2, 24, 2, 2, 2, 13, 2, 2, 2, 2, 2, 2, 2, 2, 2, 2, 2, 2, 2, 2, 2, 7, 16, 5, 2, 2, 2, 2, 2, 2, 2, 2, 2, 2, 2, 2, 2, 2, 2, 2, 2, 2, 2, 2, 9, 5, 5, 11, 30, 2, 9, 2, 8, 2, 2, 2, 2, 2, 2, 2, 2, 2, 2, 3, 2, 2, 2, 2, 12, 2, 2, 2, 2, 2, 2, 7, 2, 2, 2, 2, 2, 2, 2, 2, 2, 2, 2, 2, 2, 2, 2, 2, 2, 2, 2, 2, 2, 2, 2, 2, 2, 2, 2, 2, 2, 2, 13, 2, 2, 2, 2, 2, 2, 2, 2, 2, 2, 2, 6, 2, 3, 11, 2, 2, 2, 2, 2, 5, 5
		\\
		
		\hline
		
		\textit{Fan-Out} &	
		1, 1, 3, 2, 2, 1, 1, 1, 1, 0, 4, 2, 2, 4, 1, 0, 1, 1, 1, 5, 0, 3, 1, 0, 4, 3, 5, 1, 0, 2, 2, 1, 0, 2, 1, 4, 1, 2 &	
		
		3, 0, 3, 4, 0, 4, 3, 3, 1, 2, 3, 0, 2, 3, 2, 2, 1, 2, 2, 3, 3, 3, 3, 2, 2, 3, 1, 2, 2, 3, 3, 6, 3, 3, 3, 7, 0, 8, 2, 3, 5, 3, 0, 2, 4, 4, 1, 1, 2, 1, 5, 7, 2, 5, 3, 0, 1, 1, 1, 3\\
		
				\hline
				
		Quantidade de Métodos &	1, 8, 1, 4, 1, 2, 1, 1, 1, 0, 5, 8, 1, 1, 2, 0, 6, 8, 6, 1, 0, 0, 0, 0, 8, 1, 2, 1, 0, 1, 1, 6, 6, 8, 6, 1, 1, 1 &	
		3, 0, 4, 5, 0, 1, 0, 0, 9, 6, 4, 0, 4, 1, 1, 1, 9, 1, 1, 4, 4, 4, 4, 6, 8, 1, 3, 1, 8, 1, 6, 3, 6, 8, 6, 3, 0, 1, 0, 0, 3, 4, 6, 5, 9, 1, 6, 6, 8, 6, 4, 1, 8, 4, 5, 0, 1, 1, 1, 2\\
		
		\hline
	\caption{Valores da academia utilizados no teste estatística}
	\label{tab:valores-academia}
\end{longtable}

\begin{longtable}{| p{2cm} | p{6.5cm} | p{6.5cm} | }
		\hline
		Métrica & Valores com TDD & Valores sem TDD\\
		\hline
		Complexi- dade Ciclomática &	
		9, 0, 8, 12, 7, 12, 6, 4, 0, 5, 8, 0, 0, 2, 5, 6, 0, 3, 4, 0, 1, 7, 4, 6, 8, 3, 0, 2, 9, 1, 8, 6, 12, 6, 2, 7, 10, 8, 8, 6, 3, 5, 1, 1, 14, 1, 6, 1, 1, 1, 6, 9, 1, 1, 1, 1, 4, 8, 0, 7, 4, 0, 9, 3, 3, 0, 1, 9, 3, 1, 1, 2, 2, 1, 5, 1, 3, 0, 2, 1, 3, 1, 1, 1, 5, 7, 3, 0, 9, 8, 3, 9, 0, 4, 5, 4, 0, 5, 4, 1, 3, 3, 0, 2, 4, 3, 3, 4, 5, 5, 1, 6, 1, 1, 1, 1, 1, 1, 1, 4, 6, 6, 3, 2, 0, 4, 3, 3, 8, 1, 4, 2, 2, 7, 8, 3, 10, 3, 0, 4, 7, 11, 8, 0, 0, 1, 3, 2, 9, 13, 1, 7, 6, 8, 3, 2, 5, 2&
		
		7, 12, 6, 8, 6, 0, 1, 9, 1, 1, 7, 0, 1, 8, 16, 0, 11, 4, 4, 14, 1, 1, 8, 0, 3, 0, 0, 1, 1, 2, 6, 1, 3, 1, 1, 6, 8, 1, 4, 6, 10, 1, 8, 6, 5, 8, 9, 4, 0, 9, 8, 11, 6, 3, 8, 6, 2, 14, 1, 5, 6, 3, 0, 0, 9, 0, 15, 9, 1, 8, 4, 5, 9, 1, 1, 1, 1, 3, 5, 4, 4, 3, 4, 0, 5, 0, 1, 2, 8, 6, 10, 0, 24, 2, 1, 4, 6, 4, 0, 1, 1, 1, 1, 1, 5, 2, 2, 3, 4, 1, 3, 1, 1, 1, 3, 2, 2, 3, 1, 4, 1, 2, 2, 2, 5, 4, 6, 1, 5, 0, 0, 2, 7, 12, 7, 3, 3, 0, 12, 0, 3, 1, 1, 1, 1, 1, 5, 5, 5, 1, 2, 9, 8, 8, 7, 1, 1, 1, 10, 1, 9, 1, 1, 1, 0, 1, 3, 10, 1, 1, 2, 2, 2, 2, 8, 1, 1, 1, 1, 19, 4, 6, 10, 8, 10, 3\\
		
		\hline
		
		Falta de Coesão dos Métodos &	
		1, 1, 0.75, 1, 0.571428571, 0.76, 0.666666667, 1, 1, 0.375, 1, 1, 1, 1, 1, 1, 1, 0.5, 1, 1, 1, 1, 0.5, 0.666666667, 0.75, 1, 1, 1, 1, 1, 0.75, 0.666666667, 0.833333333, 0.666666667, 0.5, 1, 0.4, 0.75, 0.75, 0.666666667, 1, 0.666666667, 1, 1, 0.819444444, 1, 0.666666667, 1, 1, 1, 0.666666667, 0.694444444, 1, 1, 1, 1, 1, 0.75, 1, 1, 0, 1, 0.722222222, 1, 1, 1, 1, 0.722222222, 1, -1, 1, 0.5, 0.5, 0, 0.6, -1, 0.5, 1, 0, 1, 0, 1, 1, 1, 0.4, 0.571428571, 0, 1, 0.8, 0.7, 1, 0.777777778, 1, 1, 1, 0.5, 1, 0.6, 0, -1, 0, 0, 1, 1, 1, 0.333333333, 0.333333333, 1, 1, 0.75, 1, 0.777777778, 1, 1, 1, -1, 1, 1, 1, 1, 0.666666667, 0.375, 0.5, 1, 1, 1, 0.333333333, 0.5, 1, 1, 1, 0.5, 1, 0.761904762, 1, 0, 0.583333333, 0, 1, 1, 0.571428571, 0.85, 0.625, 1, 1, 1, 1, 1, 0.666666667, 0.795454545, 1, 1, 0.666666667, 0.75, 1, 1, 1, 1 &
		
		0.714285714, 0.861111111, 0.666666667, 1, 1, 1, 1, 0.666666667, 1, 1, 0.714285714, 1, 1, 1, 0.333333333, 1, 0.727272727, 1, 0.625, 0.742857143, -1, -1, 0.8, 1, 0.666666667, 1, 1, 1, 1, 1, 0.666666667, 1, 0.5, 1, 1, 0.666666667, 0.75, 1, 1, 0.666666667, 0.777777778, 1, 0.5, 0.666666667, 0.6, 0.75, 1, 0, 1, 0.714285714, 0.75, 1, 0.625, 0.5, 0.8, 0.666666667, 0, 0.816666667, 1, 1, 0.666666667, 1, 1, 1, 0.714285714, 1, 0.846153846, 1, 1, 0.833333333, 0.5, 0.628571429, 0.666666667, 1, 1, 1, -1, 1, 0, 0, 0, 0, 0.583333333, 1, 0.6, 1, 1, 0, 0.666666667, 0.5, 0.7, 1, 0.625, 0, 1, 0.5, 0.666666667, 0.5, 1, 1, -1, -1, 1, -1, 0.5, 0.5, 0.125, 1, 0.5, -1, 0, 1, 1, 1, 0.333333333, 1, 1, 1, -1, 1, 1, 1, 1, 1, 1, 0.75, 0.75, 0, 0.166666667, 1, 1, 0, 0.571428571, 0.8, 0.761904762, 0, 1, 1, 1, 1, 0.5, 1, 1, 1, 1, 1, 0.55, 0.6, 0.6, 1, 1, 0.666666667, 1, 1, 0.714285714, 1, 1, 1, 0.7, 1, 0.740740741, 1, 1, 1, 1, 1, 1, 0.7, 1, 1, 1, 1, 1, 1, 1, 1, 1, 1, 1, 0.8, 1, 0.666666667, 0.8, 0.75, 0.571428571, 1\\
		
		\hline
		
		Linhas por Método &	
		18, 2, 2, 2, 2, 2, 2, 2, 2, 11, 11, 11, 11, 11, 11, 11, 11, 11, 11, 11, 11, 2, 2, 2, 2, 2, 2, 2, 2, 2, 2, 2, 18, 2, 2, 2, 2, 2, 2, 2, 2, 2, 18, 17, 2, 2, 2, 2, 10, 2, 6, 2, 12, 9, 4, 7, 7, 4, 4, 12, 11, 2, 2, 6, 6, 6, 6, 2, 5, 8, 2, 2, 2, 2, 2, 2, 2, 2, 2, 2, 2, 2, 2, 2, 2, 2, 2, 2, 10, 18, 51, 8, 2, 2, 2, 2, 2, 2, 2, 2, 2, 2, 2, 2, 2, 2, 2, 2, 2, 2, 2, 2, 2, 2, 2, 2, 2, 2, 2, 2, 2, 2, 2, 2, 2, 2, 8, 9, 8, 6, 5, 5, 5, 5, 15, 4, 15, 5, 5, 13, 2, 2, 2, 2, 2, 2, 2, 2, 2, 2, 2, 2, 2, 2, 2, 2, 2, 2, 2, 2, 2, 2, 17, 2, 2, 2, 2, 17, 2, 2, 2, 2, 2, 2, 2, 2, 2, 2, 8, 2, 2, 11, 2, 2, 2, 2, 2, 2, 2, 2, 8, 2, 2, 2, 2, 2, 2, 2, 2, 2, 2, 2, 2, 2, 2, 3, 3, 10, 4, 4, 5, 4, 7, 2, 2, 2, 2, 2, 2, 2, 2, 1, 7, 2, 13, 8, 8, 8, 8, 2, 2, 2, 2, 2, 2, 2, 2, 8, 8, 7, 0, 2, 2, 2, 2, 2, 2, 2, 8, 0, 3, 4, 5, 2, 2, 2, 2, 0, 2, 2, 3, 4, 17, 4, 4, 4, 3, 2, 2, 2, 2, 2, 2, 2, 2, 2, 2, 2, 2, 2, 2, 2, 2, 10, 2, 14, 2, 6, 6, 6, 19, 5, 15, 6, 4, 8, 2, 2, 2, 2, 2, 2, 2, 2, 0, 2, 8, 3, 8, 8, 3, 3, 3, 2, 2, 2, 2, 2, 2, 2, 2, 2, 6, 0, 6, 4, 8, 2, 4, 4, 4, 2, 9, 14, 11, 9, 4, 5, 5, 4, 5, 5, 5, 5, 5, 2, 2, 2, 2, 13, 2, 2, 2, 2, 2, 2, 2, 2, 0, 2, 2, 2, 6, 6, 6, 6, 2, 2, 2, 2, 2, 2, 2, 7, 2, 2, 5, 15, 11, 11, 12, 2, 2, 2, 2, 2, 9, 2, 5, 3, 17, 10, 22, 2, 5, 4, 2, 2, 2, 2, 2, 2, 4, 4, 5, 4, 4, 4, 6, 5, 2, 2, 2, 2, 2, 2, 2, 9, 2, 2, 2, 17, 18, 2, 2, 2, 2, 2, 2, 2, 2, 2, 2, 2, 2, 2, 4, 2, 2, 2, 2, 2, 2, 2, 8, 9, 8, 8, 8, 6, 2, 2, 2, 2, 2, 2, 2, 2, 8, 2, 2, 2, 2, 4, 2, 2, 2, 2, 13, 6, 8, 10, 6, 10, 6, 6, 2, 2, 2, 2, 2, 2, 2, 2, 2, 2, 2, 2, 2, 2, 10, 17, 4, 5, 4, 5, 5, 16, 4, 40 &
			
2, 6, 2, 2, 2, 2, 2, 2, 2, 2, 2, 2, 2, 2, 2, 2, 2, 2, 2, 2, 2, 2, 2, 2, 2, 10, 27, 21, 21, 18, 2, 2, 2, 2, 2, 2, 2, 2, 2, 2, 2, 2, 2, 10, 2, 2, 2, 2, 2, 7, 39, 2, 2, 2, 2, 6, 15, 3, 2, 2, 2, 2, 2, 2, 3, 2, 2, 4, 4, 4, 4, 2, 2, 2, 2, 2, 2, 2, 6, 15, 0, 0, 9, 3, 7, 9, 11, 0, 3, 4, 2, 2, 2, 2, 2, 3, 2, 2, 7, 3, 2, 2, 2, 2, 2, 2, 2, 2, 2, 2, 2, 2, 2, 2, 16, 15, 2, 2, 2, 2, 2, 2, 2, 2, 6, 2, 2, 2, 2, 2, 2, 6, 5, 3, 6, 5, 5, 5, 5, 6, 2, 2, 2, 2, 2, 2, 2, 2, 2, 2, 2, 2, 2, 2, 2, 2, 2, 16, 4, 4, 4, 4, 2, 2, 2, 2, 2, 2, 2, 2, 2, 2, 2, 2, 2, 2, 2, 23, 2, 7, 8, 2, 6, 3, 5, 2, 2, 2, 2, 2, 2, 2, 2, 2, 2, 2, 2, 2, 2, 2, 2, 2, 2, 2, 2, 5, 2, 5, 2, 2, 2, 2, 6, 6, 17, 6, 21, 2, 2, 2, 2, 2, 2, 15, 2, 2, 2, 2, 2, 2, 2, 2, 3, 2, 3, 2, 2, 2, 2, 2, 2, 2, 2, 2, 20, 21, 2, 2, 2, 2, 2, 2, 2, 2, 2, 15, 2, 2, 16, 17, 6, 4, 16, 2, 9, 2, 2, 2, 2, 2, 2, 2, 2, 0, 8, 8, 8, 5, 4, 2, 2, 2, 2, 2, 2, 2, 2, 3, 18, 11, 2, 2, 2, 2, 2, 2, 2, 2, 2, 2, 2, 2, 2, 2, 2, 2, 2, 2, 8, 16, 3, 3, 8, 16, 28, 41, 26, 3, 2, 2, 2, 2, 2, 2, 2, 2, 2, 2, 2, 2, 2, 2, 2, 3, 0, 0, 3, 0, 11, 3, 3, 15, 4, 21, 2, 2, 2, 0, 4, 2, 2, 2, 3, 3, 3, 3, 3, 3, 4, 0, 5, 5, 6, 5, 12, 4, 4, 4, 4, 4, 4, 4, 4, 5, 4, 4, 2, 2, 2, 2, 2, 2, 2, 2, 2, 9, 3, 14, 2, 2, 2, 2, 2, 2, 2, 6, 2, 2, 2, 2, 2, 2, 4, 2, 2, 2, 2, 2, 2, 4, 13, 14, 12, 2, 2, 2, 2, 9, 2, 2, 4, 12, 2, 2, 19, 2, 2, 2, 2, 2, 2, 2, 2, 2, 2, 2, 2, 8, 8, 2, 2, 2, 2, 2, 2, 2, 2, 2, 4, 7, 5, 7, 7, 7, 6, 7, 6, 5, 2, 2, 2, 2, 2, 2, 3, 3, 3, 2, 2, 2, 2, 2, 2, 2, 2, 11, 2, 2, 2, 2, 2, 2, 2, 2, 2, 2, 9, 7, 4, 4, 4, 2, 2, 2, 2, 2, 2, 2, 2, 2, 2, 0, 5, 5, 5, 5, 4, 4, 4, 4, 4, 4, 4, 4, 4, 2, 4, 7, 2, 2, 3, 2, 2, 2, 2, 23, 2, 2, 2, 2, 10, 10, 10, 10, 2, 2, 2, 2, 2, 2, 2, 2, 2, 2, 2, 2, 2, 2, 2, 2, 2, 2, 2, 2, 2, 2, 2, 2, 5, 10, 2, 2, 2, 12, 2, 29, 23, 7
		\\
		
		\hline
		
		\textit{Fan-Out} &	
		2, 0, 3, 4, 3, 7, 3, 8, 4, 2, 4, 3, 0, 4, 5, 3, 0, 3, 4, 2, 2, 3, 2, 2, 3, 4, 0, 7, 3, 4, 2, 1, 4, 2, 0, 6, 6, 4, 3, 3, 3, 1, 6, 1, 1, 3, 1, 1, 1, 4, 2, 3, 0, 3, 5, 3, 6, 2, 0, 1, 3, 0, 2, 3, 3, 0, 4, 2, 3, 1, 2, 2, 2, 1, 3, 1, 2, 2, 3, 10, 3, 1, 1, 1, 1, 3, 1, 0, 5, 6, 2, 3, 1, 4, 3, 3, 1, 2, 4, 1, 2, 2, 0, 5, 3, 6, 8, 5, 3, 2, 5, 2, 1, 1, 1, 0, 1, 1, 1, 7, 2, 2, 1, 3, 0, 4, 2, 2, 4, 4, 8, 2, 4, 1, 4, 0, 4, 1, 0, 6, 3, 7, 4, 0, 0, 5, 5, 4, 1, 2, 3, 4, 2, 2, 2, 4, 5, 6 &	
		
		4, 7, 3, 4, 6, 0, 1, 3, 1, 1, 6, 1, 4, 4, 3, 0, 4, 4, 3, 6, 1, 1, 9, 0, 5, 2, 2, 2, 2, 3, 2, 2, 1, 6, 2, 1, 2, 5, 3, 0, 2, 1, 4, 3, 1, 2, 3, 2, 0, 2, 2, 3, 1, 3, 1, 3, 1, 7, 2, 10, 3, 4, 0, 0, 3, 0, 2, 3, 5, 2, 5, 8, 2, 1, 1, 1, 1, 3, 3, 4, 2, 2, 3, 1, 3, 0, 2, 12, 4, 1, 3, 1, 5, 9, 2, 1, 3, 2, 1, 2, 1, 1, 2, 1, 2, 10, 3, 11, 1, 1, 2, 2, 2, 2, 2, 2, 2, 5, 0, 6, 4, 3, 3, 3, 5, 2, 3, 3, 6, 0, 0, 6, 2, 6, 3, 3, 7, 1, 1, 0, 1, 2, 5, 1, 1, 5, 2, 2, 0, 1, 4, 1, 3, 4, 2, 1, 1, 1, 2, 5, 3, 1, 1, 1, 0, 3, 2, 3, 2, 1, 2, 2, 2, 2, 6, 3, 0, 3, 2, 1, 2, 1, 2, 2, 4, 5\\
		
				\hline
				
		Quantidade de Métodos &	
		1, 0, 8, 12, 6, 9, 6, 2, 0, 3, 4, 0, 0, 2, 5, 2, 0, 2, 4, 0, 1, 2, 4, 6, 8, 1, 0, 1, 1, 1, 8, 6, 12, 6, 2, 4, 10, 8, 8, 6, 1, 4, 1, 1, 12, 1, 6, 1, 1, 1, 6, 8, 1, 1, 1, 1, 4, 8, 0, 4, 4, 0, 8, 1, 1, 0, 1, 8, 1, 1, 1, 1, 1, 0, 4, 1, 2, 0, 1, 1, 3, 1, 1, 1, 4, 6, 2, 0, 6, 8, 1, 9, 0, 4, 4, 3, 0, 4, 1, 1, 1, 1, 0, 2, 3, 2, 3, 4, 5, 4, 1, 5, 1, 1, 1, 1, 1, 1, 1, 4, 5, 3, 2, 1, 0, 4, 2, 2, 3, 1, 4, 1, 2, 6, 8, 2, 7, 2, 0, 2, 6, 9, 6, 0, 0, 1, 3, 2, 8, 10, 1, 7, 6, 8, 1, 2, 5, 2 &	
		
		7, 12, 6, 2, 3, 0, 1, 8, 1, 1, 7, 0, 1, 1, 7, 0, 9, 4, 3, 6, 1, 1, 5, 0, 1, 0, 0, 1, 1, 1, 5, 1, 2, 1, 1, 6, 8, 1, 1, 6, 9, 1, 8, 6, 3, 8, 1, 4, 0, 7, 8, 1, 4, 3, 8, 6, 1, 12, 1, 4, 6, 1, 0, 0, 7, 0, 13, 1, 1, 8, 4, 5, 7, 1, 1, 1, 1, 1, 1, 1, 1, 1, 3, 0, 4, 0, 1, 2, 6, 4, 8, 0, 7, 2, 1, 4, 6, 4, 0, 1, 1, 1, 1, 1, 3, 1, 1, 1, 3, 1, 1, 1, 1, 1, 2, 2, 2, 1, 1, 4, 1, 2, 2, 2, 5, 3, 5, 0, 2, 0, 0, 2, 6, 9, 6, 1, 3, 0, 5, 0, 2, 1, 1, 1, 1, 1, 4, 3, 4, 1, 2, 8, 3, 8, 6, 1, 1, 1, 8, 1, 7, 1, 1, 1, 0, 1, 3, 9, 1, 1, 1, 1, 1, 1, 8, 1, 1, 1, 1, 12, 4, 6, 10, 8, 7, 3
		\\
		
		\hline
	\caption{Valores da indústria utilizados no teste estatística}
	\label{tab:valores-industria}
\end{longtable}


% ---------------------------------------------------------------------------- %
% Bibliografia
\backmatter \singlespacing   % espaçamento simples
\bibliographystyle{alpha-ime}% citação bibliográfica alpha
\bibliography{bibliografia}  % associado ao arquivo: 'bibliografia.bib'


\end{document}
