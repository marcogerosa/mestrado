% ---------------------------------------------------------------------------- %
% Mauricio Finavaro Aniche                                                     %
% mauricioaniche@gmail.com                                                     %
% ---------------------------------------------------------------------------- %

\documentclass[11pt,openany,twoside,a4paper]{book}

% ---------------------------------------------------------------------------- %
% Pacotes 
\usepackage[T1]{fontenc}
\usepackage[brazil]{babel}
\usepackage[utf8]{inputenc}
\usepackage[pdftex]{graphicx}           % usamos arquivos pdf/png como figuras
\usepackage{pifont}
\usepackage{amsfonts}
\usepackage{amssymb} 
\usepackage{setspace}                   % espaçamento flexível
\usepackage[bf,small,compact]{titlesec} % cabeçalhos dos títulos: menores e compactos
\usepackage{indentfirst}                % indentação do primeiro parágrafo
\usepackage{subfigure}                  % uso de várias figuras numa só
\usepackage{makeidx}                    % índice remissivo
\usepackage[nottoc]{tocbibind}          % acrescentamos a bibliografia/indice/conteudo no Table of Contents
\usepackage{courier}                    % usa o Adobe Courier no lugar de Computer Modern Typewriter
\usepackage{type1cm}                    % fontes realmente escaláveis
\usepackage{listings}                   % para formatar código-fonte (ex. em Java)
\usepackage{setspace}
\usepackage{longtable}
\usepackage{framed}
\usepackage{multirow}
\usepackage{titletoc}
\usepackage{lscape}
\usepackage[final]{pdfpages}
\usepackage[fixlanguage]{babelbib}
\usepackage[font=small,format=plain,labelfont=bf,up,textfont=it,up]{caption}
\usepackage[usenames,svgnames,dvipsnames,table]{xcolor}
\usepackage[a4paper,top=2.54cm,bottom=2.0cm,left=2.0cm,right=2.54cm]{geometry}% margens
\usepackage{float}
\usepackage[pdftex,plainpages=false,pdfpagelabels,pagebackref,colorlinks=true,citecolor=black,linkcolor=black,urlcolor=black,filecolor=black,bookmarksopen=true]{hyperref} % links em preto
\usepackage[all]{hypcap}                % soluciona o problema com o hyperref e capitulos
\usepackage[numbers,square,sort,nonamebreak,comma]{natbib}  % citação
                                % bibliográfica alpha (alpha-ime.bst)

\usepackage{type1cm}      % fontes realmente escaláveis
\fontsize{60}{62}\usefont{OT1}{cmr}{m}{n}{\selectfont}

% ---------------------------------------------------------------------------- %
% Cabeçalhos similares ao TAOCP de Donald E. Knuth
\usepackage{fancyhdr}
\pagestyle{fancy}
\fancyhf{}
\renewcommand{\chaptermark}[1]{\markboth{\MakeUppercase{#1}}{}}
\renewcommand{\sectionmark}[1]{\markright{\MakeUppercase{#1}}{}}
\renewcommand{\headrulewidth}{0pt}

% ---------------------------------------------------------------------------- %
\graphicspath{{./figuras/}}             % caminho das figuras (recomendável)
\frenchspacing                          % arruma o espaço: id est (i.e.) e exempli gratia (e.g.) 
\urlstyle{same}                         % URL com o mesmo estilo do texto e não mono-spaced
\makeindex                              % para o índice remissivo
\raggedbottom                           % para não permitir espaços extra no texto
\fontsize{60}{62}\usefont{OT1}{cmr}{m}{n}{\selectfont}
\cleardoublepage
\normalsize

% ---------------------------------------------------------------------------- %
% Opções de listing usados para o código fonte
% Ref: http://en.wikibooks.org/wiki/LaTeX/Packages/Listings
\lstset{ %
language=Java,                  % choose the language of the code
basicstyle=\footnotesize,       % the size of the fonts that are used for the code
numbers=left,                   % where to put the line-numbers
numberstyle=\footnotesize,      % the size of the fonts that are used for the line-numbers
stepnumber=1,                   % the step between two line-numbers. If it's 1 each line will be numbered
numbersep=5pt,                  % how far the line-numbers are from the code
showspaces=false,               % show spaces adding particular underscores
showstringspaces=false,         % underline spaces within strings
showtabs=false,                 % show tabs within strings adding particular underscores
frame=single,	                % adds a frame around the code
framerule=0.6pt,
tabsize=2,	                    % sets default tabsize to 2 spaces
captionpos=b,                   % sets the caption-position to bottom
breaklines=true,                % sets automatic line breaking
breakatwhitespace=false,        % sets if automatic breaks should only happen at whitespace
escapeinside={\%*}{*)},         % if you want to add a comment within your code
backgroundcolor=\color[rgb]{1.0,1.0,1.0}, % choose the background color.
rulecolor=\color[rgb]{0.8,0.8,0.8},
extendedchars=true,
xleftmargin=10pt,
xrightmargin=10pt,
framexleftmargin=10pt,
framexrightmargin=10pt
}

% ---------------------------------------------------------------------------- %
% Corpo do texto
\begin{document}
\frontmatter 
% cabeçalho para as páginas das seções anteriores ao capítulo 1 (frontmatter)
\fancyhead[RO]{{\footnotesize\rightmark}\hspace{2em}\thepage}
\setcounter{tocdepth}{2}
\fancyhead[LE]{\thepage\hspace{2em}\footnotesize{\leftmark}}
\fancyhead[RE,LO]{}
\fancyhead[RO]{{\footnotesize\rightmark}\hspace{2em}\thepage}

\onehalfspacing  % espaçamento

% ---------------------------------------------------------------------------- %
% Capa
% ---------------------------------------------------------------------------- %
\thispagestyle{empty}
\begin{center}
    \vspace*{2.3cm}
    \textbf{\Large{Como a prática de TDD Influencia no \\Projeto de Classes em Sistemas Orientados a Objetos}}\\
    
    \vspace*{1.2cm}
    \Large{Mauricio Finavaro Aniche}
    
    \vskip 2cm
    \textsc{
    Dissertação apresentada\\[-0.25cm] 
    ao\\[-0.25cm]
    Instituto de Matemática e Estatística\\[-0.25cm]
    da\\[-0.25cm]
    Universidade de São Paulo\\[-0.25cm]
    para\\[-0.25cm]
    obtenção do título\\[-0.25cm]
    de\\[-0.25cm]
    Mestre em Ciência da Computação}
    
    \vskip 1.5cm
    Programa: Mestrado em Ciência da Computação\\
    Orientador: Prof. Dr. Marco Aurélio Gerosa

   	\vskip 1.5cm
    \normalsize{São Paulo, Março de 2012}
\end{center}

% ---------------------------------------------------------------------------- %
% Página de rosto (só para a versão final)
% ---------------------------------------------------------------------------- %
%\newpage
%\thispagestyle{empty}
%    \begin{center}
%        \vspace*{2.3 cm}
%        \textbf{\Large{Título do trabalho a ser apresentado à \\
%        CPG para a dissertação/tese}}\\
%        \vspace*{2 cm}
%    \end{center}

%    \vskip 2cm

%    \begin{flushright}
    % Este exemplar corresponde à redação\\
    % final da dissertação/tese devidamente corrigida\\
    % e defendida por (Nome Completo do Aluno)\\
    % e aprovada pela Comissão Julgadora.
    %
%	Esta versão definitiva da tese/dissertação\\
%	contém as correções e alterações sugeridas pela\\
%	Comissão Julgadora durante a defesa realizada\\
%    por (Nome Completo do Aluno) em 4/5/2010.

%    \vskip 2cm

%    \end{flushright}
%    \vskip 4.2cm

%    \begin{quote}
%    \noindent Comissão Julgadora:
    
%    \begin{itemize}
%		\item Profa. Dra. Nome Completo (orientadora) - IME-USP [sem ponto final]
%		\item Prof. Dr. Nome Completo - IME-USP [sem ponto final]
%		\item Prof. Dr. Nome Completo - IMPA [sem ponto final]
%    \end{itemize}
      
%    \end{quote}
%\pagebreak

\pagenumbering{roman}     % começamos a numerar 

% ---------------------------------------------------------------------------- %
\chapter*{Agradecimentos}

Em primeiro lugar, gostaria de agradecer a meu orientador, Prof. Dr. Marco Aurélio
Gerosa, que, ao longo destes três anos me ensinou mais do que eu poderia imaginar.
Entrei no mestrado com o intuito de aprender sobre TDD, e saí de lá entendendo
melhor sobre ciência. 
Lições essas que levarei para toda minha vida na academia e indústria.

Gostaria também de agradecer aos amigos Gustavo Oliva e Mauricio de Diana, que
criticaram a pesquisa durante todo o tempo, me fazendo pensar novamente sobre
várias das minhas crenças em engenharia de software. Gostaria muito que continuássemos
nosso grupo de pesquisa, pois devo a vocês grande parte do que aprendi no mestrado.

Agradeço também a minha família e namorada, por terem me dado todo o suporte
emocional que precisei ao longo desta caminhada. Uma menção especial ao meu pai,
que me presentou com um livro de programação para crianças quando eu tinha por volta
de 9 anos. Talvez, sem esse presente, essa pesquisa nunca teria acontecido.

Agradeço aos meus amigos de trabalho da Locaweb e Caelum Ensino e Inovação, 
por aguentar meus discursos e palestras sobre TDD, e mostrar os diferentes
pontos de vista. Isso me ajudou a entender mais sobre a prática, o que resultou
em uma discussão mais rica ao longo do trabalho.

Por fim, agradeço as empresas que aceitaram participar do meu estudo, Bluesoft,
Amil, e WebGoal (São Paulo e Poços de Caldas). Além disso, obrigado aos meus colegas
de profissão Rafael Werner, Murilo Amêndola e Juan Lopes, por também terem
participado da pesquisa de maneira independente. Fico muito feliz por ter sido
bem recebido pela indústria, e espero que possamos continuar essa parceria entre
academia e indústria.

Um forte abraço a todos!

% ---------------------------------------------------------------------------- %
\chapter*{Resumo}

Desenvolvimento Guiado por Testes (TDD) é uma das práticas sugeridas na Programação
Extrema. A mecânica da prática é simples: o programador escreve o teste antes
de escrever o código. É, portanto, fácil inferir que a prática de TDD é uma
prática de testes de software. Entretanto, muitos autores de livros 
conhecidos pela indústria e academia, como Robert Martin, Steve Freeman e 
Dave Astels afirmam que os efeitos da prática vão além. Segundo eles,
TDD pode ajudar o desenvolvedor durante o processo de criação do projeto classes,
fazendo-os criar classes menos acopladas e mais coesas.

Entretanto, entender como a prática guia o desenvolvedor durante esse processo
não é fácil. Grande parte dos trabalhos estão interessados em descobrir se
a prática de TDD faz diferença na qualidade do código gerado, mas poucos são os 
autores que tentam discutir como a prática realmente auxilia.
Além disso, é notável a crescente adoção da prática de TDD por parte da indústria. E,
dado nosso estudo realizado dentro de um evento de métodos ágeis, percebemos que
mesmo os praticantes não entender ou conseguem expressar bem como a prática
os guia.

Este trabalho tem por objetivo compreender melhor os efeitos de TDD e como sua prática 
influencia o desenvolvedor durante o processo de projeto de sistemas orientados a objetos.
Para alcançar o objetivo, este estudo faz uso de uma combinação entre um experimento controlado 
inicial, na qual participantes foram
convidados a resolver exercícios pré-elaborados utilizando TDD e, a partir dos dados colhidos nesse estudo, um outro
estudo qualitativo detalhou como a prática influenciou as decisões de projeto de classes dos participantes.

Ao final, encontramos que a prática de TDD não melhora o projeto de classes naturalmente, mas sim
dá \textit{feedback} constante ao desenvolvedor que, por sua vez, utiliza de sua experiência
e conhecimento para melhorá-lo. Esse estudo também levantou esses possíveis \textit{feedbacks} da
prática e os catalogou. Esses padrões encontrados podem ser úteis para desenvolvedores
que queiram encontrar maneiras de validar o projeto de classes em desenvolvimento.

\noindent \textbf{Palavras-chave:} Desenvolvimento Guiado por Testes, Sistemas Orientados
a Objetos, Projeto de Classes, Qualidade Interna de Código.

% ---------------------------------------------------------------------------- %
\chapter*{Abstract}

Test-Driven Development (TDD) is one of the most known practices of Extreme
Programming (XP). The mechanical is simple: the developer writes a test before
writing the implementation. It is possible to infer that the practice is then 
a software testing technique. However, many famous book authors, such as 
Robert Martin, Steve Freeman, and Dave Astels suggest that the effects of the practice
can go beyond. According to them, TDD can help developers during the class design creation
process, enabling developers to create less coupled highly cohesive classes.

It is really hard to understand how TDD influences on the software development
process. Most part of the studies are interested on finding the difference between
a TDD'd and a non-TDD'd code. Only a few of them try to discuss how the practice
really supports class design. In addition, it is easy to notice the increasing
adoption of the practice by the industry. And, given our study
inside an agile software development conference, we noticed that even practitioners
do not understand how the practice guides them.

This work aims to understand better the effects and how TDD influences the
practitioner during the class design process in object-oriented systems. 
This work uses a combination of a initial controlled experiment, in which the participants
were invited to solve a few prepared exercises using TDD and, based on data gathered on this,
another qualitative study explored, in detail, the way TDD influenced the design decisions
taken by the participants.

At the end, we observed that the practice of TDD does not improve the class design naturally, but
gives constant feedback to developers that make use of their own experience and knowledge to
improve it. This study also categorized the possible feedback the practice can give. These
patterns can be very handful to developers that want to find another ways to validate
their class design under development.

\noindent \textbf{Keywords:} Test-Driven Development, Object-Oriented
Systems, Class Design, Internal Code Quality.

% ---------------------------------------------------------------------------- %
% Sumário
\tableofcontents    % imprime o sumário

% ---------------------------------------------------------------------------- %
\chapter{Lista de Abreviaturas}
\begin{tabular}{ll}
         TDD         & Desenvolvimento Guiado por Testes\\ 
         
         XP          & Programação Extrema (do inglês, \emph{Extreme Programming})\\
		 
		 BDUF		 & Projeto de Classe Criado de Uma Só Vez \textit{Big Design Up-Front}\\
		 
		 PRU		 & Princípio da Responsabilidade Única (\emph{Single Responsibility
		 Principle})\\
		 
		 PID		 & Princípio da Inversão de Dependências (\emph{Single Responsibility
		 Principle})\\ 
		 
		 PAF		 & Princípio do Aberto-Fechado (\emph{Open-CLosed Principle})\\
		 
		 PSL	 	 & Princípio da Substituição de Liskov (\emph{Liskov Substitution
		 Principle})\\
		 
		 PSI		 & Princípio da Segregação de Interfaces (\emph{Interface Segregation
		 Principle})\\
		 
		 OO		 	 & Orientação a Objetos
		 
\end{tabular}

% ---------------------------------------------------------------------------- %
% Listas de figuras e tabelas criadas automaticamente
\listoffigures
\begingroup
\let\cleardoublepage\relax
\let\clearpage\relax
\listoftables
\endgroup

% ---------------------------------------------------------------------------- %
% Capítulos do trabalho
\mainmatter

% cabeçalho para as páginas de todos os capítulos
\fancyhead[RE,LO]{\thesection}

%\singlespacing              % espaçamento simples
\onehalfspacing            % espaçamento um e meio

\input cap-introducao
\input cap-tdd
\input cap-quali-planejamento
\input cap-findings
\input cap-findings-quantitativo
\input cap-ameacas
\input cap-conclusoes

% cabeçalho para os apêndices
\renewcommand{\chaptermark}[1]{\markboth{\MakeUppercase{\appendixname\ \thechapter}} {\MakeUppercase{#1}} }
\fancyhead[RE,LO]{}
\appendix

\include{ape-design}
\chapter{Questionário inicial}
\label{ape:questionario-inicio}

\begin{enumerate}

\item Seu e-mail?

\item Empresa em que atua?

\item Seu nome?

\item Experiência em desenvolvimento de software \textit{(0, Entre 0 e 1 anos, Entre 1 e 2 anos, Entre 2 e 3 anos, Entre 3 e 4 anos, Entre 4 e 5 anos, Entre 5 e 6 anos, Entre 6 e 7 anos, Entre 7 e 8 anos, Entre 8 e 9 anos, Entre 9 e 10 anos, Mais que 10 anos)}

\item Experiência com TDD \textit{(0, Entre 0 e 1 anos, Entre 1 e 2 anos, Entre 2 e 3 anos, Entre 3 e 4 anos, Entre 4 e 5 anos, Entre 5 e 6 anos, Entre 6 e 7 anos, Entre 7 e 8 anos, Entre 8 e 9 anos, Entre 9 e 10 anos, Mais que 10 anos)}

\item Java é sua principal linguagem de programação? \textit{(Sim, Não)}

\item Como você avalia seus conhecimentos em Java? Fale um pouco sobre ele.	

\item Você conhece JUnit? \textit{(Sim, Não)}

\item Conhece o conceito de Mock Objects? \textit{(Sim e utilizo no meu dia a dia, Sim só na teoria, Não)}

\item Como você avalia seus conhecimentos em TDD? Fale um pouco sobre ele.	

\item Como você avalia seus conhecimentos em orientação a objetos e em design de sistemas OO? Fale um pouco sobre ele.	

\item Como você avalia sua experiência no processo de desenvolvimento de software em geral? Fale um pouco sobre ele.	

\end{enumerate}
\chapter{Exercícios}
\label{ape:exercicios}

Os exercícios são os mesmos para todos os grupos. O participante, em caso de dúvidas, poderá perguntar ao pesquisador.

\section{Lembrete ao participante}

Caro participante,

Lembre-se que os problemas aqui propostos simulam complicações do mundo real. 
Ao resolvê-los, tenha em mente que esses códigos serão futuramente mantidos
por você ou até por uma equipe maior.

Tente criar o design mais elegante possível em todas as soluções. Por serem problemas
recorrentes, imagine que amanhã esse mesmo problema se repetirá.
Escreva um código flexível o suficiente para que novas mudanças sejam fáceis de serem 
implementadas.

\section{Exercício 1 - Calculadora de Salário}

O participante deve implementar uma calculadora de salário de funcionários. Um
funcionário contém nome, e-mail, salário-base e cargo. De acordo com seu cargo,
a regra para cálculo do salário líquido é diferente:

\begin{enumerate}
	\item Caso o cargo seja DESENVOLVEDOR, o funcionário terá desconto de 20\%
	caso o salário seja maior ou igual que 3.000,00, ou apenas 10\% caso o salário seja menor 
	que isso.
	
	\item Caso o cargo seja DBA, o funcionário terá desconto de 25\%
	caso o salário seja maior ou igual que 2.000,00, ou apenas 15\% caso o salário seja menor 
	que isso.

	\item Caso o cargo seja TESTADOR, o funcionário terá desconto de 25\%
	caso o salário seja maior ou igual que 2.000,00, ou apenas 15\% caso o salário seja menor 
	que isso.
	
	\item Caso o cargo seja GERENTE, o funcionário terá desconto de 30\%
	caso o salário seja maior ou igual que 5.000,00, ou apenas 20\% caso o salário seja menor 
	que isso.
\end{enumerate}

Exemplos de cálculo do imposto:

\begin{itemize}
	\item DESENVOLVEDOR com salário-base 5,000.00. Salário final = 4.000,00
	\item GERENTE com salário-base de 2.500,00. Salário final: 2.000,00
	\item TESTADOR com salário de 550.00. Salário final: 467,50
\end{itemize}


O participante deve criar todo o código responsável para esse cálculo. Uma classe com
o método "main()" deverá ser entregue ao final, com exemplo de uso das classes criadas.

\section{Exercício 2 - Gerador de Nota Fiscal}

O participante deve implementar um sistema de geração de nota fiscal a partir de uma fatura. 
Uma fatura contém o nome e endereço do cliente, tipo do serviço e valor da fatura. O gerador de
nota fiscal deverá gerar uma nota fiscal que contém nome do cliente, valor da nota e valor
do imposto a ser pago.

O valor da nota é o mesmo do valor da fatura. Já o cálculo do imposto a ser pago deve seguir
as seguintes regras:

\begin{enumerate}
	\item Caso o serviço seja do tipo "CONSULTORIA", o valor do imposto é de 25%;
	\item Caso o serviço seja do tipo "TREINAMENTO", o valor do imposto é 15%;
	\item Qualquer outro, o valor do imposto é 6%.
\end{enumerate}

Ao final da geração da nota fiscal, o sistema ainda deve enviar essa nota por e-mail,
para o SAP, e persistir na base de dados. Por simplicidade, o desenvolvedor pode usar
os códigos abaixo, que simulam o comportamento do SMTP, SAP e banco de dados:

class NotaFiscalDao {
	public void salva(NotaFiscal nf) { System.out.println("salvando no banco"); }
}
class SAP {
	public void envia(NotaFiscal nf) { System.out.println("enviando pro sap"); }
}
class Smtp {
	public void envia(NotaFiscal nf) { System.out.println("enviando por email"); }
}

O participante é livre para alterar os métodos, parâmetros recebidos ou qualquer outra coisa das classes acima.

Ao final, o participante deve entregar todo o código responsável por geração e encaminhamento da nota fiscal 
para os processos acima citados. Uma classe com o método "main()" deverá ser entregue ao final, com
exemplo de uso das classes criadas.

\section{Exercício 3 - Processador de Boletos}

O participante deve implementar um processador de boletos. Esse processador receberá uma lista de boletos 
(que contém basicamente código do boleto, data e valor pago) e a fatura respectiva (que contém data, valor total e nome do cliente). 
O processador deve então, para cada boleto, criar um pagamento associado nessa fatura,
guardando o valor pago, a data e o tipo do pagamento (nesse caso, "BOLETO").
Além disso, caso a soma de todos os boletos ultrapasse o valor da fatura, a mesma deve ser marcada
como "PAGA".

O participante deve criar todo o código responsável pelo processador de boletos. Uma classe com
o método "main()" deverá ser entregue ao final, com exemplo de uso das classes criadas.

Exemplos de processamento:

\begin{itemize}
	\item Fatura de 1.500,00 com 3 boletos no valor de 500,00, 400,00 e 600,00: fatura marcada como PAGA, e três pagamentos do tipo BOLETO criados 
	\item Fatura de 1.500,00 com 2 boletos no valor de 500,00 e 400,00: fatura não marcada como PAGA, e dois pagamentos do tipo BOLETO criados 
\end{itemize}

\section{Exercício 4 - Saída do Quebra-Cabeça Numérico}

O participante deve de alguma forma imprimir a saída do quebra-cabeça numérico. Esse quebra-cabeça gera
uma sequência de números, que devem ser impressos no seguinte formato: "[1 -> 2 -> 3 -> 4 ->5 ->6]" (incluindo os colchetes).

O código do quebra-cabeça maluco encontra-se abaixo:

\begin{lstlisting}
public class QuebraCabecaNumerico {

	private int entrada;
	private int saida;
	private List<Numero> fila;
	private Set<Integer> visitados;
	private Numero solucao;
	
	public QuebraCabecaNumerico() {
		this.fila = new ArrayList<Numero>();
		this.visitados = new HashSet<Integer>();
	}

	public void geraCaminho(int entrada, int saida) {
		this.entrada = entrada;
		this.saida = saida;
		
		this.solucao = buscaSolucao();
	}
	
	private Numero buscaSolucao() {
		 
		adicionaRaizNaFila();
		
		while(existemNumerosNaFila()) {
			Numero numeroAtual = removeDaFila();
			
			if(encontrouSaida(numeroAtual)) return numeroAtual;
			adicionaNaFila(
				multiplicaPorDois(numeroAtual),
				(ehPar(numeroAtual)?dividePorDois(numeroAtual):null),
				somaDois(numeroAtual)
			);
		}
		
		return null;
	}

	private boolean ehPar(Numero numeroAtual) {
		return numeroAtual.getValor()%2==0;
	}

	private boolean encontrouSaida(Numero numeroAtual) {
		return numeroAtual.getValor() == saida;
	}

	private boolean existemNumerosNaFila() {
		return fila.size()!=0;
	}

	private void adicionaRaizNaFila() {
		fila.add(new Numero(entrada, null));
	}
	
	private void adicionaNaFila(Numero... numeros) {
		for(Numero numero : numeros) {
			if(numero!=null) {
				if(!visitados.contains(numero.getValor())) {
					fila.add(numero);
					visitados.add(numero.getValor());
				}
			}
		}
	}
	
	private Numero multiplicaPorDois(Numero numero) {
		return new Numero(numero.getValor()*2, numero);
	}

	private Numero dividePorDois(Numero numero) {
		return new Numero(numero.getValor()/2, numero);
	}
	
	private Numero somaDois(Numero numero) {
		return new Numero(numero.getValor()+2, numero);
	}

	private Numero removeDaFila() {
		Numero topoDaFila = fila.get(0);
		fila.remove(0);
		return topoDaFila;
	}

}

class Numero {
	private final int valor;
	private final Numero pai;
	
	public Numero(int valor, Numero pai) {
		this.valor = valor;
		this.pai = pai;
	}
	public int getValor() {
		return valor;
	}

	public Numero getPai() {
		return pai;
	}
}
\end{lstlisting}

Repare que o único método público existente "buscaSolucao()", invoca o algoritmo e guarda a solução
dentro do atributo "solucao". Um exemplo de código que visita a árvore de números gerada pelo algoritmo é:

\begin{lstlisting}
	while(solucao!=null) {
		int valor = solucao.getValor(); // esse eh o valor a ser impresso
		solucao = solucao.getPai();
	}
\end{lstlisting}

Exemplos de saídas do algoritmo:

\begin{itemize}
	\item Entrada: 2, 2 Saída: [2]
	\item Entrada: 2, 4 Saída: [2 -> 4]
	\item Entrada: 2,10 Saída: [2 -> 4 -> 8 -> 10]
	\item Entrada: 3, 10 Saída: [3 -> 5 -> 10]
\end{itemize}

O participante deve criar todo o código responsável pela saída do quebra-cabeça numérico. Uma classe com
o método "main()" deverá ser entregue ao final, com exemplo de uso das classes criadas.


\chapter{Questionário pós-experimento}
\label{ape:questionario-pos}

\section{O Estudo}

\begin{enumerate}

\item Como você avalia a clareza do primeiro exercício que você resolveu?	

\item Como você avalia a clareza do segundo exercício que você resolveu?	

\item Como você avalia o tempo que teve para resolver cada um dos exercícios?	

\item Você sentiu dificuldades para escrever código de testes para algum dos exercícios?	

\item Em sua opinião, os problemas de projeto de classes enfrentados nos exercícios se parecem com os do mundo real?

\end{enumerate}

\section{Código gerado}

\begin{enumerate}
	
\item Qual sua opinião em relação a qualidade do código que você gerou?	

\item Em relação ao projeto das classes que você criou, como eles surgiram?

\item O que você fez para avaliar a qualidade do seu projeto de classes?	

\end{enumerate}

\section{Prática de DGT}

\begin{enumerate}
	
\item Durante os exercícios nos quais deveria-se utilizar DGT, como você avalia a maneira com que praticou?	

\item Em sua opinião, a prática de DGT fez alguma diferença no projeto de classes gerado?	

\item Você resolveu exercícios com e sem a prática de DGT. Você percebeu alguma diferença em relação ao processo de criação do projeto de classes?	

\item Você percebeu alguma diferença entre a qualidade do projeto de classes dos exercícios que você resolveu sem DGT e dos exercícios que você resolveu com DGT?	

\item Ao pensar em projeto de classes, será que DGT é realmente necessário ou basta apenas o desenvolvedor usar a sua experiência?	

\item Em sua opinião, para que serve DGT?	

\item Gostaria de fazer algum comentário final sobre DGT ou o código que você escreveu?	

\end{enumerate}

\chapter{Entrevista}
\label{ape:entrevista}

\section{Dados básicos}

\begin{enumerate}
	\item Nome completo?

	\item Empresa em que atua?

	\item E-mail para contato?

\end{enumerate}

\section{Caracterização do desenvolvedor}

\begin{enumerate}
	\item Qual a sua formação?

	\item Há quanto tempo atua na área de desenvolvimento de software?

	\item Como você classificaria seu conhecimento em orientação a objetos?

	\item Você desenvolve sistemas orientados a objetos há quanto tempo?	

	\item Fale-me um pouco sobre os projetos que já desenvolveu e os desafios 
	neles encontrados?

	\item Em sua opinião, qual a maior dificuldade na criação de sistemas OO de
	qualidade?

	\item Com que frequência você costuma ler livros ou ir a congressos para se 
	atualizar sobre as novas práticas? Quais?
	
	\item Você já leu o meu blog?
\end{enumerate}

\section{A Prática de TDD}

\begin{enumerate}
	\item Você pratica TDD há quanto tempo?

	\item Em sua opinião, o que é TDD?
	
	\item Quais as suas referências em TDD (livros, blogs, etc)?

	\item Por que você pratica TDD?

	\item Como tem sido sua relação com a prática?

	\item Poderia me explicar como você pratica TDD no dia-a-dia?

	\item Você pratica TDD o tempo todo?
\end{enumerate}

\section{Relação entre TDD e Design}

\begin{enumerate}
	\item{Em sua experiência, você acredita que classes muito acopladas são realmente prejudiciais?}
		\begin{enumerate}
			\item Como você combate esse problema?

			\item \textit{(se ele comentar sobre TDD)} Como TDD te ajuda nisso?

			\item \textit{(se ele comentar sobre TDD)} E o que um programador que não faz
			TDD perde?
			
			\item Além de receber as dependências pelo construtor, você vê alguma outra
			maneira aonde TDD te ajude?
				\begin{enumerate}
					\item Um problema famoso ao começar a praticar TDD é o construtor começar a
					receber muitos parâmetros. Quando você começa a perceber isso?
					\item E como resolve esse problema?
				\end{enumerate}

			\item Você percebe alguma diferença no acoplamento que você cria quando
			pratica TDD, do acoplamento que você cria quando não pratica TDD?

			\item Você acha que mesmo com TDD é possível criar classes com alto acoplamento? 
		\end{enumerate}

	\item{E classes com baixa coesão (que têm muitas responsabilidade)? Você
	costuma ver muitas delas?}
		\begin{enumerate}
			\item Em sua opinião, por que elas aparecem?

			\item Como você combate esse problema?

			\item \textit{(se ele comentar sobre TDD)} Como TDD te ajuda nisso?
			
			\item \textit{(se ele comentar sobre TDD)} E o que um programador que não faz TDD
			perde?

			\item Se você tem classes pequenas e coesas, como fazer com que todas elas
			trabalhem juntas?
		\end{enumerate}

	\item Assim que o sistema cresce, um dos problemas que podem aparecer é o
	gerenciamento das dependências.
		
		\begin{enumerate}
		  \item Você já possou por isso?

		  \item Como você o resolveu?
		\end{enumerate}

	\item E no que o teste ajuda?
	\begin{enumerate}
		\item Ele faz alguma influencia na qualidade do seu trabalho?

		\item Qual a diferença de escrever esse teste antes?

		\item E isso não aconteceria caso você escrevesse o teste depois?

		\item \textit{(se ele disser que ajuda no design)} Como seria criar designs
		sem a ajuda dos testes? Qual a diferença?

		\item Poderia dar um exemplo (usando código se preferir) de como o TDD
		influencia o seu design? Algum outro exemplo?

	\end{enumerate}

	\item O que você acha de classes ou métodos com muitas linhas de código?

	\item O que é um método ou uma classe longa para você?

	\item Como você resolve esse problema?

\end{enumerate}

\section{Relação entre TDD e experiência}

\begin{enumerate}
	\item Como você compararia as primeiras vezes que você praticou TDD com agora?

	\item Você acha que a experiência influencia no resultado final?

	\item Em sua opinião, como você acha que seria o design de uma pessoa sem
	experiência em desenvolvimento de software e sem experiência em OO, mas praticando TDD?

	\item O programador às vezes comete desvios na prática, como não ver o teste
	falhar,  esquecer de refatorar, não rodar a bateria de testes completa,
	refatorar outro trecho de código enquanto está no vermelho, e etc. Você vê isso
	como um problema?
	
	\item TDD fala sobre \textit{baby steps} (ou passos de bebê). O que você 
	entende por dar passos de bebê?
	\begin{enumerate}
		\item É muito comum, principalmente em atividades como dojos, as pessoas
		ficarem  criando if's e mais if's (que é o código mais simples possível que
		faz o teste passar),  até o momento em que o código se torna o complexo
		suficiente para merecer uma refatoração. O que pensa sobre isso?

		\item Como você evita que esse código fique complicado?
	\end{enumerate}
\end{enumerate}

\section{TDD e outras práticas ágeis}

\begin{enumerate}
	\item Você acha que outras práticas ágeis influenciam na qualidade do seu design?

	\item Vocês praticam programação pareada? Com que frequência? Quando vocês
	fazem programação pareada, você acha que isso ajuda mais no design do que TDD?

	\item Se só praticassem programação pareada, por exemplo, você acha que mesmo
	assim precisaria de TDD?

	\item Quais outras práticas ágeis você acha que ajuda você a melhorar o design?
\end{enumerate}

\section{TDD e experiência pessoal}

(Nesse ponto o entrevistado poderá mostrar algum trecho de código do projeto em que atua.)

\begin{enumerate}
	\item Tem algum exemplo em mente aonde TDD ajudou a resolver o problema de
	maneira elegante?

	\item E aonde TDD atrapalhou?
		\begin{enumerate}
			\item Você acha que TDD se aplica à todos os casos? Se não, em quais ele não 
			se aplica e porquê?

			\item Quando TDD deve ser aplicado?
		\end{enumerate}

\end{enumerate}

\section{Opiniões finais}

\begin{enumerate}
	\item Você acha que TDD resolve de vez o problema de designs que degradam ao longo do tempo?

	\item Gostaria de dizer mais algo sobre TDD que não disse nas perguntas anteriores?
	
	\item Como você compararia a qualidade de códigos escritos com TDD e sem TDD? 

	\item Com quem posso falar para saber mais sobre o assunto?
\end{enumerate}


\chapter{Informações ao partipante}
\label{ape:informacoes-participante}

% TODO: informacoes ao participante da pesquisa

\section{Dados do Projeto}

\section{Convite}

\section{Qual o objetivo desta pesquisa?}

\section{Qual meu papel dentro dela?}

\section{Quais são os benefícios?}

\section{Minha privacidade será garantida?}

\section{Quais são os custos de participação na pesquisa?}

\section{Em caso de dúvidas, o que devo fazer?}


\chapter{Autorização}
\label{ape:autorizacao}

\section{Consentimento de Participação na Pesquisa}

Caro participante, por favor preencha atentamente as instruções abaixo:

\begin{itemize}
  \item Eu recebi, li e entendi as informações sobre essa pesquisa;
  \item Eu tive a oportunidade de tirar dúvidas sobre a pesquisa;
  \item Eu entendo que meu monitor será gravado durante a implementação dos exercícios;
  \item Eu entendo que eu posso desistir da minha participação ou de
  qualquer informação que eu provi a qualquer momento antes da finalização do processo de
  coleta de dados, sem qualquer tipo de dano ou perda;
  \item Eu entendo que, em caso de desistência, a gravação, transcrição ou
  qualquer outra informação persistida será destruída;
  \item Eu aceito fazer parte desta pesquisa;
  \item Eu gostaria de receber uma cópia do resultado final da pesquisa;
\end{itemize}

Assine este documento, informando seu nome e data corrente.

% ---------------------------------------------------------------------------- %
% Bibliografia
\backmatter \singlespacing   % espaçamento simples
\bibliographystyle{alpha-ime}% citação bibliográfica alpha
\bibliography{bibliografia}  % associado ao arquivo: 'bibliografia.bib'


\end{document}
