%% ------------------------------------------------------------------------- %%
\chapter{Introdução}
\label{cap:introducao}

Desenvolvimento Guiado por Testes, tradução do termo
em inglês \textit{Test-Driven Development (TDD)},
é uma das práticas sugeridas pela Programação
Extrema (XP) \cite{XPExplained}. A prática é baseada em um pequeno ciclo, 
no qual o desenvolvedor escreve um teste antes
de implementar a funcionalidade esperada e, depois, com o código
passando no recém-criado teste, refatora para 
remover possíveis duplicação de dados e de código \cite{TDDByExample}.

A adoção de TDD da mesma pela indústria tem crescido cada vez mais. 
Em um questionário de 2010 para descobrir quais práticas eram feitas por times
ágeis \cite{wambler-survey-agile}, Scott Ambler mostrou que 53\% dos times ágeis
adotaram TDD como uma maneira para validar o trabalho feito.
Números similares podem ser observados nos questionários anuais da Version One, que,
em sua versão de 2012 \cite{versionone-2012} mostrou que 40\% dos times ágeis têm 
feito uso da prática.

Com a prática de TDD, um desenvolvedor só escreve código
que seja coberto por um teste. Por esse motivo, é comum relacionar a prática
de TDD com a área de teste de software. Mas, além do ponto sobre
a qualidade externa do código, um discurso comum entre os praticantes de TDD
na indústria é os efeitos da prática também sobre a qualidade interna do código.
Muitos autores de livros conhecidos pela indústria e academia, como
Kent Beck \cite{TDDByExample}, Robert Martin \cite{agile-ppp}, 
Steve Freeman \cite{GOOS} e Dave Astels \cite{astels-tdd}, afirmam que a prática de TDD
promove uma melhoria significativa no projeto de classes, auxiliando
o programador a criar classes mais coesas e menos acopladas.

Entretanto, a maneira na qual a prática de TDD guia o desenvolvedor
durante o processo de criação do projeto de classes não é clara. Observamos
isso em nosso estudo qualitativo com praticantes de TDD, feito dentro de um
evento de desenvolvimento ágil brasileiro, no qual entrevistamos dez
participantes da conferência sobre os efeitos de TDD e, para nossa surpresa,
nenhum soube afirmar, com clareza, como a prática os guia em direção
a um bom projeto de classes \cite{aniche-wbma}.
Siniaalto e Abrahamsson \cite{alarming-results} também
compartilham dessa opinião e, além disso, notaram que os efeitos de TDD podem 
não ser tão automáticos ou evidentes como o esperado.
Os próprios trabalhos relacionados, discutidos na Seção \ref{cap:trabalhos-relacionados},
apenas avaliam se a prática de TDD faz diferença na qualidade dos códigos produzidos.
Poucos deles possuem um estudo qualitativo, detalhando como a prática
faz tal diferença.

Com essa informação em mãos, os desenvolvedores saberiam, de forma mais clara,
como utilizar a prática de TDD para obter uma maior qualidade no processo de criação
do projeto de classes. Entretanto, para entender essas razões é necessário
conduzir uma pesquisa no mundo real, o que  
implica um equilíbrio entre o nível de controle
e o grau de realismo. Uma situação realista é, geralmente, complexa e 
não determinística, dificultando o entendimento sobre o que acontece. Por outro
lado, aumentar o controle sobre o experimento reduz o grau de realismo, muitas
vezes fazendo com que os reais fatores de influência fiquem fora do escopo do 
estudo \cite{guidelines-case-study}.

Baseando-se no fato de que o processo de desenvolvimento de software envolve 
diversos fatores humanos e é totalmente sensível ao contexto em que ele está 
inserido, 
este estudo fez uso de uma combinação entre um experimento controlado inicial, 
no qual participantes foram convidados a resolver exercícios utilizando TDD e, 
a partir dos dados colhidos, um outro estudo qualitativo foi 
feito objetivando entender como a prática influenciou as decisões de projeto 
dos participantes.

%% ------------------------------------------------------------------------- %%
\section{Motivação}

Esta pesquisa possui diversas motivações. A primeira delas é a crescente
popularidade da prática de TDD, tanto por parte da indústria, quanto 
por parte da academia. Além disso,
conforme discutido anteriormente, os efeitos da prática de TDD são
comumente mencionados por autores de livros, mas pouco explorados. 
Entender como a prática de TDD pode influenciar no processo
de criação do projeto de classes pode trazer grandes benefícios
aos desenvolvedores de software de maneira geral.
Ao saber de maneira mais precisa
como a prática influencia, poderiam melhorar o uso da prática e obter
informações sobre seu projeto de classes mais frequentemente.

Além disso, essas informações podem ainda podem servir de apoio à decisão da 
adoção da prática de TDD nos times de desenvolvimento.


%% ------------------------------------------------------------------------- %%
\section{Caracterização da Pesquisa}

Este trabalho
visa a compreender a influencia de TDD no projeto de classes.
Para isso, avaliamos a relação entre a prática de 
TDD
e as decisões de projeto de classes tomadas pelos desenvolvedores no processo de 
criação de classes.

A análise foi feita por meio de dados que foram
capturados baseados na percepção de programadores atuantes na indústria, após
a implementação de alguns pequenos problemas especialmente criados para
esta pesquisa.

O objetivo principal deste estudo é \textbf{entender a relação da prática de TDD 
e as decisões de projeto de classes tomadas pelo programador durante o processo de 
projeto de sistemas orientados a objetos}.
Para compreendê-la, tentou-se responder às questões listadas
abaixo:

\begin{enumerate}

	\item Qual a influência de TDD no projeto de classes?

	\item Qual a relação entre TDD e as tomadas de decisões de projeto
	feitas por um desenvolvedor?

	\item Como a prática de TDD influencia o programador no processo de  
	projeto de classes, do ponto de vista do acoplamento, coesão e complexidade?

\end{enumerate}

%% ------------------------------------------------------------------------- %%
\section{Contribuições}

As contribuições deste trabalho para a área de engenharia de software
são:

\begin{enumerate}
	\item Padrões de \textit{feedback} da prática de TDD que guiam os desenvolvedores
	ao longo do processo de criação do projeto de classes;

	\item Protocolo de um estudo qualitativo sobre os efeitos da prática
	de TDD no projeto de classes, bem como lições aprendidas sobre a execução do mesmo;
		
\end{enumerate}

%% ------------------------------------------------------------------------- %%
\section{Organização do trabalho}

Este trabalho está dividido da seguinte maneira: 

\begin{itemize}
	\item O Capítulo \ref{cap:tdd} discute sobre a prática de TDD, com ênfase no
	ponto de vista do projeto de classes, e mostra trabalhos já
	realizados pela academia sobre os efeitos de TDD;

	\item O Capítulo \ref{cap:qualitativo-planejamento} discute o planejamento e execução do estudo,
	bem como o processo de captura de dados e análise;

	\item O Capítulo \ref{cap:discussao} apresenta os resultados encontrados no estudo qualitativo 
	e os discute;
	
	\item O Capítulo \ref{cap:analise-quantitativa} apresenta os resultados encontrados no 
	estudo quantitativo e os discute;	
	
	\item O Capítulo \ref{cap:ameacas} discute as possíveis ameaças aos resultados
	encontrados na pesquisa;
	
	\item O Capítulo \ref{cap:conclusoes} resume o trabalho realizado, apresenta
	as lições aprendidas, resultados esperados, produções gerados ao longo do mestrado 
	e possibilidades de trabalhos futuros.
\end{itemize}

