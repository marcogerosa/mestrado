%% ------------------------------------------------------------------------- %%
\chapter{Relação entre TDD e Projeto de Classes: Análise Quantitativa}
\label{cap:analise-quantitativa}

Na tentativa de encontrar os ditos efeitos de TDD sobre o projeto de classes,
calculamos métricas em cima dos códigos gerados, para verificar se houve
alguma diferença na qualidade dos códigos gerados com e sem a prática de TDD.

Conforme discutido no Capítulo \ref{cap:qualitativo-planejamento}, o teste
estatístico escolhido foi o Wilcoxon. 
Nas sub-seções abaixo, discutimos os números encontrados.

%% ------------------------------------------------------------------------- %%
\section{Métricas de código}

Na Tabela \ref{metricas-industria}, mostramos os \textit{p-values} encontrados para
a diferença entre códigos produzidos com e sem TDD na indústria. 
Pelos números, 
observamos que em nenhum exercício houve diferença significativa nas métricas
de complexidade ciclomática e acoplamento eferente. Já a métrica de falta
de coesão dos métodos apresentou diferenças em dois exercícios (1 e 4). 
A diferença também apareceu na quantidade de linhas por método (exercício 4)
e quantidade de métodos (exercício 1). Ao olhar os dados de todos os exercícios
juntos, nenhuma métrica apontou uma diferença significativa.
Isso nos mostra que, ao menos quantitativamente, a prática de TDD não fez
diferença nas métricas de código.

\begin{table}[h!]
	\centering
	\begin{tabular}{ | p{3cm} | p{2cm} | p{2cm} | p{2cm} | p{2cm} | p{2cm} |}
		\hline
		Exercício & Complexi- dade ciclomática & Acoplamento eferente & Falta de coesão dos métodos & Número de linhas por método 
		& Quantidade de métodos por classe \\
		\hline
		Exercício 1 &	0.8967	&	0.6741 &	\cellcolor[gray]{0.8}2.04E-07* &	0.4962 &	\cellcolor[gray]{0.8}2.99E-06* \\
		Exercício 2	& 0.7868	&	0.7640 &	0.06132 &	0.9925 &	0.7501 \\
		Exercício 3	& 0.5463	&	0.9872 &	0.5471 &	0.7216 &	0.3972\\
		Exercício 4	& 0.2198	&	0.1361 &	\cellcolor[gray]{0.8}0.04891* &	\cellcolor[gray]{0.8}0.0032* &	0.9358\\
		\hline
		Todos &	0.8123	&	0.5604 &	0.3278 &	0.06814 &	0.5849\\
		\hline
	\end{tabular}
	\caption{\textit{P-values} encontrados para a diferença entre códigos com e sem TDD na indústria}
	\label{metricas-industria}
\end{table}

Já nas Tabelas \ref{valores-exp-cc-industria}, \ref{valores-exp-fanout-industria}, \ref{valores-exp-lcom-industria}, 
\ref{valores-exp-metodos-industria} e \ref{valores-exp-linhas-industria},
calculamos os \textit{p-values} das métricas, separando-as 
por experiência em desenvolvimento de software e TDD na indústria. Os valores para o grupo
experiente em TDD e não experiente em desenvolvimento de software não foram calculados, já que nenhum
participante se enquadrou nele.

Pelos números, percebemos 
que a métrica de coesão foi a única que apresentou uma diferença significativa entre desenvolvedores
experientes, tanto em TDD quanto em desenvolvimento de software.

\begin{table}[h!]
	\centering
	\begin{tabular}{ | p{5cm} | p{5cm} | p{5cm} | }
		\hline
		 Complexidade Ciclomática & Experiente em TDD & Não experiente em TDD \\
		\hline
			Experiente em Desenvolvimento de Software 		& 0.09933	&	0.8976\\
			\hline
			Não Experiente em Desenvolvimento de Software 	& NA		&	0.4462\\
		\hline
	\end{tabular}
	\caption{\textit{P-values} encontrados para a diferença na Complexidade Ciclomática entre experientes e não experientes na indústria}
	\label{valores-exp-cc-industria}
\end{table}

\begin{table}[h!]
	\centering
	\begin{tabular}{ | p{5cm} | p{5cm} | p{5cm} | }
		\hline
		 \textit{Fan-Out} & Experiente em TDD & Não experiente em TDD \\
		\hline
			Experiente em Desenvolvimento de Software 		& 0.1401	&	0.6304\\
			\hline
			Não Experiente em Desenvolvimento de Software 	& NA		&	0.2092\\
		\hline
	\end{tabular}
	\caption{\textit{P-values} encontrados para a diferença no \textit{Fan-Out} entre experientes e não experientes na indústria}
	\label{valores-exp-fanout-industria}
\end{table}

\begin{table}[h!]
	\centering
	\begin{tabular}{ | p{5cm} | p{5cm} | p{5cm} | }
		\hline
		 Falta de Coesão nos Métodos & Experiente em TDD & Não experiente em TDD \\
		\hline
			Experiente em Desenvolvimento de Software 		& \cellcolor[gray]{0.8}0.03061*	&	0.1284\\
			\hline
			Não Experiente em Desenvolvimento de Software 	& NA		&	0.0888\\
		\hline
	\end{tabular}
	\caption{\textit{P-values} encontrados para a diferença na falta de coesão nos métodos entre experientes e não experientes na indústria}
	\label{valores-exp-lcom-industria}
\end{table}

\begin{table}[h!]
	\centering
	\begin{tabular}{ | p{5cm} | p{5cm} | p{5cm} | }
		\hline
		 Quantidade de Métodos por Classe & Experiente em TDD & Não experiente em TDD \\
		\hline
			Experiente em Desenvolvimento de Software 		& 0.09933	&	0.8976\\
			\hline
			Não Experiente em Desenvolvimento de Software 	& NA		&	0.4462\\
		\hline
	\end{tabular}
	\caption{\textit{P-values} encontrados para a diferença na quantidade de métodos por classe entre experientes e não experientes na indústria}
	\label{valores-exp-metodos-industria}
\end{table}

\begin{table}[h!]
	\centering
	\begin{tabular}{ | p{5cm} | p{5cm} | p{5cm} | }
		\hline
		 Linhas por Método & Experiente em TDD & Não experiente em TDD \\
		\hline
			Experiente em Desenvolvimento de Software 		& 0.0513	&	0.4319\\
			\hline
			Não Experiente em Desenvolvimento de Software 	& NA		&	0.5776\\
		\hline
	\end{tabular}
	\caption{\textit{P-values} encontrados para a diferença no número de linhas por método entre experientes e não experientes na indústria}
	\label{valores-exp-linhas-industria}
\end{table}

%% ------------------------------------------------------------------------- %%
\newpage
\section{Especialistas}

Ambos os especialistas não encontraram diferenças entre códigos produzidos
com e sem TDD. Na Tabela
\ref{tab:especialistas-industria},
mostramos os \textit{p-values} encontrados para a diferença de avaliação dos especialistas
entre códigos produzidos com e sem TDD.


\begin{table}[h!]
	\centering
	\begin{tabular}{| p{5cm} | c | c | c | }
		\hline
		Especialista & Projeto de classes & Testabilidade & Simplicidade\\
		\hline
		Especialista 1 &	0.4263 &	0.5235 &	0.3320\\
		Especialista 2 &	0.7447 &	0.4591 &	0.9044\\
		\hline
	\end{tabular}
	\caption{\textit{P-values} encontrados para a diferença entre as análises dos especialistas com e sem TDD na indústria}
	\label{tab:especialistas-industria}
\end{table}

%% ------------------------------------------------------------------------- %%
\subsection{Inspeção do Código-Fonte}

Nós avaliamos cada código-fonte manualmente.
Em sua maioria, os códigos eram claros e fáceis de entender,
com classes, métodos e variáveis bem nomeados.
Mas, para nossa surpresa,
poucos foram os participantes que fizeram uso de polimorfismo. A grande
maioria das implementações fazia uso de cadeias de condições para 
alcançar o objetivo.

Nós também não conseguimos identificar, por inspeção manual, quais códigos
eram produzidos com TDD e quais não eram pois, independente
da prática utilizada, ambos eram muito semelhantes.
De todos os participantes da indústria, apenas um foi completamente eliminado:
suas classes eram completamente vazias. 


%% ------------------------------------------------------------------------- %%
\section{Discussão}

Os valores apresentados anteriormente corroboram com muitos dos trabalhos relacionados. 
Aparentemente TDD não influencia a ponto de alterar 
de maneira significativa os valores das métricas de acoplamento, coesão e simplicidade.
Porém, isso é incoerente com o sentimento comum no mercado de que praticar TDD
traz benefícios para o projeto de classes. 

Conforme previsto, neste estudo conduzimos
uma etapa qualitativa para entender como se procede essa influência, do ponto
de vista dos desenvolvedores. Tal estudo parece vital para a real compreensão
dos efeitos da prática. A análise qualitativa é encontrada no capítulo a seguir.


