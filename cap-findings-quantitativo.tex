%% ------------------------------------------------------------------------- %%
\chapter{Relação entre TDD e Projeto de Classes: Análise Quantitativa}
\label{cap:analise-quantitativa}

Na tentativa de triangular as informações levantadas no capítulo anterior,
calculamos métricas em cima dos códigos gerados, para verificar se houve
alguma diferença na qualidade dos códigos gerados com e sem a prática de TDD.

Conforme discutido no Capítulo \ref{cap:qualitativo-planejamento}, o teste
estatístico escolhido foi o Wilcoxon. 
Devido ao problema que tivemos com os dados da academia, separamos a análise
dos dados, tanto das métricas de código, quanto dos especialistas, entre academia e indústria.
Nas sub-seções abaixo, discutimos os números encontrados.

%% ------------------------------------------------------------------------- %%
\section{Métricas de código}

Na Tabela \ref{metricas-industria}, mostramos os \textit{p-values} encontrados para
a diferença entre códigos produzidos com e sem TDD na indústria. Pelos números, 
observamos que em nenhum exercício houve diferença significativa nas métricas
de complexidade ciclomática e acoplamento eferente. Já a métrica de falta
de coesão dos métodos apresentou diferenças em dois exercícios (1 e 4). 
A diferença também apareceu na quantidade de linhas por método (exercício 4)
e quantidade de métodos (exercício 1). Ao olhar os dados de todos os exercícios
juntos, nenhuma métrica apontou uma diferença significativa.
Isso nos mostra que, ao menos quantitativamente, a prática de TDD não fez
diferença nas métricas de código.

\begin{table}[h!]
	\centering
	\begin{tabular}{ | p{3cm} | p{2cm} | p{2cm} | p{2cm} | p{2cm} | p{2cm} |}
		\hline
		Exercício & Complexi- dade ciclomática & Acoplamento eferente & Falta de coesão dos métodos & Número de linhas por método 
		& Quantidade de métodos por classe \\
		\hline
		Exercício 1 &	0.8967	&	0.6741 &	\cellcolor[gray]{0.8}2.04E-07* &	0.4962 &	\cellcolor[gray]{0.8}2.99E-06* \\
		Exercício 2	& 0.7868	&	0.7640 &	0.06132 &	0.9925 &	0.7501 \\
		Exercício 3	& 0.5463	&	0.9872 &	0.5471 &	0.7216 &	0.3972\\
		Exercício 4	& 0.2198	&	0.1361 &	\cellcolor[gray]{0.8}0.04891* &	\cellcolor[gray]{0.8}0.0032* &	0.9358\\
		\hline
		Todos &	0.8123	&	0.5604 &	0.3278 &	0.06814 &	0.5849\\
		\hline
	\end{tabular}
	\caption{\textit{P-values} encontrados para a diferença entre códigos com e sem TDD na indústria}
	\label{metricas-industria}
\end{table}

Na Tabela \ref{metricas-academia}, apontamos os \textit{p-values} das diferenças encontradas
entre os códigos produzidos na academia. As métricas de 
falta de coesão dos métodos e quantidade de métodos apresentaram
uma diferença significativa nos exercícios 1 e 4, respectivamente.
Ao olhar para todos os dados juntos, uma diferença significativa
apareceu também na métrica de acoplamento eferente.
Podemos então concluir que a prática de TDD também não afetou
o código produzido pelos participantes da academia.

\begin{table}[h!]
	\centering
	\begin{tabular}{ | p{3cm} | p{2cm} | p{2cm} | p{2cm} | p{2cm} | p{2cm} |}
		\hline
		Exercício & Complexi- dade ciclomática & Acoplamento eferente & Falta de coesão dos métodos & Número de linhas por método 
		& Quantidade de métodos por classe \\
		\hline
			Exercício 1	& 0.2155	&	0.0708	& \cellcolor[gray]{0.8}0.0071* &	0.1721	& \cellcolor[gray]{0.8}0.0083*\\
			Exercício 2	& 0.7244	&	0.6744	& 1 &	0.3175 &	0.5591\\
			Exercício 3	& 0.7008	&	0.1014 &	0.5007	& 0.4292	& 0.8687\\
			Exercício 4	& 0.3430	&	0.7131 &	0.7735	& 0.7833	& 0.5522\\
		\hline
			Todos &	0.2540	&	\cellcolor[gray]{0.8}0.0076* & 0.3510 & 0.2668 & 0.1706\\
		\hline
	\end{tabular}
	\caption{\textit{P-values} encontrados para a diferença entre códigos com e sem TDD na academia}
	\label{metricas-academia}
\end{table}

%% ------------------------------------------------------------------------- %%
\section{Especialistas}

Ambos os especialistas não encontraram diferenças entre códigos produzidos
com e sem TDD, tanto na indústria, quanto na academia. Nas Tabelas 
\ref{tab:especialistas-industria} e \ref{tab:especialistas-academia},
mostramos os \textit{p-values} encontrados para a diferença de avaliação dos especialistas
entre códigos produzidos com e sem TDD na indústria e academia, respectivamente.


\begin{table}[h!]
	\centering
	\begin{tabular}{| p{5cm} | c | c | c | }
		\hline
		Especialista & Projeto de classes & Testabilidade & Simplicidade\\
		\hline
		Especialista 1 &	0.4263 &	0.5235 &	0.3320\\
		Especialista 2 &	0.7447 &	0.4591 &	0.9044\\
		\hline
	\end{tabular}
	\caption{\textit{P-values} encontrados para a diferença entre as análises dos especialistas com e sem TDD na indústria}
	\label{tab:especialistas-industria}
\end{table}

\begin{table}[h!]
	\centering
	\begin{tabular}{| p{5cm} | c | c | c | }
		\hline
		Especialista & Projeto de classes & Testabilidade & Simplicidade\\
		\hline
		Especialista 1	& 0.8795 &	NA	& 0.4222\\
		Especialista 2	& 0.8800	& 0.5519 &	0.8800\\
		\hline
	\end{tabular}
	\caption{\textit{P-values} encontrados para a diferença entre as análises dos especialistas com e sem TDD na academia}
	\label{tab:especialistas-academia}
\end{table}

%% ------------------------------------------------------------------------- %%
\subsection{Inspeção do Código-Fonte}

Nós avaliamos cada código-fonte manualmente.
Em sua maioria, os códigos eram claros e fáceis de entender,
com classes, métodos e variáveis bem nomeados.
Mas, para nossa surpresa,
poucos foram os participantes que fizeram uso de polimorfismo. A grande
maioria das implementações fazia uso de cadeias de condições para 
alcançar o objetivo.

Nós também não conseguimos identificar, por inspeção manual, quais códigos
eram produzidos com TDD e quais não eram pois, independente
da prática utilizada, ambos eram muito semelhantes.
De todos os participantes da indústria, apenas um foi completamente eliminado:
suas classes eram completamente vazias. 


%% ------------------------------------------------------------------------- %%
\section{Discussão}

No capítulo anterior, discutimos sobre a maneira na qual a prática de TDD
influencia no projeto de classes. Todos os padrões levantados ajudam
o desenvolvedor a perceber possíveis problemas no projeto de classes atual.
Entretanto, a relação das métricas calculadas em relação aos códigos
produzidos usando TDD e aos códigos produzidos sem a prática,
não apresentou diferença.

Alguns motivos podem servir de explicação para este fenômeno. Conforme discutido na análise
qualitativa, os padrões levantados pelos
participantes apenas ajudam o desenvolvedor a perceber possíveis problemas
de projeto de classes. É tarefa do desenvolvedor utilizá-los para
refatorar o código atual. 

Uma possibilidade é que os desenvolvedores não perceberam ou deixaram
de interpretar o \textit{feedback} dado pelo teste no momento do desenvolvimento.
Talvez até pela falta de conhecimento mais "explícito" sobre o padrão, afinal
todos os participantes discutiram esses padrões conosco de maneira informal. 
Além disso, conforme discutido no Capítulo \ref{cap:ameacas}, um possível
problema dos números encontrados é o de que muitos participantes não terminaram
sua implementação. Isso pode confundir o valor das métricas e, consequentemente,
o teste estatístico.

Mas, indiferente do resultado quantitativo, os padrões levantados pelos 
desenvolvedores fazem sentido e, quando interpretados, podem ajudar os desenvolvedores
a melhorar o seu projeto de classes.

