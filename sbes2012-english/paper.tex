%% ------------------------------------------------------------------------- %%
\documentclass[conference]{IEEEtran}

\usepackage[numbers,square,sort,nonamebreak,comma]{natbib}  % citação
\usepackage{framed}
\usepackage{flushend}
\usepackage[table]{xcolor}
\usepackage[T1]{fontenc}
\usepackage[brazil, english]{babel}
\usepackage[utf8]{inputenc}
\usepackage[pdftex,plainpages=false,pdfpagelabels,pagebackref,colorlinks=true,citecolor=black,linkcolor=black,urlcolor=black,filecolor=black,bookmarksopen=true]{hyperref} % links em preto
\urlstyle{same}
\begin{document}
	
\title{How the Practice of TDD Influences Class Design in Object-Oriented Systems: \\Feedback Patterns to the Developer}

\author{\IEEEauthorblockN{Mauricio Finavaro Aniche, Marco Aurélio Gerosa}
\IEEEauthorblockA{Institute of Mathematics and Statistics\\
University of São Paulo\\
\{aniche, gerosa\}@ime.usp.br}}

\maketitle

\begin{abstract} 
	Despite that Test-Driven Development (TDD) appears to be a software testing practice, 
	many developers affirm that the practice influence on class design. This study aimed
	to better understand the effects of TDD and how the practice influences developers
	during class design on object-oriented systems. 
	We conducted an essentially qualitative exploratory study in which participants
	were invited to implement some exercises using TDD and, based on the data gathered,
	we gathered details of how the practice influenced design decisions from the
	participants using interviews.
	We noticed that the practice of TDD drives
	developers during class design by means of constant feedback about its quality. This study
	also named and catalogued feedback patterns perceived by the developers.
\end{abstract}

\IEEEpeerreviewmaketitle

%% ------------------------------------------------------------------------- %%
\section{Introduction}

Test-Driven Development (TDD) is one of the suggested practices
by Extreme Programming (XP) \cite{XPExplained}. The practice is based on
a small cycle, in which the developer writes a test before to implement
the expected functionality and, with the code passing on the
test, the developer refactors it to remove possible code 
and data duplication \cite{TDDByExample}.

The adoption of TDD by the industry has been growing. In a 2010 survey to discover
which practices were used by agile teams, Scott Ambler showerd that 53\% of all teams
that replied the survey adopted TDD as a way to validate the work done
\cite{wambler-survey-agile}. Similar numbers can be found on the annual surveys from
Version One that, in the 2012 version \cite{versionone-2012}, showed that
40\% of the respondent teams have been using the practice.

When using TDD, a developer only writes code that is covered by a test. Because
of that, it is common to relate the practice of TDD with software testing. However,
a common speech among TDD practitioners is its effects on the internal code quality.
Many authors, well-known by industry and academia, such as Kent Beck \cite{TDDByExample}, 
Robert Martin \cite{agile-ppp}, Steve Freeman \cite{GOOS}, and Dave Astels \cite{astels-tdd}, 
state (without scientific evidence) that the practice of TDD promotes a significative
improvement on class design, helping the developer to create more cohesive less coupled
classes.

On the other hand, the way in which the practice of TDD guides the developer
through class design is not clear. We observed that in our qualitative study
with TDD practitioners, inside of a brazilian agile event. We interview
ten participants about TDD effects \cite{aniche-wbma} and, for our surprise,
none of them could satisfactory explain how the practice guides them to better
designs.
Siniaalto and Abrahamsson \cite{alarming-results} also share this opinion. They also
noticed that the effects of TDD can not be as automatic as the expected for most people.
The related work, discussed in Section \ref{cap:trabalhos-relacionados},
only evaluates if the practice of TDD makes difference in the produced code. Only a few
of them contain a qualitative study that details how the practice does that difference.
With this information, developers would know how to use the practice of TDD in order to 
better create class design.

To understand these reasons, it is necessary to conduct a research on the real world, which
implies on a balance between control and realism. A realistic situation is usually
complex and non deterministic, which makes the comprehension of the phenomenon much
harder. On the other hand, increasing the control of the experiment reduces the
degree of realism, which can take the real factors of influence out of the study
scope \cite{guidelines-case-study}.

Based on the fact that the software development process involves many
human factors and is totally sensitive to the context that it belongs,
we conduct an essencially qualitative exploratory study in which professionals
from industry were invited to implemnet some prepared exercises using TDD and,
based on the gathered data, we investigate the details of how the practice influenced
the participants' class design decisions through interviews.

%% ------------------------------------------------------------------------- %%
\section{Test-Driven Development}

Agile software development methods focus on constant feedback. It can happen
between the team and the customer or even between the team and the
internal or external quality \cite{AgileManifesto}. Many of the suggested practices
aim to increase the quantity and quality of the feedback; the idea of pair programming,
as an example, is to give feedback about the code during its writing.

Test-Driven Development, popularized by Kent Beck through his book, \textit{TDD: By Example} (2011)
\cite{TDDByExample}, is one of the practices in which the focus is to give feedback. TDD has
a big importance during the software development cycle because, as suggested by the agile
practices, the class design of a system should emerge as software grows. And, in order
to quickly respond to this evolution, a constant feedback about internal and external code
quality is needed.

TDD is a software development practice that is based on a repetition of a small
cycle. First, developers should write a failing test. Then, they make it pass,
by implementing the desired functionality. After that, they refactor the code
to remove code or data duplication that was generated by the process. Besides that,
simplicity is intrisec to the process: the practitioner tries to write the simplest
test that fails and tries to write the simplest implementation that makes the
test pass. This cycle is also known as "Red-Green-Refactor", as it remembers the
colors that a developer usually sees when doing TDD: the red means that a test
is failing, and green means that the test passes succesfully.

It is common to relate TDD to software testing practices. Although the creation
of tests is intrisec to the process, it is also said that TDD helps developers
to create more flexible, cohesive, and less coupled classes. The tests are the
tool used by developers to evaluate the class design that is being developed.
Because of that, many developers refers to TDD 
as \textit{Test-Driven Design} \cite{tdd-taxonomy}.

Authors such as Kent Beck \cite{aim-fire}, Dave Astels \cite{astels-tdd}, and
Robert Martin \cite{bob-martin} state that TDD is, in fact, a class design
technique \cite{tdd-taxonomy} \cite{aim-fire}.
In their opinion, the change in the tradicional development cycle, although very simple,
aggregates many other benefits to the produced code: more simplicity, less coupling, and
a high cohesion. Ward Cunningham, one of the Extreme Programming pioneers, summarizes
the discussion in one statement: \textit{"Test-First programming is not a testing technique"} 
\cite{aim-fire}.
According to Janzen, a more clear definition is the one that says that TDD is the 
art of producing automated tests to production code, and use this process to guide
the class design and programming \cite{agilealliance-tdd} \cite{tdd-taxonomy}.

It is a known fact that class design tend to loose quality during the software evolution.
However, it is hard to discuss quality in terms of class design. In this study context,
we used the SOLID principles, detailed in Martin's work \cite{bob-martin}, which are
mentioned in the section below.

%% ------------------------------------------------------------------------- %%
\section{Quality of Class Design}

It is said that a class design is \textit{rotting} when it becomes hard
to evolve, the code reuse becomes more complicated than replicate the piece
of code, or the cost of doing any changes in the calss design becomes high.
Martin \cite{bob-martin} enumerated a few symptoms of a rotten class design,
also known as \textit{class design smells}. They are similar to \textit{code smells},
except that they belong to a higher level: they are present in the general structure
of the software instead of being located in just a single piece of code.

These symptoms can be measured subjectively and, sometimes, objectively. Usually
they are caused by violations of one or more class design principles.
In this study, we make use of the symptoms that were enumarated by him: 
rigidity, fragility, imobility, viscosity, opacity, needless complexity and
needless repetition. Besides, we also refer to the principles known as SOLID:
Single Responsibility Principle (SRP), Open-Closed Principle (OCP),
Liskov Substitutive Principle (LSP), Interface Segregration Principle (ISP), and
Dependency Inversion Principle (DIP) \cite{bob-martin}.

%% ------------------------------------------------------------------------- %%
\section{Related Work}
\label{cap:trabalhos-relacionados}

Many empirical studies have been done in order to evaluate the effects of TDD.
In most of them, the effects of the practice in the class design is not taken
into account, and only the effect on external quality are measured. In addition,
in contrary to what this study proposes, many of them chosed
to had a bigger control in the experiment. They also have done it inside of the academic
environment, with students that belong to the most different computer courses.

Janzen \cite{janzen-arch-improvement} showed that the complexity of the algorithms
were much smaller and the code coverage was higher in code written with TDD.
Langr \cite{langr} showed that TDD increases code quality, facilitates
maintenance, and helps to produce 33\% more tests when compared to traditional
approaches.

The study done by George e Williams \cite{george-e-williams} showed that,
although TDD can initially reduce the productivity of the more inexperienced
developers, a qualitative analysis showed that 92\% of the developers think
that TDD helps to maintain the quality of the code, and 79\% believes that
the practice promotes a simpler class design.

A study by Erdogmus \textit{et al.} \cite{erdogmus-morisio} with 24 undergraduate students
showed that TDD increases productivity. However, no difference in code quality
was found.

Another study by Janzen \cite{janzen-saiedian} with three different group
of students (each one of them using a different approach: TDD, test last, no tests)
showed that the code produced by the TDD team made better use of object-oriented
concepts, and responsibilities were better splitted into different classes,
while other teams produced a more procedural code. Also, tested classes were
104\% less coupled than non-tested classes, and methods were 43\%, on average, less complex
than the non-tested ones.

Dogsa and Batic \cite{dogsa-batic} also found an improvement on class design when
using TDD. But, according to the authors, the improvement is a consequence of the simplicity
TDD aggregates to the process. They also affirmed that the test suites created during
the practice favors the constant code refactoring.


Li \cite{angela-li} proposed a qualitative study to understand the efficacy of TDD.
In a case study, she collected the perceptions of TDD practitioners about the
benefits of the practices. To achieve that, she interviewed five developers
from software companies in Auckland, Australia. The results of the interview
were analyzed and discussed in terms of code quality, software quality, and
programmer's productivity. Regarding code quality, Li concluded that TDD
guides developers to simpler and better designed classes. In addition, the main
factors that contribute to these benefits are the confidence to refactor and
modify code, a higher code coverage, a deeper understanding of the requirements,
the ease of compreehending code, a reduced error scope, and the elevated
satisfaction of the developers.

The TDD practitioner usually makes uses of other agile practices, such as
pair programming. It difficults the evaluation of the practice.
Madeyski \cite{madeyski-package-dependencies} observed the results
among groups that practiced TDD, groups that practiced pair programming,
and the combination between them, and he was not able to show a significant
difference between teams that use TDD and teams that use pair programming
in terms of class package dependency management. However, when combining
the results, he found that TDD can help managing dependencies in class level.
According to him, developers should use TDD, but they should be aware of 
problems in class design.

Muller and Hagner \cite{muller-e-hagner} showed that TDD does not result in
better quality of productivity. Steinberg \cite{steinberg} showed that
code produced with TDD is more cohesive and less couple. Students also reported
that defect were easier to fix.

%% ------------------------------------------------------------------------- %%
\subsection{Discussion}

As presented, only a few studies evaluated the effects of TDD on class design.
And, when they do, they only discuss what the effects are and not exactly
\textbf{how} TDD has influenced. Josefsson \cite{josefsson}, in his discussion
about the need of an architectural phase and the effects of TDD in this matter,
comes to the same conclusion. According to him, the studies about TDD that are
found in the current literature are very limited and, because of that, the
"known" effects of the practice in class design can not be explained. Based on
our literature review, we believe that this limitation still exists.

Most part of these studies also does not take into account the experience
of the developer that is practicing TDD. Usually they only discuss it
during threats to validity. Janzen \cite{janzen-phd}, in his PhD, perceived
that mature developers obtain more benefits from the practice, writing simpler
classes. Besides, mature developers tend to choose TDD more often than
less experienced developers.

However, studies that analyze TDD from the class design point of view does not
come to a conclusive results; many of them even affirm that the effects of TDD
is not that different from those teams that do not practice it. Even the Janzen's PhD
thesis was inconclusive regarding the influence of the practice in coupling
and cohesion \cite{janzen-phd}. 

In addition, another strong point related to class design is simplicity and
the ease to evolve. A rigid class design that does not favor changes is hard
to be evaluated quantitatively. Needless complexity is also subjective.
Therefore, it is necessary more than an analytical comparison; the
developer's point of view should be taken into account.

%% ------------------------------------------------------------------------- %%
\section{Planning and Execution of the Study} 
\label{sec:planejamento}

Conducting an experimental exploratory study in software engineering has always
been a complex task. As suggested by the agile practices \cite{AgileManifesto}, 
one of the reasons is the human factor, very present in 
the software development process. Because of that, the analytic research
paradigm is not enough to investigate complex real cases that involves people
and their interactions with technology \cite{guidelines-case-study}.

A qualitative research is a way to explore and understand the influence that
individuals or groups put on a social or human problem. The research process
involves emerging questions and procedures, data are usually gathered through
the point of view from the participants, with the analyzis done in an inductive
way, from a very specific to a general theme, and with the researcher interpreting
the meaning of these data. Data captured by qualitative studies are usually
represented by words and figures. The final report contains a flexible structure.
Researchers that are dedicated to this kind of research support the inductive style and
the importance of showing the complexity of the situation \cite{creswell}.

As discussed in Section \ref{cap:trabalhos-relacionados}, many studies evaluated
TDD, and a few of them even found an improvement on class design, such as
less coupling, high cohesion, and more simplicity. Most of them focus on
the effects of the practice in the produced code, but only a few of them try to understand
the possible influence of the practice on the quality of it, and how TDD really guides
developers through these improvements.

To achieve our goal, we choosed to conduct an essencially 
exploratory qualitative study with developers in industry. Participants
were invited to implement a set of pre-prepared exercises. After that,
they were interview about how the practice influenced them in their
class design decisions. This section details the planning as well as
the data analysis process.


%% ------------------------------------------------------------------------- %%
\subsection{Research questions}

The main goal of this study is \textbf{to understand the relationship between
the practice of TDD and class design decisions taken by programmers in object-oriented
systems}. To compreehend them, we answered the questions below:

\begin{enumerate}

	\item What is the influence of TDD on class design?

	\item What is the relationship between TDD and the class design decisions take
	by a programmer?

	\item How does the practice of TDD influence class design in terms of coupling, 
	cohesion, and complexity?

\end{enumerate}

%% ------------------------------------------------------------------------- %%
\subsection{Research design}

Participants from different software development companies in the brazilian
market were selected. The participants' profile is discussed in sub-section
\ref{sec:planejamento-participantes}. All of them were invited to implement
a few problems using Java during a defined timebox. Participants practiced
TDD in one problem and did not practice in the other. The problems
as well as the definition of each of them should be implemented using TDD were
randomized in order to reduce bias.

All implementations were saved to posterior code metrics calculation.
Through them, we obtained relevant information about design quality (cohesion, 
coupling, and simplicity). The chosen metrics were cyclomatic complexity \cite{mccabe},
Fan-Out \cite{lorenz}, lack of cohesion of methods \cite{lcom-hs}, lines per method,
and quantity of methods. All these metrics are well-known in academia.

To calculate these metrics, we implemented our own tool. The reason is that
most part of the existant tools make use of compiled code (instead of pure source
code). Our tool contains an automated test suite and an open source
\footnote{\url{http://www.metricminer.org.br}. 
Last access on July, the 6th, 2012.}.

In addition, two specialists were invited to analyze the source code and give a score
to each one of them. Although the usefulness of the information passed by the code metrics,
the opinion of a specialist is enriching.

They evaluated the code based on three different categories: \textit{Simplicity}, \textit{Testability},
and \textit{Class Design Quality}. In each category, specialists should score from 1 (bad) to
5 (good) or choose not to evaluate that exercise. As a few participants did not finish the
implementation, specialists were warned to evaluate the intention of the participant and not
only the produced code.
To reduce the bias, the specialists did not know what code was written with TDD.

At the end of the exercise, all participants filled out a survey about their performance
on the exercises. After that, an initial analysis filtered the participants that
were interviewed posteriorly. The selection of the participants were based on
their answers on the survey plus the produced source code. Participants that implemented
a good solution on one problem but not on the other were selected. Besides, participants
that mentioned possible effects of the practice on class design were also interviewed.

The interview was semi-structured, giving some freedom to the researcher, so we were
able to change questions, if necessary. In addition, all questions were open, giving the
opportunity to the participant to talk deeper about the subject.

Once that the decisions taken by a programmer during the class design activity can be
influenced by many different factors, the questions were made to make participants
to triangulate their answers and to make them try to isolate the practice of TDD from
the other possible factors of influence. Participants that did not articulate well
the answers were removed from the analysis process.

All interviews were recorded so we were able to transcript and validate all
the captured data. In addition, we took some notes about participants' reactions
to specific questions. The interviews were made in different days,
according to their availability.

%% ------------------------------------------------------------------------- %%
\subsection{Participants' profile}
\label{sec:planejamento-participantes}

Developers from the brazilian software development market were invited to be part
of this research. The participants were evaluated according to their experience in TDD,
software development, Java, and unit tests. The only requirement was that the participant
should already knew how to write an unit test.

All participants filled out a survey before the start of the study. In this survey,
besides questioning their experience (quantitavely), it also contained open questions in which
the participant could tell about his experience in object-oriented systems, Java, and
TDD in a more detailed way.

We had 25 participants from 6 different companies. The participants, in their majority,
were not experienced in TDD. 40\% of them affirmed to use TDD for no more than a year.
52\% of them practice TDD from 1 to 3 years. Only 4\% practiced TDD from 3 to 4
years, and no participant was more experienced than that.

The numbers are a little different when talking about their experience in software
development. 24\% of them work with software development for the last 4 or 5 years.
28\% of them do that from 6 to 10 years. Only 20\% of them develop software for less
than 2 years. 64\% of the participants use Java. However, 36\% of them do not use
Java in their daily work. All of them know JUnit and 64\% use mock objects 
\footnote{Mock objects are objects created during a unit test. They mock the behavior 
of a concrete object. Usually they are used to isolate the unit test from the other classes. 
More information about it can be found on \cite{mocks}.}
during their development activities.
Only 12\% has never heard about mock objects. When talking about object-orientation,
in the open question, most part of the participants affirmed to have a good experience
on it. Some of them even affirmed to master the subject. Only a few of them affirmed
to have basic knowledge about OO.

We can also affirm that half of the participants are still trying the practice. The others
are more experienced and have already consolidated the practice. This is positive, as it
was possible to capture informations from people with different levels of maturity.

The elevated number of participants that do not use Java in their daily work is consequence
of a company that uses PHP. However, we checked that, although they do not use Java constantly,
they did not have troubles during the implementation of the exercises.

%% ------------------------------------------------------------------------- %%
\subsection{Proposed Problems}
\label{sec:exercicios}

We proposed four problems that should be implemented by the participants using Java
\footnote{The exercises can be found at \url{http://gist.github.com/3024328}. Last access on July, the 30th, 2012.}. 
The goal was to simulate recurrent class design problems in several software projects.
In Table \ref{tab:problemas-exercicios}, we present the relation of a bad implementation of
the exercises and class design principles that were not followed.

It was said to participants that the exercises simulate real world problems and that their code
would be supposely maintained by another team. Because of that, participants were asked to
implement the most elegant and flexible solution possible.

\begin{table}
	\centering
	\begin{tabular}{| l | l | l | }
		\hline
		\textbf{Exercise} & \textbf{Smell} & \textbf{Principles to}\\
		& & \textbf{be Followed}\\
		
		\hline
		
		Exercise 1 & Rigidity, Needless Complexity & SRP, OCP \\
		Exercise 2 & Fragility, Viscosity, Imobility & SRP, DIP, OCP \\
		Exercise 3 & Rigidity, Fragility & SRP\\
		Exercise 4 & Fragility, Viscosity, Imobility & OCP, SRP, DIP \\
		
		\hline
	\end{tabular}
	\caption{Proposed exercises and class design smells}
	\label{tab:problemas-exercicios}
\end{table}

%% ------------------------------------------------------------------------- %%
\section{Quantitative Analysis}

To triangulate the information of the qualitative analysis, we calculated
code metrics to check whether there was a difference in code produced
with the practice of TDD.
We analysed
264 production classes (831 methods, 2520 lines) and
73 test classes (225 methods, 1832 lines).

The statistical test used was the Wilcoxon. It is a non parametric hypothesis test
used to compare two samples and to check whether there is a difference in the mean
of both samples. Therefore, we used Wilcoxon to compare the mean between
the code metrics in code produced with and without TDD. The significance level
used was the standard (0.05).


%% ------------------------------------------------------------------------- %%
\subsection{Code metrics}

In Table \ref{metricas-industria}, we showed the \textit{p-values} found for
the difference between code produced with and without TDD. By the numbers,
we observed that there was no significant difference in cyclomatic complexity
and efferent coupling.
Lack of cohesion of methods was different in two exercises (1 and 4). A difference
also appeared on the quantity of lines in a method (exercise 4) and quantity of
methods (exercise 1).

When looking to the data of all exercises together, no metric presented a significant
difference. We concluded that, at least quantitatively, the practice of TDD did not
make any difference in code metrics.

\begin{table*}
	\centering
	\begin{tabular}{ | p{3cm} | p{2cm} | p{2cm} | p{2cm} | p{2cm} | p{2cm} |}
		\hline
		\textbf{Exercise} & \textbf{Cyclomatic Complexity} & \textbf{Efferent coupling} & \textbf{Lack of Cohesion of Methods} & \textbf{Number of lines per method} 
		& \textbf{Quantity of methods per class} \\
		\hline
		Exercise 1 &	0.8967	&	0.6741 &	\cellcolor[gray]{0.8}2.04E-07* &	0.4962 &	\cellcolor[gray]{0.8}2.99E-06* \\
		Exercise 2	& 0.7868	&	0.7640 &	0.06132 &	0.9925 &	0.7501 \\
		Exercise 3	& 0.5463	&	0.9872 &	0.5471 &	0.7216 &	0.3972\\
		Exercise 4	& 0.2198	&	0.1361 &	\cellcolor[gray]{0.8}0.04891* &	\cellcolor[gray]{0.8}0.0032* &	0.9358\\
		\hline
		All together &	0.8123	&	0.5604 &	0.3278 &	0.06814 &	0.5849\\
		\hline
	\end{tabular}
	\caption{\textit{P-values} found for the difference between code produced with and without TDD}
	\label{metricas-industria}
\end{table*}


In Table \ref{valores-separados}, we calculated the \textit{p-values} of the metric,
splliting them by experience in software development and TDD. The values for the group that
was experienced in TDD and not experience in software development was not calculated as no
participant fits on it.

By the numbers, we perceived that cohesion was the only metric that presented a significant
different in developers that are experienced in software development and TDD.

\begin{table}
	\centering
	\begin{tabular}{ | p{3cm} | p{2cm} | p{2cm} | }
		\hline
		  & \textbf{Experience in TDD} & \textbf{No experience in TDD} \\
		\hline
			\multicolumn{3}{|l|}{Cyclomatic Complexity} \\
		\hline
			Experience in Software Development 		& 0.09933	&	0.8976\\
			\hline
			No Experience in Software Development 	& NA		&	0.4462\\
		\hline
			\multicolumn{3}{|l|}{Fan-Out}\\
		\hline
			Experience in Software Development 		& 0.1401	&	0.6304\\
			\hline
			No Experience in Software Development 	& NA		&	0.2092\\
		\hline
			\multicolumn{3}{|l|}{Lack of Cohesion of Methods}\\
		\hline
			Experience in Software Development 		& \cellcolor[gray]{0.8}0.03061*	&	0.1284\\
			\hline
			No Experience in Software Development 	& NA		&	0.0888\\
		\hline
			\multicolumn{3}{|l|}{Quantity of Methods per Class} \\
		\hline
			Experience in Software Development 		& 0.09933	&	0.8976\\
			\hline
			No Experience in Software Development 	& NA		&	0.4462\\
		\hline
			\multicolumn{3}{|l|}{Lines per Method}\\
		\hline
			Experience in Software Development 		& 0.0513	&	0.4319\\
			\hline
			No Experience in Software Development 	& NA		&	0.5776\\
		\hline
	\end{tabular}
	\caption{\textit{P-values} found for the difference in metrics between experienced and non experienced participants}
	\label{valores-separados}
\end{table}

%% ------------------------------------------------------------------------- %%
\subsection{Specialists}

In Table
\ref{tab:especialistas-industria}, we presented the \textit{p-values} found for the
difference in the specialists' evaluation on code produced with and without TDD.
By the numbers, we observed that both specialists did not find any difference.

\begin{table}[h!]
	\centering
	\begin{tabular}{| p{2cm} | c | c | c | }
		\hline
		\textbf{Specialist} & \textbf{Class Design} & \textbf{Testability} & \textbf{Simplicity}\\
		\hline
		Specialist 1 &	0.4263 &	0.5235 &	0.3320\\
		Specialist 2 &	0.7447 &	0.4591 &	0.9044\\
		\hline
	\end{tabular}
	\caption{\textit{P-values} found for the difference in specialists' analysis}
	\label{tab:especialistas-industria}
\end{table}

%% ------------------------------------------------------------------------- %%
\section{Qualitative Analysis}

The values presented before corroborate with many of the related work.
Apparently TDD does not influence the code to the point of changing
the metric values for coupling, cohesion, and simplicity. However, this is
incoerent to the industry common feeling of the positive effects of TDD on class design.
As expected, in this study we conducted a qualitative analysis to understand
how this influence goes from the point of view of the developers.

In this section we present and discuss about the analysis and interpretation of
the qualitative data gathered during the execution of this study. In particular,
in Section \ref{padroes-tdd}, we defined some patterns about the feedback the practice
of TDD gives to the developer.

An interesting point is that the participants, regardless their experience on TDD or
software development, commented similar points. Because of that, we did not separate
discussions in the categories presented in Section \ref{sec:planejamento}.

%% ------------------------------------------------------------------------- %%
\subsection{Interview analysis}

Different from what was expected, a majority of the participants affirmed that the
practice of TDD would not change the class design produced by them during the exercises
they did not use the practice.
The main justification was that experience and previous knowledge in object-orientation
guided them during the class design. All participants agreed that a developer with no
knowledge in any of the mentioned areas would create a good class design only by practicing
TDD.

Two good examples that reinforce the point were given by the participants. One of them
said that made use of a design pattern \cite{gof} that has learned a few days before. Another
participant mentioned that his studies on SOLID principles helped him during the exercises.
The transcription below were extracted from the interviews:

\begin{framed}
	\textit{"It was even funny. I am reading the Design Patterns (book), and it discusses about polymorphism. 
	My implementation was based on that, because I've never done something like that before (...), here I rarely
	create new stuff, I just maintain legacy code."}
\end{framed}

In addition, the only participant from industry that has never practiced TDD before
affirmed that he did not feel any improvement in the class design when practicing.
Curiously, this participant said that he already knew that "TDD was a design technique".
It somehow indicates that the popularity of the effects of TDD in class design
is big.
That opinion wa slightly different from the experienced participants, 
that affirm that TDD is not only about design, but also about testing.

However, in contrary to the fact that TDD does not guide developers directly to a
good class design, all participants believe that TDD has positive effects in class design.
Many of them even mentioned the difficult when trying to stop using TDD:

\begin{framed}

	\textit{"When you are about to implement something, you end up thinking on the tests that you will do. 
	It is hard to think like "write code without thinking on tests!". As soon as you get used to it, 
	you just don't know another way to write code..."}
	
\end{framed}

According to them, TDD can help during class design process but, in order to achieve that,
the developer should have certain experience in software development. Most of participants
affirmed that their class designs were based on their experiences and past learnings.
In their opinion, the best option is to join the practice of TDD and experience:

\begin{framed}

	\textit{"The ideal is to put both things together [experience and TDD] (...) 
	I don't believe that TDD by itself could make things get better. There are many other
	concepts [that a developer should know] to make things good."}

\end{framed}

%% ------------------------------------------------------------------------- %%
\subsection{Rapid Feedback}

A grande maioria dos participantes também comentaram que uma diferença que percebem
no momento que praticam TDD é o \textit{feedback} mais constante. Na maneira
tradicional, o tempo entre a escrita do código de produção e o código
de testes é muito grande. O TDD, ao solicitar que o desenvolvedor
escreva o teste antes, também faz com que o desenvolvedor receba o \textit{feedback} que
os testes podem dar mais cedo:

\begin{framed}
\textit{"Você ia olhar para o teste, e falar: "Está legal? Não está?", e ia fazer de novo."}
\end{framed}

Um participante comentou que, com o teste, o desenvolvedor pode observar
e criticar o código que escreveu no momento logo após a escrita.
Essa crítica, de forma contínua, faz com que o desenvolvedor acabe
por pensar constantemente no código que está produzindo:

\begin{framed}
\textit{"Quando você faz o teste, você vê logo o que não gostou do método daquele jeito (...), você
não percebe isso até que você use o teste."}
\end{framed}

Diminuir o tempo entre a escrita do código e a escrita do teste também o ajuda a desenvolver código
que efetivamente resolve o problema. Segundo os participantes, na maneira tradicional, 
o desenvolvedor escreve muito código antes de saber se ele funciona:

\begin{framed}
\textit{"[O teste] não é só uma especificação; ele tem que de fato funcionar. Então,
como você diminui muito o tempo entre escrever um programa que funcione e testar aquilo,
você consegue mais rápido ver se aquela parte pequena funciona ou não (...)"}
\end{framed}

%% ------------------------------------------------------------------------- %%
\subsection{The search for testability}

Maybe the main reason the practice of TDD helps developers in their class design is
by the constant search for testability. It is possible to infer that,
when starting to produce code by its test, the production code should be, necessarily,
possible to be tested.

On the other hand, when code is not easy to be tested, developers understand that
as a class design smell. When it happens, developers usually try to refactor
the code to make it easier to be tested. One participant also affirmed that
take it as a rule: if it is hard to test, then the code can be improved.

\begin{framed}
	\textit{"I take it as a rule: everything it is very complex [the test],
	I think we should stop and refactor because, in my opinion, it can be simpler."}	
\end{framed}

That point was already raised by Feathers \cite{feathers-synergy}.
The harder is to write the test, the bigger the chance of a class design smell.
According to him, there is a big synergy between a high testable class and
a good class design: if the developers is looking for testability, he ends up
creating a good class design; if he is looking for a good class design, he ends
up writing testable code.

In that search for testability, developers are encouraged to write code
that is easily testable. Codes like that contain a few interesting
characteristics, such as the ease of invoke the expected behavior,
the non need of complicated pre-conditions, and the explicit declaration
of all dependencies.

The participants went deep further. During the interviews, many of them
mentioned about many patterns they have found in unit tests. These patterns
make them think about possible design problems in the class they are building.
All these patterns are discussed in the following sub-sections.

%% ------------------------------------------------------------------------- %%
\subsection{TDD Feedback Patterns}
\label{padroes-tdd}

As mentioned before, most part of the feedback a test can give occurs in the moment
that the developer finds a difficulty when writing the test. This sub-section discusses
patterns mentioned by the participants that make them believe that there is a
problem in the class design.

%% ------------------------------------------------------------------------- %%
\subsubsection{Patterns Related to Cohesion}

When only one method requires too many unit tests to assure its behavior,
it is probably too complex and/or contains many different responsibilities.
Codes like that usually contain many different paths and tend to update
many internal attributes of the object. In those cases, the developer
are required to write many tests to cover all the possible scenarios.
We call this pattern \textbf{Many Tests to a Method}.

The same idea can be generalized to the class. Classes that expose too 
many methods to the external world also tend to have many responsibilities.
Classes like that should be refactored as well.
We call this pattern \textbf{Many Tests to a Class}.

Another cohesion problem can be found when a developer feels the need
of writing big scenarios for only one class or method. It is possible to infer
that this need emerges in classes that deal with too many objects or
contain too many responsibilities. We call it \textbf{Too Big Scenario}.

One pattern that was not explicitly mentioned by the participants, but
noticed by us, is when a developer feels the need to test a method
that is not public. Private methods usually helps to transform a public method
into something more easier to be read. When trying to test it in an isolated
way, the developer may have found a method that contains too much responsibility.
That responsibility can be moved to a different class. We call it \textbf{Tests in Non Public Methods}.

%% ------------------------------------------------------------------------- %%
\subsubsection{Patterns Related to Coupling}

The abusive use of mock objects to test a single class indicates that
the class under test has coupling issues. It is possible to infer that a class
that makes use of many mock objects depends upon many classes and, therefore,
tend to be an unstable class. We call it \textbf{Too Many Mock Objects}.

Another pattern perceived by us is the creation of mock objects that are
not used in some test methods. That usually happens when the class
is highly coupled, and the result of one dependency does not infer in
the other. When it happens, the programmer ends up writing a set of tests,
in which some of them deal with a subset of mock objects, while other
tests deal with the other subset of mock objects. It indicates that
the class is highly coupled and needs to be refactored. We call it
\textbf{Non Used Mock Objects}.

%% ------------------------------------------------------------------------- %%
\subsubsection{Patterns Related to Abstractions}



A falta de abstração geralmente faz com que uma simples mudança precise
ser feita em diferentes pontos do código. Quando uma mudança acontece e 
o programador é obrigado a fazer a mesma alteração em diferentes testes,
isso indica a falta de uma abstração correta para evitar a 
repetição desnecessária de código.
A esse padrão damos o nome de \textbf{Mesma Alteração Em Diferentes Testes}.
Analogamente, o programador pode perceber a mesma coisa
quando ele começa a criar testes repetidos para entidades diferentes.
Chamamos esse padrão de \textbf{Testes Repetidos Para Entidades Diferentes}.

Quando o desenvolvedor começa o teste e percebe que a interface pública da classe
não é fácil de ser utilizada, pode indicar que abstração
corrente não é clara o suficiente e poderia ser melhorada. A esse padrão,
chamamos de \textbf{Interface Não Amigável}.

Outro padrão não mencionado explícitamente pelos participantes 
é a existência da palavra \textit{"se"} no nome do teste. Testes que
possuem nomes como esse geralmente indicam a existência de um \textit{"if"} na implementação
do código de produção. Essas diversas condições podem, geralmente, ser refatoradas e,
por meio do uso de poliformismo, serem eliminadas. A falta de abstração nesse caso
é evidenciada pelo padrão \textbf{Condicional No Nome Do Teste}.

%% ------------------------------------------------------------------------- %%
\subsection{Relação dos padrões com os princípios de projeto de classes}

É possível relacionar os padrões de \textit{feedback} levantados pelos participantes
com os mau cheiros de projeto de classes comentados neste trabalho. 
Na Tabela \ref{tab:relacao-padroes},
mostramos essa relação, e como esses padrões podem efetivamente ajudar o desenvolvedor
a procurar por problemas no seu projeto de classes.


\begin{table}[h!]
	\centering
	\begin{tabular}{| p{2.5cm} | p{2.5cm} | p{2cm} | }
		\hline

		\textbf{Padrão} & \textbf{Possíveis Mau Cheiros de Projeto de Classes} & \textbf{Possíveis Princípios Feridos}\\
		
		\hline

		Muitos Testes Para Um Método                   & Complexidade Desnecessária, Opacidade   & PRU \\ \hline
		Muitos Testes Para Uma Classe                  & Complexidade Desnecessária, Opacidade   & PRU \\ \hline
		Cenário Muito Grande                           & Opacidade, Fragilidade                  & PRU \\ \hline
		Testes Em Método Que Não É Público             & Complexidade Desnecessária              & PRU, PAF \\ \hline
		Objetos Dublê em Excesso                       & Fragilidade                             & PID, PAF \\ \hline
		Objetos Dublês Não Utilizados                  & Fragilidade                             & PID, PAF \\ \hline
		Mesma Alteração Em Diferentes Testes           & Fragilidade, Rigidez                    & PRU \\ \hline
		Testes Idênticos Para Entidades Diferentes     & Repetição Desnecessária, Rigidez        & PRU  \\ \hline
		Interface Não Amigável                         & Opacidade                               & ISP \\ \hline
		Condicional No Nome Do Teste                   & Rigidez, Fragilidade                    & PRU, PAF \\

		\hline
		
	\end{tabular}
	\caption{Relação entre os padrões de \textit{feedback} de TDD e mau cheiros de projeto de classes}
	\label{tab:relacao-padroes}
\end{table}

%% ------------------------------------------------------------------------- %%
\section{Ameaças à Validade}
\label{cap:ameacas}

%% ------------------------------------------------------------------------- %%
\subsection{Validade de Construção}

%% ------------------------------------------------------------------------- %%
\subsubsection{Exercícios de pequeno porte}

Os exercícios propostos são pequenos perto de um projeto real. Entretanto, todos os exercícios propostos contém
problemas localizados de projeto de classes. Uma vez que esta pesquisa avalia os 
efeitos de TDD no projeto de classes, 
acreditamos que os problemas conseguem simular de forma satisfatória
problemas de projeto de classes que desenvolvedores encaram no dia a dia de trabalho.

Além disso, ao final do exercício, os participantes responderam a uma pergunta sobre a semelhança
entre os problemas de projeto de classes propostos e os problemas encontrados no mundo real.
Todos os participantes da indústria afirmaram que os problemas se parecem com os que eles enfrentam
no dia a dia de trabalho. 

%% ------------------------------------------------------------------------- %%
\subsection{Validade interna}

%% ------------------------------------------------------------------------- %%
\subsubsection{Efeitos recentes de TDD na memória}

Muitos dos participantes da indústria afirmaram que utilizam TDD no seu dia a dia de trabalho.
Isso pode fazer com que o participante
não avalie friamente as vantagens e desvantagens do desenvolvimento sem TDD. 

Para diminuir esse viés, os participantes fizeram alguns exercícios também
sem TDD, para que ambos os estilos de desenvolvimento (com e sem TDD) estivessem
recentes em sua memória.

%% ------------------------------------------------------------------------- %%
\subsubsection{Exercícios}

Alguns participantes também não terminaram suas implementações dos exercícios. Isso
pode influenciar na análise quantitativa, afinal, um projeto de classes que
seria complexo assim que pronto, ao olho da métrica, pode aparentar ser simples.

%% ------------------------------------------------------------------------- %%
\subsubsection{Influência do pesquisador}

O pesquisador possui
um papel fundamental em pesquisas qualitativas. Mas isso pode fazer com que
a interpretação dos resultados seja influenciada pelo contexto, experiências,
e até viéses do próprio pesquisador.
Neste estudo, a nossa opinião teve forte influência na seleção dos candidatos
para a entrevista.
Além disso, os exercícios foram criados pelos pesquisadores dessa pesquisa e, 
de certa forma, podem ter influenciado a prática de TDD.
Para diminuir esse problema, revisamos todas as análises,
buscando por conclusões incorretas ou não tão claras. 

%% ------------------------------------------------------------------------- %%
\subsection{Validade externa}

%% ------------------------------------------------------------------------- %%
\subsubsection{Desejabilidade social}

Enviesamento pela desejabilidade social é o termo científico usado para descrever
a tendência de que alguns participantes respondam a questões de modo que serão
bem vistos pelos outros membros da comunidade \cite{crowne}.
Métodos ágeis e TDD possuem um discurso forte. A comunidade brasileira de métodos
ágeis ainda é nova e percebe-se de maneira empírica que muitos repetem o discurso
sem grande experiência ou embasamento no assunto.
No caso desta pesquisa, um possível viés é o participante responder o que
a literatura diz sobre TDD, e não exatamente o que ele pratica e sente sobre
os efeitos da prática. 

Para diminuir esse viés, eliminaríamos do processo de análise os participantes
que responderam às perguntas de forma superficial, apenas repetindo a literatura. Na prática,
isso não aconteceu. Em sua maioria, poucas foram as respostas nas quais os participantes
foram superficiais. Nesses casos, essas respostas foram eliminadas da análise.

%% ------------------------------------------------------------------------- %%
\subsubsection{Quantidade de participantes insuficiente}

Apesar de termos feito contato
com diversas empresas e grupos de desenvolvimento de software,
objetivando encontrar um bom número de participantes para a pesquisa,
a quantidade de participantes final do estudo pode não ser suficiente para generalizar
os resultados encontrados. 

%% ------------------------------------------------------------------------- %%
\section{Conclusion}
%\section{Conclusão}


In this study, we discuss and understand how the TDD practices can make the
difference in the routine of a software developer, bringing a better meaning
to the statement that the practice improves the classes design.
In addition to endorse the quantitative studies form the literature, this study
goes beyond and analyzes feedback patterns that are showed in the moment when
the developer uses TDD and, in practice, guides him during the development.

%Neste trabalho, discutimos e entendemos como a prática de TDD pode
%fazer a diferença no dia a dia de um desenvolvedor de software,
%trazendo um melhor significado à afirmação de que a prática melhora o projeto de classes.
%Além de corroborar com os estudos quantativos da literatura, este estudo
%foi além, e observou padrões de \textit{feedback} que aparecem
%no momento em que o desenvolvedor utiliza TDD, e que, na prática, o guia durante
%o desenvolvimento. 

It is expected, with this cataloged patterns, that developers pay more attention
to the potential feedbacks that TDD brings and, use them to improving their class
design. The teaching of TDD can also be be assisted by this study, considering
that teachers can use the patterns raised here and show to students
how the practice can, actually guide the developer to a better class design.


%Espera-se que, com esses padrões catalogados, desenvolvedores atentem-se mais
%aos possíveis \textit{feedbacks} que a prática de TDD fornece e, utilizem-os
%para melhorar a qualidade dos seus projetos de classe. O ensino de TDD também
%pode ser beneficiado por este trabalho, já que professores podem fazer uso dos padrões
%aqui levantados e mostrar ao aluno como a prática pode efetivamente guiar
%o desenvolvedor para um projeto de classes melhor.

A possible future work is the developing of tools that automatically detects this patterns
and warn the developer about a possible code smell in design. Besides that,
the search for more patterns like the showed here can improve even more the feedback
provided by the practice.

%Um possível trabalho futuro
%é a criação de ferramentas que automaticamente detectam esses padrões e notifiquem
%o desenvolvedor sobre o possível mau cheiro de projeto. 
%Além disso, a busca por mais padrões como os levantados aqui pode aumentar ainda 
%mais o \textit{feedback} dado pela prática. 

In below subsections, we answer each of the questions raised by this study.

%Nas sub-seções abaixo, respondemos cada uma das questões levantadas
%por este trabalho.

\subsection{What is the influence of TDD in class design?}
%\subsection{Qual a influência de TDD no projeto de classes?}

The practice of TDD \textbf{can} affect the creating process of class design. However,
unlike of the comments made by the industry, \textbf{the TDD practice cannot guide the
developer to a good class design automatically}; the experience and knowledge of
developer are essential to develop of oriented object software.

%A prática de TDD \textbf{pode} influenciar no processo de criação do projeto de classes.
%No entanto, ao contrário do que é comentado pela indústria,
%\textbf{a prática de TDD não guia o desenvolvedor para um bom projeto de classes
%de forma automática}; a experiência e conhecimento 
%do desenvolvedor são fundamentais ao criar software orientado a objetos.

The practice, through the possibles feedbacks about of class design, heavily
discussed in Section \ref{padroes-tdd}, can be a guide to the developer.
This feedbacks, when noted, make the developer realize problems in class
design beforehand, easing code refactoring (Q1 and Q2).

%A prática, por meio dos seus possíveis \textit{feedback} em relação ao
%projeto de classes, discutidos
%em profundidade na Seção \ref{padroes-tdd}, pode servir de guia
%para o desenvolvedor. Esses \textit{feedback}, quando observados, fazem
%com que o desenvolvedor perceba problemas de projeto de classes de
%forma antecipada, facilitando a refatoração do código (Q1 e Q2).

\textbf{Therefore, this is the way how the practice guides the developer to
a better class design; giving constant feedback about potential issues in
current class design. It is a developer's task to realize this issues and
respectively, improve the design. }

%\textbf{Portanto, esta é a forma na qual a prática guia o desenvolvedor para
%um melhor projeto de classes: dando retorno constante sobre os possíveis problemas
%existentes no atual projeto de classes. É tarefa do desenvolvedor perceber
%esses problemas e melhorar o projeto de acordo.}

\subsection{What is the relation between TDD and design decisions made by
a developer?}

%\subsection{Qual a relação entre TDD e as tomadas de decisões de projeto
%feitas por um desenvolvedor?}

The developer that uses TDD writes tests before the code. This make
the current unit test be treated as a draft by the developer.
\textbf{Noting carefully the code of the unit test, the developer can
realize issues in class design that he is writing. Issues like 
classes with too many responsibilities or with too many dependencies}.

%O desenvolvedor que pratica TDD escreve os testes antes do código.
%Isso faz com que o teste de unidade que está sendo escrito sirva de
%rascunho para o desenvolvedor. \textbf{Ao observar o código do teste de unidade com atenção,
%o desenvolvedor pode perceber problemas no projeto de classes que está criando.
%Problemas esses como classes que possuem diversas responsabilidades
%ou que possuem muitas dependências}. 

\subsection{How the TDD practice affects the programmer in the process of designing
classes, about the coupling, cohesion and complexity?}
%\subsection{Como a prática de TDD influencia o programador no processo de  
%projeto de classes, do ponto de vista do acoplamento, coesão e complexidade?}

Writing a unit test to a specific class, the developer is forced to follow
the same steps: the writing of scenery, the execution of the action being
tested and lastly the assurance of the behavior was executed like the
expected.
\textbf{Any trouble writing this steps, can imply issues in the class design}. 
The aware developer, notices and improves the respectively design class.

%Ao escrever um teste de unidade para uma determinada classe, o desenvolvedor
%é obrigado a passar sempre pelos mesmos passos: a escrita do cenário, a execução
%da ação sob teste e, por fim, a garantia que o comportamento foi executado 
%de acordo com o esperado.
%\textbf{Uma dificuldade na escrita de qualquer uma dessas partes pode implicar
%em problemas no projeto de classes}. O desenvolvedor, atento, percebe e 
%melhora seu projeto de classes de acordo.

%% ------------------------------------------------------------------------- %%
\section*{Acknowledgement}

Agradeçemos às empresas que aceitaram participar deste estudo: Caelum, Bluesoft,
Amil e WebGoal (São Paulo e Poços de Caldas). Além disso, agradecemos aos
desenvolvedores que participaram da pesquisa de forma independente.

%% ------------------------------------------------------------------------- %%
\bibliographystyle{IEEEtran}
\footnotesize{\bibliography{paper}}

% that's all folks
\end{document}

%% ------------------------------------------------------------------------- %