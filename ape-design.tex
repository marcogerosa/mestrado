%% ------------------------------------------------------------------------- %%
\chapter{Projeto de Classes em Sistemas Orientados a Objetos}
\label{ape:design}

Conforme sugerido por esta pesquisa, para escrever classes com alta
testabilidade, o profissional da prática de TDD acaba por fazer uso de boas práticas de
desenvolvimento de software. 
Esse apêndice discute algumas dessas boas práticas, que
foram utilizadas durante o processo de avaliação desta pesquisa.

%% ------------------------------------------------------------------------- %%
\section{Sintomas de Projetos de Classes em Degradação}
\label{sec:design-degradacao}

Diz-se que um projeto de classes está \textit{degradando}
quando o mesmo começa a ficar difícil de evoluir, o reúso de código se 
torna mais complicado do que repetir o trecho de código, ou o custo de se fazer 
qualquer alteração no projeto de classes se torna alto.

Robert Martin \cite{bob-martin} enumerou alguns sintomas de projeto de classes em degradação, 
chamados também de \textit{"maus cheiros"} de projeto de classes. Esses sintomas são parecidos com os 
maus cheiros de código (\textit{"code smells"}), mas em um nível mais alto: eles
estão presentes na estrutura geral do software ao invés de estarem localizados
em apenas um pequeno trecho de código.

Esses sintomas podem ser medidos de forma subjetiva e algumas vezes de forma 
até objetiva. Geralmente, esses sintomas são causados por violações de um ou 
mais princípios de projeto de classes, apresentados na seção \ref{sec:design-oo-principios}.
Muitos desses problemas são relacionados à gerência de dependências. Quando essa
atividade não é feita corretamente, o código gerado torna-se difícil de manter e
reusar. Entretanto, quando bem feita, o software tende a ser flexível, robusto 
e suas partes reusáveis.

\subsection{Rigidez}
\label{subsec:rigidez}

Rigidez é a tendência do software em se tornar difícil de mudar, mesmo de 
maneiras simples. Toda mudança causa uma cascata de mudanças subsequentes  em
módulos dependentes. Quanto mais módulos precisam ser modificados, maior é a
rigidez do projeto de classes. 

Quando um projeto de classes está muito rígido, não se sabe com segurança quando uma 
mudança terá fim. Mudanças simples passam a demorar muito tempo até serem 
aplicadas no código e frequentemente acabam superando em várias vezes a 
estimativa de esforço inicial. 
Frases como \textit{"isto foi muito mais complicado do que eu imaginei"} 
tornam-se populares. Neste momento, gerentes de desenvolvimento começam a ficar
receosos em permitir que os desenvolvedores consertem problemas não críticos.

\subsection{Fragilidade}
\label{subsec:fragilidade}

Fragilidade é a tendência do software em quebrar em muitos lugares diferentes 
toda vez que uma única mudança acontece. Frequentemente, os novos problemas
ocorrem  em áreas não relacionadas conceitualmente com a área que foi mudada, 
tornando o processo de manutenção demasiadamente custoso, complexo e tedioso. 

Consertar os novos problemas usualmente passa a resultar em outros novos
problemas e assim por diante. Infelizmente, módulos frágeis são comuns em
sistemas de software. São estes os módulos que sempre aparecem na lista de defeitos
a serem corrigidos. 
Além disto, desenvolvedores começam a ficar receosos de alterar certos trechos 
de código, pois sabem que estes estão tão frágeis que qualquer mudança simples 
fatalmente acarretará na introdução de problemas inesperados e de naturezas
diversas.

\subsection{Imobilidade}

Imobilidade é a impossibilidade de se reusar software de outros projetos ou de
partes do mesmo projeto. Neste cenário, o módulo que se deseja reutilizar 
frequentemente tem uma bagagem muito grande de dependências e não possui código 
claro. 
Depois de muita investigação, os arquitetos descobrem que o trabalho e o risco 
de separar as partes desejáveis das indesejáveis são tão grandes, que o módulo 
acaba sendo reescrito ao invés de reutilizado.

\subsection{Viscosidade}

Quando uma mudança deve ser realizada, usualmente há várias opções para realizar
tal mudança. Quando as opções que preservam o projeto de classes são mais difíceis de serem
implementadas do que aquelas que não o preservam, há alta viscosidade 
de projeto de classes. Neste cenário, é fácil fazer a "coisa errada" e é difícil fazer a
"coisa certa", ou seja, é difícil preservar e aprimorar o projeto de classes.

\subsection{Complexidade Desnecessária}

Detecta-se complexidade desnecessária no projeto de classes quando ele contém muitos 
elementos inúteis ou não utilizados (\textit{dead code}). Geralmente ocorre
quando há muito projeto inicial (\textit{up-front design}) e não se segue uma 
abordagem de desenvolvimento iterativa e incremental, de modo que os projetistas
tentam prever uma série de futuros requisitos para o sistema e concebem um 
projeto de classes demasiadamente flexível ou desnecessariamente sofisticado. 

Frequentemente apenas algumas previsões acabam se concretizando ao longo do
tempo e, neste meio período, o projeto de classes carrega o peso de elementos e construções 
não utilizados. O software então se torna complexo e difícil de ser entendido. 
Projetos com complexidade muito alta comumente afetam a produtividade, porque 
quando os desenvolvedores herdam tal projeto, eles gastam muito tempo 
aprendendo as nuances do projeto de classes antes que possam efetivamente  estendê-lo ou
mantê-lo confortavelmente \cite{kerievsky}.

\subsection{Repetição Desnecessária}

Quando há repetição de trechos de código, é sinal de que uma abstração
apropriada não foi capturada durante o processo de projeto de classes (ou inclusive na
análise). Esse problema é frequente e é comum encontrar softwares que contenham 
dezenas e até centenas de elementos com códigos repetidos. 

Descobrir a melhor abstração para eliminar a repetição de código geralmente não 
está na lista de itens de alta prioridade dos desenvolvedores, de maneira que a 
resolução do problema acaba sendo eternamente postergada. Também, o sistema se
torna cada vez mais difícil de entender e principalmente de manter, pois os 
problemas encontrados em uma unidade de repetição devem ser corrigidos
potencialmente  em toda repetição, com o agravante de que uma repetição pode
ter forma ligeiramente diferente de outra.

\subsection{Opacidade}

Opacidade é a tendência de um módulo ser difícil de ser entendido. Códigos podem
ser escritos de maneira clara e expressiva ou de maneira "opaca" e complicada. A
tendência de um código é se tornar mais e mais opaco à medida que o tempo passa
e, para que isso seja evitado, é necessário um esforço constante em manter esse 
código claro e expressivo. 

Uma maneira para prevenir isso é fazer com que os desenvolvedores se ponham no
papel de leitores do código e refatorem esse código de maneira que qualquer
outro  leitor poderia entender. Além disso, revisões de código feita por outros
desenvolvedores é também uma possível solução para manter o código menos opaco.

%% ------------------------------------------------------------------------- %%
\section{Princípios de Projeto de Classes}
\label{sec:design-oo-principios}

Todos os problemas citados na seção \ref{sec:design-degradacao} podem ser
evitados pelo uso puro e simples de orientação a objetos. A máxima da
programação orientada a objetos diz que classes devem possuir um baixo
acoplamento e uma alta coesão.

Alcançando esse objetivo, mudanças no código seriam executadas mais facilmente;
alterações seriam feitas em pontos únicos e a propagação de mudanças seria bem
menor. Com as abstrações bem estabelecidas, novas funcionalidades seriam
implementadas através de novo código, sem a necessidade de alterações no código
já existente. Necessidades de evolução do projeto de classes seriam feitas com pouco
esforço, já que módulos dependeriam apenas de abstrações.

Mas, alcançar tal objetivo não é tarefa fácil. Criar classes pouco acopladas e
altamente coesas demanda um grande esforço por parte do desenvolvedor e requer
grande conhecimento e experiência no paradigma da orientação a objetos.

Os princípios comentados nesta seção são muito discutidos por Robert Martin
em vários de seus livros e artigos publicados \cite{bob-martin}.
Esses princípios são produto de décadas de experiência em engenharia de
software. Segundo ele, esses princípios não são produto de uma única 
pessoa, mas sim a integração de pensamentos e trabalhos de um grande número de 
desenvolvedores de software  e pesquisadores, e visam combater todos os sintomas
de degradação discutidos na Seção \ref{sec:design-degradacao}.

Conhecidos pelo acrônimo \textit{SOLID} (sólido, em português), são eles:

\begin{itemize}
	\item Princípio da Responsabilidade Única (\textit{Single-Responsibility 
	Principle (SRP)})
	\item Princípio do Aberto-Fechado (\textit{Open-Closed Principle (OCP)})
	\item Princípio de Substituição de Liskov (\textit{Liskov Substitution 
	Principle (LSP)})
	\item Princípio da Inversão de Dependência (\textit{Dependency Inversion
	Principle (DIP)})
	\item Princípio da Segregação de Interfaces (\textit{Interface Segregation 
	Principle (ISP)})
\end{itemize}

\subsection{Princípio da Responsabilidade Única}
\label{subsec:principio-srp}

O termo coesão define a relação entre os elementos de um mesmo módulo
\cite{demarco} \cite{pagejones}. Isso significa que os todos elementos de uma 
classe que tem apenas uma responsabilidade tendem a se relacionar. Diz-se que
uma classe como essa é uma classe que possui alta coesão (ou que é coesa). 
Já em uma classe com muitas responsabilidades diferentes, os elementos tendem a
se relacionar apenas em "grupos", ou seja, com os elementos que tratam de uma 
das responsabilidades da classe. A esse tipo de classe, diz-se que ela possui 
uma baixa coesão (ou que não é coesa).
Robert Martin altera esse conceito de coesão e a relaciona com as forças
que  causam um módulo ou uma classe a mudar. No caso, o Princípio de
Responsabilidade Única diz que uma classe deve ter apenas uma única razão para 
mudar \cite{bob-martin}.

Esse princípio é importante no momento em que há uma alteração em alguma 
funcionalidade do software. Quando isso ocorre, o programador precisa procurar 
pelas classes que possuem a responsabilidade a ser modificada. Supondo uma
classe  que possua mais de uma razão para mudar, isso significa que ela é
acessada  por duas partes do software que fazem coisas diferentes.  Fazer uma
alteração em uma das responsabilidades dessa classe pode, de maneira não
intencional, quebrar a outra parte de maneira inesperada. Isso torna o projeto de classes
frágil, como comentado na sub-seção \ref{subsec:fragilidade}.

\subsection{Princípio do Aberto-Fechado}
\label{subsec:ocp}

O Princípio do Aberto-Fechado, cunhado por Bertrand Meyer, diz que as entidades
do software (como classes, módulos, funções, etc) devem ser abertas para
extensão, mas fechadas para alteração \cite{meyer-ocp}. 
Se uma simples alteração resulta em uma cascata de alterações em módulos
dependentes, isso cheira à rigidez, conforme descrito na sub-seção 
\ref{subsec:rigidez}. O princípio pede então para que o programador sempre 
refatore as classes de modo que mudanças desse tipo não causem mais modificações.

Quando esse princípio é aplicado de maneira correta, novas alterações fazem com
que o programador adicione novo código, e não modifique o anterior. Isso é
alcançado através da criação de abstrações para o problema. Linguagens
orientadas a objetos possuem mecanismos para criá-las (conhecido com interfaces
em linguagens como Java ou C\#). Através dessas abstrações, o programador consegue 
descrever a maneira em que uma determinada classe deve se portar, mas sem se
preocupar em como essa classe faz isso.

\subsection{Princípio de Substituição de Liskov}
\label{subsec:lsp}

Esse princípio, que discute sobre tipos e sub-tipos, criado por Barbara Liskov
em 1988 \cite{liskov}, é importante já que herança é uma das maneiras para se 
suportar abstrações e polimorfismo em linguagens orientadas a objetos e, como 
visto na seção \ref{subsec:ocp}, o Princípio do Aberto-Fechado se baseia 
fortemente na utilização desses recursos.

O problema é que utilizar herança não é tarefa fácil, pois o acoplamento criado
entre classe filha e classe pai é grande. Fazer as classes filhas respeitarem o
contrato do pai, e ainda permitir que mudanças na classe pai não influenciem nas
classes filhas requer trabalho.

O princípio de Liskov diz que, se um tipo S é sub-classe de um tipo T,
então objetos do tipo T podem ser substituídos por objetos do tipo S, sem
alterar nenhuma das propriedades desejadas daquele programa.

Um clássico exemplo sobre Princípio de Substituição de Liskov é o exemplo dos
Quadrados e Retângulos. Imagine uma classe Retângulo. Um retângulo possui dois 
lados de tamanhos diferentes. Imagine agora uma classe Quadrado (figura
geométrica que possui todos os lados com o mesmo tamanho) que herde de
Retângulo. A única alteração é fazer com que os dois lados tenham o mesmo 
tamanho. 
Apesar de parecer lógico, afinal um Quadrado é um Retângulo com apenas uma
condição diferente, a classe Quadrado quebra o Princípio de Liskov: a
pré-condição dela é mais forte do que a do quadrado, afinal os dois lados devem 
ter o mesmo tamanho.

Quebras do princípio de Liskov geralmente levam o programador a quebrar o
princípio do OCP também. Ele percebe que, para determinados sub-tipos, ele 
precisa fazer um tratamento especial, e acaba escrevendo condições nas classes 
clientes que fazem uso disso.

\subsection{Princípio da Inversão de Dependências}
\label{subsec:dip}

Classes de baixo nível, que fazem uso de infraestrutura ou de outros detalhes
de implementação podem facilmente sofrer modificações. E, se classes de mais alto
nível dependerem dessas classes, essas modificações podem se propagar, tornando
o código frágil.

O Princípio de Inversão de Dependências se baseia em duas afirmações:

\begin{itemize}
	\item Módulos de alto nível não devem depender de módulos de baixo nível. 
	Ambos devem depender de abstrações
	\item Abstrações não devem depender de detalhes. Detalhes devem depender de
	abstrações
\end{itemize}

Em resumo, as classes devem, na medida do possível, acoplar-se sempre com módulos mais
estáveis do que ela própria, já que, como as mudanças em módulos estáveis são
menos prováveis, raramente essa classe precisará ser alterada por mudanças em
suas dependências \cite{bobmartin-oodmetrics}.

\subsection{Princípio da Segregação de Interfaces}
\label{subsec:isp}

Acoplar-se com uma interface de baixa granularidade (ou gordas, do termo
em inglês \textit{fat interfaces}) pode ser perigoso, já que qualquer alteração
que um outro cliente forçar nessa interface poderá ser propagada para essa
classe.

O princípio da segregação de interfaces diz que classes cliente não devem ser
forçados a depender de métodos que eles não usam. Quando uma interface não é
coesa, ela contém métodos que são usados por um grupo de clientes, e outros 
métodos que são usados por outro grupo de clientes. Apesar de uma classe poder 
implementar mais de uma interface, o princípio diz que o cliente da classe deve
apenas depender de interfaces coesas.

%% ------------------------------------------------------------------------- %%
\section{Conclusão}

Todos os princípios discutidos na seção \ref{sec:design-oo-principios} tentam
diminuir os possíveis problemas de projeto de classes que possam eventualmente aparecer.
Discutir o que é um bom projeto de classes é algo difícil; mas é possível enumerar algumas
das características desejáveis: isolar elementos reusáveis de elementos não
reusáveis, diminuir a propagação de alterações em caso de uma nova
funcionalidade.

Esses serão os princípios de projeto de classes levados em conta no momento da análise
dos dados colhidos.