%% ------------------------------------------------------------------------- %%
\chapter{Ameaças à Validade}
\label{cap:ameacas}

\section{Validade de Construção}

Uma pesquisa é válida do ponto de vista de construção quando seus instrumentos realmente
medem as informações necessárias para o estudo.

\subsection{Exercícios de pequeno porte}

Os exercícios propostos são pequenos perto de um projeto real. Todos os exercícios propostos contém
problemas localizados de projeto de classes. E, uma vez que esta pesquisa tenta avaliar os efeitos de TDD no projeto de classes, 
acreditamos que os problemas conseguem simular de forma satisfatória
problemas de projeto de classes que desenvolvedores encaram no dia a dia de trabalho.

Além disso, ao final do exercício, os participantes responderam uma pergunta sobre a semelhança
entre os problemas de projeto de classes propostos e os problemas encontrados no mundo real.
Todos os participantes da indústria afirmaram que os problemas se parecem com os que eles enfrentam
no dia a dia de trabalho. Já os participantes da academia afirmaram que não sabiam dizer, afinal não
possuíam experiência prática.

\section{Validade interna}

Uma pesquisa tem alta validade interna quando ela é capaz de diminuir o valor das hipóteses alternativas, mostrando
que a hipótese estudada é a explicação mais plausível dos dados. Para isso, a pesquisa precisa controlar as possíveis
variáveis que poderiam influenciar na coleta, análise e interpretação dos dados. A validade interna é portanto
garantida quando o planejamento do estudo nos possibilita ter certeza de que as relações observadas de
forma empírica não podem ser explicadas por outros fatores.

As sub-seções abaixo discutem as possíveis ameaças à validade interna.

\subsection{Efeitos recentes de TDD na memória}

Por serem entrevistados pouco tempo depois da resolução dos exercícios, os participantes tínham
em suas mentes os efeitos recentes de TDD no código. Isso pode fazer com que o participante
não avalie friamente as vantagens e desvantagens do desenvolvimento sem TDD. 

Para diminuir esse viés, os participantes fizeram alguns exercícios também
sem TDD, para que ambos os estilos de desenvolvimento (com e sem TDD) estejam
recentes em sua memória.

\subsection{Exercícios inacabados}

Alguns participantes não terminaram suas implementações dos exercícios. Isso
pode influenciar na análise quantitativa, afinal, um projeto de classes que
seria complexo assim que pronto, ao olho da métrica, pode aparentar ser simples.

\subsection{Influência do pesquisador}

Como discutido no capítulo \ref{cap:qualitativo-planejamento}, o pesquisador possui
um papel fundamental em pesquisas qualitativas. Mas isso pode fazer com que
a interpretação dos resultados seja influenciada pelo contexto, experiências,
e até viéses do próprio pesquisador.
Neste estudo, a nossa opinião teve forte influência na seleção dos candidatos
para a entrevista.
Para diminuir esse problema, revisamos todas as análises,
buscando por conclusões incorretas ou não tão claras. 

\section{Validade externa}

Uma pesquisa possui validade externa quando ela possibilita ao pesquisador 
generalizar os resultados obtidos à outras populações ou outros contextos.

As sub-seções abaixo discutem as possíveis ameaças à validade externa
desta pesquisa.

\subsection{Desejabilidade social}

Enviesamento pela desejabilidade social é o termo científico usado para descrever
a tendência de que alguns participantes respondam questões de modo que serão
bem vistos pelos outros membros da comunidade \cite{crowne}.
Métodos ágeis e TDD possuem um discurso forte. A comunidade brasileira de métodos
ágeis ainda é nova e percebe-se de maneira empírica que muitos repetem o discurso
sem grande experiência ou embasamento no assunto.
No caso desta pesquisa, um possível viés é o participante responder o que
a literatura diz sobre TDD, e não exatamente o que ele pratica e sente sobre
os efeitos da prática. 

Para diminuir esse viés, eliminaríamos do processo de análise os participantes
que responderam as perguntas de forma superficial, apenas repetindo a literatura. Na prática,
isso não aconteceu. Em sua maioria, poucas foram as respostas nas quais os participantes
foram superficiais. Nestes casos, essas respostas foram eliminadas da análise.

\subsection{Quantidade de participantes insuficiente}

Apesar de termos feito contato
com diversas empresas e grupos de desenvolvimento de software,
objetivando encontrar um bom número de participantes para a pesquisa,
a quantidade de participantes final do estudo pode não ser suficiente para generalizar
os resultados encontrados. 

\section{Validade de Conclusão}

A validade de conclusão discute se os pontos as quais a pesquisa chegou realmente
fazem sentido.

\subsection{Padrões encontrados}

Os padrões levantados pelos participantes durante o processo de entrevistas
foi revisado pelos autores desta pesquisa e, ao final, consideramos que todos
eles fazem sentido.
No entanto, podem haver ainda mais padrões a serem descobertos.

