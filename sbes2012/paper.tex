%% ------------------------------------------------------------------------- %%
\documentclass[conference]{IEEEtran}

\usepackage[numbers,square,sort,nonamebreak,comma]{natbib}  % citação
\usepackage{framed}
\usepackage[table]{xcolor}
\usepackage[T1]{fontenc}
\usepackage[brazil, english]{babel}
\usepackage[utf8]{inputenc}
\usepackage[pdftex,plainpages=false,pdfpagelabels,pagebackref,colorlinks=true,citecolor=black,linkcolor=black,urlcolor=black,filecolor=black,bookmarksopen=true]{hyperref} % links em preto
\urlstyle{same}

\begin{document}
	
\title{Como a Prática de TDD Influencia o Projeto de Classes em Sistemas Orientados a Objetos: \\Padrões de \textit{Feedback} para o Desenvolvedor}

\author{\IEEEauthorblockN{Mauricio Finavaro Aniche, Marco Aurélio Gerosa}
\IEEEauthorblockA{Instituto de Matemática e Estatística\\
Universidade de São Paulo\\
\{aniche, gerosa\}@ime.usp.br}}

\maketitle

\selectlanguage{brazil}

\begin{abstract} 
	Apesar de Desenvolvimento Guiado por Testes (TDD) soar como uma prática de testes de software, 
	muitos desenvolvedores afirmam que a prática influencia no projeto de classes.
	Este trabalho tem por objetivo compreender melhor os efeitos de TDD e como sua prática 
	influencia o desenvolvedor durante o projeto de sistemas orientados a objetos.
	Conduzimos um estudo exploratório que faz uso de uma combinação entre um experimento controlado 
	inicial e um estudo qualitativo com desenvolvedores da indústria de software brasileira.
	Ao final, observamos que a prática de TDD guia o desenvolvedor durante o processo
	de criação do projeto de classes por meio de constante \textit{feedback} sobre a qualidade
	do projeto. Este trabalho catalogou e nomeou os padrões de
	\textit{feedback} percebidos pelos participantes.
\end{abstract}

\vspace{0.2cm}
\selectlanguage{english}

\begin{abstract} 
	Despite Test-Driven Development (TDD) appears to be a software testing practice, 
	many developers affirm that the practice influence on class design. This study aims
	to better understand the effects of TDD and how the practice influences developers
	during class design on object-oriented systems. We conducted an exploratory study that makes
	use of a combination between an initial controlled experiment and a qualitative study
	with developers of the brazilian industry. We notice that the practice of TDD drives
	developers during class design by means of constant feedback about its quality. This study
	also named and catalogued feedback patterns perceived by the developers.
\end{abstract}

\selectlanguage{brazil}

\IEEEpeerreviewmaketitle

%% ------------------------------------------------------------------------- %%
\section{Introdução}

Desenvolvimento Guiado por Testes, tradução do termo
em inglês \textit{Test-Driven Development (TDD)},
é uma das práticas sugeridas pela Programação
Extrema (XP) \cite{XPExplained}. A prática é baseada em um pequeno ciclo, 
no qual o desenvolvedor escreve um teste antes
de implementar a funcionalidade esperada e, depois, com o código
passando no recém-criado teste, refatora para 
remover possíveis duplicação de dados e de código \cite{TDDByExample}.

A adoção de TDD pela indústria tem crescido. 
Em um questionário de 2010 para descobrir quais práticas eram feitas por times
ágeis, Scott Ambler mostrou que 53\% dos times ágeis
adotaram TDD como uma maneira para validar o trabalho feito \cite{wambler-survey-agile}.
Números similares podem ser observados nos questionários anuais da Version One, que,
em sua versão de 2012 \cite{versionone-2012} mostrou que 40\% dos times ágeis têm 
feito uso da prática.

Com a prática de TDD, um desenvolvedor só escreve código
que seja coberto por um teste. Por esse motivo, é comum relacionar a prática
de TDD com testes de software. Mas um discurso comum entre os praticantes de TDD
na indústria é os efeitos da prática também sobre a qualidade interna do código.
Muitos autores de livros conhecidos pela indústria e academia, como
Kent Beck \cite{TDDByExample}, Robert Martin \cite{agile-ppp}, 
Steve Freeman \cite{GOOS} e Dave Astels \cite{astels-tdd}, afirmam que a prática de TDD
promove uma melhoria significativa no projeto de classes, auxiliando
o programador a criar classes mais coesas e menos acopladas.

Entretanto, a maneira na qual a prática de TDD guia o desenvolvedor
durante o processo de criação do projeto de classes não é clara. Observamos
isso em nosso estudo qualitativo com praticantes de TDD, feito dentro de um
evento de desenvolvimento ágil brasileiro, no qual entrevistamos dez
participantes da conferência sobre os efeitos de TDD. E, para nossa surpresa,
nenhum soube afirmar, com clareza, como a prática os guia em direção
a um bom projeto de classes \cite{aniche-wbma}.
Siniaalto e Abrahamsson \cite{alarming-results} também
compartilham dessa opinião e, além disso, notaram que os efeitos de TDD podem 
não ser tão automáticos ou evidentes como o esperado.
Os trabalhos relacionados, discutidos na Seção \ref{cap:trabalhos-relacionados},
apenas avaliam se a prática de TDD faz diferença na qualidade dos códigos produzidos.
Poucos deles possuem um estudo qualitativo, detalhando como a prática
faz tal diferença.
Com essa informação em mãos, os desenvolvedores saberiam, de forma mais clara,
como utilizar a prática de TDD para obter uma maior qualidade no processo de criação
do projeto de classes. 

Para entender essas razões é necessário
conduzir uma pesquisa no mundo real, o que  
implica um equilíbrio entre o nível de controle
e o grau de realismo. Uma situação realista é, geralmente, complexa e 
não determinística, dificultando o entendimento sobre o que acontece. Por outro
lado, aumentar o controle sobre o experimento reduz o grau de realismo, muitas
vezes fazendo com que os reais fatores de influência fiquem fora do escopo do 
estudo \cite{guidelines-case-study}.

Baseando-se no fato de que o processo de desenvolvimento de software envolve 
diversos fatores humanos e é totalmente sensível ao contexto em que ele está 
inserido, 
este estudo exploratório fez uso de uma combinação entre um experimento controlado inicial, 
no qual participantes foram convidados a resolver exercícios utilizando TDD e, 
a partir dos dados colhidos, um outro estudo qualitativo foi 
feito objetivando entender como a prática influenciou as decisões de projeto 
dos participantes.

%% ------------------------------------------------------------------------- %%
\section{Desenvolvimento Guiado por Testes}

Métodos ágeis de desenvolvimento de software focam em constante
\textit{feedback}, seja ele da equipe em relação ao cliente, seja da
qualidade (interna e externa) do código produzido à equipe \cite{AgileManifesto}.
Com isso, muitas das práticas sugeridas por métodos ágeis visam aumentar a 
quantidade e a qualidade desse \textit{feedback}; a ideia da programação pareada, por
exemplo, é dar \textit{feedback} sobre o código durante sua escrita.

Desenvolvimento Guiado por Testes (TDD), prática popularizada por Kent Beck por meio de seu livro
\textit{TDD: By Example} em 2001 \cite{TDDByExample}, é mais uma das práticas
ágeis na qual o foco é dar \textit{feedback}. TDD tem grande importância durante o ciclo
de desenvolvimento uma vez que, conforme sugerido pelas práticas ágeis, o projeto de classes de um
software deve emergir à medida que o software cresce. E, para responder
rapidamente a essa evolução, é necessário um constante \textit{feedback} sobre a
qualidade interna e externa do código.

TDD é uma prática de desenvolvimento de software que se baseia na repetição de
um pequeno ciclo de atividades. Primeiro, o desenvolvedor escreve um
teste que falha. Em seguida, o faz passar, implementando a
funcionalidade desejada. Por fim, refatora o código para remover
duplicações de dados ou de código geradas pelo processo.
Além disso, simplicidade deve ser também algo intrínseco ao processo; o praticante
busca escrever o teste mais simples que falhe e escrever a implementação mais simples
que faça o teste passar.
Esse ciclo
é também conhecido como 
"Vermelho-Verde-Refatora" (ou \textit{"Red-Green-Refactor"}), uma vez que lembra as cores que um 
desenvolvedor normalmente vê quando faz TDD: o vermelho significa que
o teste está falhando, e o verde que o teste foi executado com sucesso.

É comum relacionar TDD a práticas de testes de software. 
Embora a criação de testes seja algo intrínseco ao processo, é dito que TDD também 
auxilia o desenvolvedor a criar classes mais flexíveis, mais coesas e
menos acopladas. Os testes são a ferramenta que o programador utiliza para
validar o projeto de classes criado. Por esse motivo, muitos se referem a TDD como
\textit{Projeto de Classes Guiado por Testes} \cite{tdd-taxonomy}.

Autores como Kent Beck \cite{aim-fire}, Dave Astels \cite{astels-tdd} e
Robert Martin \cite{bob-martin} afirmam que TDD é, na verdade, uma prática de
projeto de classes \cite{tdd-taxonomy} \cite{aim-fire}.
Na opinião desses autores, a mudança na ordem do ciclo de
desenvolvimento tradicional, apesar de simples, agrega diversos outros
benefícios ao código produzido: maior simplicidade, menor acoplamento e maior
coesão das classes criadas, levando a um melhor projeto de classes. 
Ward Cunningham, um dos pioneiros da Programação Extrema, resume essa 
discussão em uma frase: \textit{"Test-First programming is not a testing technique"} 
que, em uma tradução livre, significa \textit{"Escrever primeiro os testes
não é uma prática de testes"} \cite{aim-fire}.
Segundo Janzen, uma definição mais clara é a de que TDD é a arte de produzir testes
automatizados para código de produção, usando esse processo para guiar o 
projeto e a programação \cite{agilealliance-tdd} \cite{tdd-taxonomy}.

É um fato conhecido que projetos de classe tendem a perder qualidade conforme o projeto
evolui. Qualquer prática que visa reduzir a perda de qualidade deve ser avaliado afim
de diminuir o custo de manutenção do projeto. No entanto, é difícil discutir qualidade
em projetos de classe. No contexto deste trabalho, utilizamos então os princípios de projeto
de classes levantados por Martin \cite{bob-martin}, e que são citados na seção abaixo.

%% ------------------------------------------------------------------------- %%
\section{Degradação do Projeto de Classes}

Diz-se que um projeto de classes está \textit{degradando}
quando o mesmo começa a ficar difícil de evoluir, o reúso de código se 
torna mais complicado do que repetir o trecho de código, ou o custo de se fazer 
qualquer alteração no projeto de classes se torna alto.

Martin \cite{bob-martin} enumerou alguns sintomas de projeto de classes em degradação, 
chamados também de \textit{"maus cheiros"} de projeto de classes. Esses sintomas são parecidos com os 
maus cheiros de código (\textit{"code smells"}), mas em um nível mais alto: eles
estão presentes na estrutura geral do software em vez de estarem localizados
em apenas um pequeno trecho de código.

Esses sintomas podem ser medidos de forma subjetiva e algumas vezes de forma 
até objetiva. Geralmente, esses sintomas são causados por violações de um ou 
mais princípios de projeto de classes.
Neste trabalho, fazemos uso dos sintomas por ele levantados: rigidez, fragilidade, imobilidade, 
viscosidade, complexidade desnecessária e repetição desnecessária. Além disso, referenciamos também
os princípios de projeto conhecidos como SOLID, 
como o Princípio da Responsabilidade Única (PRU), Princípio do Aberto-Fechado (PAF),
Princípio da Substituição de Liskov (PSL), Princípio da Segregação de Interfaces (PSI) e Princípio da
Inversão de Dependências (PID) \cite{bob-martin}.

%% ------------------------------------------------------------------------- %%
\section{Trabalhos Relacionados}
\label{cap:trabalhos-relacionados}

Muitos estudos empíricos já foram realizados para avaliar os efeitos de TDD.
Em grande parte deles, os efeitos da prática no projeto de classes não é 
levado em conta, e apenas os efeitos da prática na qualidade externa são medidos.
Além disso, diferentemente
do que esta pesquisa propõe, muitos desses estudos optaram por um
maior controle no experimento, e os realizaram dentro de ambientes acadêmicos 
com estudantes dos mais diversos cursos de computação.

Janzen \cite{janzen-arch-improvement} 
apontou que a complexidade dos algoritmos era muito menor e a quantidade e
cobertura dos testes era maior nos códigos escritos com TDD.
Langr \cite{langr} apontou que TDD aumenta a qualidade do código, provê uma 
facilidade maior de manutenção e ajuda a produzir 33\% mais testes comparado a
abordagens tradicionais.

O estudo feito por George e Williams \cite{george-e-williams} mostrou que,
apesar de TDD poder reduzir inicialmente a produtividade dos desenvolvedores 
mais inexperientes, uma análise
qualitativa mostrou que 92\% pensam que TDD ajuda a manter um
código de maior qualidade e 79\% acreditam que ele promove um projeto 
de classes mais simples.

Um estudo feito por Erdogmus \textit{et al.} \cite{erdogmus-morisio} com 24 estudantes de
graduação mostrou que TDD aumenta a produtividade. Entretanto, nenhuma diferença 
de qualidade no código foi encontrada.

Outro estudo feito por Janzen \cite{janzen-saiedian} com três diferentes grupos
de alunos (cada um deles usando uma abordagem diferente: TDD, testes depois, sem
testes) mostrou que o código produzido pelo time que fez TDD usou melhor os
conceitos de orientação a objetos e as responsabilidades foram separadas em 
diferentes classes, enquanto os outros times produziram um código mais
procedural. 
As classes testadas tinham valores de acoplamento 104\% menor do 
que as classes não testadas e os métodos eram, na média, 43\% menos complexos 
do que os não-testados.

Dogsa e Batic \cite{dogsa-batic} também encontraram uma melhora no
projeto de classes feita com TDD. Mas, segundo os autores, essa melhora é 
consequência da simplicidade que a prática de TDD agrega ao processo. Eles
também  afirmaram que a bateria de testes de regressão gerada durante a prática 
favorece a constante refatoração do código.

Li \cite{angela-li} propôs um estudo qualitativo para
entender a eficácia de TDD. Por meio de um estudo de caso, ela coletou as 
percepções de benefícios que praticantes de TDD têm sobre a prática. Para isso ela
fez uso de cinco entrevistas semi-estruturadas realizadas em empresas de software de 
Auckland, Nova Zelândia. Os resultados das entrevistas foram analisados e alinhados
com qualidade de código,
qualidade da aplicação e produtividade do desenvolvedor.
No que diz respeito à qualidade de código, Li chegou a conclusão de
que TDD guia o desenvolvedor para classes mais simples e com melhor projeto de classes. 
Além disso, o código tende a ser mais simples e fácil de ler.
De acordo com o trabalho, os principais fatores que contribuem para esses benefícios
é a maior confiança em refatorar e modificar código, uma maior cobertura de testes,
entendimento mais profundo dos requisitos, maior facilidade na compreensão do código,
grau e escopo de erros reduzidos, além de uma maior satisfação pessoal do desenvolvedor.

O praticante de TDD geralmente faz uso também de outras práticas ágeis, como
programação pareada, o que dificulta a avaliação dos benefícios
de TDD. Madeyski \cite{madeyski-package-dependencies} observou os resultados
entre grupos que praticavam TDD, grupos que praticavam programação pareada, 
e a combinação entre elas,
e não conseguiu mostrar grande diferença entre equipes que utilizam programação 
pareada e equipes que utilizam TDD, no que diz respeito ao gerenciamento de dependências entre 
pacotes de classes. Entretanto, ao combinar os resultados, Madeyski encontrou que TDD pode 
ajudar no nível de gerenciamento de dependências entre classes. Segundo ele, o 
programador deve utilizar TDD, mas ficar atento a possíveis problemas de projeto de classes.

O estudo de Muller e Hagner \cite{muller-e-hagner} apontou que TDD não resulta
em melhor qualidade ou produtividade. Entretanto, os estudantes avaliados perceberam um 
melhor reúso dos códigos produzidos com TDD. Steinberg \cite{steinberg} mostrou
que código produzido com TDD é mais coeso e menos acoplado. Os estudantes também
reportaram que os defeitos eram mais fáceis de serem corrigidos. 

%% ------------------------------------------------------------------------- %%
\subsection{Discussão}

Como apresentado, poucos trabalhos avaliam os efeitos de TDD sobre o
projeto de classes. Quando o fazem, apenas discutem quais os efeitos da prática
e não exatamente \textbf{como} TDD os influencia. Josefsson
\cite{josefsson}, em sua discussão sobre a necessidade de uma fase de projeto
arquitetural e os efeitos de TDD nesse quesito, chega à mesma conclusão. Segundo
ele, os estudos sobre TDD encontrados na literatura atual são muito limitados, e
por esse motivo, os ditos efeitos que TDD têm 
sobre o projeto de classes não podem ser explicados. Com base no levantamento
bibliográfico realizado, acreditamos que esta limitação se mantém.

Grande parte desses estudos também não levam em conta a experiência do
programador que está praticando TDD. Geralmente esse ponto é discutido apenas 
na seção de ameaças à validade do estudo. Janzen, em seu doutorado, percebeu que
desenvolvedores mais maduros obtêm mais benefícios de TDD, escrevendo classes
mais simples. Além disso, desenvolvedores maduros que experimentam a prática
tendem a optar por TDD mais do que desenvolvedores menos experientes
\cite{janzen-phd}.

Os trabalhos que analisam TDD do ponto de vista de projeto de classes, no entanto, não
chegam a resultados conclusivos; muitos deles dizem que os efeitos
de TDD não são tão diferentes daqueles dos times que não praticam TDD.  A própria tese de
doutorado de Janzen foi inconclusiva no que diz respeito à influência de TDD no 
acoplamento e na coesão \cite{janzen-phd}. 

Além disso, outro ponto fortemente relacionado com projeto de classes é a simplicidade e
facilidade de evolução. Um projeto de classes rígido, não favorável a mudanças,  é difícil de ser 
avaliado de maneira quantitativa. Complexidade
desnecessária também é totalmente subjetiva. 

Portanto, é necessário mais do que uma comparação analítica; o
ponto de vista dos desenvolvedores deve ser levado em consideração.

%% ------------------------------------------------------------------------- %%
\section{Planejamento e Execução do Estudo} 
\label{sec:planejamento}

Conduzir um estudo exploratório experimental em engenharia de software sempre foi uma
atividade difícil. Uma das razões para isso é o fator humano, muito presente 
no processo de desenvolvimento de software, como sugerido por métodos ágeis  em
geral \cite{AgileManifesto}. Dessa maneira, o paradigma de pesquisa analítico 
não é suficiente para investigar casos reais complexos envolvendo pessoas e 
suas interações com a tecnologia \cite{guidelines-case-study}.

Uma pesquisa qualitativa é um meio para se explorar e entender a influência que 
indivíduos ou grupos atribuem a um problema social ou humano. O processo de
pesquisa envolve questões emergentes e procedimentos, dados geralmente colhidos
sob o ponto de vista do participante, com a análise feita de maneira indutiva
indo geralmente de um tema específico para um tema geral e com o pesquisador
fazendo interpretações do significado desses dados. Dados capturados por estudos
qualitativos são representados por palavras e figuras.
O relatório final tem uma estrutura flexível e os pesquisadores que se
dedicam a essa forma de pesquisa apoiam uma maneira de olhar para a pesquisa que
honra o estilo indutivo e a importância de mostrar a 
complexidade de uma situação \cite{creswell}.

Conforme discutido na Seção \ref{cap:trabalhos-relacionados}, muitos 
trabalhos avaliaram TDD, e alguns deles relatam inclusive uma melhora
no projeto de classes, como um menor acoplamento, uma maior coesão, e até mesmo
mais simplicidade. 
Grande parte deles focam nos efeitos da prática no código final, mas poucos 
estudos tentam entender a possível influência da experiência
nos resultados encontrados, e como TDD realmente guia o programador 
em direção a essas melhorias.

Para entendê-las, este estudo faz uso de uma combinação entre um experimento 
controlado inicial, no qual participantes foram convidados a resolver exercícios 
pré-preparados utilizando TDD e, a partir dos dados colhidos nesse estudo, um outro
estudo qualitativo entrou em detalhes sobre como a prática influenciou as decisões de 
projeto de classes dos participantes. Este capítulo detalha o planejamento do estudo, 
bem como o processo de análise dos dados colhidos.

%% ------------------------------------------------------------------------- %%
\subsection{Questões de pesquisa}

Os objetivo principal deste estudo é \textbf{entender a relação da prática de TDD 
e as decisões de projeto de classes tomadas pelo programador durante o 
projeto de sistemas orientados a objetos}.
Para compreendê-la, objetivamos responder às questões listadas
abaixo:

\begin{enumerate}

	\item Qual a influência de TDD no projeto de classes?

	\item Qual a relação entre TDD e as tomadas de decisões de projeto de classes
	feitas por um desenvolvedor?

	\item Como a prática de TDD influencia o programador no 
	projeto de classes, do ponto de vista do acoplamento, coesão e complexidade?

\end{enumerate}

%% ------------------------------------------------------------------------- %%
\subsection{Projeto da pesquisa}

Participantes de diferentes empresas de desenvolvimento de software do mercado
brasileiro foram selecionados. O perfil dos participantes é 
discutido na sub-seção \ref{sec:planejamento-participantes}. 
Todos eles foram solicitados a resolver 
alguns problemas utilizando Java, dentro de um período de tempo limitado. 
Os participantes utilizaram TDD em um problema, e não utilizaram
no outro. Os problemas resolvidos bem como em qual deles o participante
deveria utilizar TDD foram randomizados, a fim de diminuir o problema do aprendizado.

Todas as implementações feitas foram salvas, para posterior
cálculo de métricas de código. 
As métricas utilizadas foram: complexidade ciclomática \cite{mccabe}, \textit{Fan-Out} \cite{lorenz},
falta de coesão dos métodos \cite{lcom-hs}, quantidade de linhas por método e quantidade de métodos.

Todas as métricas citadas já são de uso conhecido na academia. Para calcular essas
métricas, nós implementamos nossa própria ferramenta. O motivo para tal é que
grande parte das ferramentas existentes fazem uso de código compilado, e não
apenas do código-fonte. Nossa ferramenta possui bateria de testes automatizados
e código-fonte aberto \footnote{\url{http://www.github.com/mauricioaniche/msr-asserts}. 
Último acesso em 10 de Fevereiro de 2012.}.

Além disso, dois especialistas foram convidados a analisar os códigos-fonte e a dar notas para cada
um deles. Apesar das métricas de código nos darem informações
preciosas sobre a qualidade do código, a opinião de um especialista, baseada
em sua experiência passada, é bastante enriquecedora.

As categorias nas quais eles deveriam avaliar eram: \textit{Simplicidade}, \textit{Testabilidade} e
\textit{Qualidade do Projeto de Classes}.
Em cada uma dessas categorias, os especialistas puderam dar notas entre
1 (ruim) e 5 (bom), ou optar por não avaliar aquele exercício.
Como alguns participantes não terminaram o exercício, o especialista
foi avisado de que ele deveria avaliar inclusive a intenção de projeto de classes criado
pelo participante, e não só o código atual. 
Para que a opinião do especialista fosse imparcial, ele \textbf{não} sabia a qual grupo
pertencia e como cada código-fonte analisado foi desenvolvido (com ou sem a prática de TDD).

Ao final do exercício, todos participantes
responderam um questionário sobre seu desempenho na resolução dos problemas.
Em seguida, uma análise inicial serviu para filtrar os candidatos
que foram posteriormente entrevistados. A seleção dos candidatos foi baseada
através do conjunto de respostas dadas no questionário mais código-fonte gerado. Candidatos
que criaram boas soluções em um exercício, mas não em outro, eram selecionados, por exemplo.
Além disso, candidatos que mencionaram possíveis efeitos da prática de TDD no projeto de classes
também foram incluídos.

A entrevista foi semi-estruturada, dando liberdade ao
pesquisador para mudar o rumo das perguntas, caso se fizesse necessário.
Além disso, todas as perguntas foram abertas, permitindo que o desenvolvedor desse
uma resposta ampla sobre o assunto.

Uma vez que as decisões tomadas por um programador durante a atividade de projeto de classes
podem ser influenciadas por vários diferentes fatores, 
as perguntas foram feitas de modo que o participante triangulasse suas respostas,
e tentasse isolar o máximo possível a atividade de TDD dos outros possíveis fatores
de influência. Participantes que não articulassem bem suas respostas seriam eliminados
durante o processo de análise.

Todas as entrevistas foram gravadas para que pudéssemos fazer a
transcrição e rever os dados a qualquer momento durante o processo. Além disso,
também tomamos notas, capturando informações como reações dos 
participantes a determinadas perguntas, ou qualquer outra informação relevante. 
As entrevistas também foram feitas em dias diferentes de acordo com a disponibilidade
de cada participante.

%% ------------------------------------------------------------------------- %%
\subsection{Participantes da pesquisa}
\label{sec:planejamento-participantes}

Desenvolvedores atuantes no mercado de 
software brasileiro foram selecionados para participarem da pesquisa.
Os participantes foram convidados e avaliados de acordo com sua experiência em TDD,
em desenvolvimento de software, em Java e em testes de unidade. A única exigência
era que o desenvolvedor já soubesse como escrever testes de unidade.

Esses pontos foram avaliados por meio de um questionário, 
respondido por todos os participantes antes do início do estudo. 
Esse questionário, além de perguntar qual a experiência
do participante (de maneira quantitativa, em anos), 
continha questões nas quais o participante
podia falar sobre sua expêriencia em projeto orientado a objetos,
Java e TDD de forma mais aberta.

Ao todo tivemos 25 participantes, de 6 diferentes empresas.
Os participantes, em sua maioria, eram pessoas com pouca experiência em TDD.
40\% deles disseram utilizar a prática há no máximo um ano. 52\% deles praticam TDD
entre 1 e 3 anos. Apenas 4\% praticou entre três e quatro anos, e nenhum participante
possuía mais experiência do que isso. 

Os números são um pouco diferentes quando se trata da experiência em desenvolvimento
de software. 24\% dos participantes desenvolve software entre 4 e 5 anos.
28\% deles faz isso entre 6 e 10 anos. 20\% possui até 2 anos de experiência.

Entrando em aspectos mais técnicos, 64\% dos participantes afirmaram programar em Java. Entretanto,
36\% disseram que não trabalham com Java no seu dia a dia. Todos eles afirmam conhecer JUnit,
e 64\% deles aplicam objetos dublês\footnote{Objetos dublê ou, do inglês, 
\textit{mock objects}, são objetos criados durante um teste de unidade, e que imitam o comportamento de um
outro objeto concreto. Geralmente são muito utilizados para isolar um teste de outras classes
do sistema. Mais informações sobre objetos dublês são encontradas em \cite{mocks}.}
durante suas atividades de desenvolvimento. Só 12\% dizem nunca ter ouvido falar sobre o conceito de objetos dublês. 
Com relação a conhecimentos
em orientação a objetos, na pergunta aberta do questionário, grande parte deles 
afirmou que possuem uma boa experiência e alguns
chegam até a afirmar que dominam o assunto. Poucos disseram que possuem conhecimentos
básicos.

Em relação à experiência com TDD,
podemos afirmar que metade dos participantes ainda está experimentando a prática, enquanto
outros já a tem mais consolidada. Isso é positivo, já que foi possível capturar informações
da prática de TDD por pessoas com diferentes níveis de maturidade.

Em relação ao alto número de pessoas que não utilizam Java, isso se deve ao fato de uma das
empresas fazer uso de PHP para seu trabalho do dia a dia. No entanto, nós conhecemos a equipe
e verificamos que, apesar de não utilizarem a linguagem constantemente, eles não tiveram
problema algum durante a execução dos exercícios.

%% ------------------------------------------------------------------------- %%
\subsection{Problemas Propostos}
\label{sec:exercicios}

Foram propostos quatro problemas que deveriam ser resolvidos pelos participantes, utilizando
linguagem Java. O objetivo desses exercícios foi simular problemas de projeto de classes 
recorrentes em diversos projetos de software. 
Na Tabela \ref{tab:problemas-exercicios}, apresentamos a relação entre uma má
implementação dos exercícios e os princípios de projeto de classes feridos por
ela.

Foi dito ao participante que os exercícios simulam problemas do mundo real, e ele deveria
ter em mente que as soluções geradas supostamente seriam mantidas por uma outra equipe.
Por esse motivo, foi solicitado ao participante que implementasse a solução mais elegante e flexível 
possível.

\begin{table}
	\centering
	\begin{tabular}{| l | l | l | }
		\hline
		Exercício & Mau Cheiro & Princípios A\\
		& & Serem Seguidos\\
		
		\hline
		
		Exercício 1 & Rigidez, Complexidade Desnecessária & PRU, PAF \\
		Exercício 2 & Fragilidade, Viscosidade, Imobilidade & PRU, PID, PAF \\
		Exercício 3 & Rigidez, Fragilidade & PRU\\
		Exercício 4 & Fragilidade, Viscosidade, Imobilidade & PAF, PRU, PID \\
		
		\hline
	\end{tabular}
	\caption{Exercícios propostos e mau cheiros de projeto de classes}
	\label{tab:problemas-exercicios}
\end{table}

Os exercícios propostos foram baseados em um workshop criado pelo autor desta pesquisa, e o mesmo
foi aplicado para 2 turmas diferentes, uma delas dentro do Agile Brazil 2011, o
maior evento brasileiro de métodos ágeis, que tinha um público heterogêneo, e uma delas para
uma das turmas do curso de Ciência da Computação do Instituto de Matemática e Estatística da Universidade
de São Paulo, na qual o público era constituído em sua maioria de alunos de graduação. 

%% ------------------------------------------------------------------------- %%
\section{Análise Quantitativa}

Na tentativa de triangular as informações levantadas pela análise qualitativa,
calculamos métricas em cima dos códigos gerados, para verificar se houve
alguma diferença na qualidade dos códigos gerados com e sem a prática de TDD.
Ao total, analisamos
264 classes de produção (831 metodos, totalizando 2520 linhas) e
73 classes de teste (225 metodos, totalizando 1832 linhas).

O teste estatístico escolhido foi o Wilcoxon. Ele é um teste de hipótese não paramétrico,
utilizado para comparar duas amostras e verificar se as médias entre elas
são diferentes. Portanto, utilizamos Wilcoxon para comparar se a diferença entre a média
das métricas dos códigos gerados com TDD e sem TDD é significativamente diferente.


%% ------------------------------------------------------------------------- %%
\subsection{Métricas de código}

Na Tabela \ref{metricas-industria}, mostramos os \textit{p-values} encontrados para
a diferença entre códigos produzidos com e sem TDD. 
Pelos números, 
observamos que em nenhum exercício houve diferença significativa nas métricas
de complexidade ciclomática e acoplamento eferente. Já a métrica de falta
de coesão dos métodos apresentou diferenças em dois exercícios (1 e 4). 
A diferença também apareceu na quantidade de linhas por método (exercício 4)
e quantidade de métodos (exercício 1). Ao olhar os dados de todos os exercícios
juntos, nenhuma métrica apontou uma diferença significativa.
Isso nos mostra que, ao menos quantitativamente, a prática de TDD não fez
diferença nas métricas de código.

\begin{table*}
	\centering
	\begin{tabular}{ | p{3cm} | p{2cm} | p{2cm} | p{2cm} | p{2cm} | p{2cm} |}
		\hline
		Exercício & Complexi- dade ciclomática & Acoplamento eferente & Falta de coesão dos métodos & Número de linhas por método 
		& Quantidade de métodos por classe \\
		\hline
		Exercício 1 &	0.8967	&	0.6741 &	\cellcolor[gray]{0.8}2.04E-07* &	0.4962 &	\cellcolor[gray]{0.8}2.99E-06* \\
		Exercício 2	& 0.7868	&	0.7640 &	0.06132 &	0.9925 &	0.7501 \\
		Exercício 3	& 0.5463	&	0.9872 &	0.5471 &	0.7216 &	0.3972\\
		Exercício 4	& 0.2198	&	0.1361 &	\cellcolor[gray]{0.8}0.04891* &	\cellcolor[gray]{0.8}0.0032* &	0.9358\\
		\hline
		Todos &	0.8123	&	0.5604 &	0.3278 &	0.06814 &	0.5849\\
		\hline
	\end{tabular}
	\caption{\textit{P-values} encontrados para a diferença entre códigos com e sem TDD na indústria}
	\label{metricas-industria}
\end{table*}

Já na Tabela \ref{valores-separados},
calculamos os \textit{p-values} das métricas, separando-as 
por experiência em desenvolvimento de software e TDD. Os valores para o grupo
experiente em TDD e não experiente em desenvolvimento de software não foram calculados, já que nenhum
participante se enquadrou nele.

Pelos números, percebemos 
que a métrica de coesão foi a única que apresentou uma diferença significativa entre desenvolvedores
experientes, tanto em TDD quanto em desenvolvimento de software.

\begin{table}
	\centering
	\begin{tabular}{ | p{3cm} | p{2cm} | p{2cm} | }
		\hline
		  & Experiente em TDD & Não experiente em TDD \\
		\hline
			\multicolumn{3}{|l|}{Complexidade Ciclomática} \\
		\hline
			Experiente em Desenvolvimento de Software 		& 0.09933	&	0.8976\\
			\hline
			Não Experiente em Desenvolvimento de Software 	& NA		&	0.4462\\
		\hline
			\multicolumn{3}{|l|}{\textit{Fan-Out}}\\
		\hline
			Experiente em Desenvolvimento de Software 		& 0.1401	&	0.6304\\
			\hline
			Não Experiente em Desenvolvimento de Software 	& NA		&	0.2092\\
		\hline
			\multicolumn{3}{|l|}{Falta de Coesão dos Métodos}\\
		\hline
			Experiente em Desenvolvimento de Software 		& \cellcolor[gray]{0.8}0.03061*	&	0.1284\\
			\hline
			Não Experiente em Desenvolvimento de Software 	& NA		&	0.0888\\
		\hline
			\multicolumn{3}{|l|}{Quantidade de Métodos por Classe} \\
		\hline
			Experiente em Desenvolvimento de Software 		& 0.09933	&	0.8976\\
			\hline
			Não Experiente em Desenvolvimento de Software 	& NA		&	0.4462\\
		\hline
			\multicolumn{3}{|l|}{Linhas por Método}\\
		\hline
			Experiente em Desenvolvimento de Software 		& 0.0513	&	0.4319\\
			\hline
			Não Experiente em Desenvolvimento de Software 	& NA		&	0.5776\\
		\hline
	\end{tabular}
	\caption{\textit{P-values} encontrados para a diferença das métricas entre experientes e não experientes na indústria}
	\label{valores-separados}
\end{table}

%% ------------------------------------------------------------------------- %%
\subsection{Especialistas}

Na Tabela
\ref{tab:especialistas-industria},
mostramos os \textit{p-values} encontrados para a diferença de avaliação dos especialistas
entre códigos produzidos com e sem TDD.

Pelos números, observamos que ambos os especialistas não encontraram diferenças 
entre códigos produzidos com e sem TDD. 

\begin{table}[h!]
	\centering
	\begin{tabular}{| p{2cm} | c | c | c | }
		\hline
		Especialista & Projeto de classes & Testabilidade & Simplicidade\\
		\hline
		Especialista 1 &	0.4263 &	0.5235 &	0.3320\\
		Especialista 2 &	0.7447 &	0.4591 &	0.9044\\
		\hline
	\end{tabular}
	\caption{\textit{P-values} encontrados para a diferença entre as análises dos especialistas com e sem TDD na indústria}
	\label{tab:especialistas-industria}
\end{table}

%% ------------------------------------------------------------------------- %%
\section{Análise Qualitativa}

Os valores apresentados anteriormente corroboram com muitos dos trabalhos relacionados. 
Aparentemente TDD não influencia a ponto de alterar 
de maneira significativa os valores das métricas de acoplamento, coesão e simplicidade.
Porém, isso é incoerente com o sentimento comum no mercado de que praticar TDD
traz benefícios para o projeto de classes. Conforme previsto, neste estudo conduzimos
uma etapa qualitativa para entender como se procede essa influência, do ponto
de vista dos desenvolvedores.

Nesta seção apresentamos e discutimos sobre a análise e interpretação dos dados qualitativos colhidos
na execução deste estudo. Em particular, na Seção 
\ref{padroes-tdd}, levantamos os padrões de \textit{feedback} que a prática de TDD
dá ao desenvolvedor.

Um ponto interessante a ser notado é que os participantes, independente de experiência
em TDD ou em desenvolvimento de software, comentaram pontos similares. Por esse motivo,
não separamos a discussão pelas categorias levantadas na Seção \ref{sec:planejamento}.

%% ------------------------------------------------------------------------- %%
\subsection{Análise das Entrevistas}

Diferente do esperado, a maioria absoluta dos participantes afirmou que 
a prática de TDD não faria com que seus projetos de classes fosse de alguma forma diferentes, caso tivessem
feito ambos os exercícios com a prática.
A principal justificativa dada pelos participantes foi que a experiência e o conhecimento prévio
em orientação a objetos os guiaram durante o processo de criação do projeto de classes. Nenhum dos
participantes, por exemplo, afirmou que um desenvolvedor sem conhecimento em alguma das áreas
citadas criaria um bom projeto de classes somente por praticar TDD.

Dois bons exemplos foram dados pelos participantes, que ajudam a reforçar esse ponto. Um deles
comentou que fez uso de um padrão de projetos \cite{gof} que aprendeu apenas alguns dias antes.
Outro participante mencionou que seus estudos sobre os princípios SOLID
o ajudaram durante os exercícios. Segue o trecho mencionado pelo participante:

\begin{framed}
\textit{"Até foi engraçado, eu estou lendo o Design Patterns (livro), e ele fala de polimorfismo, e foi
lá que eu mirei para fazer, porque eu nunca tinha feito nada assim (...), aqui dificilmente eu crio
coisa nova, só dou manutenção no código."}
\end{framed}

Além do mais, o único participante da indústria que nunca havia
praticado TDD afirmou que não sentiu diferença no processo de criação de classes durante
a prática.
Curioso é que esse mesmo participante que nunca praticou TDD afirmou que "sabia que TDD era uma prática de projeto de classes",
diferentemente dos participantes mais experientes que sempre afirmavam que TDD não é só uma prática de projeto de classes,
mas também de testes. Isso indica, de certa forma, que a popularidade dos efeitos de TDD no projeto de classes
é grande.

Entretanto, apesar de TDD não guiar o desenvolvedor diretamente para um bom projeto de classes,
todos eles afirmaram que enxergam benefícios na prática de TDD, mesmo do
ponto de vista de projeto de classes. Muitos deles, inclusive, mencionaram a dificuldade
de parar de usar TDD:

\begin{framed}
\textit{"Você vai fazer alguma coisa, você acaba pensando já nos testes que você vai fazer. É difícil 
falar assim: "programa sem pensar nos testes!" Depois que você acostuma, você não sabe outra
maneira de programar..."}
\end{framed}

Segundo eles, TDD pode ajudar no processo de projeto de classes, mas, para isso,
o desenvolvedor deve possuir certa experiência em desenvolvimento de software. 
Grande parte dos participantes afirmou que o 
projeto de classes criado surgiu de experiências e aprendizados passados.
Segundo eles, a melhor opção é unir a prática de TDD com a experiência:

\begin{framed}
\textit{"O ideal é somar as duas coisas [experiência e TDD] (...) 
Não acredito que TDD sozinho consiga fazer as coisas ficarem boas. Tem outros conceitos
para as coisas ficarem boas."}
\end{framed}

%% ------------------------------------------------------------------------- %%
\subsection{\textit{Feedback} mais rápido}

A grande maioria dos participantes também comentaram que uma diferença que percebem
no momento que praticam TDD é o \textit{feedback} mais constante. Na maneira
tradicional, o tempo entre a escrita do código de produção e o código
de testes é muito grande. O TDD, ao solicitar que o desenvolvedor
escreva o teste antes, também faz com que o desenvolvedor receba o \textit{feedback} que
os testes podem dar mais cedo:

\begin{framed}
\textit{"Você ia olhar para o teste, e falar: "Está legal? Não está?", e ia fazer de novo."}
\end{framed}

Um participante comentou que, com o teste, o desenvolvedor pode observar
e criticar o código que escreveu no momento logo após a escrita.
E essa crítica, de forma contínua, faz com que o desenvolvedor acabe
por pensar constantemente no código que está produzindo:

\begin{framed}
\textit{"Quando você faz o teste, você vê logo o que não gostou do método daquele jeito (...), você
não percebe isso até que você use o teste."}
\end{framed}

Diminuir o tempo entre a escrita do código e a escrita do teste também o ajuda a desenvolver código
que efetivamente resolve o problema. Segundo os participantes, na maneira tradicional, 
o desenvolvedor escreve muito código antes de saber se o mesmo funciona:

\begin{framed}
\textit{"[O teste] não é só uma especificação; ele tem que de fato funcionar. Então,
como você diminui muito o tempo entre escrever um programa que funcione e testar aquilo,
você consegue mais rápido ver se aquela parte pequena funciona ou não (...)"}
\end{framed}

%% ------------------------------------------------------------------------- %%
\subsection{Busca pela testabilidade}

Talvez o principal ponto pelo qual a prática ajude os desenvolvedores no projeto de classes 
seja pela constante busca pela testabilidade. É possível inferir que, quando se 
começa a escrita do código pelo seu teste, o código de produção deve ser, necessariamente,
possível de testar.

Por outro lado, quando o código não é fácil de ser testado, os desenvolvedores
entendem isso como um mau cheiro de projeto de classes. Quando isso acontece,
os desenvolvedores geralmente tentam refatorar o código para possibilitar que
os mesmos sejam testados mais facilmente.

Um dos participantes, inclusive, afirmou que leva isso como uma regra:
se está difícil testar, é possível melhorar:

\begin{framed}
\textit{"Eu utilizo isso como uma regra: sempre que está muito complexo [o teste],
acho que nós temos que parar e refatorar, porque, na minha opinião, dá
pra ficar mais simples."}
\end{framed}

Esse ponto, na verdade, já foi levantado antes por Feathers \cite{feathers-synergy}.
Quanto mais difícil for a escrita do teste, maior a chance da existência de
algum problema de projeto de classes. Segundo ele, 
existe uma sinergia muito grande entre uma classe com alta testabilidade e um bom projeto de classes: 
se o programador busca por testabilidade, acaba criando um bom projeto de classes; se 
busca por um bom projeto de classes, acaba escrevendo código mais
testável.

Na busca pela testabilidade, o desenvolvedor é encorajado a escrever um
código que seja facilmente testável. Códigos assim possuem algumas
características interessantes, como a facilidade para invocar o comportamento
esperado, a não necessidade de pré-condições complicadas e a explicitação de
todas as dependências que a classe possui.

Mas, os participantes foram ainda mais longe. Durante as entrevistas,
vários deles mencionaram diversos padrões que encontram no \textit{feedback} dos testes,
e que os fazem pensar sobre os possíveis problemas de acoplamento,
coesão, falta de abstração, etc., na classe que estão criando.
Esses padrões são melhor discutidos a seguir.

%% ------------------------------------------------------------------------- %%
\subsection{Padrões de \textit{Feedback} de TDD}
\label{padroes-tdd}

Como mencionado anteriormente, grande parte do \textit{feedback} que os testes
dão, acontece no momento em que o programador encontra dificuldades para a
escrita dos mesmos. Esta seção discute padrões levantados pelos praticantes
que os levam a crer que há um problema de projeto de classes no código
que está sendo testado.

%% ------------------------------------------------------------------------- %%
\subsubsection{Padrões Ligados à Coesão}

Quando um único método necessita de diversos testes para garantir seu comportamento,
o método em questão provavelmente é complexo e/ou possui diversas responsabilidades.
Códigos assim possuem geralmente diversos caminhos
diferentes e tendem a alterar muitos atributos internos do objeto, obrigando o desenvolvedor
a criar muitos testes, caso queira ter uma alta cobertura de testes.
A esse padrão, demos o nome de \textbf{Muitos Testes Para Um Método}.

O mesmo pode ser entendido quando o desenvolvedor escreve muitos testes para a 
classe como um todo. Classes que expõem muitos métodos para o mundo de fora
também tendem a possuir muitas responsabilidades. Chamamos este padrão
de \textbf{Muitos Testes Para Uma Classe}.

Outro problema de coesão pode ser encontrado quando o programador
sente a necessidade de escrever cenários de teste muito grandes para uma
única classe ou método. É possível inferir que essa necessidade surge 
em códigos que lidam com muitos objetos e fazem muita coisa. Nomeamos
esse padrão de \textbf{Cenário Muito Grande}.

Um padrão não explicitamente levantado pelos participantes, mas notado
por nós, é quando o desenvolvedor sente a necessidade de se testar
um método que não é público. Métodos privados geralmente servem para 
transformar o método público em algo mais fácil de ler. Ao desejar
testá-lo de maneira isolada, o programador pode estar de frente a
um método que possua uma responsabilidade suficiente para ser
alocada em uma outra classe. A esse padrão, chamamos de 
\textbf{Testes em Método Que Não É Público}.

%% ------------------------------------------------------------------------- %%
\subsubsection{Padrões Ligados ao Acoplamento}

O uso abusivo de objetos dublês para testar uma
única classe indica que a classe sob teste possui problemas
de acoplamento. É possível deduzir que uma classe que faz uso de muitos
objetos dublês depende de muitas classes, e portanto, tende a ser
uma classe instável. A esse padrão, demos o nome de \textbf{Objetos Dublê em Excesso}.

Outro padrão percebido por nós é a criação de objetos dublês que não
são utilizados em alguns métodos de testes. Isso geralmente acontece quando
a classe é altamente acoplada, e o resultado da ação de uma dependência não
interfere na outra. Quando isso acontece, o programador acaba por escrever
conjuntos de testes, sendo que alguns deles lidam com um sub-conjunto dos objetos dublês,
enquanto outros testes lidam com o outro sub-conjunto de objetos dublês. 
Isso indica um alto acoplamento 
da classe, que precisa ser refatorada. A esse padrão demos o nome de
\textbf{Objetos Dublê Não Utilizados}. Este padrão poderia também ser classificado
como um padrão de coesão, já que essa classe claramente possui também mais de uma
única responsabilidade.

%% ------------------------------------------------------------------------- %%
\subsubsection{Padrões Ligados à Abstrações Incorretas}

A falta de abstração geralmente faz com que uma simples mudança precise
ser feita em diferentes pontos do código. Quando uma mudança acontece e 
o programador é obrigado a fazer a mesma alteração em diferentes testes,
isso indica a falta de uma abstração correta para evitar a 
repetição desnecessária de código.
A esse padrão damos o nome de \textbf{Mesma Alteração Em Diferentes Testes}.
Analogamente, o programador pode perceber a mesma coisa
quando ele começa a criar testes repetidos para entidades diferentes.
Chamamos esse padrão de \textbf{Testes Repetidos Para Entidades Diferentes}.

Quando o desenvolvedor começa o teste e percebe que a interface pública da classe
não é fácil de ser utilizada, pode indicar que abstração
corrente não é clara o suficiente e poderia ser melhorada. A esse padrão,
chamamos de \textbf{Interface Não Amigável}.

Outro padrão não mencionado explícitamente pelos participantes 
é a existência da palavra \textit{"se"} no nome do teste. Testes que
possuem nomes como esse geralmente indicam a existência de um \textit{"if"} na implementação
do código de produção. Essas diversas condições podem, geralmente, ser refatoradas e,
por meio do uso de poliformismo, serem eliminadas. A falta de abstração nesse caso
é evidenciada pelo padrão \textbf{Condicional No Nome Do Teste}.

%% ------------------------------------------------------------------------- %%
\subsection{Relação dos padrões com os princípios de projeto de classes}

É possível relacionar os padrões de \textit{feedback} levantados pelos participantes
com os mau cheiros de projeto de classes comentados neste trabalho. 
Na Tabela \ref{tab:relacao-padroes},
mostramos essa relação, e como esses padrões podem efetivamente ajudar o desenvolvedor
a procurar por problemas no seu projeto de classes.


\begin{table}[h!]
	\centering
	\begin{tabular}{| p{2.5cm} | p{2.5cm} | p{2cm} | }
		\hline

		Padrão & Possíveis Mau Cheiros de Projeto de Classes & Possíveis Princípios Feridos\\
		
		\hline

		Muitos Testes Para Um Método                   & Complexidade Desnecessária, Opacidade   & PRU \\ \hline
		Muitos Testes Para Uma Classe                  & Complexidade Desnecessária, Opacidade   & PRU \\ \hline
		Cenário Muito Grande                           & Opacidade, Fragilidade                  & PRU \\ \hline
		Testes Em Método Que Não É Público             & Complexidade Desnecessária              & PRU, PAF \\ \hline
		Objetos Dublê em Excesso                       & Fragilidade                             & PID, PAF \\ \hline
		Objetos Dublês Não Utilizados                  & Fragilidade                             & PID, PAF \\ \hline
		Mesma Alteração Em Diferentes Testes           & Fragilidade, Rigidez                    & PRU \\ \hline
		Testes Idênticos Para Entidades Diferentes     & Repetição Desnecessária, Rigidez        & PRU  \\ \hline
		Interface Não Amigável                         & Opacidade                               & ISP \\ \hline
		Condicional No Nome Do Teste                   & Rigidez, Fragilidade                    & PRU, PAF \\

		\hline
		
	\end{tabular}
	\caption{Relação entre os padrões de \textit{feedback} de TDD e mau cheiros de projeto de classes}
	\label{tab:relacao-padroes}
\end{table}

%% ------------------------------------------------------------------------- %%
\section{Ameaças à Validade}
\label{cap:ameacas}

%% ------------------------------------------------------------------------- %%
\subsection{Validade de Construção}

%% ------------------------------------------------------------------------- %%
\subsubsection{Exercícios de pequeno porte}

Os exercícios propostos são pequenos perto de um projeto real. Entretanto, todos os exercícios propostos contém
problemas localizados de projeto de classes. E, uma vez que esta pesquisa tenta avaliar os 
efeitos de TDD no projeto de classes, 
acreditamos que os problemas conseguem simular de forma satisfatória
problemas de projeto de classes que desenvolvedores encaram no dia a dia de trabalho.

Além disso, ao final do exercício, os participantes responderam uma pergunta sobre a semelhança
entre os problemas de projeto de classes propostos e os problemas encontrados no mundo real.
Todos os participantes da indústria afirmaram que os problemas se parecem com os que eles enfrentam
no dia a dia de trabalho. 

%% ------------------------------------------------------------------------- %%
\subsection{Validade interna}

%% ------------------------------------------------------------------------- %%
\subsubsection{Efeitos recentes de TDD na memória}

Muitos dos participantes da indústria afirmaram que utilizam TDD no seu dia a dia de trabalho.
Isso pode fazer com que o participante
não avalie friamente as vantagens e desvantagens do desenvolvimento sem TDD. 

Para diminuir esse viés, os participantes fizeram alguns exercícios também
sem TDD, para que ambos os estilos de desenvolvimento (com e sem TDD) estivessem
recentes em sua memória.

%% ------------------------------------------------------------------------- %%
\subsubsection{Exercícios inacabados}

Alguns participantes não terminaram suas implementações dos exercícios. Isso
pode influenciar na análise quantitativa, afinal, um projeto de classes que
seria complexo assim que pronto, ao olho da métrica, pode aparentar ser simples.

%% ------------------------------------------------------------------------- %%
\subsubsection{Influência do pesquisador}

O pesquisador possui
um papel fundamental em pesquisas qualitativas. Mas isso pode fazer com que
a interpretação dos resultados seja influenciada pelo contexto, experiências,
e até viéses do próprio pesquisador.
Neste estudo, a nossa opinião teve forte influência na seleção dos candidatos
para a entrevista.
Para diminuir esse problema, revisamos todas as análises,
buscando por conclusões incorretas ou não tão claras. 

%% ------------------------------------------------------------------------- %%
\subsection{Validade externa}

%% ------------------------------------------------------------------------- %%
\subsubsection{Desejabilidade social}

Enviesamento pela desejabilidade social é o termo científico usado para descrever
a tendência de que alguns participantes respondam questões de modo que serão
bem vistos pelos outros membros da comunidade \cite{crowne}.
Métodos ágeis e TDD possuem um discurso forte. A comunidade brasileira de métodos
ágeis ainda é nova e percebe-se de maneira empírica que muitos repetem o discurso
sem grande experiência ou embasamento no assunto.
No caso desta pesquisa, um possível viés é o participante responder o que
a literatura diz sobre TDD, e não exatamente o que ele pratica e sente sobre
os efeitos da prática. 

Para diminuir esse viés, eliminaríamos do processo de análise os participantes
que responderam as perguntas de forma superficial, apenas repetindo a literatura. Na prática,
isso não aconteceu. Em sua maioria, poucas foram as respostas nas quais os participantes
foram superficiais. Nesses casos, essas respostas foram eliminadas da análise.

%% ------------------------------------------------------------------------- %%
\subsubsection{Quantidade de participantes insuficiente}

Apesar de termos feito contato
com diversas empresas e grupos de desenvolvimento de software,
objetivando encontrar um bom número de participantes para a pesquisa,
a quantidade de participantes final do estudo pode não ser suficiente para generalizar
os resultados encontrados. 

%% ------------------------------------------------------------------------- %%
\section{Conclusão}

Neste trabalho, discutimos e entendemos como a prática de TDD pode
fazer a diferença no dia a dia de um desenvolvedor de software,
trazendo um melhor significado à afirmação de que a prática melhora o projeto de classes.
Além de corroborar com os estudos quantativos da literatura, este estudo
foi além, e observou padrões de \textit{feedback} que aparecem
no momento em que o desenvolvedor utiliza TDD, e que, na prática, o guia durante
o desenvolvimento.

A prática de TDD \textbf{pode} influenciar no processo de criação do projeto de classes.
No entanto, ao contrário do que é comentado pela indústria,
\textbf{a prática de TDD não guia o desenvolvedor para um bom projeto de classes
de forma automática}; a experiência e conhecimento 
do desenvolvedor são fundamentais ao criar software orientado a objetos. 

A prática, por meio dos seus possíveis \textit{feedbacks} em relação ao
projeto de classes, discutidos
em profundidade na Seção \ref{padroes-tdd}, pode servir de guia
para o desenvolvedor. Esses \textit{feedbacks}, quando observados, fazem
com que o desenvolvedor perceba problemas de projeto de classes de
forma antecipada, facilitando a refatoração do código.

Ao escrever um teste de unidade para uma determinada classe, o desenvolvedor
é obrigado a passar sempre pelos mesmos passos. Todo teste de unidade é composto
de um conjunto de linhas responsáveis por montar o cenário do teste, um conjunto
de linhas que executam a ação sob teste e, por fim, um conjunto de linhas que
garantem que o comportamento foi executado de acordo com o esperado.

Uma dificuldade na escrita de qualquer um desses conjuntos pode implicar
em problemas no projeto de classes. Por exemplo, uma classe que para
ser testada necessita de grandes cenários, pode nos indicar que a classe
sob teste possui pré-condições muito complicadas. Já dificuldades na hora
de executar a ação sob teste pode nos indicar que a interface pública dessa
classe não é amigável. 

Espera-se que, com estes padrões catalogados, desenvolvedores atentem-se mais
aos possíveis \textit{feedbacks} que a prática de TDD fornece e, utilizem-os
para melhorar a qualidade dos seus projetos de classe. Um possível trabalho futuro
é a criação de ferramentas que automaticamente detectam esses padrões e notifiquem
o desenvolvedor sobre o possível mau cheiro de projeto. Além disso, a busca por
mais padrões como os levantados aqui pode aumentar ainda mais o \textit{feedback} dado
pela prática.

\textbf{Portanto, essa é a forma na qual a prática guia o desenvolvedor para
um melhor projeto de classes: dando retorno constante sobre os possíveis problemas
existentes no atual projeto de classes. É tarefa do desenvolvedor perceber
estes problemas e melhorar o projeto de acordo.}

%% ------------------------------------------------------------------------- %%
\section*{Agradecimentos}

Agradeçemos às empresas que aceitaram participar deste estudo: Bluesoft,
Amil, e WebGoal (São Paulo e Poços de Caldas). Além disso, agradecemos aos
desenvolvedores que participaram da pesquisa de forma independente.

%% ------------------------------------------------------------------------- %%
\bibliographystyle{IEEEtran}
\footnotesize{\bibliography{paper}}

% that's all folks
\end{document}

%% ------------------------------------------------------------------------- %